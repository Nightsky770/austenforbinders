%!TeX root=../sensetop.tex
\chapter[Chapter \thechapter]{}
\lettrine[lraise=0.3]{B}{arton} Park was about half a mile from the cottage. The ladies had passed near it in their way along the valley, but it was screened from their view at home by the projection of a hill. The house was large and handsome; and the Middletons lived in a style of equal hospitality and elegance. The former was for Sir John's gratification, the latter for that of his lady. They were scarcely ever without some friends staying with them in the house, and they kept more company of every kind than any other family in the neighbourhood. It was necessary to the happiness of both; for however dissimilar in temper and outward behaviour, they strongly resembled each other in that total want of talent and taste which confined their employments, unconnected with such as society produced, within a very narrow compass. Sir John was a sportsman, Lady Middleton a mother. He hunted and shot, and she humoured her children; and these were their only resources. Lady Middleton had the advantage of being able to spoil her children all the year round, while Sir John's independent employments were in existence only half the time. Continual engagements at home and abroad, however, supplied all the deficiencies of nature and education; supported the good spirits of Sir John, and gave exercise to the good breeding of his wife.

Lady Middleton piqued herself upon the elegance of her table, and of all her domestic arrangements; and from this kind of vanity was her greatest enjoyment in any of their parties. But Sir John's satisfaction in society was much more real; he delighted in collecting about him more young people than his house would hold, and the noisier they were the better was he pleased. He was a blessing to all the juvenile part of the neighbourhood, for in summer he was for ever forming parties to eat cold ham and chicken out of doors, and in winter his private balls were numerous enough for any young lady who was not suffering under the unsatiable appetite of fifteen.

The arrival of a new family in the country was always a matter of joy to him, and in every point of view he was charmed with the inhabitants he had now procured for his cottage at Barton. The Miss Dashwoods were young, pretty, and unaffected. It was enough to secure his good opinion; for to be unaffected was all that a pretty girl could want to make her mind as captivating as her person. The friendliness of his disposition made him happy in accommodating those, whose situation might be considered, in comparison with the past, as unfortunate. In showing kindness to his cousins therefore he had the real satisfaction of a good heart; and in settling a family of females only in his cottage, he had all the satisfaction of a sportsman; for a sportsman, though he esteems only those of his sex who are sportsmen likewise, is not often desirous of encouraging their taste by admitting them to a residence within his own manor.

Mrs Dashwood and her daughters were met at the door of the house by Sir John, who welcomed them to Barton Park with unaffected sincerity; and as he attended them to the drawing room repeated to the young ladies the concern which the same subject had drawn from him the day before, at being unable to get any smart young men to meet them. They would see, he said, only one gentleman there besides himself; a particular friend who was staying at the park, but who was neither very young nor very gay. He hoped they would all excuse the smallness of the party, and could assure them it should never happen so again. He had been to several families that morning in hopes of procuring some addition to their number, but it was moonlight and every body was full of engagements. Luckily Lady Middleton's mother had arrived at Barton within the last hour, and as she was a very cheerful agreeable woman, he hoped the young ladies would not find it so very dull as they might imagine. The young ladies, as well as their mother, were perfectly satisfied with having two entire strangers of the party, and wished for no more.

Mrs Jennings, Lady Middleton's mother, was a good-humoured, merry, fat, elderly woman, who talked a great deal, seemed very happy, and rather vulgar. She was full of jokes and laughter, and before dinner was over had said many witty things on the subject of lovers and husbands; hoped they had not left their hearts behind them in Sussex, and pretended to see them blush whether they did or not. Marianne was vexed at it for her sister's sake, and turned her eyes towards Elinor to see how she bore these attacks, with an earnestness which gave Elinor far more pain than could arise from such common-place raillery as Mrs Jennings's.

Colonel Brandon, the friend of Sir John, seemed no more adapted by resemblance of manner to be his friend, than Lady Middleton was to be his wife, or Mrs Jennings to be Lady Middleton's mother. He was silent and grave. His appearance however was not unpleasing, in spite of his being in the opinion of Marianne and Margaret an absolute old bachelor, for he was on the wrong side of five and thirty; but though his face was not handsome, his countenance was sensible, and his address was particularly gentlemanlike.

There was nothing in any of the party which could recommend them as companions to the Dashwoods; but the cold insipidity of Lady Middleton was so particularly repulsive, that in comparison of it the gravity of Colonel Brandon, and even the boisterous mirth of Sir John and his mother-in-law was interesting. Lady Middleton seemed to be roused to enjoyment only by the entrance of her four noisy children after dinner, who pulled her about, tore her clothes, and put an end to every kind of discourse except what related to themselves.

In the evening, as Marianne was discovered to be musical, she was invited to play. The instrument was unlocked, every body prepared to be charmed, and Marianne, who sang very well, at their request went through the chief of the songs which Lady Middleton had brought into the family on her marriage, and which perhaps had lain ever since in the same position on the pianoforte, for her ladyship had celebrated that event by giving up music, although by her mother's account, she had played extremely well, and by her own was very fond of it.

Marianne's performance was highly applauded. Sir John was loud in his admiration at the end of every song, and as loud in his conversation with the others while every song lasted. Lady Middleton frequently called him to order, wondered how any one's attention could be diverted from music for a moment, and asked Marianne to sing a particular song which Marianne had just finished. Colonel Brandon alone, of all the party, heard her without being in raptures. He paid her only the compliment of attention; and she felt a respect for him on the occasion, which the others had reasonably forfeited by their shameless want of taste. His pleasure in music, though it amounted not to that ecstatic delight which alone could sympathize with her own, was estimable when contrasted against the horrible insensibility of the others; and she was reasonable enough to allow that a man of five and thirty might well have outlived all acuteness of feeling and every exquisite power of enjoyment. She was perfectly disposed to make every allowance for the colonel's advanced state of life which humanity required.