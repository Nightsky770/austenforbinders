%!TeX root=../sensetop.tex
\chapter[Chapter \thechapter]{}
\lettrine[lines=4,lraise=0.3]{M}{rs} Jennings was a widow with an ample jointure. She had only two daughters, both of whom she had lived to see respectably married, and she had now therefore nothing to do but to marry all the rest of the world. In the promotion of this object she was zealously active, as far as her ability reached; and missed no opportunity of projecting weddings among all the young people of her acquaintance. She was remarkably quick in the discovery of attachments, and had enjoyed the advantage of raising the blushes and the vanity of many a young lady by insinuations of her power over such a young man; and this kind of discernment enabled her soon after her arrival at Barton decisively to pronounce that Colonel Brandon was very much in love with Marianne Dashwood. She rather suspected it to be so, on the very first evening of their being together, from his listening so attentively while she sang to them; and when the visit was returned by the Middletons’ dining at the cottage, the fact was ascertained by his listening to her again. It must be so. She was perfectly convinced of it. It would be an excellent match, for \textit{he} was rich, and \textit{she} was handsome. Mrs Jennings had been anxious to see Colonel Brandon well married, ever since her connection with Sir John first brought him to her knowledge; and she was always anxious to get a good husband for every pretty girl.

The immediate advantage to herself was by no means inconsiderable, for it supplied her with endless jokes against them both. At the park she laughed at the colonel, and in the cottage at Marianne. To the former her raillery was probably, as far as it regarded only himself, perfectly indifferent; but to the latter it was at first incomprehensible; and when its object was understood, she hardly knew whether most to laugh at its absurdity, or censure its impertinence, for she considered it as an unfeeling reflection on the colonel’s advanced years, and on his forlorn condition as an old bachelor.

Mrs Dashwood, who could not think a man five years younger than herself, so exceedingly ancient as he appeared to the youthful fancy of her daughter, ventured to clear Mrs Jennings from the probability of wishing to throw ridicule on his age.

»But at least, Mama, you cannot deny the absurdity of the accusation, though you may not think it intentionally ill-natured. Colonel Brandon is certainly younger than Mrs Jennings, but he is old enough to be \textit{my} father; and if he were ever animated enough to be in love, must have long outlived every sensation of the kind. It is too ridiculous! When is a man to be safe from such wit, if age and infirmity will not protect him?«

»Infirmity!« said Elinor, »do you call Colonel Brandon infirm? I can easily suppose that his age may appear much greater to you than to my mother; but you can hardly deceive yourself as to his having the use of his limbs!«

»Did not you hear him complain of the rheumatism? and is not that the commonest infirmity of declining life?«

»My dearest child,« said her mother, laughing, »at this rate you must be in continual terror of \textit{my} decay; and it must seem to you a miracle that my life has been extended to the advanced age of forty.«

»Mama, you are not doing me justice. I know very well that Colonel Brandon is not old enough to make his friends yet apprehensive of losing him in the course of nature. He may live twenty years longer. But thirty-five has nothing to do with matrimony.«

»Perhaps,« said Elinor, »thirty-five and seventeen had better not have any thing to do with matrimony together. But if there should by any chance happen to be a woman who is single at seven and twenty, I should not think Colonel Brandon’s being thirty-five any objection to his marrying \textit{her}.«

»A woman of seven and twenty,« said Marianne, after pausing a moment, »can never hope to feel or inspire affection again, and if her home be uncomfortable, or her fortune small, I can suppose that she might bring herself to submit to the offices of a nurse, for the sake of the provision and security of a wife. In his marrying such a woman therefore there would be nothing unsuitable. It would be a compact of convenience, and the world would be satisfied. In my eyes it would be no marriage at all, but that would be nothing. To me it would seem only a commercial exchange, in which each wished to be benefited at the expense of the other.«

»It would be impossible, I know,« replied Elinor, »to convince you that a woman of seven and twenty could feel for a man of thirty-five anything near enough to love, to make him a desirable companion to her. But I must object to your dooming Colonel Brandon and his wife to the constant confinement of a sick chamber, merely because he chanced to complain yesterday (a very cold damp day) of a slight rheumatic feel in one of his shoulders.«

»But he talked of flannel waistcoats,« said Marianne; »and with me a flannel waistcoat is invariably connected with aches, cramps, rheumatisms, and every species of ailment that can afflict the old and the feeble.«

»Had he been only in a violent fever, you would not have despised him half so much. Confess, Marianne, is not there something interesting to you in the flushed cheek, hollow eye, and quick pulse of a fever?«

Soon after this, upon Elinor’s leaving the room, »Mama,« said Marianne, »I have an alarm on the subject of illness which I cannot conceal from you. I am sure Edward Ferrars is not well. We have now been here almost a fortnight, and yet he does not come. Nothing but real indisposition could occasion this extraordinary delay. What else can detain him at Norland?«

»Had you any idea of his coming so soon?« said Mrs Dashwood. »\textit{I} had none. On the contrary, if I have felt any anxiety at all on the subject, it has been in recollecting that he sometimes showed a want of pleasure and readiness in accepting my invitation, when I talked of his coming to Barton. Does Elinor expect him already?«

»I have never mentioned it to her, but of course she must.«

»I rather think you are mistaken, for when I was talking to her yesterday of getting a new grate for the spare bedchamber, she observed that there was no immediate hurry for it, as it was not likely that the room would be wanted for some time.«

»How strange this is! what can be the meaning of it! But the whole of their behaviour to each other has been unaccountable! How cold, how composed were their last adieus! How languid their conversation the last evening of their being together! In Edward’s farewell there was no distinction between Elinor and me: it was the good wishes of an affectionate brother to both. Twice did I leave them purposely together in the course of the last morning, and each time did he most unaccountably follow me out of the room. And Elinor, in quitting Norland and Edward, cried not as I did. Even now her self-command is invariable. When is she dejected or melancholy? When does she try to avoid society, or appear restless and dissatisfied in it?«