%!TeX root=../sensetop.tex
\chapter[Chapter \thechapter]{}
\lettrine[lraise=0.3]{T}{he} Miss Dashwoods had now been rather more than two months in town, and Marianne's impatience to be gone increased every day. She sighed for the air, the liberty, the quiet of the country; and fancied that if any place could give her ease, Barton must do it. Elinor was hardly less anxious than herself for their removal, and only so much less bent on its being effected immediately, as that she was conscious of the difficulties of so long a journey, which Marianne could not be brought to acknowledge. She began, however, seriously to turn her thoughts towards its accomplishment, and had already mentioned their wishes to their kind hostess, who resisted them with all the eloquence of her good-will, when a plan was suggested, which, though detaining them from home yet a few weeks longer, appeared to Elinor altogether much more eligible than any other. The Palmers were to remove to Cleveland about the end of March, for the Easter holidays; and Mrs Jennings, with both her friends, received a very warm invitation from Charlotte to go with them. This would not, in itself, have been sufficient for the delicacy of Miss Dashwood;—but it was inforced with so much real politeness by Mr Palmer himself, as, joined to the very great amendment of his manners towards them since her sister had been known to be unhappy, induced her to accept it with pleasure.

When she told Marianne what she had done, however, her first reply was not very auspicious.

<Cleveland!>—she cried, with great agitation. <No, I cannot go to Cleveland.>

<You forget,> said Elinor gently, <that its situation is not—that it is not in the neighbourhood of\longdash>

<But it is in Somersetshire.—I cannot go into Somersetshire.—There, where I looked forward to going...No, Elinor, you cannot expect me to go there.>

Elinor would not argue upon the propriety of overcoming such feelings;—she only endeavoured to counteract them by working on others;—represented it, therefore, as a measure which would fix the time of her returning to that dear mother, whom she so much wished to see, in a more eligible, more comfortable manner, than any other plan could do, and perhaps without any greater delay. From Cleveland, which was within a few miles of Bristol, the distance to Barton was not beyond one day, though a long day's journey; and their mother's servant might easily come there to attend them down; and as there could be no occasion of their staying above a week at Cleveland, they might now be at home in little more than three weeks' time. As Marianne's affection for her mother was sincere, it must triumph with little difficulty, over the imaginary evils she had started.

Mrs Jennings was so far from being weary of her guests, that she pressed them very earnestly to return with her again from Cleveland. Elinor was grateful for the attention, but it could not alter her design; and their mother's concurrence being readily gained, every thing relative to their return was arranged as far as it could be;—and Marianne found some relief in drawing up a statement of the hours that were yet to divide her from Barton.

<Ah! Colonel, I do not know what you and I shall do without the Miss Dashwoods;>—was Mrs Jennings's address to him when he first called on her, after their leaving her was settled—<for they are quite resolved upon going home from the Palmers;—and how forlorn we shall be, when I come back!—Lord! we shall sit and gape at one another as dull as two cats.>

Perhaps Mrs Jennings was in hopes, by this vigorous sketch of their future ennui, to provoke him to make that offer, which might give himself an escape from it; and if so, she had soon afterwards good reason to think her object gained; for, on Elinor's moving to the window to take more expeditiously the dimensions of a print, which she was going to copy for her friend, he followed her to it with a look of particular meaning, and conversed with her there for several minutes. The effect of his discourse on the lady too, could not escape her observation, for though she was too honourable to listen, and had even changed her seat, on purpose that she might \textit{not} hear, to one close by the piano forte on which Marianne was playing, she could not keep herself from seeing that Elinor changed colour, attended with agitation, and was too intent on what he said to pursue her employment. Still farther in confirmation of her hopes, in the interval of Marianne's turning from one lesson to another, some words of the Colonel's inevitably reached her ear, in which he seemed to be apologising for the badness of his house. This set the matter beyond a doubt. She wondered, indeed, at his thinking it necessary to do so; but supposed it to be the proper etiquette. What Elinor said in reply she could not distinguish, but judged from the motion of her lips, that she did not think \textit{that} any material objection; and Mrs Jennings commended her in her heart for being so honest. They then talked on for a few minutes longer without her catching a syllable, when another lucky stop in Marianne's performance brought her these words in the Colonel's calm voice,—

<I am afraid it cannot take place very soon.>

Astonished and shocked at so unlover-like a speech, she was almost ready to cry out, <Lord! what should hinder it?>—but checking her desire, confined herself to this silent ejaculation.

<This is very strange!—sure he need not wait to be older.>

This delay on the Colonel's side, however, did not seem to offend or mortify his fair companion in the least, for on their breaking up the conference soon afterwards, and moving different ways, Mrs Jennings very plainly heard Elinor say, and with a voice which showed her to feel what she said,

<I shall always think myself very much obliged to you.>

Mrs Jennings was delighted with her gratitude, and only wondered that after hearing such a sentence, the Colonel should be able to take leave of them, as he immediately did, with the utmost \textit{sang-froid}, and go away without making her any reply! She had not thought her old friend could have made so indifferent a suitor.

What had really passed between them was to this effect.

<I have heard,> said he, with great compassion, <of the injustice your friend Mr Ferrars has suffered from his family; for if I understand the matter right, he has been entirely cast off by them for persevering in his engagement with a very deserving young woman. Have I been rightly informed? Is it so?;>

Elinor told him that it was.

<The cruelty, the impolitic cruelty,> he replied, with great feeling, <of dividing, or attempting to divide, two young people long attached to each other, is terrible. Mrs Ferrars does not know what she may be doing—what she may drive her son to. I have seen Mr Ferrars two or three times in Harley Street, and am much pleased with him. He is not a young man with whom one can be intimately acquainted in a short time, but I have seen enough of him to wish him well for his own sake, and as a friend of yours, I wish it still more. I understand that he intends to take orders. Will you be so good as to tell him that the living of Delaford, now just vacant, as I am informed by this day's post, is his, if he think it worth his acceptance; but \textit{that}, perhaps, so unfortunately circumstanced as he is now, it may be nonsense to appear to doubt; I only wish it were more valuable. It is a rectory, but a small one; the late incumbent, I believe, did not make more than \textsterling 200 per annum, and though it is certainly capable of improvement, I fear, not to such an amount as to afford him a very comfortable income. Such as it is, however, my pleasure in presenting it to him, will be very great. Pray assure him of it.>

Elinor's astonishment at this commission could hardly have been greater, had the Colonel been really making her an offer of his hand. The preferment, which only two days before she had considered as hopeless for Edward, was already provided to enable him to marry; and \textit{she}, of all people in the world, was fixed on to bestow it! Her emotion was such as Mrs Jennings had attributed to a very different cause; but whatever minor feelings less pure, less pleasing, might have a share in that emotion, her esteem for the general benevolence, and her gratitude for the particular friendship, which together prompted Colonel Brandon to this act, were strongly felt, and warmly expressed. She thanked him for it with all her heart, spoke of Edward's principles and disposition with that praise which she knew them to deserve; and promised to undertake the commission with pleasure, if it were really his wish to put off so agreeable an office to another. But at the same time, she could not help thinking that no one could so well perform it as himself. It was an office in short, from which, unwilling to give Edward the pain of receiving an obligation from \textit{her}, she would have been very glad to be spared herself; but Colonel Brandon, on motives of equal delicacy, declining it likewise, still seemed so desirous of its being given through her means, that she would not on any account make farther opposition. Edward, she believed, was still in town, and fortunately she had heard his address from Miss Steele. She could undertake therefore to inform him of it, in the course of the day. After this had been settled, Colonel Brandon began to talk of his own advantage in securing so respectable and agreeable a neighbour, and \textit{then} it was that he mentioned with regret, that the house was small and indifferent; an evil which Elinor, as Mrs Jennings had supposed her to do, made very light of, at least as far as regarded its size.

<The smallness of the house,> said she, <I cannot imagine any inconvenience to them, for it will be in proportion to their family and income.>

By which the Colonel was surprised to find that \textit{she} was considering Mr Ferrars's marriage as the certain consequence of the presentation; for he did not suppose it possible that Delaford living could supply such an income, as anybody in his style of life would venture to settle on, and he said so.

<This little rectory \textit{can} do no more than make Mr Ferrars comfortable as a bachelor; it cannot enable him to marry. I am sorry to say that my patronage ends with this; and my interest is hardly more extensive. If, however, by an unforeseen chance it should be in my power to serve him farther, I must think very differently of him from what I now do, if I am not as ready to be useful to him then as I sincerely wish I could be at present. What I am now doing indeed, seems nothing at all, since it can advance him so little towards what must be his principal, his only object of happiness. His marriage must still be a distant good; at least, I am afraid it cannot take place very soon.>

Such was the sentence which, when misunderstood, so justly offended the delicate feelings of Mrs Jennings; but after this narration of what really passed between Colonel Brandon and Elinor, while they stood at the window, the gratitude expressed by the latter on their parting, may perhaps appear in general, not less reasonably excited, nor less properly worded than if it had arisen from an offer of marriage.