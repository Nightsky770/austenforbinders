%!TeX root=../emmatop.tex
\chapter[Chapter \thechapter]{}
\lettrine[lines=4,lraise=0.3]{O}{ne} morning, about ten days after Mrs Churchill's decease, Emma was called downstairs to Mr Weston, who »could not stay five minutes, and wanted particularly to speak with her.«—He met her at the parlour-door, and hardly asking her how she did, in the natural key of his voice, sunk it immediately, to say, unheard by her father,

»Can you come to Randalls at any time this morning?—Do, if it be possible. Mrs Weston wants to see you. She must see you.«

»Is she unwell?«

»No, no, not at all—only a little agitated. She would have ordered the carriage, and come to you, but she must see you alone, and that you know—(nodding towards her father)—Humph!—Can you come?«

»Certainly. This moment, if you please. It is impossible to refuse what you ask in such a way. But what can be the matter?—Is she really not ill?«

»Depend upon me—but ask no more questions. You will know it all in time. The most unaccountable business! But hush, hush!«

To guess what all this meant, was impossible even for Emma. Something really important seemed announced by his looks; but, as her friend was well, she endeavoured not to be uneasy, and settling it with her father, that she would take her walk now, she and Mr Weston were soon out of the house together and on their way at a quick pace for Randalls.

»Now,«—said Emma, when they were fairly beyond the sweep gates,—»now Mr Weston, do let me know what has happened.«

»No, no,«—he gravely replied.—»Don't ask me. I promised my wife to leave it all to her. She will break it to you better than I can. Do not be impatient, Emma; it will all come out too soon.«

»Break it to me,« cried Emma, standing still with terror.—»Good God!—Mr Weston, tell me at once.—Something has happened in Brunswick Square. I know it has. Tell me, I charge you tell me this moment what it is.«

»No, indeed you are mistaken.«—

»Mr Weston do not trifle with me.—Consider how many of my dearest friends are now in Brunswick Square. Which of them is it?—I charge you by all that is sacred, not to attempt concealment.«

»Upon my word, Emma.«—

»Your word!—why not your honour!—why not say upon your honour, that it has nothing to do with any of them? Good Heavens!—What can be to be broke to me, that does not relate to one of that family?«

»Upon my honour,« said he very seriously, »it does not. It is not in the smallest degree connected with any human being of the name of Knightley.«

Emma's courage returned, and she walked on.

»I was wrong,« he continued, »in talking of its being broke to you. I should not have used the expression. In fact, it does not concern you—it concerns only myself,—that is, we hope.—Humph!—In short, my dear Emma, there is no occasion to be so uneasy about it. I don't say that it is not a disagreeable business—but things might be much worse.—If we walk fast, we shall soon be at Randalls.«

Emma found that she must wait; and now it required little effort. She asked no more questions therefore, merely employed her own fancy, and that soon pointed out to her the probability of its being some money concern—something just come to light, of a disagreeable nature in the circumstances of the family,—something which the late event at Richmond had brought forward. Her fancy was very active. Half a dozen natural children, perhaps—and poor Frank cut off!—This, though very undesirable, would be no matter of agony to her. It inspired little more than an animating curiosity.

»Who is that gentleman on horseback?« said she, as they proceeded—speaking more to assist Mr Weston in keeping his secret, than with any other view.

»I do not know.—One of the Otways.—Not Frank;—it is not Frank, I assure you. You will not see him. He is half way to Windsor by this time.«

»Has your son been with you, then?«

»Oh! yes—did not you know?—Well, well, never mind.«

For a moment he was silent; and then added, in a tone much more guarded and demure,

»Yes, Frank came over this morning, just to ask us how we did.«

They hurried on, and were speedily at Randalls.—»Well, my dear,« said he, as they entered the room—»I have brought her, and now I hope you will soon be better. I shall leave you together. There is no use in delay. I shall not be far off, if you want me.«—And Emma distinctly heard him add, in a lower tone, before he quitted the room,—»I have been as good as my word. She has not the least idea.«

Mrs Weston was looking so ill, and had an air of so much perturbation, that Emma's uneasiness increased; and the moment they were alone, she eagerly said,

»What is it my dear friend? Something of a very unpleasant nature, I find, has occurred;—do let me know directly what it is. I have been walking all this way in complete suspense. We both abhor suspense. Do not let mine continue longer. It will do you good to speak of your distress, whatever it may be.«

»Have you indeed no idea?« said Mrs Weston in a trembling voice. »Cannot you, my dear Emma—cannot you form a guess as to what you are to hear?«

»So far as that it relates to Mr Frank Churchill, I do guess.«

»You are right. It does relate to him, and I will tell you directly;« (resuming her work, and seeming resolved against looking up.) »He has been here this very morning, on a most extraordinary errand. It is impossible to express our surprize. He came to speak to his father on a subject,—to announce an attachment\longdash«

She stopped to breathe. Emma thought first of herself, and then of Harriet.

»More than an attachment, indeed,« resumed Mrs Weston; »an engagement—a positive engagement.—What will you say, Emma—what will any body say, when it is known that Frank Churchill and Miss Fairfax are engaged;—nay, that they have been long engaged!«

Emma even jumped with surprize;—and, horror-struck, exclaimed,

»Jane Fairfax!—Good God! You are not serious? You do not mean it?«

»You may well be amazed,« returned Mrs Weston, still averting her eyes, and talking on with eagerness, that Emma might have time to recover— »You may well be amazed. But it is even so. There has been a solemn engagement between them ever since October—formed at Weymouth, and kept a secret from every body. Not a creature knowing it but themselves—neither the Campbells, nor her family, nor his.—It is so wonderful, that though perfectly convinced of the fact, it is yet almost incredible to myself. I can hardly believe it.—I thought I knew him.«

Emma scarcely heard what was said.—Her mind was divided between two ideas—her own former conversations with him about Miss Fairfax; and poor Harriet;—and for some time she could only exclaim, and require confirmation, repeated confirmation.

»Well,« said she at last, trying to recover herself; »this is a circumstance which I must think of at least half a day, before I can at all comprehend it. What!—engaged to her all the winter—before either of them came to Highbury?«

»Engaged since October,—secretly engaged.—It has hurt me, Emma, very much. It has hurt his father equally. Some part of his conduct we cannot excuse.«

Emma pondered a moment, and then replied, »I will not pretend not to understand you; and to give you all the relief in my power, be assured that no such effect has followed his attentions to me, as you are apprehensive of.«

Mrs Weston looked up, afraid to believe; but Emma's countenance was as steady as her words.

»That you may have less difficulty in believing this boast, of my present perfect indifference,« she continued, »I will farther tell you, that there was a period in the early part of our acquaintance, when I did like him, when I was very much disposed to be attached to him—nay, was attached—and how it came to cease, is perhaps the wonder. Fortunately, however, it did cease. I have really for some time past, for at least these three months, cared nothing about him. You may believe me, Mrs Weston. This is the simple truth.«

Mrs Weston kissed her with tears of joy; and when she could find utterance, assured her, that this protestation had done her more good than any thing else in the world could do.

»Mr Weston will be almost as much relieved as myself,« said she. »On this point we have been wretched. It was our darling wish that you might be attached to each other—and we were persuaded that it was so.— Imagine what we have been feeling on your account.«

»I have escaped; and that I should escape, may be a matter of grateful wonder to you and myself. But this does not acquit him, Mrs Weston; and I must say, that I think him greatly to blame. What right had he to come among us with affection and faith engaged, and with manners so very disengaged? What right had he to endeavour to please, as he certainly did—to distinguish any one young woman with persevering attention, as he certainly did—while he really belonged to another?—How could he tell what mischief he might be doing?—How could he tell that he might not be making me in love with him?—very wrong, very wrong indeed.«

»From something that he said, my dear Emma, I rather imagine\longdash«

»And how could she bear such behaviour! Composure with a witness! to look on, while repeated attentions were offering to another woman, before her face, and not resent it.—That is a degree of placidity, which I can neither comprehend nor respect.«

»There were misunderstandings between them, Emma; he said so expressly. He had not time to enter into much explanation. He was here only a quarter of an hour, and in a state of agitation which did not allow the full use even of the time he could stay—but that there had been misunderstandings he decidedly said. The present crisis, indeed, seemed to be brought on by them; and those misunderstandings might very possibly arise from the impropriety of his conduct.«

»Impropriety! Oh! Mrs Weston—it is too calm a censure. Much, much beyond impropriety!—It has sunk him, I cannot say how it has sunk him in my opinion. So unlike what a man should be!—None of that upright integrity, that strict adherence to truth and principle, that disdain of trick and littleness, which a man should display in every transaction of his life.«

»Nay, dear Emma, now I must take his part; for though he has been wrong in this instance, I have known him long enough to answer for his having many, very many, good qualities; and\longdash«

»Good God!« cried Emma, not attending to her.—»Mrs Smallridge, too! Jane actually on the point of going as governess! What could he mean by such horrible indelicacy? To suffer her to engage herself—to suffer her even to think of such a measure!«

»He knew nothing about it, Emma. On this article I can fully acquit him. It was a private resolution of hers, not communicated to him—or at least not communicated in a way to carry conviction.—Till yesterday, I know he said he was in the dark as to her plans. They burst on him, I do not know how, but by some letter or message—and it was the discovery of what she was doing, of this very project of hers, which determined him to come forward at once, own it all to his uncle, throw himself on his kindness, and, in short, put an end to the miserable state of concealment that had been carrying on so long.«

Emma began to listen better.

»I am to hear from him soon,« continued Mrs Weston. »He told me at parting, that he should soon write; and he spoke in a manner which seemed to promise me many particulars that could not be given now. Let us wait, therefore, for this letter. It may bring many extenuations. It may make many things intelligible and excusable which now are not to be understood. Don't let us be severe, don't let us be in a hurry to condemn him. Let us have patience. I must love him; and now that I am satisfied on one point, the one material point, I am sincerely anxious for its all turning out well, and ready to hope that it may. They must both have suffered a great deal under such a system of secresy and concealment.«

»His sufferings,« replied Emma dryly, »do not appear to have done him much harm. Well, and how did Mr Churchill take it?«

»Most favourably for his nephew—gave his consent with scarcely a difficulty. Conceive what the events of a week have done in that family! While poor Mrs Churchill lived, I suppose there could not have been a hope, a chance, a possibility;—but scarcely are her remains at rest in the family vault, than her husband is persuaded to act exactly opposite to what she would have required. What a blessing it is, when undue influence does not survive the grave!—He gave his consent with very little persuasion.«

»Ah!« thought Emma, »he would have done as much for Harriet.«

»This was settled last night, and Frank was off with the light this morning. He stopped at Highbury, at the Bates's, I fancy, some time—and then came on hither; but was in such a hurry to get back to his uncle, to whom he is just now more necessary than ever, that, as I tell you, he could stay with us but a quarter of an hour.—He was very much agitated—very much, indeed—to a degree that made him appear quite a different creature from any thing I had ever seen him before.—In addition to all the rest, there had been the shock of finding her so very unwell, which he had had no previous suspicion of—and there was every appearance of his having been feeling a great deal.«

»And do you really believe the affair to have been carrying on with such perfect secresy?—The Campbells, the Dixons, did none of them know of the engagement?«

Emma could not speak the name of Dixon without a little blush.

»None; not one. He positively said that it had been known to no being in the world but their two selves.«

»Well,« said Emma, »I suppose we shall gradually grow reconciled to the idea, and I wish them very happy. But I shall always think it a very abominable sort of proceeding. What has it been but a system of hypocrisy and deceit,—espionage, and treachery?—To come among us with professions of openness and simplicity; and such a league in secret to judge us all!—Here have we been, the whole winter and spring, completely duped, fancying ourselves all on an equal footing of truth and honour, with two people in the midst of us who may have been carrying round, comparing and sitting in judgment on sentiments and words that were never meant for both to hear.—They must take the consequence, if they have heard each other spoken of in a way not perfectly agreeable!«

»I am quite easy on that head,« replied Mrs Weston. »I am very sure that I never said any thing of either to the other, which both might not have heard.«

»You are in luck.—Your only blunder was confined to my ear, when you imagined a certain friend of ours in love with the lady.«

»True. But as I have always had a thoroughly good opinion of Miss Fairfax, I never could, under any blunder, have spoken ill of her; and as to speaking ill of him, there I must have been safe.«

At this moment Mr Weston appeared at a little distance from the window, evidently on the watch. His wife gave him a look which invited him in; and, while he was coming round, added, »Now, dearest Emma, let me intreat you to say and look every thing that may set his heart at ease, and incline him to be satisfied with the match. Let us make the best of it—and, indeed, almost every thing may be fairly said in her favour. It is not a connexion to gratify; but if Mr Churchill does not feel that, why should we? and it may be a very fortunate circumstance for him, for Frank, I mean, that he should have attached himself to a girl of such steadiness of character and good judgment as I have always given her credit for—and still am disposed to give her credit for, in spite of this one great deviation from the strict rule of right. And how much may be said in her situation for even that error!«

»Much, indeed!« cried Emma feelingly. »If a woman can ever be excused for thinking only of herself, it is in a situation like Jane Fairfax's.—Of such, one may almost say, that »the world is not their«s, nor the world's law.'«

She met Mr Weston on his entrance, with a smiling countenance, exclaiming,

»A very pretty trick you have been playing me, upon my word! This was a device, I suppose, to sport with my curiosity, and exercise my talent of guessing. But you really frightened me. I thought you had lost half your property, at least. And here, instead of its being a matter of condolence, it turns out to be one of congratulation.—I congratulate you, Mr Weston, with all my heart, on the prospect of having one of the most lovely and accomplished young women in England for your daughter.«

A glance or two between him and his wife, convinced him that all was as right as this speech proclaimed; and its happy effect on his spirits was immediate. His air and voice recovered their usual briskness: he shook her heartily and gratefully by the hand, and entered on the subject in a manner to prove, that he now only wanted time and persuasion to think the engagement no very bad thing. His companions suggested only what could palliate imprudence, or smooth objections; and by the time they had talked it all over together, and he had talked it all over again with Emma, in their walk back to Hartfield, he was become perfectly reconciled, and not far from thinking it the very best thing that Frank could possibly have done.