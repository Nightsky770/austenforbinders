%!TeX root=../emmatop.tex
\chapter[Chapter \thechapter]{}
\lettrine[lraise=0.3]{M}{r} and Mrs John Knightley were not detained long at Hartfield. The weather soon improved enough for those to move who must move; and Mr Woodhouse having, as usual, tried to persuade his daughter to stay behind with all her children, was obliged to see the whole party set off, and return to his lamentations over the destiny of poor Isabella;—which poor Isabella, passing her life with those she doated on, full of their merits, blind to their faults, and always innocently busy, might have been a model of right feminine happiness.

The evening of the very day on which they went brought a note from Mr Elton to Mr Woodhouse, a long, civil, ceremonious note, to say, with Mr Elton's best compliments, <that he was proposing to leave Highbury the following morning in his way to Bath; where, in compliance with the pressing entreaties of some friends, he had engaged to spend a few weeks, and very much regretted the impossibility he was under, from various circumstances of weather and business, of taking a personal leave of Mr Woodhouse, of whose friendly civilities he should ever retain a grateful sense—and had Mr Woodhouse any commands, should be happy to attend to them.>

Emma was most agreeably surprized.—Mr Elton's absence just at this time was the very thing to be desired. She admired him for contriving it, though not able to give him much credit for the manner in which it was announced. Resentment could not have been more plainly spoken than in a civility to her father, from which she was so pointedly excluded. She had not even a share in his opening compliments.—Her name was not mentioned;—and there was so striking a change in all this, and such an ill-judged solemnity of leave-taking in his graceful acknowledgments, as she thought, at first, could not escape her father's suspicion.

It did, however.—Her father was quite taken up with the surprize of so sudden a journey, and his fears that Mr Elton might never get safely to the end of it, and saw nothing extraordinary in his language. It was a very useful note, for it supplied them with fresh matter for thought and conversation during the rest of their lonely evening. Mr Woodhouse talked over his alarms, and Emma was in spirits to persuade them away with all her usual promptitude.

She now resolved to keep Harriet no longer in the dark. She had reason to believe her nearly recovered from her cold, and it was desirable that she should have as much time as possible for getting the better of her other complaint before the gentleman's return. She went to Mrs Goddard's accordingly the very next day, to undergo the necessary penance of communication; and a severe one it was.—She had to destroy all the hopes which she had been so industriously feeding—to appear in the ungracious character of the one preferred—and acknowledge herself grossly mistaken and mis-judging in all her ideas on one subject, all her observations, all her convictions, all her prophecies for the last six weeks.

The confession completely renewed her first shame—and the sight of Harriet's tears made her think that she should never be in charity with herself again.

Harriet bore the intelligence very well—blaming nobody—and in every thing testifying such an ingenuousness of disposition and lowly opinion of herself, as must appear with particular advantage at that moment to her friend.

Emma was in the humour to value simplicity and modesty to the utmost; and all that was amiable, all that ought to be attaching, seemed on Harriet's side, not her own. Harriet did not consider herself as having any thing to complain of. The affection of such a man as Mr Elton would have been too great a distinction.—She never could have deserved him—and nobody but so partial and kind a friend as Miss Woodhouse would have thought it possible.

Her tears fell abundantly—but her grief was so truly artless, that no dignity could have made it more respectable in Emma's eyes—and she listened to her and tried to console her with all her heart and understanding—really for the time convinced that Harriet was the superior creature of the two—and that to resemble her would be more for her own welfare and happiness than all that genius or intelligence could do.

It was rather too late in the day to set about being simple-minded and ignorant; but she left her with every previous resolution confirmed of being humble and discreet, and repressing imagination all the rest of her life. Her second duty now, inferior only to her father's claims, was to promote Harriet's comfort, and endeavour to prove her own affection in some better method than by match-making. She got her to Hartfield, and shewed her the most unvarying kindness, striving to occupy and amuse her, and by books and conversation, to drive Mr Elton from her thoughts.

Time, she knew, must be allowed for this being thoroughly done; and she could suppose herself but an indifferent judge of such matters in general, and very inadequate to sympathise in an attachment to Mr Elton in particular; but it seemed to her reasonable that at Harriet's age, and with the entire extinction of all hope, such a progress might be made towards a state of composure by the time of Mr Elton's return, as to allow them all to meet again in the common routine of acquaintance, without any danger of betraying sentiments or increasing them.

Harriet did think him all perfection, and maintained the non-existence of any body equal to him in person or goodness—and did, in truth, prove herself more resolutely in love than Emma had foreseen; but yet it appeared to her so natural, so inevitable to strive against an inclination of that sort unrequited, that she could not comprehend its continuing very long in equal force.

If Mr Elton, on his return, made his own indifference as evident and indubitable as she could not doubt he would anxiously do, she could not imagine Harriet's persisting to place her happiness in the sight or the recollection of him.

Their being fixed, so absolutely fixed, in the same place, was bad for each, for all three. Not one of them had the power of removal, or of effecting any material change of society. They must encounter each other, and make the best of it.

Harriet was farther unfortunate in the tone of her companions at Mrs Goddard's; Mr Elton being the adoration of all the teachers and great girls in the school; and it must be at Hartfield only that she could have any chance of hearing him spoken of with cooling moderation or repellent truth. Where the wound had been given, there must the cure be found if anywhere; and Emma felt that, till she saw her in the way of cure, there could be no true peace for herself.