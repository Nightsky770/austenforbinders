%!TeX root=../emmatop.tex
\chapter[Chapter \thechapter]{}
\lettrine[lines=4,lraise=0.3]{H}{uman} nature is so well disposed towards those who are in interesting situations, that a young person, who either marries or dies, is sure of being kindly spoken of.

\zz
A week had not passed since Miss Hawkins's name was first mentioned in Highbury, before she was, by some means or other, discovered to have every recommendation of person and mind; to be handsome, elegant, highly accomplished, and perfectly amiable: and when Mr Elton himself arrived to triumph in his happy prospects, and circulate the fame of her merits, there was very little more for him to do, than to tell her Christian name, and say whose music she principally played.

Mr Elton returned, a very happy man. He had gone away rejected and mortified—disappointed in a very sanguine hope, after a series of what appeared to him strong encouragement; and not only losing the right lady, but finding himself debased to the level of a very wrong one. He had gone away deeply offended—he came back engaged to another—and to another as superior, of course, to the first, as under such circumstances what is gained always is to what is lost. He came back gay and self-satisfied, eager and busy, caring nothing for Miss Woodhouse, and defying Miss Smith.

The charming Augusta Hawkins, in addition to all the usual advantages of perfect beauty and merit, was in possession of an independent fortune, of so many thousands as would always be called ten; a point of some dignity, as well as some convenience: the story told well; he had not thrown himself away—he had gained a woman of 10,000 l. or thereabouts; and he had gained her with such delightful rapidity—the first hour of introduction had been so very soon followed by distinguishing notice; the history which he had to give Mrs Cole of the rise and progress of the affair was so glorious—the steps so quick, from the accidental rencontre, to the dinner at Mr Green's, and the party at Mrs Brown's—smiles and blushes rising in importance—with consciousness and agitation richly scattered—the lady had been so easily impressed—so sweetly disposed—had in short, to use a most intelligible phrase, been so very ready to have him, that vanity and prudence were equally contented.

He had caught both substance and shadow—both fortune and affection, and was just the happy man he ought to be; talking only of himself and his own concerns—expecting to be congratulated—ready to be laughed at—and, with cordial, fearless smiles, now addressing all the young ladies of the place, to whom, a few weeks ago, he would have been more cautiously gallant.

The wedding was no distant event, as the parties had only themselves to please, and nothing but the necessary preparations to wait for; and when he set out for Bath again, there was a general expectation, which a certain glance of Mrs Cole's did not seem to contradict, that when he next entered Highbury he would bring his bride.

During his present short stay, Emma had barely seen him; but just enough to feel that the first meeting was over, and to give her the impression of his not being improved by the mixture of pique and pretension, now spread over his air. She was, in fact, beginning very much to wonder that she had ever thought him pleasing at all; and his sight was so inseparably connected with some very disagreeable feelings, that, except in a moral light, as a penance, a lesson, a source of profitable humiliation to her own mind, she would have been thankful to be assured of never seeing him again. She wished him very well; but he gave her pain, and his welfare twenty miles off would administer most satisfaction.

The pain of his continued residence in Highbury, however, must certainly be lessened by his marriage. Many vain solicitudes would be prevented—many awkwardnesses smoothed by it. A Mrs Elton would be an excuse for any change of intercourse; former intimacy might sink without remark. It would be almost beginning their life of civility again.

Of the lady, individually, Emma thought very little. She was good enough for Mr Elton, no doubt; accomplished enough for Highbury—handsome enough—to look plain, probably, by Harriet's side. As to connexion, there Emma was perfectly easy; persuaded, that after all his own vaunted claims and disdain of Harriet, he had done nothing. On that article, truth seemed attainable. What she was, must be uncertain; but who she was, might be found out; and setting aside the 10,000 l., it did not appear that she was at all Harriet's superior. She brought no name, no blood, no alliance. Miss Hawkins was the youngest of the two daughters of a Bristol—merchant, of course, he must be called; but, as the whole of the profits of his mercantile life appeared so very moderate, it was not unfair to guess the dignity of his line of trade had been very moderate also. Part of every winter she had been used to spend in Bath; but Bristol was her home, the very heart of Bristol; for though the father and mother had died some years ago, an uncle remained—in the law line—nothing more distinctly honourable was hazarded of him, than that he was in the law line; and with him the daughter had lived. Emma guessed him to be the drudge of some attorney, and too stupid to rise. And all the grandeur of the connexion seemed dependent on the elder sister, who was very well married, to a gentleman in a great way, near Bristol, who kept two carriages! That was the wind-up of the history; that was the glory of Miss Hawkins.

Could she but have given Harriet her feelings about it all! She had talked her into love; but, alas! she was not so easily to be talked out of it. The charm of an object to occupy the many vacancies of Harriet's mind was not to be talked away. He might be superseded by another; he certainly would indeed; nothing could be clearer; even a Robert Martin would have been sufficient; but nothing else, she feared, would cure her. Harriet was one of those, who, having once begun, would be always in love. And now, poor girl! she was considerably worse from this reappearance of Mr Elton. She was always having a glimpse of him somewhere or other. Emma saw him only once; but two or three times every day Harriet was sure just to meet with him, or just to miss him, just to hear his voice, or see his shoulder, just to have something occur to preserve him in her fancy, in all the favouring warmth of surprize and conjecture. She was, moreover, perpetually hearing about him; for, excepting when at Hartfield, she was always among those who saw no fault in Mr Elton, and found nothing so interesting as the discussion of his concerns; and every report, therefore, every guess—all that had already occurred, all that might occur in the arrangement of his affairs, comprehending income, servants, and furniture, was continually in agitation around her. Her regard was receiving strength by invariable praise of him, and her regrets kept alive, and feelings irritated by ceaseless repetitions of Miss Hawkins's happiness, and continual observation of, how much he seemed attached!—his air as he walked by the house—the very sitting of his hat, being all in proof of how much he was in love!

Had it been allowable entertainment, had there been no pain to her friend, or reproach to herself, in the waverings of Harriet's mind, Emma would have been amused by its variations. Sometimes Mr Elton predominated, sometimes the Martins; and each was occasionally useful as a check to the other. Mr Elton's engagement had been the cure of the agitation of meeting Mr Martin. The unhappiness produced by the knowledge of that engagement had been a little put aside by Elizabeth Martin's calling at Mrs Goddard's a few days afterwards. Harriet had not been at home; but a note had been prepared and left for her, written in the very style to touch; a small mixture of reproach, with a great deal of kindness; and till Mr Elton himself appeared, she had been much occupied by it, continually pondering over what could be done in return, and wishing to do more than she dared to confess. But Mr Elton, in person, had driven away all such cares. While he staid, the Martins were forgotten; and on the very morning of his setting off for Bath again, Emma, to dissipate some of the distress it occasioned, judged it best for her to return Elizabeth Martin's visit.

How that visit was to be acknowledged—what would be necessary—and what might be safest, had been a point of some doubtful consideration. Absolute neglect of the mother and sisters, when invited to come, would be ingratitude. It must not be: and yet the danger of a renewal of the acquaintance—!

After much thinking, she could determine on nothing better, than Harriet's returning the visit; but in a way that, if they had understanding, should convince them that it was to be only a formal acquaintance. She meant to take her in the carriage, leave her at the Abbey Mill, while she drove a little farther, and call for her again so soon, as to allow no time for insidious applications or dangerous recurrences to the past, and give the most decided proof of what degree of intimacy was chosen for the future.

She could think of nothing better: and though there was something in it which her own heart could not approve—something of ingratitude, merely glossed over—it must be done, or what would become of Harriet?