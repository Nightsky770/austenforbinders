%!TeX root=../emmatop.tex
\chapter[Chapter \thechapter]{}
	\lettrine[lraise=0.3,ante=`]{H}{arriet,} poor Harriet!'—Those were the words; in them lay the tormenting ideas which Emma could not get rid of, and which constituted the real misery of the business to her. Frank Churchill had behaved very ill by herself—very ill in many ways,—but it was not so much his behaviour as her own, which made her so angry with him. It was the scrape which he had drawn her into on Harriet's account, that gave the deepest hue to his offence.—Poor Harriet! to be a second time the dupe of her misconceptions and flattery. Mr Knightley had spoken prophetically, when he once said, <Emma, you have been no friend to Harriet Smith.>—She was afraid she had done her nothing but disservice.—It was true that she had not to charge herself, in this instance as in the former, with being the sole and original author of the mischief; with having suggested such feelings as might otherwise never have entered Harriet's imagination; for Harriet had acknowledged her admiration and preference of Frank Churchill before she had ever given her a hint on the subject; but she felt completely guilty of having encouraged what she might have repressed. She might have prevented the indulgence and increase of such sentiments. Her influence would have been enough. And now she was very conscious that she ought to have prevented them.—She felt that she had been risking her friend's happiness on most insufficient grounds. Common sense would have directed her to tell Harriet, that she must not allow herself to think of him, and that there were five hundred chances to one against his ever caring for her.—<But, with common sense,> she added, <I am afraid I have had little to do.>

She was extremely angry with herself. If she could not have been angry with Frank Churchill too, it would have been dreadful.—As for Jane Fairfax, she might at least relieve her feelings from any present solicitude on her account. Harriet would be anxiety enough; she need no longer be unhappy about Jane, whose troubles and whose ill-health having, of course, the same origin, must be equally under cure.—Her days of insignificance and evil were over.—She would soon be well, and happy, and prosperous.—Emma could now imagine why her own attentions had been slighted. This discovery laid many smaller matters open. No doubt it had been from jealousy.—In Jane's eyes she had been a rival; and well might any thing she could offer of assistance or regard be repulsed. An airing in the Hartfield carriage would have been the rack, and arrowroot from the Hartfield storeroom must have been poison. She understood it all; and as far as her mind could disengage itself from the injustice and selfishness of angry feelings, she acknowledged that Jane Fairfax would have neither elevation nor happiness beyond her desert. But poor Harriet was such an engrossing charge! There was little sympathy to be spared for any body else. Emma was sadly fearful that this second disappointment would be more severe than the first. Considering the very superior claims of the object, it ought; and judging by its apparently stronger effect on Harriet's mind, producing reserve and self-command, it would.—She must communicate the painful truth, however, and as soon as possible. An injunction of secresy had been among Mr Weston's parting words. <For the present, the whole affair was to be completely a secret. Mr Churchill had made a point of it, as a token of respect to the wife he had so very recently lost; and every body admitted it to be no more than due decorum.>—Emma had promised; but still Harriet must be excepted. It was her superior duty.

In spite of her vexation, she could not help feeling it almost ridiculous, that she should have the very same distressing and delicate office to perform by Harriet, which Mrs Weston had just gone through by herself. The intelligence, which had been so anxiously announced to her, she was now to be anxiously announcing to another. Her heart beat quick on hearing Harriet's footstep and voice; so, she supposed, had poor Mrs Weston felt when she was approaching Randalls. Could the event of the disclosure bear an equal resemblance!—But of that, unfortunately, there could be no chance.

<Well, Miss Woodhouse!> cried Harriet, coming eagerly into the room—<is not this the oddest news that ever was?>

<What news do you mean?> replied Emma, unable to guess, by look or voice, whether Harriet could indeed have received any hint.

<About Jane Fairfax. Did you ever hear any thing so strange? Oh!—you need not be afraid of owning it to me, for Mr Weston has told me himself. I met him just now. He told me it was to be a great secret; and, therefore, I should not think of mentioning it to any body but you, but he said you knew it.>

<What did Mr Weston tell you?>—said Emma, still perplexed.

<Oh! he told me all about it; that Jane Fairfax and Mr Frank Churchill are to be married, and that they have been privately engaged to one another this long while. How very odd!>

It was, indeed, so odd; Harriet's behaviour was so extremely odd, that Emma did not know how to understand it. Her character appeared absolutely changed. She seemed to propose shewing no agitation, or disappointment, or peculiar concern in the discovery. Emma looked at her, quite unable to speak.

<Had you any idea,> cried Harriet, <of his being in love with her?—You, perhaps, might.—You (blushing as she spoke) who can see into every body's heart; but nobody else—>

<Upon my word,> said Emma, <I begin to doubt my having any such talent. Can you seriously ask me, Harriet, whether I imagined him attached to another woman at the very time that I was—tacitly, if not openly—encouraging you to give way to your own feelings?—I never had the slightest suspicion, till within the last hour, of Mr Frank Churchill's having the least regard for Jane Fairfax. You may be very sure that if I had, I should have cautioned you accordingly.>

<Me!> cried Harriet, colouring, and astonished. <Why should you caution me?—You do not think I care about Mr Frank Churchill.>

<I am delighted to hear you speak so stoutly on the subject,> replied Emma, smiling; <but you do not mean to deny that there was a time—and not very distant either—when you gave me reason to understand that you did care about him?>

<Him!—never, never. Dear Miss Woodhouse, how could you so mistake me?> turning away distressed.

<Harriet!> cried Emma, after a moment's pause—<What do you mean?—Good Heaven! what do you mean?—Mistake you!—Am I to suppose then?\longdash>

She could not speak another word.—Her voice was lost; and she sat down, waiting in great terror till Harriet should answer.

Harriet, who was standing at some distance, and with face turned from her, did not immediately say any thing; and when she did speak, it was in a voice nearly as agitated as Emma's.

<I should not have thought it possible,> she began, <that you could have misunderstood me! I know we agreed never to name him—but considering how infinitely superior he is to every body else, I should not have thought it possible that I could be supposed to mean any other person. Mr Frank Churchill, indeed! I do not know who would ever look at him in the company of the other. I hope I have a better taste than to think of Mr Frank Churchill, who is like nobody by his side. And that you should have been so mistaken, is amazing!—I am sure, but for believing that you entirely approved and meant to encourage me in my attachment, I should have considered it at first too great a presumption almost, to dare to think of him. At first, if you had not told me that more wonderful things had happened; that there had been matches of greater disparity (those were your very words);—I should not have dared to give way to—I should not have thought it possible—But if you, who had been always acquainted with him—>

<Harriet!> cried Emma, collecting herself resolutely—<Let us understand each other now, without the possibility of farther mistake. Are you speaking of—Mr Knightley?>

<To be sure I am. I never could have an idea of any body else—and so I thought you knew. When we talked about him, it was as clear as possible.>

<Not quite,> returned Emma, with forced calmness, <for all that you then said, appeared to me to relate to a different person. I could almost assert that you had named Mr Frank Churchill. I am sure the service Mr Frank Churchill had rendered you, in protecting you from the gipsies, was spoken of.>

<Oh! Miss Woodhouse, how you do forget!>

<My dear Harriet, I perfectly remember the substance of what I said on the occasion. I told you that I did not wonder at your attachment; that considering the service he had rendered you, it was extremely natural:—and you agreed to it, expressing yourself very warmly as to your sense of that service, and mentioning even what your sensations had been in seeing him come forward to your rescue.—The impression of it is strong on my memory.>

<Oh, dear,> cried Harriet, <now I recollect what you mean; but I was thinking of something very different at the time. It was not the gipsies—it was not Mr Frank Churchill that I meant. No! (with some elevation) I was thinking of a much more precious circumstance—of Mr Knightley's coming and asking me to dance, when Mr Elton would not stand up with me; and when there was no other partner in the room. That was the kind action; that was the noble benevolence and generosity; that was the service which made me begin to feel how superior he was to every other being upon earth.>

<Good God!> cried Emma, <this has been a most unfortunate—most deplorable mistake!—What is to be done?>

<You would not have encouraged me, then, if you had understood me? At least, however, I cannot be worse off than I should have been, if the other had been the person; and now—it is possible\longdash>

She paused a few moments. Emma could not speak.

<I do not wonder, Miss Woodhouse,> she resumed, <that you should feel a great difference between the two, as to me or as to any body. You must think one five hundred million times more above me than the other. But I hope, Miss Woodhouse, that supposing—that if—strange as it may appear—. But you know they were your own words, that more wonderful things had happened, matches of greater disparity had taken place than between Mr Frank Churchill and me; and, therefore, it seems as if such a thing even as this, may have occurred before—and if I should be so fortunate, beyond expression, as to—if Mr Knightley should really—if he does not mind the disparity, I hope, dear Miss Woodhouse, you will not set yourself against it, and try to put difficulties in the way. But you are too good for that, I am sure.>

Harriet was standing at one of the windows. Emma turned round to look at her in consternation, and hastily said,

<Have you any idea of Mr Knightley's returning your affection?>

<Yes,> replied Harriet modestly, but not fearfully—<I must say that I have.>

Emma's eyes were instantly withdrawn; and she sat silently meditating, in a fixed attitude, for a few minutes. A few minutes were sufficient for making her acquainted with her own heart. A mind like hers, once opening to suspicion, made rapid progress. She touched—she admitted—she acknowledged the whole truth. Why was it so much worse that Harriet should be in love with Mr Knightley, than with Frank Churchill? Why was the evil so dreadfully increased by Harriet's having some hope of a return? It darted through her, with the speed of an arrow, that Mr Knightley must marry no one but herself!

Her own conduct, as well as her own heart, was before her in the same few minutes. She saw it all with a clearness which had never blessed her before. How improperly had she been acting by Harriet! How inconsiderate, how indelicate, how irrational, how unfeeling had been her conduct! What blindness, what madness, had led her on! It struck her with dreadful force, and she was ready to give it every bad name in the world. Some portion of respect for herself, however, in spite of all these demerits—some concern for her own appearance, and a strong sense of justice by Harriet—(there would be no need of compassion to the girl who believed herself loved by Mr Knightley—but justice required that she should not be made unhappy by any coldness now,) gave Emma the resolution to sit and endure farther with calmness, with even apparent kindness.—For her own advantage indeed, it was fit that the utmost extent of Harriet's hopes should be enquired into; and Harriet had done nothing to forfeit the regard and interest which had been so voluntarily formed and maintained—or to deserve to be slighted by the person, whose counsels had never led her right.—Rousing from reflection, therefore, and subduing her emotion, she turned to Harriet again, and, in a more inviting accent, renewed the conversation; for as to the subject which had first introduced it, the wonderful story of Jane Fairfax, that was quite sunk and lost.—Neither of them thought but of Mr Knightley and themselves.

Harriet, who had been standing in no unhappy reverie, was yet very glad to be called from it, by the now encouraging manner of such a judge, and such a friend as Miss Woodhouse, and only wanted invitation, to give the history of her hopes with great, though trembling delight.—Emma's tremblings as she asked, and as she listened, were better concealed than Harriet's, but they were not less. Her voice was not unsteady; but her mind was in all the perturbation that such a development of self, such a burst of threatening evil, such a confusion of sudden and perplexing emotions, must create.—She listened with much inward suffering, but with great outward patience, to Harriet's detail.—Methodical, or well arranged, or very well delivered, it could not be expected to be; but it contained, when separated from all the feebleness and tautology of the narration, a substance to sink her spirit—especially with the corroborating circumstances, which her own memory brought in favour of Mr Knightley's most improved opinion of Harriet.

Harriet had been conscious of a difference in his behaviour ever since those two decisive dances.—Emma knew that he had, on that occasion, found her much superior to his expectation. From that evening, or at least from the time of Miss Woodhouse's encouraging her to think of him, Harriet had begun to be sensible of his talking to her much more than he had been used to do, and of his having indeed quite a different manner towards her; a manner of kindness and sweetness!—Latterly she had been more and more aware of it. When they had been all walking together, he had so often come and walked by her, and talked so very delightfully!—He seemed to want to be acquainted with her. Emma knew it to have been very much the case. She had often observed the change, to almost the same extent.—Harriet repeated expressions of approbation and praise from him—and Emma felt them to be in the closest agreement with what she had known of his opinion of Harriet. He praised her for being without art or affectation, for having simple, honest, generous, feelings.—She knew that he saw such recommendations in Harriet; he had dwelt on them to her more than once.—Much that lived in Harriet's memory, many little particulars of the notice she had received from him, a look, a speech, a removal from one chair to another, a compliment implied, a preference inferred, had been unnoticed, because unsuspected, by Emma. Circumstances that might swell to half an hour's relation, and contained multiplied proofs to her who had seen them, had passed undiscerned by her who now heard them; but the two latest occurrences to be mentioned, the two of strongest promise to Harriet, were not without some degree of witness from Emma herself.—The first, was his walking with her apart from the others, in the lime-walk at Donwell, where they had been walking some time before Emma came, and he had taken pains (as she was convinced) to draw her from the rest to himself—and at first, he had talked to her in a more particular way than he had ever done before, in a very particular way indeed!—(Harriet could not recall it without a blush.) He seemed to be almost asking her, whether her affections were engaged.—But as soon as she (Miss Woodhouse) appeared likely to join them, he changed the subject, and began talking about farming:—The second, was his having sat talking with her nearly half an hour before Emma came back from her visit, the very last morning of his being at Hartfield—though, when he first came in, he had said that he could not stay five minutes—and his having told her, during their conversation, that though he must go to London, it was very much against his inclination that he left home at all, which was much more (as Emma felt) than he had acknowledged to her. The superior degree of confidence towards Harriet, which this one article marked, gave her severe pain.

On the subject of the first of the two circumstances, she did, after a little reflection, venture the following question. <Might he not?—Is not it possible, that when enquiring, as you thought, into the state of your affections, he might be alluding to Mr Martin—he might have Mr Martin's interest in view?> But Harriet rejected the suspicion with spirit.

<Mr Martin! No indeed!—There was not a hint of Mr Martin. I hope I know better now, than to care for Mr Martin, or to be suspected of it.>

When Harriet had closed her evidence, she appealed to her dear Miss Woodhouse, to say whether she had not good ground for hope.

<I never should have presumed to think of it at first,> said she, <but for you. You told me to observe him carefully, and let his behaviour be the rule of mine—and so I have. But now I seem to feel that I may deserve him; and that if he does chuse me, it will not be any thing so very wonderful.>

The bitter feelings occasioned by this speech, the many bitter feelings, made the utmost exertion necessary on Emma's side, to enable her to say on reply,

<Harriet, I will only venture to declare, that Mr Knightley is the last man in the world, who would intentionally give any woman the idea of his feeling for her more than he really does.>

Harriet seemed ready to worship her friend for a sentence so satisfactory; and Emma was only saved from raptures and fondness, which at that moment would have been dreadful penance, by the sound of her father's footsteps. He was coming through the hall. Harriet was too much agitated to encounter him. <She could not compose herself— Mr Woodhouse would be alarmed—she had better go;>—with most ready encouragement from her friend, therefore, she passed off through another door—and the moment she was gone, this was the spontaneous burst of Emma's feelings: <Oh God! that I had never seen her!>

The rest of the day, the following night, were hardly enough for her thoughts.—She was bewildered amidst the confusion of all that had rushed on her within the last few hours. Every moment had brought a fresh surprize; and every surprize must be matter of humiliation to her.—How to understand it all! How to understand the deceptions she had been thus practising on herself, and living under!—The blunders, the blindness of her own head and heart!—she sat still, she walked about, she tried her own room, she tried the shrubbery—in every place, every posture, she perceived that she had acted most weakly; that she had been imposed on by others in a most mortifying degree; that she had been imposing on herself in a degree yet more mortifying; that she was wretched, and should probably find this day but the beginning of wretchedness.

To understand, thoroughly understand her own heart, was the first endeavour. To that point went every leisure moment which her father's claims on her allowed, and every moment of involuntary absence of mind.

How long had Mr Knightley been so dear to her, as every feeling declared him now to be? When had his influence, such influence begun?— When had he succeeded to that place in her affection, which Frank Churchill had once, for a short period, occupied?—She looked back; she compared the two—compared them, as they had always stood in her estimation, from the time of the latter's becoming known to her—and as they must at any time have been compared by her, had it—oh! had it, by any blessed felicity, occurred to her, to institute the comparison.—She saw that there never had been a time when she did not consider Mr Knightley as infinitely the superior, or when his regard for her had not been infinitely the most dear. She saw, that in persuading herself, in fancying, in acting to the contrary, she had been entirely under a delusion, totally ignorant of her own heart—and, in short, that she had never really cared for Frank Churchill at all!

This was the conclusion of the first series of reflection. This was the knowledge of herself, on the first question of inquiry, which she reached; and without being long in reaching it.—She was most sorrowfully indignant; ashamed of every sensation but the one revealed to her—her affection for Mr Knightley.—Every other part of her mind was disgusting.

With insufferable vanity had she believed herself in the secret of every body's feelings; with unpardonable arrogance proposed to arrange every body's destiny. She was proved to have been universally mistaken; and she had not quite done nothing—for she had done mischief. She had brought evil on Harriet, on herself, and she too much feared, on Mr Knightley.—Were this most unequal of all connexions to take place, on her must rest all the reproach of having given it a beginning; for his attachment, she must believe to be produced only by a consciousness of Harriet's;—and even were this not the case, he would never have known Harriet at all but for her folly.

Mr Knightley and Harriet Smith!—It was a union to distance every wonder of the kind.—The attachment of Frank Churchill and Jane Fairfax became commonplace, threadbare, stale in the comparison, exciting no surprize, presenting no disparity, affording nothing to be said or thought.—Mr Knightley and Harriet Smith!—Such an elevation on her side! Such a debasement on his! It was horrible to Emma to think how it must sink him in the general opinion, to foresee the smiles, the sneers, the merriment it would prompt at his expense; the mortification and disdain of his brother, the thousand inconveniences to himself.—Could it be?—No; it was impossible. And yet it was far, very far, from impossible.—Was it a new circumstance for a man of first-rate abilities to be captivated by very inferior powers? Was it new for one, perhaps too busy to seek, to be the prize of a girl who would seek him?—Was it new for any thing in this world to be unequal, inconsistent, incongruous—or for chance and circumstance (as second causes) to direct the human fate?

Oh! had she never brought Harriet forward! Had she left her where she ought, and where he had told her she ought!—Had she not, with a folly which no tongue could express, prevented her marrying the unexceptionable young man who would have made her happy and respectable in the line of life to which she ought to belong—all would have been safe; none of this dreadful sequel would have been.

How Harriet could ever have had the presumption to raise her thoughts to Mr Knightley!—How she could dare to fancy herself the chosen of such a man till actually assured of it!—But Harriet was less humble, had fewer scruples than formerly.—Her inferiority, whether of mind or situation, seemed little felt.—She had seemed more sensible of Mr Elton's being to stoop in marrying her, than she now seemed of Mr Knightley's.—Alas! was not that her own doing too? Who had been at pains to give Harriet notions of self-consequence but herself?—Who but herself had taught her, that she was to elevate herself if possible, and that her claims were great to a high worldly establishment?—If Harriet, from being humble, were grown vain, it was her doing too.