%!TeX root=../emmatop.tex
\chapter[Chapter \thechapter]{}
\lettrine[lines=4,lraise=0.3]{E}{mma's} very good opinion of Frank Churchill was a little shaken the following day, by hearing that he was gone off to London, merely to have his hair cut. A sudden freak seemed to have seized him at breakfast, and he had sent for a chaise and set off, intending to return to dinner, but with no more important view that appeared than having his hair cut. There was certainly no harm in his travelling sixteen miles twice over on such an errand; but there was an air of foppery and nonsense in it which she could not approve. It did not accord with the rationality of plan, the moderation in expense, or even the unselfish warmth of heart, which she had believed herself to discern in him yesterday. Vanity, extravagance, love of change, restlessness of temper, which must be doing something, good or bad; heedlessness as to the pleasure of his father and Mrs Weston, indifferent as to how his conduct might appear in general; he became liable to all these charges. His father only called him a coxcomb, and thought it a very good story; but that Mrs Weston did not like it, was clear enough, by her passing it over as quickly as possible, and making no other comment than that »all young people would have their little whims.«

With the exception of this little blot, Emma found that his visit hitherto had given her friend only good ideas of him. Mrs Weston was very ready to say how attentive and pleasant a companion he made himself—how much she saw to like in his disposition altogether. He appeared to have a very open temper—certainly a very cheerful and lively one; she could observe nothing wrong in his notions, a great deal decidedly right; he spoke of his uncle with warm regard, was fond of talking of him—said he would be the best man in the world if he were left to himself; and though there was no being attached to the aunt, he acknowledged her kindness with gratitude, and seemed to mean always to speak of her with respect. This was all very promising; and, but for such an unfortunate fancy for having his hair cut, there was nothing to denote him unworthy of the distinguished honour which her imagination had given him; the honour, if not of being really in love with her, of being at least very near it, and saved only by her own indifference—(for still her resolution held of never marrying)—the honour, in short, of being marked out for her by all their joint acquaintance.

Mr Weston, on his side, added a virtue to the account which must have some weight. He gave her to understand that Frank admired her extremely—thought her very beautiful and very charming; and with so much to be said for him altogether, she found she must not judge him harshly. As Mrs Weston observed, »all young people would have their little whims.«

There was one person among his new acquaintance in Surry, not so leniently disposed. In general he was judged, throughout the parishes of Donwell and Highbury, with great candour; liberal allowances were made for the little excesses of such a handsome young man—one who smiled so often and bowed so well; but there was one spirit among them not to be softened, from its power of censure, by bows or smiles—Mr Knightley. The circumstance was told him at Hartfield; for the moment, he was silent; but Emma heard him almost immediately afterwards say to himself, over a newspaper he held in his hand, »Hum! just the trifling, silly fellow I took him for.« She had half a mind to resent; but an instant's observation convinced her that it was really said only to relieve his own feelings, and not meant to provoke; and therefore she let it pass.

Although in one instance the bearers of not good tidings, Mr and Mrs Weston's visit this morning was in another respect particularly opportune. Something occurred while they were at Hartfield, to make Emma want their advice; and, which was still more lucky, she wanted exactly the advice they gave.

This was the occurrence:—The Coles had been settled some years in Highbury, and were very good sort of people—friendly, liberal, and unpretending; but, on the other hand, they were of low origin, in trade, and only moderately genteel. On their first coming into the country, they had lived in proportion to their income, quietly, keeping little company, and that little unexpensively; but the last year or two had brought them a considerable increase of means—the house in town had yielded greater profits, and fortune in general had smiled on them. With their wealth, their views increased; their want of a larger house, their inclination for more company. They added to their house, to their number of servants, to their expenses of every sort; and by this time were, in fortune and style of living, second only to the family at Hartfield. Their love of society, and their new dining-room, prepared every body for their keeping dinner-company; and a few parties, chiefly among the single men, had already taken place. The regular and best families Emma could hardly suppose they would presume to invite—neither Donwell, nor Hartfield, nor Randalls. Nothing should tempt her to go, if they did; and she regretted that her father's known habits would be giving her refusal less meaning than she could wish. The Coles were very respectable in their way, but they ought to be taught that it was not for them to arrange the terms on which the superior families would visit them. This lesson, she very much feared, they would receive only from herself; she had little hope of Mr Knightley, none of Mr Weston.

But she had made up her mind how to meet this presumption so many weeks before it appeared, that when the insult came at last, it found her very differently affected. Donwell and Randalls had received their invitation, and none had come for her father and herself; and Mrs Weston's accounting for it with »I suppose they will not take the liberty with you; they know you do not dine out,« was not quite sufficient. She felt that she should like to have had the power of refusal; and afterwards, as the idea of the party to be assembled there, consisting precisely of those whose society was dearest to her, occurred again and again, she did not know that she might not have been tempted to accept. Harriet was to be there in the evening, and the Bateses. They had been speaking of it as they walked about Highbury the day before, and Frank Churchill had most earnestly lamented her absence. Might not the evening end in a dance? had been a question of his. The bare possibility of it acted as a farther irritation on her spirits; and her being left in solitary grandeur, even supposing the omission to be intended as a compliment, was but poor comfort.

It was the arrival of this very invitation while the Westons were at Hartfield, which made their presence so acceptable; for though her first remark, on reading it, was that »of course it must be declined,« she so very soon proceeded to ask them what they advised her to do, that their advice for her going was most prompt and successful.

She owned that, considering every thing, she was not absolutely without inclination for the party. The Coles expressed themselves so properly—there was so much real attention in the manner of it—so much consideration for her father. »They would have solicited the honour earlier, but had been waiting the arrival of a folding-screen from London, which they hoped might keep Mr Woodhouse from any draught of air, and therefore induce him the more readily to give them the honour of his company.« Upon the whole, she was very persuadable; and it being briefly settled among themselves how it might be done without neglecting his comfort—how certainly Mrs Goddard, if not Mrs Bates, might be depended on for bearing him company—Mr Woodhouse was to be talked into an acquiescence of his daughter's going out to dinner on a day now near at hand, and spending the whole evening away from him. As for his going, Emma did not wish him to think it possible, the hours would be too late, and the party too numerous. He was soon pretty well resigned.

»I am not fond of dinner-visiting,« said he—»I never was. No more is Emma. Late hours do not agree with us. I am sorry Mr and Mrs Cole should have done it. I think it would be much better if they would come in one afternoon next summer, and take their tea with us—take us in their afternoon walk; which they might do, as our hours are so reasonable, and yet get home without being out in the damp of the evening. The dews of a summer evening are what I would not expose any body to. However, as they are so very desirous to have dear Emma dine with them, and as you will both be there, and Mr Knightley too, to take care of her, I cannot wish to prevent it, provided the weather be what it ought, neither damp, nor cold, nor windy.« Then turning to Mrs Weston, with a look of gentle reproach—»Ah! Miss Taylor, if you had not married, you would have staid at home with me.«

»Well, sir,« cried Mr Weston, »as I took Miss Taylor away, it is incumbent on me to supply her place, if I can; and I will step to Mrs Goddard in a moment, if you wish it.«

But the idea of any thing to be done in a moment, was increasing, not lessening, Mr Woodhouse's agitation. The ladies knew better how to allay it. Mr Weston must be quiet, and every thing deliberately arranged.

With this treatment, Mr Woodhouse was soon composed enough for talking as usual. »He should be happy to see Mrs Goddard. He had a great regard for Mrs Goddard; and Emma should write a line, and invite her. James could take the note. But first of all, there must be an answer written to Mrs Cole.«

»You will make my excuses, my dear, as civilly as possible. You will say that I am quite an invalid, and go no where, and therefore must decline their obliging invitation; beginning with my compliments, of course. But you will do every thing right. I need not tell you what is to be done. We must remember to let James know that the carriage will be wanted on Tuesday. I shall have no fears for you with him. We have never been there above once since the new approach was made; but still I have no doubt that James will take you very safely. And when you get there, you must tell him at what time you would have him come for you again; and you had better name an early hour. You will not like staying late. You will get very tired when tea is over.«

»But you would not wish me to come away before I am tired, papa?«

»Oh! no, my love; but you will soon be tired. There will be a great many people talking at once. You will not like the noise.«

»But, my dear sir,« cried Mr Weston, »if Emma comes away early, it will be breaking up the party.«

»And no great harm if it does,« said Mr Woodhouse. »The sooner every party breaks up, the better.«

»But you do not consider how it may appear to the Coles. Emma's going away directly after tea might be giving offence. They are good-natured people, and think little of their own claims; but still they must feel that any body's hurrying away is no great compliment; and Miss Woodhouse's doing it would be more thought of than any other person's in the room. You would not wish to disappoint and mortify the Coles, I am sure, sir; friendly, good sort of people as ever lived, and who have been your neighbours these ten years.«

»No, upon no account in the world, Mr Weston; I am much obliged to you for reminding me. I should be extremely sorry to be giving them any pain. I know what worthy people they are. Perry tells me that Mr Cole never touches malt liquor. You would not think it to look at him, but he is bilious—Mr Cole is very bilious. No, I would not be the means of giving them any pain. My dear Emma, we must consider this. I am sure, rather than run the risk of hurting Mr and Mrs Cole, you would stay a little longer than you might wish. You will not regard being tired. You will be perfectly safe, you know, among your friends.«

»Oh yes, papa. I have no fears at all for myself; and I should have no scruples of staying as late as Mrs Weston, but on your account. I am only afraid of your sitting up for me. I am not afraid of your not being exceedingly comfortable with Mrs Goddard. She loves piquet, you know; but when she is gone home, I am afraid you will be sitting up by yourself, instead of going to bed at your usual time—and the idea of that would entirely destroy my comfort. You must promise me not to sit up.«

He did, on the condition of some promises on her side: such as that, if she came home cold, she would be sure to warm herself thoroughly; if hungry, that she would take something to eat; that her own maid should sit up for her; and that Serle and the butler should see that every thing were safe in the house, as usual.