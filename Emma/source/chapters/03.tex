%!TeX root=../emmatop.tex
\chapter[Chapter \thechapter]{}
\lettrine[lraise=0.3]{M}{r} Woodhouse was fond of society in his own way. He liked very much to have his friends come and see him; and from various united causes, from his long residence at Hartfield, and his good nature, from his fortune, his house, and his daughter, he could command the visits of his own little circle, in a great measure, as he liked. He had not much intercourse with any families beyond that circle; his horror of late hours, and large dinner-parties, made him unfit for any acquaintance but such as would visit him on his own terms. Fortunately for him, Highbury, including Randalls in the same parish, and Donwell Abbey in the parish adjoining, the seat of Mr Knightley, comprehended many such. Not unfrequently, through Emma's persuasion, he had some of the chosen and the best to dine with him: but evening parties were what he preferred; and, unless he fancied himself at any time unequal to company, there was scarcely an evening in the week in which Emma could not make up a card-table for him.

Real, long-standing regard brought the Westons and Mr Knightley; and by Mr Elton, a young man living alone without liking it, the privilege of exchanging any vacant evening of his own blank solitude for the elegancies and society of Mr Woodhouse's drawing-room, and the smiles of his lovely daughter, was in no danger of being thrown away.

After these came a second set; among the most come-at-able of whom were Mrs and Miss Bates, and Mrs Goddard, three ladies almost always at the service of an invitation from Hartfield, and who were fetched and carried home so often, that Mr Woodhouse thought it no hardship for either James or the horses. Had it taken place only once a year, it would have been a grievance.

Mrs Bates, the widow of a former vicar of Highbury, was a very old lady, almost past every thing but tea and quadrille. She lived with her single daughter in a very small way, and was considered with all the regard and respect which a harmless old lady, under such untoward circumstances, can excite. Her daughter enjoyed a most uncommon degree of popularity for a woman neither young, handsome, rich, nor married. Miss Bates stood in the very worst predicament in the world for having much of the public favour; and she had no intellectual superiority to make atonement to herself, or frighten those who might hate her into outward respect. She had never boasted either beauty or cleverness. Her youth had passed without distinction, and her middle of life was devoted to the care of a failing mother, and the endeavour to make a small income go as far as possible. And yet she was a happy woman, and a woman whom no one named without good-will. It was her own universal good-will and contented temper which worked such wonders. She loved every body, was interested in every body's happiness, quicksighted to every body's merits; thought herself a most fortunate creature, and surrounded with blessings in such an excellent mother, and so many good neighbours and friends, and a home that wanted for nothing. The simplicity and cheerfulness of her nature, her contented and grateful spirit, were a recommendation to every body, and a mine of felicity to herself. She was a great talker upon little matters, which exactly suited Mr Woodhouse, full of trivial communications and harmless gossip.

Mrs Goddard was the mistress of a School—not of a seminary, or an establishment, or any thing which professed, in long sentences of refined nonsense, to combine liberal acquirements with elegant morality, upon new principles and new systems—and where young ladies for enormous pay might be screwed out of health and into vanity—but a real, honest, old-fashioned Boarding-school, where a reasonable quantity of accomplishments were sold at a reasonable price, and where girls might be sent to be out of the way, and scramble themselves into a little education, without any danger of coming back prodigies. Mrs Goddard's school was in high repute—and very deservedly; for Highbury was reckoned a particularly healthy spot: she had an ample house and garden, gave the children plenty of wholesome food, let them run about a great deal in the summer, and in winter dressed their chilblains with her own hands. It was no wonder that a train of twenty young couple now walked after her to church. She was a plain, motherly kind of woman, who had worked hard in her youth, and now thought herself entitled to the occasional holiday of a tea-visit; and having formerly owed much to Mr Woodhouse's kindness, felt his particular claim on her to leave her neat parlour, hung round with fancy-work, whenever she could, and win or lose a few sixpences by his fireside.

These were the ladies whom Emma found herself very frequently able to collect; and happy was she, for her father's sake, in the power; though, as far as she was herself concerned, it was no remedy for the absence of Mrs Weston. She was delighted to see her father look comfortable, and very much pleased with herself for contriving things so well; but the quiet prosings of three such women made her feel that every evening so spent was indeed one of the long evenings she had fearfully anticipated.

As she sat one morning, looking forward to exactly such a close of the present day, a note was brought from Mrs Goddard, requesting, in most respectful terms, to be allowed to bring Miss Smith with her; a most welcome request: for Miss Smith was a girl of seventeen, whom Emma knew very well by sight, and had long felt an interest in, on account of her beauty. A very gracious invitation was returned, and the evening no longer dreaded by the fair mistress of the mansion.

Harriet Smith was the natural daughter of somebody. Somebody had placed her, several years back, at Mrs Goddard's school, and somebody had lately raised her from the condition of scholar to that of parlour-boarder. This was all that was generally known of her history. She had no visible friends but what had been acquired at Highbury, and was now just returned from a long visit in the country to some young ladies who had been at school there with her.

She was a very pretty girl, and her beauty happened to be of a sort which Emma particularly admired. She was short, plump, and fair, with a fine bloom, blue eyes, light hair, regular features, and a look of great sweetness, and, before the end of the evening, Emma was as much pleased with her manners as her person, and quite determined to continue the acquaintance.

She was not struck by any thing remarkably clever in Miss Smith's conversation, but she found her altogether very engaging—not inconveniently shy, not unwilling to talk—and yet so far from pushing, shewing so proper and becoming a deference, seeming so pleasantly grateful for being admitted to Hartfield, and so artlessly impressed by the appearance of every thing in so superior a style to what she had been used to, that she must have good sense, and deserve encouragement. Encouragement should be given. Those soft blue eyes, and all those natural graces, should not be wasted on the inferior society of Highbury and its connexions. The acquaintance she had already formed were unworthy of her. The friends from whom she had just parted, though very good sort of people, must be doing her harm. They were a family of the name of Martin, whom Emma well knew by character, as renting a large farm of Mr Knightley, and residing in the parish of Donwell—very creditably, she believed—she knew Mr Knightley thought highly of them—but they must be coarse and unpolished, and very unfit to be the intimates of a girl who wanted only a little more knowledge and elegance to be quite perfect. \textit{She} would notice her; she would improve her; she would detach her from her bad acquaintance, and introduce her into good society; she would form her opinions and her manners. It would be an interesting, and certainly a very kind undertaking; highly becoming her own situation in life, her leisure, and powers.

She was so busy in admiring those soft blue eyes, in talking and listening, and forming all these schemes in the in-betweens, that the evening flew away at a very unusual rate; and the supper-table, which always closed such parties, and for which she had been used to sit and watch the due time, was all set out and ready, and moved forwards to the fire, before she was aware. With an alacrity beyond the common impulse of a spirit which yet was never indifferent to the credit of doing every thing well and attentively, with the real good-will of a mind delighted with its own ideas, did she then do all the honours of the meal, and help and recommend the minced chicken and scalloped oysters, with an urgency which she knew would be acceptable to the early hours and civil scruples of their guests.

Upon such occasions poor Mr Woodhouse's feelings were in sad warfare. He loved to have the cloth laid, because it had been the fashion of his youth, but his conviction of suppers being very unwholesome made him rather sorry to see any thing put on it; and while his hospitality would have welcomed his visitors to every thing, his care for their health made him grieve that they would eat.

Such another small basin of thin gruel as his own was all that he could, with thorough self-approbation, recommend; though he might constrain himself, while the ladies were comfortably clearing the nicer things, to say:

<Mrs Bates, let me propose your venturing on one of these eggs. An egg boiled very soft is not unwholesome. Serle understands boiling an egg better than any body. I would not recommend an egg boiled by any body else; but you need not be afraid, they are very small, you see—one of our small eggs will not hurt you. Miss Bates, let Emma help you to a \textit{little} bit of tart—a \textit{very} little bit. Ours are all apple-tarts. You need not be afraid of unwholesome preserves here. I do not advise the custard. Mrs Goddard, what say you to \textit{half} a glass of wine? A \textit{small} half-glass, put into a tumbler of water? I do not think it could disagree with you.>

Emma allowed her father to talk—but supplied her visitors in a much more satisfactory style, and on the present evening had particular pleasure in sending them away happy. The happiness of Miss Smith was quite equal to her intentions. Miss Woodhouse was so great a personage in Highbury, that the prospect of the introduction had given as much panic as pleasure; but the humble, grateful little girl went off with highly gratified feelings, delighted with the affability with which Miss Woodhouse had treated her all the evening, and actually shaken hands with her at last!