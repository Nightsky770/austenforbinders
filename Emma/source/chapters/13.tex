%!TeX root=../emmatop.tex
\chapter[Chapter \thechapter]{}
\lettrine[lraise=0.3]{T}{here} could hardly be a happier creature in the world than Mrs John Knightley, in this short visit to Hartfield, going about every morning among her old acquaintance with her five children, and talking over what she had done every evening with her father and sister. She had nothing to wish otherwise, but that the days did not pass so swiftly. It was a delightful visit;—perfect, in being much too short.

In general their evenings were less engaged with friends than their mornings; but one complete dinner engagement, and out of the house too, there was no avoiding, though at Christmas. Mr Weston would take no denial; they must all dine at Randalls one day;—even Mr Woodhouse was persuaded to think it a possible thing in preference to a division of the party.

How they were all to be conveyed, he would have made a difficulty if he could, but as his son and daughter's carriage and horses were actually at Hartfield, he was not able to make more than a simple question on that head; it hardly amounted to a doubt; nor did it occupy Emma long to convince him that they might in one of the carriages find room for Harriet also.

Harriet, Mr Elton, and Mr Knightley, their own especial set, were the only persons invited to meet them;—the hours were to be early, as well as the numbers few; Mr Woodhouse's habits and inclination being consulted in every thing.

The evening before this great event (for it was a very great event that Mr Woodhouse should dine out, on the 24th of December) had been spent by Harriet at Hartfield, and she had gone home so much indisposed with a cold, that, but for her own earnest wish of being nursed by Mrs Goddard, Emma could not have allowed her to leave the house. Emma called on her the next day, and found her doom already signed with regard to Randalls. She was very feverish and had a bad sore throat: Mrs Goddard was full of care and affection, Mr Perry was talked of, and Harriet herself was too ill and low to resist the authority which excluded her from this delightful engagement, though she could not speak of her loss without many tears.

Emma sat with her as long as she could, to attend her in Mrs Goddard's unavoidable absences, and raise her spirits by representing how much Mr Elton's would be depressed when he knew her state; and left her at last tolerably comfortable, in the sweet dependence of his having a most comfortless visit, and of their all missing her very much. She had not advanced many yards from Mrs Goddard's door, when she was met by Mr Elton himself, evidently coming towards it, and as they walked on slowly together in conversation about the invalid—of whom he, on the rumour of considerable illness, had been going to inquire, that he might carry some report of her to Hartfield—they were overtaken by Mr John Knightley returning from the daily visit to Donwell, with his two eldest boys, whose healthy, glowing faces shewed all the benefit of a country run, and seemed to ensure a quick despatch of the roast mutton and rice pudding they were hastening home for. They joined company and proceeded together. Emma was just describing the nature of her friend's complaint;—<a throat very much inflamed, with a great deal of heat about her, a quick, low pulse, \&c. and she was sorry to find from Mrs Goddard that Harriet was liable to very bad sore-throats, and had often alarmed her with them.> Mr Elton looked all alarm on the occasion, as he exclaimed,

<A sore-throat!—I hope not infectious. I hope not of a putrid infectious sort. Has Perry seen her? Indeed you should take care of yourself as well as of your friend. Let me entreat you to run no risks. Why does not Perry see her?>

Emma, who was not really at all frightened herself, tranquillised this excess of apprehension by assurances of Mrs Goddard's experience and care; but as there must still remain a degree of uneasiness which she could not wish to reason away, which she would rather feed and assist than not, she added soon afterwards—as if quite another subject,

<It is so cold, so very cold—and looks and feels so very much like snow, that if it were to any other place or with any other party, I should really try not to go out to-day—and dissuade my father from venturing; but as he has made up his mind, and does not seem to feel the cold himself, I do not like to interfere, as I know it would be so great a disappointment to Mr and Mrs Weston. But, upon my word, Mr Elton, in your case, I should certainly excuse myself. You appear to me a little hoarse already, and when you consider what demand of voice and what fatigues to-morrow will bring, I think it would be no more than common prudence to stay at home and take care of yourself to-night.>

Mr Elton looked as if he did not very well know what answer to make; which was exactly the case; for though very much gratified by the kind care of such a fair lady, and not liking to resist any advice of hers, he had not really the least inclination to give up the visit;—but Emma, too eager and busy in her own previous conceptions and views to hear him impartially, or see him with clear vision, was very well satisfied with his muttering acknowledgment of its being <very cold, certainly very cold,> and walked on, rejoicing in having extricated him from Randalls, and secured him the power of sending to inquire after Harriet every hour of the evening.

<You do quite right,> said she;—<we will make your apologies to Mr and Mrs Weston.>

But hardly had she so spoken, when she found her brother was civilly offering a seat in his carriage, if the weather were Mr Elton's only objection, and Mr Elton actually accepting the offer with much prompt satisfaction. It was a done thing; Mr Elton was to go, and never had his broad handsome face expressed more pleasure than at this moment; never had his smile been stronger, nor his eyes more exulting than when he next looked at her.

<Well,> said she to herself, <this is most strange!—After I had got him off so well, to chuse to go into company, and leave Harriet ill behind!—Most strange indeed!—But there is, I believe, in many men, especially single men, such an inclination—such a passion for dining out—a dinner engagement is so high in the class of their pleasures, their employments, their dignities, almost their duties, that any thing gives way to it—and this must be the case with Mr Elton; a most valuable, amiable, pleasing young man undoubtedly, and very much in love with Harriet; but still, he cannot refuse an invitation, he must dine out wherever he is asked. What a strange thing love is! he can see ready wit in Harriet, but will not dine alone for her.>

Soon afterwards Mr Elton quitted them, and she could not but do him the justice of feeling that there was a great deal of sentiment in his manner of naming Harriet at parting; in the tone of his voice while assuring her that he should call at Mrs Goddard's for news of her fair friend, the last thing before he prepared for the happiness of meeting her again, when he hoped to be able to give a better report; and he sighed and smiled himself off in a way that left the balance of approbation much in his favour.

After a few minutes of entire silence between them, John Knightley began with—

<I never in my life saw a man more intent on being agreeable than Mr Elton. It is downright labour to him where ladies are concerned. With men he can be rational and unaffected, but when he has ladies to please, every feature works.>

<Mr Elton's manners are not perfect,> replied Emma; <but where there is a wish to please, one ought to overlook, and one does overlook a great deal. Where a man does his best with only moderate powers, he will have the advantage over negligent superiority. There is such perfect good-temper and good-will in Mr Elton as one cannot but value.>

<Yes,> said Mr John Knightley presently, with some slyness, <he seems to have a great deal of good-will towards you.>

<Me!> she replied with a smile of astonishment, <are you imagining me to be Mr Elton's object?>

<Such an imagination has crossed me, I own, Emma; and if it never occurred to you before, you may as well take it into consideration now.>

<Mr Elton in love with me!—What an idea!>

<I do not say it is so; but you will do well to consider whether it is so or not, and to regulate your behaviour accordingly. I think your manners to him encouraging. I speak as a friend, Emma. You had better look about you, and ascertain what you do, and what you mean to do.>

<I thank you; but I assure you you are quite mistaken. Mr Elton and I are very good friends, and nothing more;> and she walked on, amusing herself in the consideration of the blunders which often arise from a partial knowledge of circumstances, of the mistakes which people of high pretensions to judgment are for ever falling into; and not very well pleased with her brother for imagining her blind and ignorant, and in want of counsel. He said no more.

Mr Woodhouse had so completely made up his mind to the visit, that in spite of the increasing coldness, he seemed to have no idea of shrinking from it, and set forward at last most punctually with his eldest daughter in his own carriage, with less apparent consciousness of the weather than either of the others; too full of the wonder of his own going, and the pleasure it was to afford at Randalls to see that it was cold, and too well wrapt up to feel it. The cold, however, was severe; and by the time the second carriage was in motion, a few flakes of snow were finding their way down, and the sky had the appearance of being so overcharged as to want only a milder air to produce a very white world in a very short time.

Emma soon saw that her companion was not in the happiest humour. The preparing and the going abroad in such weather, with the sacrifice of his children after dinner, were evils, were disagreeables at least, which Mr John Knightley did not by any means like; he anticipated nothing in the visit that could be at all worth the purchase; and the whole of their drive to the vicarage was spent by him in expressing his discontent.

<A man,> said he, <must have a very good opinion of himself when he asks people to leave their own fireside, and encounter such a day as this, for the sake of coming to see him. He must think himself a most agreeable fellow; I could not do such a thing. It is the greatest absurdity—Actually snowing at this moment!—The folly of not allowing people to be comfortable at home—and the folly of people's not staying comfortably at home when they can! If we were obliged to go out such an evening as this, by any call of duty or business, what a hardship we should deem it;—and here are we, probably with rather thinner clothing than usual, setting forward voluntarily, without excuse, in defiance of the voice of nature, which tells man, in every thing given to his view or his feelings, to stay at home himself, and keep all under shelter that he can;—here are we setting forward to spend five dull hours in another man's house, with nothing to say or to hear that was not said and heard yesterday, and may not be said and heard again to-morrow. Going in dismal weather, to return probably in worse;—four horses and four servants taken out for nothing but to convey five idle, shivering creatures into colder rooms and worse company than they might have had at home.>

Emma did not find herself equal to give the pleased assent, which no doubt he was in the habit of receiving, to emulate the <Very true, my love,> which must have been usually administered by his travelling companion; but she had resolution enough to refrain from making any answer at all. She could not be complying, she dreaded being quarrelsome; her heroism reached only to silence. She allowed him to talk, and arranged the glasses, and wrapped herself up, without opening her lips.

They arrived, the carriage turned, the step was let down, and Mr Elton, spruce, black, and smiling, was with them instantly. Emma thought with pleasure of some change of subject. Mr Elton was all obligation and cheerfulness; he was so very cheerful in his civilities indeed, that she began to think he must have received a different account of Harriet from what had reached her. She had sent while dressing, and the answer had been, <Much the same—not better.>

<My report from Mrs Goddard's,> said she presently, <was not so pleasant as I had hoped\longdash <Not better> was my answer.>

His face lengthened immediately; and his voice was the voice of sentiment as he answered.

<Oh! no—I am grieved to find—I was on the point of telling you that when I called at Mrs Goddard's door, which I did the very last thing before I returned to dress, I was told that Miss Smith was not better, by no means better, rather worse. Very much grieved and concerned—I had flattered myself that she must be better after such a cordial as I knew had been given her in the morning.>

Emma smiled and answered—<My visit was of use to the nervous part of her complaint, I hope; but not even I can charm away a sore throat; it is a most severe cold indeed. Mr Perry has been with her, as you probably heard.>

<Yes—I imagined—that is—I did not\longdash>

<He has been used to her in these complaints, and I hope to-morrow morning will bring us both a more comfortable report. But it is impossible not to feel uneasiness. Such a sad loss to our party to-day!>

<Dreadful!—Exactly so, indeed.—She will be missed every moment.>

This was very proper; the sigh which accompanied it was really estimable; but it should have lasted longer. Emma was rather in dismay when only half a minute afterwards he began to speak of other things, and in a voice of the greatest alacrity and enjoyment.

<What an excellent device,> said he, <the use of a sheepskin for carriages. How very comfortable they make it;—impossible to feel cold with such precautions. The contrivances of modern days indeed have rendered a gentleman's carriage perfectly complete. One is so fenced and guarded from the weather, that not a breath of air can find its way unpermitted. Weather becomes absolutely of no consequence. It is a very cold afternoon—but in this carriage we know nothing of the matter.—Ha! snows a little I see.>

<Yes,> said John Knightley, <and I think we shall have a good deal of it.>

<Christmas weather,> observed Mr Elton. <Quite seasonable; and extremely fortunate we may think ourselves that it did not begin yesterday, and prevent this day's party, which it might very possibly have done, for Mr Woodhouse would hardly have ventured had there been much snow on the ground; but now it is of no consequence. This is quite the season indeed for friendly meetings. At Christmas every body invites their friends about them, and people think little of even the worst weather. I was snowed up at a friend's house once for a week. Nothing could be pleasanter. I went for only one night, and could not get away till that very day se'nnight.>

Mr John Knightley looked as if he did not comprehend the pleasure, but said only, coolly,

<I cannot wish to be snowed up a week at Randalls.>

At another time Emma might have been amused, but she was too much astonished now at Mr Elton's spirits for other feelings. Harriet seemed quite forgotten in the expectation of a pleasant party.

<We are sure of excellent fires,> continued he, <and every thing in the greatest comfort. Charming people, Mr and Mrs Weston;—Mrs Weston indeed is much beyond praise, and he is exactly what one values, so hospitable, and so fond of society;—it will be a small party, but where small parties are select, they are perhaps the most agreeable of any. Mr Weston's dining-room does not accommodate more than ten comfortably; and for my part, I would rather, under such circumstances, fall short by two than exceed by two. I think you will agree with me, (turning with a soft air to Emma,) I think I shall certainly have your approbation, though Mr Knightley perhaps, from being used to the large parties of London, may not quite enter into our feelings.>

<I know nothing of the large parties of London, sir—I never dine with any body.>

<Indeed! (in a tone of wonder and pity,) I had no idea that the law had been so great a slavery. Well, sir, the time must come when you will be paid for all this, when you will have little labour and great enjoyment.>

<My first enjoyment,> replied John Knightley, as they passed through the sweep-gate, <will be to find myself safe at Hartfield again.>