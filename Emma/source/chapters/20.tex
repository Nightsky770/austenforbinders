%!TeX root=../emmatop.tex
\chapter[Chapter \thechapter]{}
\lettrine[lines=4,lraise=0.3]{J}{ane} Fairfax was an orphan, the only child of Mrs Bates's youngest daughter.

\zz
The marriage of Lieut. Fairfax of the ——regiment of infantry, and Miss Jane Bates, had had its day of fame and pleasure, hope and interest; but nothing now remained of it, save the melancholy remembrance of him dying in action abroad—of his widow sinking under consumption and grief soon afterwards—and this girl.

By birth she belonged to Highbury: and when at three years old, on losing her mother, she became the property, the charge, the consolation, the foundling of her grandmother and aunt, there had seemed every probability of her being permanently fixed there; of her being taught only what very limited means could command, and growing up with no advantages of connexion or improvement, to be engrafted on what nature had given her in a pleasing person, good understanding, and warm-hearted, well-meaning relations.

But the compassionate feelings of a friend of her father gave a change to her destiny. This was Colonel Campbell, who had very highly regarded Fairfax, as an excellent officer and most deserving young man; and farther, had been indebted to him for such attentions, during a severe camp-fever, as he believed had saved his life. These were claims which he did not learn to overlook, though some years passed away from the death of poor Fairfax, before his own return to England put any thing in his power. When he did return, he sought out the child and took notice of her. He was a married man, with only one living child, a girl, about Jane's age: and Jane became their guest, paying them long visits and growing a favourite with all; and before she was nine years old, his daughter's great fondness for her, and his own wish of being a real friend, united to produce an offer from Colonel Campbell of undertaking the whole charge of her education. It was accepted; and from that period Jane had belonged to Colonel Campbell's family, and had lived with them entirely, only visiting her grandmother from time to time.

The plan was that she should be brought up for educating others; the very few hundred pounds which she inherited from her father making independence impossible. To provide for her otherwise was out of Colonel Campbell's power; for though his income, by pay and appointments, was handsome, his fortune was moderate and must be all his daughter's; but, by giving her an education, he hoped to be supplying the means of respectable subsistence hereafter.

Such was Jane Fairfax's history. She had fallen into good hands, known nothing but kindness from the Campbells, and been given an excellent education. Living constantly with right-minded and well-informed people, her heart and understanding had received every advantage of discipline and culture; and Colonel Campbell's residence being in London, every lighter talent had been done full justice to, by the attendance of first-rate masters. Her disposition and abilities were equally worthy of all that friendship could do; and at eighteen or nineteen she was, as far as such an early age can be qualified for the care of children, fully competent to the office of instruction herself; but she was too much beloved to be parted with. Neither father nor mother could promote, and the daughter could not endure it. The evil day was put off. It was easy to decide that she was still too young; and Jane remained with them, sharing, as another daughter, in all the rational pleasures of an elegant society, and a judicious mixture of home and amusement, with only the drawback of the future, the sobering suggestions of her own good understanding to remind her that all this might soon be over.

The affection of the whole family, the warm attachment of Miss Campbell in particular, was the more honourable to each party from the circumstance of Jane's decided superiority both in beauty and acquirements. That nature had given it in feature could not be unseen by the young woman, nor could her higher powers of mind be unfelt by the parents. They continued together with unabated regard however, till the marriage of Miss Campbell, who by that chance, that luck which so often defies anticipation in matrimonial affairs, giving attraction to what is moderate rather than to what is superior, engaged the affections of Mr Dixon, a young man, rich and agreeable, almost as soon as they were acquainted; and was eligibly and happily settled, while Jane Fairfax had yet her bread to earn.

This event had very lately taken place; too lately for any thing to be yet attempted by her less fortunate friend towards entering on her path of duty; though she had now reached the age which her own judgment had fixed on for beginning. She had long resolved that one-and-twenty should be the period. With the fortitude of a devoted novitiate, she had resolved at one-and-twenty to complete the sacrifice, and retire from all the pleasures of life, of rational intercourse, equal society, peace and hope, to penance and mortification for ever.

The good sense of Colonel and Mrs Campbell could not oppose such a resolution, though their feelings did. As long as they lived, no exertions would be necessary, their home might be hers for ever; and for their own comfort they would have retained her wholly; but this would be selfishness:—what must be at last, had better be soon. Perhaps they began to feel it might have been kinder and wiser to have resisted the temptation of any delay, and spared her from a taste of such enjoyments of ease and leisure as must now be relinquished. Still, however, affection was glad to catch at any reasonable excuse for not hurrying on the wretched moment. She had never been quite well since the time of their daughter's marriage; and till she should have completely recovered her usual strength, they must forbid her engaging in duties, which, so far from being compatible with a weakened frame and varying spirits, seemed, under the most favourable circumstances, to require something more than human perfection of body and mind to be discharged with tolerable comfort.

With regard to her not accompanying them to Ireland, her account to her aunt contained nothing but truth, though there might be some truths not told. It was her own choice to give the time of their absence to Highbury; to spend, perhaps, her last months of perfect liberty with those kind relations to whom she was so very dear: and the Campbells, whatever might be their motive or motives, whether single, or double, or treble, gave the arrangement their ready sanction, and said, that they depended more on a few months spent in her native air, for the recovery of her health, than on any thing else. Certain it was that she was to come; and that Highbury, instead of welcoming that perfect novelty which had been so long promised it—Mr Frank Churchill—must put up for the present with Jane Fairfax, who could bring only the freshness of a two years' absence.

Emma was sorry;—to have to pay civilities to a person she did not like through three long months!—to be always doing more than she wished, and less than she ought! Why she did not like Jane Fairfax might be a difficult question to answer; Mr Knightley had once told her it was because she saw in her the really accomplished young woman, which she wanted to be thought herself; and though the accusation had been eagerly refuted at the time, there were moments of self-examination in which her conscience could not quite acquit her. But »she could never get acquainted with her: she did not know how it was, but there was such coldness and reserve—such apparent indifference whether she pleased or not—and then, her aunt was such an eternal talker!—and she was made such a fuss with by every body!—and it had been always imagined that they were to be so intimate—because their ages were the same, every body had supposed they must be so fond of each other.« These were her reasons—she had no better.

It was a dislike so little just—every imputed fault was so magnified by fancy, that she never saw Jane Fairfax the first time after any considerable absence, without feeling that she had injured her; and now, when the due visit was paid, on her arrival, after a two years' interval, she was particularly struck with the very appearance and manners, which for those two whole years she had been depreciating. Jane Fairfax was very elegant, remarkably elegant; and she had herself the highest value for elegance. Her height was pretty, just such as almost every body would think tall, and nobody could think very tall; her figure particularly graceful; her size a most becoming medium, between fat and thin, though a slight appearance of ill-health seemed to point out the likeliest evil of the two. Emma could not but feel all this; and then, her face—her features—there was more beauty in them altogether than she had remembered; it was not regular, but it was very pleasing beauty. Her eyes, a deep grey, with dark eye-lashes and eyebrows, had never been denied their praise; but the skin, which she had been used to cavil at, as wanting colour, had a clearness and delicacy which really needed no fuller bloom. It was a style of beauty, of which elegance was the reigning character, and as such, she must, in honour, by all her principles, admire it:—elegance, which, whether of person or of mind, she saw so little in Highbury. There, not to be vulgar, was distinction, and merit.

In short, she sat, during the first visit, looking at Jane Fairfax with twofold complacency; the sense of pleasure and the sense of rendering justice, and was determining that she would dislike her no longer. When she took in her history, indeed, her situation, as well as her beauty; when she considered what all this elegance was destined to, what she was going to sink from, how she was going to live, it seemed impossible to feel any thing but compassion and respect; especially, if to every well-known particular entitling her to interest, were added the highly probable circumstance of an attachment to Mr Dixon, which she had so naturally started to herself. In that case, nothing could be more pitiable or more honourable than the sacrifices she had resolved on. Emma was very willing now to acquit her of having seduced Mr Dixon's affections from his wife, or of any thing mischievous which her imagination had suggested at first. If it were love, it might be simple, single, successless love on her side alone. She might have been unconsciously sucking in the sad poison, while a sharer of his conversation with her friend; and from the best, the purest of motives, might now be denying herself this visit to Ireland, and resolving to divide herself effectually from him and his connexions by soon beginning her career of laborious duty.

Upon the whole, Emma left her with such softened, charitable feelings, as made her look around in walking home, and lament that Highbury afforded no young man worthy of giving her independence; nobody that she could wish to scheme about for her.

These were charming feelings—but not lasting. Before she had committed herself by any public profession of eternal friendship for Jane Fairfax, or done more towards a recantation of past prejudices and errors, than saying to Mr Knightley, »She certainly is handsome; she is better than handsome!« Jane had spent an evening at Hartfield with her grandmother and aunt, and every thing was relapsing much into its usual state. Former provocations reappeared. The aunt was as tiresome as ever; more tiresome, because anxiety for her health was now added to admiration of her powers; and they had to listen to the description of exactly how little bread and butter she ate for breakfast, and how small a slice of mutton for dinner, as well as to see exhibitions of new caps and new workbags for her mother and herself; and Jane's offences rose again. They had music; Emma was obliged to play; and the thanks and praise which necessarily followed appeared to her an affectation of candour, an air of greatness, meaning only to shew off in higher style her own very superior performance. She was, besides, which was the worst of all, so cold, so cautious! There was no getting at her real opinion. Wrapt up in a cloak of politeness, she seemed determined to hazard nothing. She was disgustingly, was suspiciously reserved.

If any thing could be more, where all was most, she was more reserved on the subject of Weymouth and the Dixons than any thing. She seemed bent on giving no real insight into Mr Dixon's character, or her own value for his company, or opinion of the suitableness of the match. It was all general approbation and smoothness; nothing delineated or distinguished. It did her no service however. Her caution was thrown away. Emma saw its artifice, and returned to her first surmises. There probably was something more to conceal than her own preference; Mr Dixon, perhaps, had been very near changing one friend for the other, or been fixed only to Miss Campbell, for the sake of the future twelve thousand pounds.

The like reserve prevailed on other topics. She and Mr Frank Churchill had been at Weymouth at the same time. It was known that they were a little acquainted; but not a syllable of real information could Emma procure as to what he truly was. »Was he handsome?«—»She believed he was reckoned a very fine young man.« »Was he agreeable?«—»He was generally thought so.« »Did he appear a sensible young man; a young man of information?«—»At a watering-place, or in a common London acquaintance, it was difficult to decide on such points. Manners were all that could be safely judged of, under a much longer knowledge than they had yet had of Mr Churchill. She believed every body found his manners pleasing.« Emma could not forgive her.