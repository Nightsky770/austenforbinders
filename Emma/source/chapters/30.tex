%!TeX root=../emmatop.tex
\chapter[Chapter \thechapter]{}
\lettrine[lraise=0.3]{O}{ne} thing only was wanting to make the prospect of the ball completely satisfactory to Emma—its being fixed for a day within the granted term of Frank Churchill's stay in Surry; for, in spite of Mr Weston's confidence, she could not think it so very impossible that the Churchills might not allow their nephew to remain a day beyond his fortnight. But this was not judged feasible. The preparations must take their time, nothing could be properly ready till the third week were entered on, and for a few days they must be planning, proceeding and hoping in uncertainty—at the risk—in her opinion, the great risk, of its being all in vain.

Enscombe however was gracious, gracious in fact, if not in word. His wish of staying longer evidently did not please; but it was not opposed. All was safe and prosperous; and as the removal of one solicitude generally makes way for another, Emma, being now certain of her ball, began to adopt as the next vexation Mr Knightley's provoking indifference about it. Either because he did not dance himself, or because the plan had been formed without his being consulted, he seemed resolved that it should not interest him, determined against its exciting any present curiosity, or affording him any future amusement. To her voluntary communications Emma could get no more approving reply, than,

<Very well. If the Westons think it worth while to be at all this trouble for a few hours of noisy entertainment, I have nothing to say against it, but that they shall not chuse pleasures for me.—Oh! yes, I must be there; I could not refuse; and I will keep as much awake as I can; but I would rather be at home, looking over William Larkins's week's account; much rather, I confess.—Pleasure in seeing dancing!—not I, indeed—I never look at it—I do not know who does.—Fine dancing, I believe, like virtue, must be its own reward. Those who are standing by are usually thinking of something very different.>

This Emma felt was aimed at her; and it made her quite angry. It was not in compliment to Jane Fairfax however that he was so indifferent, or so indignant; he was not guided by her feelings in reprobating the ball, for she enjoyed the thought of it to an extraordinary degree. It made her animated—open hearted—she voluntarily said;—

<Oh! Miss Woodhouse, I hope nothing may happen to prevent the ball. What a disappointment it would be! I do look forward to it, I own, with very great pleasure.>

It was not to oblige Jane Fairfax therefore that he would have preferred the society of William Larkins. No!—she was more and more convinced that Mrs Weston was quite mistaken in that surmise. There was a great deal of friendly and of compassionate attachment on his side—but no love.

Alas! there was soon no leisure for quarrelling with Mr Knightley. Two days of joyful security were immediately followed by the over-throw of every thing. A letter arrived from Mr Churchill to urge his nephew's instant return. Mrs Churchill was unwell—far too unwell to do without him; she had been in a very suffering state (so said her husband) when writing to her nephew two days before, though from her usual unwillingness to give pain, and constant habit of never thinking of herself, she had not mentioned it; but now she was too ill to trifle, and must entreat him to set off for Enscombe without delay.

The substance of this letter was forwarded to Emma, in a note from Mrs Weston, instantly. As to his going, it was inevitable. He must be gone within a few hours, though without feeling any real alarm for his aunt, to lessen his repugnance. He knew her illnesses; they never occurred but for her own convenience.

Mrs Weston added, <that he could only allow himself time to hurry to Highbury, after breakfast, and take leave of the few friends there whom he could suppose to feel any interest in him; and that he might be expected at Hartfield very soon.>

This wretched note was the finale of Emma's breakfast. When once it had been read, there was no doing any thing, but lament and exclaim. The loss of the ball—the loss of the young man—and all that the young man might be feeling!—It was too wretched!—Such a delightful evening as it would have been!—Every body so happy! and she and her partner the happiest!—<I said it would be so,> was the only consolation.

Her father's feelings were quite distinct. He thought principally of Mrs Churchill's illness, and wanted to know how she was treated; and as for the ball, it was shocking to have dear Emma disappointed; but they would all be safer at home.

Emma was ready for her visitor some time before he appeared; but if this reflected at all upon his impatience, his sorrowful look and total want of spirits when he did come might redeem him. He felt the going away almost too much to speak of it. His dejection was most evident. He sat really lost in thought for the first few minutes; and when rousing himself, it was only to say,

<Of all horrid things, leave-taking is the worst.>

<But you will come again,> said Emma. <This will not be your only visit to Randalls.>

<Ah!—(shaking his head)—the uncertainty of when I may be able to return!—I shall try for it with a zeal!—It will be the object of all my thoughts and cares!—and if my uncle and aunt go to town this spring—but I am afraid—they did not stir last spring—I am afraid it is a custom gone for ever.>

<Our poor ball must be quite given up.>

<Ah! that ball!—why did we wait for any thing?—why not seize the pleasure at once?—How often is happiness destroyed by preparation, foolish preparation!—You told us it would be so.—Oh! Miss Woodhouse, why are you always so right?>

<Indeed, I am very sorry to be right in this instance. I would much rather have been merry than wise.>

<If I can come again, we are still to have our ball. My father depends on it. Do not forget your engagement.>

Emma looked graciously.

<Such a fortnight as it has been!> he continued; <every day more precious and more delightful than the day before!—every day making me less fit to bear any other place. Happy those, who can remain at Highbury!>

<As you do us such ample justice now,> said Emma, laughing, <I will venture to ask, whether you did not come a little doubtfully at first? Do not we rather surpass your expectations? I am sure we do. I am sure you did not much expect to like us. You would not have been so long in coming, if you had had a pleasant idea of Highbury.>

He laughed rather consciously; and though denying the sentiment, Emma was convinced that it had been so.

<And you must be off this very morning?>

<Yes; my father is to join me here: we shall walk back together, and I must be off immediately. I am almost afraid that every moment will bring him.>

<Not five minutes to spare even for your friends Miss Fairfax and Miss Bates? How unlucky! Miss Bates's powerful, argumentative mind might have strengthened yours.>

<Yes—I have called there; passing the door, I thought it better. It was a right thing to do. I went in for three minutes, and was detained by Miss Bates's being absent. She was out; and I felt it impossible not to wait till she came in. She is a woman that one may, that one must laugh at; but that one would not wish to slight. It was better to pay my visit, then>—

He hesitated, got up, walked to a window.

<In short,> said he, <perhaps, Miss Woodhouse—I think you can hardly be quite without suspicion>—

He looked at her, as if wanting to read her thoughts. She hardly knew what to say. It seemed like the forerunner of something absolutely serious, which she did not wish. Forcing herself to speak, therefore, in the hope of putting it by, she calmly said,

<You are quite in the right; it was most natural to pay your visit, then>—

He was silent. She believed he was looking at her; probably reflecting on what she had said, and trying to understand the manner. She heard him sigh. It was natural for him to feel that he had cause to sigh. He could not believe her to be encouraging him. A few awkward moments passed, and he sat down again; and in a more determined manner said,

<It was something to feel that all the rest of my time might be given to Hartfield. My regard for Hartfield is most warm>—

He stopt again, rose again, and seemed quite embarrassed.—He was more in love with her than Emma had supposed; and who can say how it might have ended, if his father had not made his appearance? Mr Woodhouse soon followed; and the necessity of exertion made him composed.

A very few minutes more, however, completed the present trial. Mr Weston, always alert when business was to be done, and as incapable of procrastinating any evil that was inevitable, as of foreseeing any that was doubtful, said, <It was time to go;> and the young man, though he might and did sigh, could not but agree, to take leave.

<I shall hear about you all,> said he; <that is my chief consolation. I shall hear of every thing that is going on among you. I have engaged Mrs Weston to correspond with me. She has been so kind as to promise it. Oh! the blessing of a female correspondent, when one is really interested in the absent!—she will tell me every thing. In her letters I shall be at dear Highbury again.>

A very friendly shake of the hand, a very earnest <Good-bye,> closed the speech, and the door had soon shut out Frank Churchill. Short had been the notice—short their meeting; he was gone; and Emma felt so sorry to part, and foresaw so great a loss to their little society from his absence as to begin to be afraid of being too sorry, and feeling it too much.

It was a sad change. They had been meeting almost every day since his arrival. Certainly his being at Randalls had given great spirit to the last two weeks—indescribable spirit; the idea, the expectation of seeing him which every morning had brought, the assurance of his attentions, his liveliness, his manners! It had been a very happy fortnight, and forlorn must be the sinking from it into the common course of Hartfield days. To complete every other recommendation, he had almost told her that he loved her. What strength, or what constancy of affection he might be subject to, was another point; but at present she could not doubt his having a decidedly warm admiration, a conscious preference of herself; and this persuasion, joined to all the rest, made her think that she must be a little in love with him, in spite of every previous determination against it.

<I certainly must,> said she. <This sensation of listlessness, weariness, stupidity, this disinclination to sit down and employ myself, this feeling of every thing's being dull and insipid about the house!— I must be in love; I should be the oddest creature in the world if I were not—for a few weeks at least. Well! evil to some is always good to others. I shall have many fellow-mourners for the ball, if not for Frank Churchill; but Mr Knightley will be happy. He may spend the evening with his dear William Larkins now if he likes.>

Mr Knightley, however, shewed no triumphant happiness. He could not say that he was sorry on his own account; his very cheerful look would have contradicted him if he had; but he said, and very steadily, that he was sorry for the disappointment of the others, and with considerable kindness added,

<You, Emma, who have so few opportunities of dancing, you are really out of luck; you are very much out of luck!>

It was some days before she saw Jane Fairfax, to judge of her honest regret in this woeful change; but when they did meet, her composure was odious. She had been particularly unwell, however, suffering from headache to a degree, which made her aunt declare, that had the ball taken place, she did not think Jane could have attended it; and it was charity to impute some of her unbecoming indifference to the languor of ill-health.

