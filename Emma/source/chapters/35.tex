%!TeX root=../emmatop.tex
\chapter[Chapter \thechapter]{}
\lettrine[lines=4,lraise=0.35]{W}{hen} the ladies returned to the drawing-room after dinner, Emma found it hardly possible to prevent their making two distinct parties;—with so much perseverance in judging and behaving ill did Mrs Elton engross Jane Fairfax and slight herself. She and Mrs Weston were obliged to be almost always either talking together or silent together. Mrs Elton left them no choice. If Jane repressed her for a little time, she soon began again; and though much that passed between them was in a half-whisper, especially on Mrs Elton's side, there was no avoiding a knowledge of their principal subjects: The post-office—catching cold—fetching letters—and friendship, were long under discussion; and to them succeeded one, which must be at least equally unpleasant to Jane—inquiries whether she had yet heard of any situation likely to suit her, and professions of Mrs Elton's meditated activity.

»Here is April come!« said she, »I get quite anxious about you. June will soon be here.«

»But I have never fixed on June or any other month—merely looked forward to the summer in general.«

»But have you really heard of nothing?«

»I have not even made any inquiry; I do not wish to make any yet.«

»Oh! my dear, we cannot begin too early; you are not aware of the difficulty of procuring exactly the desirable thing.«

»I not aware!« said Jane, shaking her head; »dear Mrs Elton, who can have thought of it as I have done?«

»But you have not seen so much of the world as I have. You do not know how many candidates there always are for the first situations. I saw a vast deal of that in the neighbourhood round Maple Grove. A cousin of Mr Suckling, Mrs Bragge, had such an infinity of applications; every body was anxious to be in her family, for she moves in the first circle. Wax-candles in the schoolroom! You may imagine how desirable! Of all houses in the kingdom Mrs Bragge's is the one I would most wish to see you in.«

»Colonel and Mrs Campbell are to be in town again by midsummer,« said Jane. »I must spend some time with them; I am sure they will want it;—afterwards I may probably be glad to dispose of myself. But I would not wish you to take the trouble of making any inquiries at present.«

»Trouble! aye, I know your scruples. You are afraid of giving me trouble; but I assure you, my dear Jane, the Campbells can hardly be more interested about you than I am. I shall write to Mrs Partridge in a day or two, and shall give her a strict charge to be on the look-out for any thing eligible.«

»Thank you, but I would rather you did not mention the subject to her; till the time draws nearer, I do not wish to be giving any body trouble.«

»But, my dear child, the time is drawing near; here is April, and June, or say even July, is very near, with such business to accomplish before us. Your inexperience really amuses me! A situation such as you deserve, and your friends would require for you, is no everyday occurrence, is not obtained at a moment's notice; indeed, indeed, we must begin inquiring directly.«

»Excuse me, ma'am, but this is by no means my intention; I make no inquiry myself, and should be sorry to have any made by my friends. When I am quite determined as to the time, I am not at all afraid of being long unemployed. There are places in town, offices, where inquiry would soon produce something—Offices for the sale—not quite of human flesh—but of human intellect.«

»Oh! my dear, human flesh! You quite shock me; if you mean a fling at the slave-trade, I assure you Mr Suckling was always rather a friend to the abolition.«

»I did not mean, I was not thinking of the slave-trade,« replied Jane; »governess-trade, I assure you, was all that I had in view; widely different certainly as to the guilt of those who carry it on; but as to the greater misery of the victims, I do not know where it lies. But I only mean to say that there are advertising offices, and that by applying to them I should have no doubt of very soon meeting with something that would do.«

»Something that would do!« repeated Mrs Elton. »Aye, that may suit your humble ideas of yourself;—I know what a modest creature you are; but it will not satisfy your friends to have you taking up with any thing that may offer, any inferior, commonplace situation, in a family not moving in a certain circle, or able to command the elegancies of life.«

»You are very obliging; but as to all that, I am very indifferent; it would be no object to me to be with the rich; my mortifications, I think, would only be the greater; I should suffer more from comparison. A gentleman's family is all that I should condition for.«

»I know you, I know you; you would take up with any thing; but I shall be a little more nice, and I am sure the good Campbells will be quite on my side; with your superior talents, you have a right to move in the first circle. Your musical knowledge alone would entitle you to name your own terms, have as many rooms as you like, and mix in the family as much as you chose;—that is—I do not know—if you knew the harp, you might do all that, I am very sure; but you sing as well as play;—yes, I really believe you might, even without the harp, stipulate for what you chose;—and you must and shall be delightfully, honourably and comfortably settled before the Campbells or I have any rest.«

»You may well class the delight, the honour, and the comfort of such a situation together,« said Jane, »they are pretty sure to be equal; however, I am very serious in not wishing any thing to be attempted at present for me. I am exceedingly obliged to you, Mrs Elton, I am obliged to any body who feels for me, but I am quite serious in wishing nothing to be done till the summer. For two or three months longer I shall remain where I am, and as I am.«

»And I am quite serious too, I assure you,« replied Mrs Elton gaily, »in resolving to be always on the watch, and employing my friends to watch also, that nothing really unexceptionable may pass us.«

In this style she ran on; never thoroughly stopped by any thing till Mr Woodhouse came into the room; her vanity had then a change of object, and Emma heard her saying in the same half-whisper to Jane,

»Here comes this dear old beau of mine, I protest!—Only think of his gallantry in coming away before the other men!—what a dear creature he is;—I assure you I like him excessively. I admire all that quaint, old-fashioned politeness; it is much more to my taste than modern ease; modern ease often disgusts me. But this good old Mr Woodhouse, I wish you had heard his gallant speeches to me at dinner. Oh! I assure you I began to think my caro sposo would be absolutely jealous. I fancy I am rather a favourite; he took notice of my gown. How do you like it?—Selina's choice—handsome, I think, but I do not know whether it is not over-trimmed; I have the greatest dislike to the idea of being over-trimmed—quite a horror of finery. I must put on a few ornaments now, because it is expected of me. A bride, you know, must appear like a bride, but my natural taste is all for simplicity; a simple style of dress is so infinitely preferable to finery. But I am quite in the minority, I believe; few people seem to value simplicity of dress,—show and finery are every thing. I have some notion of putting such a trimming as this to my white and silver poplin. Do you think it will look well?«

The whole party were but just reassembled in the drawing-room when Mr Weston made his appearance among them. He had returned to a late dinner, and walked to Hartfield as soon as it was over. He had been too much expected by the best judges, for surprize—but there was great joy. Mr Woodhouse was almost as glad to see him now, as he would have been sorry to see him before. John Knightley only was in mute astonishment.—That a man who might have spent his evening quietly at home after a day of business in London, should set off again, and walk half a mile to another man's house, for the sake of being in mixed company till bed-time, of finishing his day in the efforts of civility and the noise of numbers, was a circumstance to strike him deeply. A man who had been in motion since eight o'clock in the morning, and might now have been still, who had been long talking, and might have been silent, who had been in more than one crowd, and might have been alone!—Such a man, to quit the tranquillity and independence of his own fireside, and on the evening of a cold sleety April day rush out again into the world!—Could he by a touch of his finger have instantly taken back his wife, there would have been a motive; but his coming would probably prolong rather than break up the party. John Knightley looked at him with amazement, then shrugged his shoulders, and said, »I could not have believed it even of him.«

Mr Weston meanwhile, perfectly unsuspicious of the indignation he was exciting, happy and cheerful as usual, and with all the right of being principal talker, which a day spent anywhere from home confers, was making himself agreeable among the rest; and having satisfied the inquiries of his wife as to his dinner, convincing her that none of all her careful directions to the servants had been forgotten, and spread abroad what public news he had heard, was proceeding to a family communication, which, though principally addressed to Mrs Weston, he had not the smallest doubt of being highly interesting to every body in the room. He gave her a letter, it was from Frank, and to herself; he had met with it in his way, and had taken the liberty of opening it.

»Read it, read it,« said he, »it will give you pleasure; only a few lines—will not take you long; read it to Emma.«

The two ladies looked over it together; and he sat smiling and talking to them the whole time, in a voice a little subdued, but very audible to every body.

»Well, he is coming, you see; good news, I think. Well, what do you say to it?—I always told you he would be here again soon, did not I?—Anne, my dear, did not I always tell you so, and you would not believe me?—In town next week, you see—at the latest, I dare say; for she is as impatient as the black gentleman when any thing is to be done; most likely they will be there to-morrow or Saturday. As to her illness, all nothing of course. But it is an excellent thing to have Frank among us again, so near as town. They will stay a good while when they do come, and he will be half his time with us. This is precisely what I wanted. Well, pretty good news, is not it? Have you finished it? Has Emma read it all? Put it up, put it up; we will have a good talk about it some other time, but it will not do now. I shall only just mention the circumstance to the others in a common way.«

Mrs Weston was most comfortably pleased on the occasion. Her looks and words had nothing to restrain them. She was happy, she knew she was happy, and knew she ought to be happy. Her congratulations were warm and open; but Emma could not speak so fluently. She was a little occupied in weighing her own feelings, and trying to understand the degree of her agitation, which she rather thought was considerable.

Mr Weston, however, too eager to be very observant, too communicative to want others to talk, was very well satisfied with what she did say, and soon moved away to make the rest of his friends happy by a partial communication of what the whole room must have overheard already.

It was well that he took every body's joy for granted, or he might not have thought either Mr Woodhouse or Mr Knightley particularly delighted. They were the first entitled, after Mrs Weston and Emma, to be made happy;—from them he would have proceeded to Miss Fairfax, but she was so deep in conversation with John Knightley, that it would have been too positive an interruption; and finding himself close to Mrs Elton, and her attention disengaged, he necessarily began on the subject with her.