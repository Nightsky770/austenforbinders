%!TeX root=../emmatop.tex
\chapter[Chapter \thechapter]{}
\lettrine[lines=4,lraise=0.3]{T}{hough} now the middle of December, there had yet been no weather to prevent the young ladies from tolerably regular exercise; and on the morrow, Emma had a charitable visit to pay to a poor sick family, who lived a little way out of Highbury.

Their road to this detached cottage was down Vicarage Lane, a lane leading at right angles from the broad, though irregular, main street of the place; and, as may be inferred, containing the blessed abode of Mr Elton. A few inferior dwellings were first to be passed, and then, about a quarter of a mile down the lane rose the Vicarage, an old and not very good house, almost as close to the road as it could be. It had no advantage of situation; but had been very much smartened up by the present proprietor; and, such as it was, there could be no possibility of the two friends passing it without a slackened pace and observing eyes.—Emma's remark was—

»There it is. There go you and your riddle-book one of these days.«—Harriet's was—

»Oh, what a sweet house!—How very beautiful!—There are the yellow curtains that Miss Nash admires so much.«

»I do not often walk this way \textit{now},« said Emma, as they proceeded, »but \textit{then} there will be an inducement, and I shall gradually get intimately acquainted with all the hedges, gates, pools and pollards of this part of Highbury.«

Harriet, she found, had never in her life been inside the Vicarage, and her curiosity to see it was so extreme, that, considering exteriors and probabilities, Emma could only class it, as a proof of love, with Mr Elton's seeing ready wit in her.

»I wish we could contrive it,« said she; »but I cannot think of any tolerable pretence for going in;—no servant that I want to inquire about of his housekeeper—no message from my father.«

She pondered, but could think of nothing. After a mutual silence of some minutes, Harriet thus began again—

»I do so wonder, Miss Woodhouse, that you should not be married, or going to be married! so charming as you are!«—

Emma laughed, and replied,

»My being charming, Harriet, is not quite enough to induce me to marry; I must find other people charming—one other person at least. And I am not only, not going to be married, at present, but have very little intention of ever marrying at all.«

»Ah!—so you say; but I cannot believe it.«

»I must see somebody very superior to any one I have seen yet, to be tempted; Mr Elton, you know, (recollecting herself,) is out of the question: and I do \textit{not} wish to see any such person. I would rather not be tempted. I cannot really change for the better. If I were to marry, I must expect to repent it.«

»Dear me!—it is so odd to hear a woman talk so!«—

»I have none of the usual inducements of women to marry. Were I to fall in love, indeed, it would be a different thing! but I never have been in love; it is not my way, or my nature; and I do not think I ever shall. And, without love, I am sure I should be a fool to change such a situation as mine. Fortune I do not want; employment I do not want; consequence I do not want: I believe few married women are half as much mistress of their husband's house as I am of Hartfield; and never, never could I expect to be so truly beloved and important; so always first and always right in any man's eyes as I am in my father's.«

»But then, to be an old maid at last, like Miss Bates!«

»That is as formidable an image as you could present, Harriet; and if I thought I should ever be like Miss Bates! so silly—so satisfied—so smiling—so prosing—so undistinguishing and unfastidious—and so apt to tell every thing relative to every body about me, I would marry to-morrow. But between \textit{us}, I am convinced there never can be any likeness, except in being unmarried.«

»But still, you will be an old maid! and that's so dreadful!«

»Never mind, Harriet, I shall not be a poor old maid; and it is poverty only which makes celibacy contemptible to a generous public! A single woman, with a very narrow income, must be a ridiculous, disagreeable old maid! the proper sport of boys and girls, but a single woman, of good fortune, is always respectable, and may be as sensible and pleasant as any body else. And the distinction is not quite so much against the candour and common sense of the world as appears at first; for a very narrow income has a tendency to contract the mind, and sour the temper. Those who can barely live, and who live perforce in a very small, and generally very inferior, society, may well be illiberal and cross. This does not apply, however, to Miss Bates; she is only too good natured and too silly to suit me; but, in general, she is very much to the taste of every body, though single and though poor. Poverty certainly has not contracted her mind: I really believe, if she had only a shilling in the world, she would be very likely to give away sixpence of it; and nobody is afraid of her: that is a great charm.«

»Dear me! but what shall you do? how shall you employ yourself when you grow old?«

»If I know myself, Harriet, mine is an active, busy mind, with a great many independent resources; and I do not perceive why I should be more in want of employment at forty or fifty than one-and-twenty. Woman's usual occupations of hand and mind will be as open to me then as they are now; or with no important variation. If I draw less, I shall read more; if I give up music, I shall take to carpet-work. And as for objects of interest, objects for the affections, which is in truth the great point of inferiority, the want of which is really the great evil to be avoided in \textit{not} marrying, I shall be very well off, with all the children of a sister I love so much, to care about. There will be enough of them, in all probability, to supply every sort of sensation that declining life can need. There will be enough for every hope and every fear; and though my attachment to none can equal that of a parent, it suits my ideas of comfort better than what is warmer and blinder. My nephews and nieces!—I shall often have a niece with me.«

»Do you know Miss Bates's niece? That is, I know you must have seen her a hundred times—but are you acquainted?«

»Oh! yes; we are always forced to be acquainted whenever she comes to Highbury. By the bye, \textit{that} is almost enough to put one out of conceit with a niece. Heaven forbid! at least, that I should ever bore people half so much about all the Knightleys together, as she does about Jane Fairfax. One is sick of the very name of Jane Fairfax. Every letter from her is read forty times over; her compliments to all friends go round and round again; and if she does but send her aunt the pattern of a stomacher, or knit a pair of garters for her grandmother, one hears of nothing else for a month. I wish Jane Fairfax very well; but she tires me to death.«

They were now approaching the cottage, and all idle topics were superseded. Emma was very compassionate; and the distresses of the poor were as sure of relief from her personal attention and kindness, her counsel and her patience, as from her purse. She understood their ways, could allow for their ignorance and their temptations, had no romantic expectations of extraordinary virtue from those for whom education had done so little; entered into their troubles with ready sympathy, and always gave her assistance with as much intelligence as good-will. In the present instance, it was sickness and poverty together which she came to visit; and after remaining there as long as she could give comfort or advice, she quitted the cottage with such an impression of the scene as made her say to Harriet, as they walked away,

»These are the sights, Harriet, to do one good. How trifling they make every thing else appear!—I feel now as if I could think of nothing but these poor creatures all the rest of the day; and yet, who can say how soon it may all vanish from my mind?«

»Very true,« said Harriet. »Poor creatures! one can think of nothing else.«

»And really, I do not think the impression will soon be over,« said Emma, as she crossed the low hedge, and tottering footstep which ended the narrow, slippery path through the cottage garden, and brought them into the lane again. »I do not think it will,« stopping to look once more at all the outward wretchedness of the place, and recall the still greater within.

»Oh! dear, no,« said her companion.

They walked on. The lane made a slight bend; and when that bend was passed, Mr Elton was immediately in sight; and so near as to give Emma time only to say farther,

»Ah! Harriet, here comes a very sudden trial of our stability in good thoughts. Well, (smiling,) I hope it may be allowed that if compassion has produced exertion and relief to the sufferers, it has done all that is truly important. If we feel for the wretched, enough to do all we can for them, the rest is empty sympathy, only distressing to ourselves.«

Harriet could just answer, »Oh! dear, yes,« before the gentleman joined them. The wants and sufferings of the poor family, however, were the first subject on meeting. He had been going to call on them. His visit he would now defer; but they had a very interesting parley about what could be done and should be done. Mr Elton then turned back to accompany them.

»To fall in with each other on such an errand as this,« thought Emma; »to meet in a charitable scheme; this will bring a great increase of love on each side. I should not wonder if it were to bring on the declaration. It must, if I were not here. I wish I were anywhere else.«

Anxious to separate herself from them as far as she could, she soon afterwards took possession of a narrow footpath, a little raised on one side of the lane, leaving them together in the main road. But she had not been there two minutes when she found that Harriet's habits of dependence and imitation were bringing her up too, and that, in short, they would both be soon after her. This would not do; she immediately stopped, under pretence of having some alteration to make in the lacing of her half-boot, and stooping down in complete occupation of the footpath, begged them to have the goodness to walk on, and she would follow in half a minute. They did as they were desired; and by the time she judged it reasonable to have done with her boot, she had the comfort of farther delay in her power, being overtaken by a child from the cottage, setting out, according to orders, with her pitcher, to fetch broth from Hartfield. To walk by the side of this child, and talk to and question her, was the most natural thing in the world, or would have been the most natural, had she been acting just then without design; and by this means the others were still able to keep ahead, without any obligation of waiting for her. She gained on them, however, involuntarily: the child's pace was quick, and theirs rather slow; and she was the more concerned at it, from their being evidently in a conversation which interested them. Mr Elton was speaking with animation, Harriet listening with a very pleased attention; and Emma, having sent the child on, was beginning to think how she might draw back a little more, when they both looked around, and she was obliged to join them.

Mr Elton was still talking, still engaged in some interesting detail; and Emma experienced some disappointment when she found that he was only giving his fair companion an account of the yesterday's party at his friend Cole's, and that she was come in herself for the Stilton cheese, the north Wiltshire, the butter, the celery, the beet-root, and all the dessert.

»This would soon have led to something better, of course,« was her consoling reflection; »any thing interests between those who love; and any thing will serve as introduction to what is near the heart. If I could but have kept longer away!«

They now walked on together quietly, till within view of the vicarage pales, when a sudden resolution, of at least getting Harriet into the house, made her again find something very much amiss about her boot, and fall behind to arrange it once more. She then broke the lace off short, and dexterously throwing it into a ditch, was presently obliged to entreat them to stop, and acknowledged her inability to put herself to rights so as to be able to walk home in tolerable comfort.

»Part of my lace is gone,« said she, »and I do not know how I am to contrive. I really am a most troublesome companion to you both, but I hope I am not often so ill-equipped. Mr Elton, I must beg leave to stop at your house, and ask your housekeeper for a bit of ribband or string, or any thing just to keep my boot on.«

Mr Elton looked all happiness at this proposition; and nothing could exceed his alertness and attention in conducting them into his house and endeavouring to make every thing appear to advantage. The room they were taken into was the one he chiefly occupied, and looking forwards; behind it was another with which it immediately communicated; the door between them was open, and Emma passed into it with the housekeeper to receive her assistance in the most comfortable manner. She was obliged to leave the door ajar as she found it; but she fully intended that Mr Elton should close it. It was not closed, however, it still remained ajar; but by engaging the housekeeper in incessant conversation, she hoped to make it practicable for him to chuse his own subject in the adjoining room. For ten minutes she could hear nothing but herself. It could be protracted no longer. She was then obliged to be finished, and make her appearance.

The lovers were standing together at one of the windows. It had a most favourable aspect; and, for half a minute, Emma felt the glory of having schemed successfully. But it would not do; he had not come to the point. He had been most agreeable, most delightful; he had told Harriet that he had seen them go by, and had purposely followed them; other little gallantries and allusions had been dropt, but nothing serious.

»Cautious, very cautious,« thought Emma; »he advances inch by inch, and will hazard nothing till he believes himself secure.«

Still, however, though every thing had not been accomplished by her ingenious device, she could not but flatter herself that it had been the occasion of much present enjoyment to both, and must be leading them forward to the great event.