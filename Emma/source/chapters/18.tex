%!TeX root=../emmatop.tex
\chapter[Chapter \thechapter]{}
\lettrine[lraise=0.3]{M}{r} Frank Churchill did not come. When the time proposed drew near, Mrs Weston's fears were justified in the arrival of a letter of excuse. For the present, he could not be spared, to his <very great mortification and regret; but still he looked forward with the hope of coming to Randalls at no distant period.>

Mrs Weston was exceedingly disappointed—much more disappointed, in fact, than her husband, though her dependence on seeing the young man had been so much more sober: but a sanguine temper, though for ever expecting more good than occurs, does not always pay for its hopes by any proportionate depression. It soon flies over the present failure, and begins to hope again. For half an hour Mr Weston was surprized and sorry; but then he began to perceive that Frank's coming two or three months later would be a much better plan; better time of year; better weather; and that he would be able, without any doubt, to stay considerably longer with them than if he had come sooner.

These feelings rapidly restored his comfort, while Mrs Weston, of a more apprehensive disposition, foresaw nothing but a repetition of excuses and delays; and after all her concern for what her husband was to suffer, suffered a great deal more herself.

Emma was not at this time in a state of spirits to care really about Mr Frank Churchill's not coming, except as a disappointment at Randalls. The acquaintance at present had no charm for her. She wanted, rather, to be quiet, and out of temptation; but still, as it was desirable that she should appear, in general, like her usual self, she took care to express as much interest in the circumstance, and enter as warmly into Mr and Mrs Weston's disappointment, as might naturally belong to their friendship.

She was the first to announce it to Mr Knightley; and exclaimed quite as much as was necessary, (or, being acting a part, perhaps rather more,) at the conduct of the Churchills, in keeping him away. She then proceeded to say a good deal more than she felt, of the advantage of such an addition to their confined society in Surry; the pleasure of looking at somebody new; the gala-day to Highbury entire, which the sight of him would have made; and ending with reflections on the Churchills again, found herself directly involved in a disagreement with Mr Knightley; and, to her great amusement, perceived that she was taking the other side of the question from her real opinion, and making use of Mrs Weston's arguments against herself.

<The Churchills are very likely in fault,> said Mr Knightley, coolly; <but I dare say he might come if he would.>

<I do not know why you should say so. He wishes exceedingly to come; but his uncle and aunt will not spare him.>

<I cannot believe that he has not the power of coming, if he made a point of it. It is too unlikely, for me to believe it without proof.>

<How odd you are! What has Mr Frank Churchill done, to make you suppose him such an unnatural creature?>

<I am not supposing him at all an unnatural creature, in suspecting that he may have learnt to be above his connexions, and to care very little for any thing but his own pleasure, from living with those who have always set him the example of it. It is a great deal more natural than one could wish, that a young man, brought up by those who are proud, luxurious, and selfish, should be proud, luxurious, and selfish too. If Frank Churchill had wanted to see his father, he would have contrived it between September and January. A man at his age—what is he?—three or four-and-twenty—cannot be without the means of doing as much as that. It is impossible.>

<That's easily said, and easily felt by you, who have always been your own master. You are the worst judge in the world, Mr Knightley, of the difficulties of dependence. You do not know what it is to have tempers to manage.>

<It is not to be conceived that a man of three or four-and-twenty should not have liberty of mind or limb to that amount. He cannot want money—he cannot want leisure. We know, on the contrary, that he has so much of both, that he is glad to get rid of them at the idlest haunts in the kingdom. We hear of him for ever at some watering-place or other. A little while ago, he was at Weymouth. This proves that he can leave the Churchills.>

<Yes, sometimes he can.>

<And those times are whenever he thinks it worth his while; whenever there is any temptation of pleasure.>

<It is very unfair to judge of any body's conduct, without an intimate knowledge of their situation. Nobody, who has not been in the interior of a family, can say what the difficulties of any individual of that family may be. We ought to be acquainted with Enscombe, and with Mrs Churchill's temper, before we pretend to decide upon what her nephew can do. He may, at times, be able to do a great deal more than he can at others.>

<There is one thing, Emma, which a man can always do, if he chuses, and that is, his duty; not by manoeuvring and finessing, but by vigour and resolution. It is Frank Churchill's duty to pay this attention to his father. He knows it to be so, by his promises and messages; but if he wished to do it, it might be done. A man who felt rightly would say at once, simply and resolutely, to Mrs Churchill—<Every sacrifice of mere pleasure you will always find me ready to make to your convenience; but I must go and see my father immediately. I know he would be hurt by my failing in such a mark of respect to him on the present occasion. I shall, therefore, set off to-morrow.>—If he would say so to her at once, in the tone of decision becoming a man, there would be no opposition made to his going.>

<No,> said Emma, laughing; <but perhaps there might be some made to his coming back again. Such language for a young man entirely dependent, to use!—Nobody but you, Mr Knightley, would imagine it possible. But you have not an idea of what is requisite in situations directly opposite to your own. Mr Frank Churchill to be making such a speech as that to the uncle and aunt, who have brought him up, and are to provide for him!—Standing up in the middle of the room, I suppose, and speaking as loud as he could!—How can you imagine such conduct practicable?>

<Depend upon it, Emma, a sensible man would find no difficulty in it. He would feel himself in the right; and the declaration—made, of course, as a man of sense would make it, in a proper manner—would do him more good, raise him higher, fix his interest stronger with the people he depended on, than all that a line of shifts and expedients can ever do. Respect would be added to affection. They would feel that they could trust him; that the nephew who had done rightly by his father, would do rightly by them; for they know, as well as he does, as well as all the world must know, that he ought to pay this visit to his father; and while meanly exerting their power to delay it, are in their hearts not thinking the better of him for submitting to their whims. Respect for right conduct is felt by every body. If he would act in this sort of manner, on principle, consistently, regularly, their little minds would bend to his.>

<I rather doubt that. You are very fond of bending little minds; but where little minds belong to rich people in authority, I think they have a knack of swelling out, till they are quite as unmanageable as great ones. I can imagine, that if you, as you are, Mr Knightley, were to be transported and placed all at once in Mr Frank Churchill's situation, you would be able to say and do just what you have been recommending for him; and it might have a very good effect. The Churchills might not have a word to say in return; but then, you would have no habits of early obedience and long observance to break through. To him who has, it might not be so easy to burst forth at once into perfect independence, and set all their claims on his gratitude and regard at nought. He may have as strong a sense of what would be right, as you can have, without being so equal, under particular circumstances, to act up to it.>

<Then it would not be so strong a sense. If it failed to produce equal exertion, it could not be an equal conviction.>

<Oh, the difference of situation and habit! I wish you would try to understand what an amiable young man may be likely to feel in directly opposing those, whom as child and boy he has been looking up to all his life.>

<Our amiable young man is a very weak young man, if this be the first occasion of his carrying through a resolution to do right against the will of others. It ought to have been a habit with him by this time, of following his duty, instead of consulting expediency. I can allow for the fears of the child, but not of the man. As he became rational, he ought to have roused himself and shaken off all that was unworthy in their authority. He ought to have opposed the first attempt on their side to make him slight his father. Had he begun as he ought, there would have been no difficulty now.>

<We shall never agree about him,> cried Emma; <but that is nothing extraordinary. I have not the least idea of his being a weak young man: I feel sure that he is not. Mr Weston would not be blind to folly, though in his own son; but he is very likely to have a more yielding, complying, mild disposition than would suit your notions of man's perfection. I dare say he has; and though it may cut him off from some advantages, it will secure him many others.>

<Yes; all the advantages of sitting still when he ought to move, and of leading a life of mere idle pleasure, and fancying himself extremely expert in finding excuses for it. He can sit down and write a fine flourishing letter, full of professions and falsehoods, and persuade himself that he has hit upon the very best method in the world of preserving peace at home and preventing his father's having any right to complain. His letters disgust me.>

<Your feelings are singular. They seem to satisfy every body else.>

<I suspect they do not satisfy Mrs Weston. They hardly can satisfy a woman of her good sense and quick feelings: standing in a mother's place, but without a mother's affection to blind her. It is on her account that attention to Randalls is doubly due, and she must doubly feel the omission. Had she been a person of consequence herself, he would have come I dare say; and it would not have signified whether he did or no. Can you think your friend behindhand in these sort of considerations? Do you suppose she does not often say all this to herself? No, Emma, your amiable young man can be amiable only in French, not in English. He may be very <amiable,> have very good manners, and be very agreeable; but he can have no English delicacy towards the feelings of other people: nothing really amiable about him.>

<You seem determined to think ill of him.>

<Me!—not at all,> replied Mr Knightley, rather displeased; <I do not want to think ill of him. I should be as ready to acknowledge his merits as any other man; but I hear of none, except what are merely personal; that he is well-grown and good-looking, with smooth, plausible manners.>

<Well, if he have nothing else to recommend him, he will be a treasure at Highbury. We do not often look upon fine young men, well-bred and agreeable. We must not be nice and ask for all the virtues into the bargain. Cannot you imagine, Mr Knightley, what a sensation his coming will produce? There will be but one subject throughout the parishes of Donwell and Highbury; but one interest—one object of curiosity; it will be all Mr Frank Churchill; we shall think and speak of nobody else.>

<You will excuse my being so much over-powered. If I find him conversable, I shall be glad of his acquaintance; but if he is only a chattering coxcomb, he will not occupy much of my time or thoughts.>

<My idea of him is, that he can adapt his conversation to the taste of every body, and has the power as well as the wish of being universally agreeable. To you, he will talk of farming; to me, of drawing or music; and so on to every body, having that general information on all subjects which will enable him to follow the lead, or take the lead, just as propriety may require, and to speak extremely well on each; that is my idea of him.>

<And mine,> said Mr Knightley warmly, <is, that if he turn out any thing like it, he will be the most insufferable fellow breathing! What! at three-and-twenty to be the king of his company—the great man—the practised politician, who is to read every body's character, and make every body's talents conduce to the display of his own superiority; to be dispensing his flatteries around, that he may make all appear like fools compared with himself! My dear Emma, your own good sense could not endure such a puppy when it came to the point.>

<I will say no more about him,> cried Emma, <you turn every thing to evil. We are both prejudiced; you against, I for him; and we have no chance of agreeing till he is really here.>

<Prejudiced! I am not prejudiced.>

<But I am very much, and without being at all ashamed of it. My love for Mr and Mrs Weston gives me a decided prejudice in his favour.>

<He is a person I never think of from one month's end to another,> said Mr Knightley, with a degree of vexation, which made Emma immediately talk of something else, though she could not comprehend why he should be angry.

To take a dislike to a young man, only because he appeared to be of a different disposition from himself, was unworthy the real liberality of mind which she was always used to acknowledge in him; for with all the high opinion of himself, which she had often laid to his charge, she had never before for a moment supposed it could make him unjust to the merit of another.