%!TeX root=../emmatop.tex
\chapter[Chapter \thechapter]{}
\lettrine[lraise=0.3]{E}{mma} continued to entertain no doubt of her being in love. Her ideas only varied as to the how much. At first, she thought it was a good deal; and afterwards, but little. She had great pleasure in hearing Frank Churchill talked of; and, for his sake, greater pleasure than ever in seeing Mr and Mrs Weston; she was very often thinking of him, and quite impatient for a letter, that she might know how he was, how were his spirits, how was his aunt, and what was the chance of his coming to Randalls again this spring. But, on the other hand, she could not admit herself to be unhappy, nor, after the first morning, to be less disposed for employment than usual; she was still busy and cheerful; and, pleasing as he was, she could yet imagine him to have faults; and farther, though thinking of him so much, and, as she sat drawing or working, forming a thousand amusing schemes for the progress and close of their attachment, fancying interesting dialogues, and inventing elegant letters; the conclusion of every imaginary declaration on his side was that she refused him. Their affection was always to subside into friendship. Every thing tender and charming was to mark their parting; but still they were to part. When she became sensible of this, it struck her that she could not be very much in love; for in spite of her previous and fixed determination never to quit her father, never to marry, a strong attachment certainly must produce more of a struggle than she could foresee in her own feelings.

<I do not find myself making any use of the word sacrifice,> said she.—<In not one of all my clever replies, my delicate negatives, is there any allusion to making a sacrifice. I do suspect that he is not really necessary to my happiness. So much the better. I certainly will not persuade myself to feel more than I do. I am quite enough in love. I should be sorry to be more.>

Upon the whole, she was equally contented with her view of his feelings.

<He is undoubtedly very much in love—every thing denotes it—very much in love indeed!—and when he comes again, if his affection continue, I must be on my guard not to encourage it.—It would be most inexcusable to do otherwise, as my own mind is quite made up. Not that I imagine he can think I have been encouraging him hitherto. No, if he had believed me at all to share his feelings, he would not have been so wretched. Could he have thought himself encouraged, his looks and language at parting would have been different.—Still, however, I must be on my guard. This is in the supposition of his attachment continuing what it now is; but I do not know that I expect it will; I do not look upon him to be quite the sort of man—I do not altogether build upon his steadiness or constancy.—His feelings are warm, but I can imagine them rather changeable.—Every consideration of the subject, in short, makes me thankful that my happiness is not more deeply involved.—I shall do very well again after a little while—and then, it will be a good thing over; for they say every body is in love once in their lives, and I shall have been let off easily.>

When his letter to Mrs Weston arrived, Emma had the perusal of it; and she read it with a degree of pleasure and admiration which made her at first shake her head over her own sensations, and think she had undervalued their strength. It was a long, well-written letter, giving the particulars of his journey and of his feelings, expressing all the affection, gratitude, and respect which was natural and honourable, and describing every thing exterior and local that could be supposed attractive, with spirit and precision. No suspicious flourishes now of apology or concern; it was the language of real feeling towards Mrs Weston; and the transition from Highbury to Enscombe, the contrast between the places in some of the first blessings of social life was just enough touched on to shew how keenly it was felt, and how much more might have been said but for the restraints of propriety.—The charm of her own name was not wanting. Miss Woodhouse appeared more than once, and never without a something of pleasing connexion, either a compliment to her taste, or a remembrance of what she had said; and in the very last time of its meeting her eye, unadorned as it was by any such broad wreath of gallantry, she yet could discern the effect of her influence and acknowledge the greatest compliment perhaps of all conveyed. Compressed into the very lowest vacant corner were these words—<I had not a spare moment on Tuesday, as you know, for Miss Woodhouse's beautiful little friend. Pray make my excuses and adieus to her.> This, Emma could not doubt, was all for herself. Harriet was remembered only from being her friend. His information and prospects as to Enscombe were neither worse nor better than had been anticipated; Mrs Churchill was recovering, and he dared not yet, even in his own imagination, fix a time for coming to Randalls again.

Gratifying, however, and stimulative as was the letter in the material part, its sentiments, she yet found, when it was folded up and returned to Mrs Weston, that it had not added any lasting warmth, that she could still do without the writer, and that he must learn to do without her. Her intentions were unchanged. Her resolution of refusal only grew more interesting by the addition of a scheme for his subsequent consolation and happiness. His recollection of Harriet, and the words which clothed it, the <beautiful little friend,> suggested to her the idea of Harriet's succeeding her in his affections. Was it impossible?—No.—Harriet undoubtedly was greatly his inferior in understanding; but he had been very much struck with the loveliness of her face and the warm simplicity of her manner; and all the probabilities of circumstance and connexion were in her favour.—For Harriet, it would be advantageous and delightful indeed.

<I must not dwell upon it,> said she.—<I must not think of it. I know the danger of indulging such speculations. But stranger things have happened; and when we cease to care for each other as we do now, it will be the means of confirming us in that sort of true disinterested friendship which I can already look forward to with pleasure.>

It was well to have a comfort in store on Harriet's behalf, though it might be wise to let the fancy touch it seldom; for evil in that quarter was at hand. As Frank Churchill's arrival had succeeded Mr Elton's engagement in the conversation of Highbury, as the latest interest had entirely borne down the first, so now upon Frank Churchill's disappearance, Mr Elton's concerns were assuming the most irresistible form.—His wedding-day was named. He would soon be among them again; Mr Elton and his bride. There was hardly time to talk over the first letter from Enscombe before <Mr Elton and his bride> was in every body's mouth, and Frank Churchill was forgotten. Emma grew sick at the sound. She had had three weeks of happy exemption from Mr Elton; and Harriet's mind, she had been willing to hope, had been lately gaining strength. With Mr Weston's ball in view at least, there had been a great deal of insensibility to other things; but it was now too evident that she had not attained such a state of composure as could stand against the actual approach—new carriage, bell-ringing, and all.

Poor Harriet was in a flutter of spirits which required all the reasonings and soothings and attentions of every kind that Emma could give. Emma felt that she could not do too much for her, that Harriet had a right to all her ingenuity and all her patience; but it was heavy work to be for ever convincing without producing any effect, for ever agreed to, without being able to make their opinions the same. Harriet listened submissively, and said <it was very true—it was just as Miss Woodhouse described—it was not worth while to think about them—and she would not think about them any longer> but no change of subject could avail, and the next half-hour saw her as anxious and restless about the Eltons as before. At last Emma attacked her on another ground.

<Your allowing yourself to be so occupied and so unhappy about Mr Elton's marrying, Harriet, is the strongest reproach you can make me. You could not give me a greater reproof for the mistake I fell into. It was all my doing, I know. I have not forgotten it, I assure you.—Deceived myself, I did very miserably deceive you—and it will be a painful reflection to me for ever. Do not imagine me in danger of forgetting it.>

Harriet felt this too much to utter more than a few words of eager exclamation. Emma continued,

<I have not said, exert yourself Harriet for my sake; think less, talk less of Mr Elton for my sake; because for your own sake rather, I would wish it to be done, for the sake of what is more important than my comfort, a habit of self-command in you, a consideration of what is your duty, an attention to propriety, an endeavour to avoid the suspicions of others, to save your health and credit, and restore your tranquillity. These are the motives which I have been pressing on you. They are very important—and sorry I am that you cannot feel them sufficiently to act upon them. My being saved from pain is a very secondary consideration. I want you to save yourself from greater pain. Perhaps I may sometimes have felt that Harriet would not forget what was due—or rather what would be kind by me.>

This appeal to her affections did more than all the rest. The idea of wanting gratitude and consideration for Miss Woodhouse, whom she really loved extremely, made her wretched for a while, and when the violence of grief was comforted away, still remained powerful enough to prompt to what was right and support her in it very tolerably.

<You, who have been the best friend I ever had in my life—Want gratitude to you!—Nobody is equal to you!—I care for nobody as I do for you!—Oh! Miss Woodhouse, how ungrateful I have been!>

Such expressions, assisted as they were by every thing that look and manner could do, made Emma feel that she had never loved Harriet so well, nor valued her affection so highly before.

<There is no charm equal to tenderness of heart,> said she afterwards to herself. <There is nothing to be compared to it. Warmth and tenderness of heart, with an affectionate, open manner, will beat all the clearness of head in the world, for attraction, I am sure it will. It is tenderness of heart which makes my dear father so generally beloved—which gives Isabella all her popularity.—I have it not—but I know how to prize and respect it.—Harriet is my superior in all the charm and all the felicity it gives. Dear Harriet!—I would not change you for the clearest-headed, longest-sighted, best-judging female breathing. Oh! the coldness of a Jane Fairfax!—Harriet is worth a hundred such—And for a wife—a sensible man's wife—it is invaluable. I mention no names; but happy the man who changes Emma for Harriet!>