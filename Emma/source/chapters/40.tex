%!TeX root=../emmatop.tex
\chapter[Chapter \thechapter]{}
\lettrine[lraise=0.3]{A}{}  very few days had passed after this adventure, when Harriet came one morning to Emma with a small parcel in her hand, and after sitting down and hesitating, thus began:

\zz
<Miss Woodhouse—if you are at leisure—I have something that I should like to tell you—a sort of confession to make—and then, you know, it will be over.>

Emma was a good deal surprized; but begged her to speak. There was a seriousness in Harriet's manner which prepared her, quite as much as her words, for something more than ordinary.

<It is my duty, and I am sure it is my wish,> she continued, <to have no reserves with you on this subject. As I am happily quite an altered creature in one respect, it is very fit that you should have the satisfaction of knowing it. I do not want to say more than is necessary—I am too much ashamed of having given way as I have done, and I dare say you understand me.>

<Yes,> said Emma, <I hope I do.>

<How I could so long a time be fancying myself!...> cried Harriet, warmly. <It seems like madness! I can see nothing at all extraordinary in him now.—I do not care whether I meet him or not—except that of the two I had rather not see him—and indeed I would go any distance round to avoid him—but I do not envy his wife in the least; I neither admire her nor envy her, as I have done: she is very charming, I dare say, and all that, but I think her very ill-tempered and disagreeable—I shall never forget her look the other night!—However, I assure you, Miss Woodhouse, I wish her no evil.—No, let them be ever so happy together, it will not give me another moment's pang: and to convince you that I have been speaking truth, I am now going to destroy—what I ought to have destroyed long ago—what I ought never to have kept—I know that very well (blushing as she spoke).—However, now I will destroy it all—and it is my particular wish to do it in your presence, that you may see how rational I am grown. Cannot you guess what this parcel holds?> said she, with a conscious look.

<Not the least in the world.—Did he ever give you any thing?>

<No—I cannot call them gifts; but they are things that I have valued very much.>

She held the parcel towards her, and Emma read the words Most precious treasures on the top. Her curiosity was greatly excited. Harriet unfolded the parcel, and she looked on with impatience. Within abundance of silver paper was a pretty little Tunbridge-ware box, which Harriet opened: it was well lined with the softest cotton; but, excepting the cotton, Emma saw only a small piece of court-plaister.

<Now,> said Harriet, <you must recollect.>

<No, indeed I do not.>

<Dear me! I should not have thought it possible you could forget what passed in this very room about court-plaister, one of the very last times we ever met in it!—It was but a very few days before I had my sore throat—just before Mr and Mrs John Knightley came—I think the very evening.—Do not you remember his cutting his finger with your new penknife, and your recommending court-plaister?—But, as you had none about you, and knew I had, you desired me to supply him; and so I took mine out and cut him a piece; but it was a great deal too large, and he cut it smaller, and kept playing some time with what was left, before he gave it back to me. And so then, in my nonsense, I could not help making a treasure of it—so I put it by never to be used, and looked at it now and then as a great treat.>

<My dearest Harriet!> cried Emma, putting her hand before her face, and jumping up, <you make me more ashamed of myself than I can bear. Remember it? Aye, I remember it all now; all, except your saving this relic—I knew nothing of that till this moment—but the cutting the finger, and my recommending court-plaister, and saying I had none about me!—Oh! my sins, my sins!—And I had plenty all the while in my pocket!—One of my senseless tricks!—I deserve to be under a continual blush all the rest of my life.—Well—(sitting down again)—go on—what else?>

<And had you really some at hand yourself? I am sure I never suspected it, you did it so naturally.>

<And so you actually put this piece of court-plaister by for his sake!> said Emma, recovering from her state of shame and feeling divided between wonder and amusement. And secretly she added to herself, <Lord bless me! when should I ever have thought of putting by in cotton a piece of court-plaister that Frank Churchill had been pulling about! I never was equal to this.>

<Here,> resumed Harriet, turning to her box again, <here is something still more valuable, I mean that has been more valuable, because this is what did really once belong to him, which the court-plaister never did.>

Emma was quite eager to see this superior treasure. It was the end of an old pencil,—the part without any lead.

<This was really his,> said Harriet.—<Do not you remember one morning?—no, I dare say you do not. But one morning—I forget exactly the day—but perhaps it was the Tuesday or Wednesday before that evening, he wanted to make a memorandum in his pocket-book; it was about spruce-beer. Mr Knightley had been telling him something about brewing spruce-beer, and he wanted to put it down; but when he took out his pencil, there was so little lead that he soon cut it all away, and it would not do, so you lent him another, and this was left upon the table as good for nothing. But I kept my eye on it; and, as soon as I dared, caught it up, and never parted with it again from that moment.>

<I do remember it,> cried Emma; <I perfectly remember it.—Talking about spruce-beer.—Oh! yes—Mr Knightley and I both saying we liked it, and Mr Elton's seeming resolved to learn to like it too. I perfectly remember it.—Stop; Mr Knightley was standing just here, was not he? I have an idea he was standing just here.>

<Ah! I do not know. I cannot recollect.—It is very odd, but I cannot recollect.—Mr Elton was sitting here, I remember, much about where I am now.>—

<Well, go on.>

<Oh! that's all. I have nothing more to shew you, or to say—except that I am now going to throw them both behind the fire, and I wish you to see me do it.>

<My poor dear Harriet! and have you actually found happiness in treasuring up these things?>

<Yes, simpleton as I was!—but I am quite ashamed of it now, and wish I could forget as easily as I can burn them. It was very wrong of me, you know, to keep any remembrances, after he was married. I knew it was—but had not resolution enough to part with them.>

<But, Harriet, is it necessary to burn the court-plaister?—I have not a word to say for the bit of old pencil, but the court-plaister might be useful.>

<I shall be happier to burn it,> replied Harriet. <It has a disagreeable look to me. I must get rid of every thing.—There it goes, and there is an end, thank Heaven! of Mr Elton.>

<And when,> thought Emma, <will there be a beginning of Mr Churchill?>

She had soon afterwards reason to believe that the beginning was already made, and could not but hope that the gipsy, though she had told no fortune, might be proved to have made Harriet's.—About a fortnight after the alarm, they came to a sufficient explanation, and quite undesignedly. Emma was not thinking of it at the moment, which made the information she received more valuable. She merely said, in the course of some trivial chat, <Well, Harriet, whenever you marry I would advise you to do so and so>—and thought no more of it, till after a minute's silence she heard Harriet say in a very serious tone, <I shall never marry.>

Emma then looked up, and immediately saw how it was; and after a moment's debate, as to whether it should pass unnoticed or not, replied,

<Never marry!—This is a new resolution.>

<It is one that I shall never change, however.>

After another short hesitation, <I hope it does not proceed from—I hope it is not in compliment to Mr Elton?>

<Mr Elton indeed!> cried Harriet indignantly.—<Oh! no>—and Emma could just catch the words, <so superior to Mr Elton!>

She then took a longer time for consideration. Should she proceed no farther?—should she let it pass, and seem to suspect nothing?—Perhaps Harriet might think her cold or angry if she did; or perhaps if she were totally silent, it might only drive Harriet into asking her to hear too much; and against any thing like such an unreserve as had been, such an open and frequent discussion of hopes and chances, she was perfectly resolved.—She believed it would be wiser for her to say and know at once, all that she meant to say and know. Plain dealing was always best. She had previously determined how far she would proceed, on any application of the sort; and it would be safer for both, to have the judicious law of her own brain laid down with speed.—She was decided, and thus spoke—

<Harriet, I will not affect to be in doubt of your meaning. Your resolution, or rather your expectation of never marrying, results from an idea that the person whom you might prefer, would be too greatly your superior in situation to think of you. Is not it so?>

<Oh! Miss Woodhouse, believe me I have not the presumption to suppose— Indeed I am not so mad.—But it is a pleasure to me to admire him at a distance—and to think of his infinite superiority to all the rest of the world, with the gratitude, wonder, and veneration, which are so proper, in me especially.>

<I am not at all surprized at you, Harriet. The service he rendered you was enough to warm your heart.>

<Service! oh! it was such an inexpressible obligation!—The very recollection of it, and all that I felt at the time—when I saw him coming—his noble look—and my wretchedness before. Such a change! In one moment such a change! From perfect misery to perfect happiness!>

<It is very natural. It is natural, and it is honourable.—Yes, honourable, I think, to chuse so well and so gratefully.—But that it will be a fortunate preference is more than I can promise. I do not advise you to give way to it, Harriet. I do not by any means engage for its being returned. Consider what you are about. Perhaps it will be wisest in you to check your feelings while you can: at any rate do not let them carry you far, unless you are persuaded of his liking you. Be observant of him. Let his behaviour be the guide of your sensations. I give you this caution now, because I shall never speak to you again on the subject. I am determined against all interference. Henceforward I know nothing of the matter. Let no name ever pass our lips. We were very wrong before; we will be cautious now.—He is your superior, no doubt, and there do seem objections and obstacles of a very serious nature; but yet, Harriet, more wonderful things have taken place, there have been matches of greater disparity. But take care of yourself. I would not have you too sanguine; though, however it may end, be assured your raising your thoughts to him, is a mark of good taste which I shall always know how to value.>

Harriet kissed her hand in silent and submissive gratitude. Emma was very decided in thinking such an attachment no bad thing for her friend. Its tendency would be to raise and refine her mind—and it must be saving her from the danger of degradation.