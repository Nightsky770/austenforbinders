%!TeX root=../emmatop.tex
\chapter[Chapter \thechapter]{}
\lettrine[lines=4,lraise=0.3]{T}{ill} now that she was threatened with its loss, Emma had never known how much of her happiness depended on being first with Mr Knightley, first in interest and affection.—Satisfied that it was so, and feeling it her due, she had enjoyed it without reflection; and only in the dread of being supplanted, found how inexpressibly important it had been.—Long, very long, she felt she had been first; for, having no female connexions of his own, there had been only Isabella whose claims could be compared with hers, and she had always known exactly how far he loved and esteemed Isabella. She had herself been first with him for many years past. She had not deserved it; she had often been negligent or perverse, slighting his advice, or even wilfully opposing him, insensible of half his merits, and quarrelling with him because he would not acknowledge her false and insolent estimate of her own—but still, from family attachment and habit, and thorough excellence of mind, he had loved her, and watched over her from a girl, with an endeavour to improve her, and an anxiety for her doing right, which no other creature had at all shared. In spite of all her faults, she knew she was dear to him; might she not say, very dear?—When the suggestions of hope, however, which must follow here, presented themselves, she could not presume to indulge them. Harriet Smith might think herself not unworthy of being peculiarly, exclusively, passionately loved by Mr Knightley. She could not. She could not flatter herself with any idea of blindness in his attachment to her. She had received a very recent proof of its impartiality.—How shocked had he been by her behaviour to Miss Bates! How directly, how strongly had he expressed himself to her on the subject!—Not too strongly for the offence—but far, far too strongly to issue from any feeling softer than upright justice and clear-sighted goodwill.—She had no hope, nothing to deserve the name of hope, that he could have that sort of affection for herself which was now in question; but there was a hope (at times a slight one, at times much stronger,) that Harriet might have deceived herself, and be overrating his regard for her.—Wish it she must, for his sake—be the consequence nothing to herself, but his remaining single all his life. Could she be secure of that, indeed, of his never marrying at all, she believed she should be perfectly satisfied.—Let him but continue the same Mr Knightley to her and her father, the same Mr Knightley to all the world; let Donwell and Hartfield lose none of their precious intercourse of friendship and confidence, and her peace would be fully secured.—Marriage, in fact, would not do for her. It would be incompatible with what she owed to her father, and with what she felt for him. Nothing should separate her from her father. She would not marry, even if she were asked by Mr Knightley.

It must be her ardent wish that Harriet might be disappointed; and she hoped, that when able to see them together again, she might at least be able to ascertain what the chances for it were.—She should see them henceforward with the closest observance; and wretchedly as she had hitherto misunderstood even those she was watching, she did not know how to admit that she could be blinded here.—He was expected back every day. The power of observation would be soon given—frightfully soon it appeared when her thoughts were in one course. In the meanwhile, she resolved against seeing Harriet.—It would do neither of them good, it would do the subject no good, to be talking of it farther.—She was resolved not to be convinced, as long as she could doubt, and yet had no authority for opposing Harriet's confidence. To talk would be only to irritate.—She wrote to her, therefore, kindly, but decisively, to beg that she would not, at present, come to Hartfield; acknowledging it to be her conviction, that all farther confidential discussion of one topic had better be avoided; and hoping, that if a few days were allowed to pass before they met again, except in the company of others—she objected only to a tête-à-tête—they might be able to act as if they had forgotten the conversation of yesterday.—Harriet submitted, and approved, and was grateful.

This point was just arranged, when a visitor arrived to tear Emma's thoughts a little from the one subject which had engrossed them, sleeping or waking, the last twenty-four hours—Mrs Weston, who had been calling on her daughter-in-law elect, and took Hartfield in her way home, almost as much in duty to Emma as in pleasure to herself, to relate all the particulars of so interesting an interview.

Mr Weston had accompanied her to Mrs Bates's, and gone through his share of this essential attention most handsomely; but she having then induced Miss Fairfax to join her in an airing, was now returned with much more to say, and much more to say with satisfaction, than a quarter of an hour spent in Mrs Bates's parlour, with all the encumbrance of awkward feelings, could have afforded.

A little curiosity Emma had; and she made the most of it while her friend related. Mrs Weston had set off to pay the visit in a good deal of agitation herself; and in the first place had wished not to go at all at present, to be allowed merely to write to Miss Fairfax instead, and to defer this ceremonious call till a little time had passed, and Mr Churchill could be reconciled to the engagement's becoming known; as, considering every thing, she thought such a visit could not be paid without leading to reports:—but Mr Weston had thought differently; he was extremely anxious to shew his approbation to Miss Fairfax and her family, and did not conceive that any suspicion could be excited by it; or if it were, that it would be of any consequence; for »such things,« he observed, »always got about.« Emma smiled, and felt that Mr Weston had very good reason for saying so. They had gone, in short—and very great had been the evident distress and confusion of the lady. She had hardly been able to speak a word, and every look and action had shewn how deeply she was suffering from consciousness. The quiet, heart-felt satisfaction of the old lady, and the rapturous delight of her daughter—who proved even too joyous to talk as usual, had been a gratifying, yet almost an affecting, scene. They were both so truly respectable in their happiness, so disinterested in every sensation; thought so much of Jane; so much of every body, and so little of themselves, that every kindly feeling was at work for them. Miss Fairfax's recent illness had offered a fair plea for Mrs Weston to invite her to an airing; she had drawn back and declined at first, but, on being pressed had yielded; and, in the course of their drive, Mrs Weston had, by gentle encouragement, overcome so much of her embarrassment, as to bring her to converse on the important subject. Apologies for her seemingly ungracious silence in their first reception, and the warmest expressions of the gratitude she was always feeling towards herself and Mr Weston, must necessarily open the cause; but when these effusions were put by, they had talked a good deal of the present and of the future state of the engagement. Mrs Weston was convinced that such conversation must be the greatest relief to her companion, pent up within her own mind as every thing had so long been, and was very much pleased with all that she had said on the subject.

»On the misery of what she had suffered, during the concealment of so many months,« continued Mrs Weston, »she was energetic. This was one of her expressions. »I will not say, that since I entered into the engagement I have not had some happy moments; but I can say, that I have never known the blessing of one tranquil hour:«—and the quivering lip, Emma, which uttered it, was an attestation that I felt at my heart.«

»Poor girl!« said Emma. »She thinks herself wrong, then, for having consented to a private engagement?«

»Wrong! No one, I believe, can blame her more than she is disposed to blame herself. »The consequence,« said she, »has been a state of perpetual suffering to me; and so it ought. But after all the punishment that misconduct can bring, it is still not less misconduct. Pain is no expiation. I never can be blameless. I have been acting contrary to all my sense of right; and the fortunate turn that every thing has taken, and the kindness I am now receiving, is what my conscience tells me ought not to be.« »Do not imagine, madam,« she continued, »that I was taught wrong. Do not let any reflection fall on the principles or the care of the friends who brought me up. The error has been all my own; and I do assure you that, with all the excuse that present circumstances may appear to give, I shall yet dread making the story known to Colonel Campbell.««

»Poor girl!« said Emma again. »She loves him then excessively, I suppose. It must have been from attachment only, that she could be led to form the engagement. Her affection must have overpowered her judgment.«

»Yes, I have no doubt of her being extremely attached to him.«

»I am afraid,« returned Emma, sighing, »that I must often have contributed to make her unhappy.«

»On your side, my love, it was very innocently done. But she probably had something of that in her thoughts, when alluding to the misunderstandings which he had given us hints of before. One natural consequence of the evil she had involved herself in,« she said, »was that of making her unreasonable. The consciousness of having done amiss, had exposed her to a thousand inquietudes, and made her captious and irritable to a degree that must have been—that had been—hard for him to bear. »I did not make the allowances,« said she, »which I ought to have done, for his temper and spirits—his delightful spirits, and that gaiety, that playfulness of disposition, which, under any other circumstances, would, I am sure, have been as constantly bewitching to me, as they were at first.« She then began to speak of you, and of the great kindness you had shewn her during her illness; and with a blush which shewed me how it was all connected, desired me, whenever I had an opportunity, to thank you—I could not thank you too much—for every wish and every endeavour to do her good. She was sensible that you had never received any proper acknowledgment from herself.«

»If I did not know her to be happy now,« said Emma, seriously, »which, in spite of every little drawback from her scrupulous conscience, she must be, I could not bear these thanks;—for, oh! Mrs Weston, if there were an account drawn up of the evil and the good I have done Miss Fairfax!—Well (checking herself, and trying to be more lively), this is all to be forgotten. You are very kind to bring me these interesting particulars. They shew her to the greatest advantage. I am sure she is very good—I hope she will be very happy. It is fit that the fortune should be on his side, for I think the merit will be all on hers.«

Such a conclusion could not pass unanswered by Mrs Weston. She thought well of Frank in almost every respect; and, what was more, she loved him very much, and her defence was, therefore, earnest. She talked with a great deal of reason, and at least equal affection—but she had too much to urge for Emma's attention; it was soon gone to Brunswick Square or to Donwell; she forgot to attempt to listen; and when Mrs Weston ended with, »We have not yet had the letter we are so anxious for, you know, but I hope it will soon come,« she was obliged to pause before she answered, and at last obliged to answer at random, before she could at all recollect what letter it was which they were so anxious for.

»Are you well, my Emma?« was Mrs Weston's parting question.

»Oh! perfectly. I am always well, you know. Be sure to give me intelligence of the letter as soon as possible.«

Mrs Weston's communications furnished Emma with more food for unpleasant reflection, by increasing her esteem and compassion, and her sense of past injustice towards Miss Fairfax. She bitterly regretted not having sought a closer acquaintance with her, and blushed for the envious feelings which had certainly been, in some measure, the cause. Had she followed Mr Knightley's known wishes, in paying that attention to Miss Fairfax, which was every way her due; had she tried to know her better; had she done her part towards intimacy; had she endeavoured to find a friend there instead of in Harriet Smith; she must, in all probability, have been spared from every pain which pressed on her now.—Birth, abilities, and education, had been equally marking one as an associate for her, to be received with gratitude; and the other—what was she?—Supposing even that they had never become intimate friends; that she had never been admitted into Miss Fairfax's confidence on this important matter—which was most probable—still, in knowing her as she ought, and as she might, she must have been preserved from the abominable suspicions of an improper attachment to Mr Dixon, which she had not only so foolishly fashioned and harboured herself, but had so unpardonably imparted; an idea which she greatly feared had been made a subject of material distress to the delicacy of Jane's feelings, by the levity or carelessness of Frank Churchill's. Of all the sources of evil surrounding the former, since her coming to Highbury, she was persuaded that she must herself have been the worst. She must have been a perpetual enemy. They never could have been all three together, without her having stabbed Jane Fairfax's peace in a thousand instances; and on Box Hill, perhaps, it had been the agony of a mind that would bear no more.

The evening of this day was very long, and melancholy, at Hartfield. The weather added what it could of gloom. A cold stormy rain set in, and nothing of July appeared but in the trees and shrubs, which the wind was despoiling, and the length of the day, which only made such cruel sights the longer visible.

The weather affected Mr Woodhouse, and he could only be kept tolerably comfortable by almost ceaseless attention on his daughter's side, and by exertions which had never cost her half so much before. It reminded her of their first forlorn tête-à-tête, on the evening of Mrs Weston's wedding-day; but Mr Knightley had walked in then, soon after tea, and dissipated every melancholy fancy. Alas! such delightful proofs of Hartfield's attraction, as those sort of visits conveyed, might shortly be over. The picture which she had then drawn of the privations of the approaching winter, had proved erroneous; no friends had deserted them, no pleasures had been lost.—But her present forebodings she feared would experience no similar contradiction. The prospect before her now, was threatening to a degree that could not be entirely dispelled—that might not be even partially brightened. If all took place that might take place among the circle of her friends, Hartfield must be comparatively deserted; and she left to cheer her father with the spirits only of ruined happiness.

The child to be born at Randalls must be a tie there even dearer than herself; and Mrs Weston's heart and time would be occupied by it. They should lose her; and, probably, in great measure, her husband also.—Frank Churchill would return among them no more; and Miss Fairfax, it was reasonable to suppose, would soon cease to belong to Highbury. They would be married, and settled either at or near Enscombe. All that were good would be withdrawn; and if to these losses, the loss of Donwell were to be added, what would remain of cheerful or of rational society within their reach? Mr Knightley to be no longer coming there for his evening comfort!—No longer walking in at all hours, as if ever willing to change his own home for their's!—How was it to be endured? And if he were to be lost to them for Harriet's sake; if he were to be thought of hereafter, as finding in Harriet's society all that he wanted; if Harriet were to be the chosen, the first, the dearest, the friend, the wife to whom he looked for all the best blessings of existence; what could be increasing Emma's wretchedness but the reflection never far distant from her mind, that it had been all her own work?

When it came to such a pitch as this, she was not able to refrain from a start, or a heavy sigh, or even from walking about the room for a few seconds—and the only source whence any thing like consolation or composure could be drawn, was in the resolution of her own better conduct, and the hope that, however inferior in spirit and gaiety might be the following and every future winter of her life to the past, it would yet find her more rational, more acquainted with herself, and leave her less to regret when it were gone.