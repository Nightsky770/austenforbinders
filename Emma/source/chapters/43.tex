%!TeX root=../emmatop.tex
\chapter[Chapter \thechapter]{}
\lettrine[lraise=0.3]{T}{hey} had a very fine day for Box Hill; and all the other outward circumstances of arrangement, accommodation, and punctuality, were in favour of a pleasant party. Mr Weston directed the whole, officiating safely between Hartfield and the Vicarage, and every body was in good time. Emma and Harriet went together; Miss Bates and her niece, with the Eltons; the gentlemen on horseback. Mrs Weston remained with Mr Woodhouse. Nothing was wanting but to be happy when they got there. Seven miles were travelled in expectation of enjoyment, and every body had a burst of admiration on first arriving; but in the general amount of the day there was deficiency. There was a languor, a want of spirits, a want of union, which could not be got over. They separated too much into parties. The Eltons walked together; Mr Knightley took charge of Miss Bates and Jane; and Emma and Harriet belonged to Frank Churchill. And Mr Weston tried, in vain, to make them harmonise better. It seemed at first an accidental division, but it never materially varied. Mr and Mrs Elton, indeed, shewed no unwillingness to mix, and be as agreeable as they could; but during the two whole hours that were spent on the hill, there seemed a principle of separation, between the other parties, too strong for any fine prospects, or any cold collation, or any cheerful Mr Weston, to remove.

At first it was downright dulness to Emma. She had never seen Frank Churchill so silent and stupid. He said nothing worth hearing—looked without seeing—admired without intelligence—listened without knowing what she said. While he was so dull, it was no wonder that Harriet should be dull likewise; and they were both insufferable.

When they all sat down it was better; to her taste a great deal better, for Frank Churchill grew talkative and gay, making her his first object. Every distinguishing attention that could be paid, was paid to her. To amuse her, and be agreeable in her eyes, seemed all that he cared for—and Emma, glad to be enlivened, not sorry to be flattered, was gay and easy too, and gave him all the friendly encouragement, the admission to be gallant, which she had ever given in the first and most animating period of their acquaintance; but which now, in her own estimation, meant nothing, though in the judgment of most people looking on it must have had such an appearance as no English word but flirtation could very well describe. <Mr Frank Churchill and Miss Woodhouse flirted together excessively.> They were laying themselves open to that very phrase—and to having it sent off in a letter to Maple Grove by one lady, to Ireland by another. Not that Emma was gay and thoughtless from any real felicity; it was rather because she felt less happy than she had expected. She laughed because she was disappointed; and though she liked him for his attentions, and thought them all, whether in friendship, admiration, or playfulness, extremely judicious, they were not winning back her heart. She still intended him for her friend.

<How much I am obliged to you,> said he, <for telling me to come to-day!—If it had not been for you, I should certainly have lost all the happiness of this party. I had quite determined to go away again.>

<Yes, you were very cross; and I do not know what about, except that you were too late for the best strawberries. I was a kinder friend than you deserved. But you were humble. You begged hard to be commanded to come.>

<Don't say I was cross. I was fatigued. The heat overcame me.>

<It is hotter to-day.>

<Not to my feelings. I am perfectly comfortable to-day.>

<You are comfortable because you are under command.>

<Your command?—Yes.>

<Perhaps I intended you to say so, but I meant self-command. You had, somehow or other, broken bounds yesterday, and run away from your own management; but to-day you are got back again—and as I cannot be always with you, it is best to believe your temper under your own command rather than mine.>

<It comes to the same thing. I can have no self-command without a motive. You order me, whether you speak or not. And you can be always with me. You are always with me.>

<Dating from three o'clock yesterday. My perpetual influence could not begin earlier, or you would not have been so much out of humour before.>

<Three o'clock yesterday! That is your date. I thought I had seen you first in February.>

<Your gallantry is really unanswerable. But (lowering her voice)—nobody speaks except ourselves, and it is rather too much to be talking nonsense for the entertainment of seven silent people.>

<I say nothing of which I am ashamed,> replied he, with lively impudence. <I saw you first in February. Let every body on the Hill hear me if they can. Let my accents swell to Mickleham on one side, and Dorking on the other. I saw you first in February.> And then whispering—<Our companions are excessively stupid. What shall we do to rouse them? Any nonsense will serve. They shall talk. Ladies and gentlemen, I am ordered by Miss Woodhouse (who, wherever she is, presides) to say, that she desires to know what you are all thinking of?>

Some laughed, and answered good-humouredly. Miss Bates said a great deal; Mrs Elton swelled at the idea of Miss Woodhouse's presiding; Mr Knightley's answer was the most distinct.

<Is Miss Woodhouse sure that she would like to hear what we are all thinking of?>

<Oh! no, no>—cried Emma, laughing as carelessly as she could—<Upon no account in the world. It is the very last thing I would stand the brunt of just now. Let me hear any thing rather than what you are all thinking of. I will not say quite all. There are one or two, perhaps, (glancing at Mr Weston and Harriet,) whose thoughts I might not be afraid of knowing.>

<It is a sort of thing,> cried Mrs Elton emphatically, <which I should not have thought myself privileged to inquire into. Though, perhaps, as the Chaperon of the party—I never was in any circle—exploring parties—young ladies—married women\longdash>

Her mutterings were chiefly to her husband; and he murmured, in reply,

<Very true, my love, very true. Exactly so, indeed—quite unheard of—but some ladies say any thing. Better pass it off as a joke. Every body knows what is due to you.>

<It will not do,> whispered Frank to Emma; <they are most of them affronted. I will attack them with more address. Ladies and gentlemen—I am ordered by Miss Woodhouse to say, that she waives her right of knowing exactly what you may all be thinking of, and only requires something very entertaining from each of you, in a general way. Here are seven of you, besides myself, (who, she is pleased to say, am very entertaining already,) and she only demands from each of you either one thing very clever, be it prose or verse, original or repeated—or two things moderately clever—or three things very dull indeed, and she engages to laugh heartily at them all.>

<Oh! very well,> exclaimed Miss Bates, <then I need not be uneasy. <Three things very dull indeed.> That will just do for me, you know. I shall be sure to say three dull things as soon as ever I open my mouth, shan't I\@? (looking round with the most good-humoured dependence on every body's assent)—Do not you all think I shall?>

Emma could not resist.

<Ah! ma'am, but there may be a difficulty. Pardon me—but you will be limited as to number—only three at once.>

Miss Bates, deceived by the mock ceremony of her manner, did not immediately catch her meaning; but, when it burst on her, it could not anger, though a slight blush shewed that it could pain her.

<Ah!—well—to be sure. Yes, I see what she means, (turning to Mr Knightley,) and I will try to hold my tongue. I must make myself very disagreeable, or she would not have said such a thing to an old friend.>

<I like your plan,> cried Mr Weston. <Agreed, agreed. I will do my best. I am making a conundrum. How will a conundrum reckon?>

<Low, I am afraid, sir, very low,> answered his son;—<but we shall be indulgent—especially to any one who leads the way.>

<No, no,> said Emma, <it will not reckon low. A conundrum of Mr Weston's shall clear him and his next neighbour. Come, sir, pray let me hear it.>

<I doubt its being very clever myself,> said Mr Weston. <It is too much a matter of fact, but here it is.—What two letters of the alphabet are there, that express perfection?>

<What two letters!—express perfection! I am sure I do not know.>

<Ah! you will never guess. You, (to Emma), I am certain, will never guess.—I will tell you.—M. and A.—Em-ma.—Do you understand?>

Understanding and gratification came together. It might be a very indifferent piece of wit, but Emma found a great deal to laugh at and enjoy in it—and so did Frank and Harriet.—It did not seem to touch the rest of the party equally; some looked very stupid about it, and Mr Knightley gravely said,

<This explains the sort of clever thing that is wanted, and Mr Weston has done very well for himself; but he must have knocked up every body else. Perfection should not have come quite so soon.>

<Oh! for myself, I protest I must be excused,> said Mrs Elton; <I really cannot attempt—I am not at all fond of the sort of thing. I had an acrostic once sent to me upon my own name, which I was not at all pleased with. I knew who it came from. An abominable puppy!—You know who I mean (nodding to her husband). These kind of things are very well at Christmas, when one is sitting round the fire; but quite out of place, in my opinion, when one is exploring about the country in summer. Miss Woodhouse must excuse me. I am not one of those who have witty things at every body's service. I do not pretend to be a wit. I have a great deal of vivacity in my own way, but I really must be allowed to judge when to speak and when to hold my tongue. Pass us, if you please, Mr Churchill. Pass Mr E., Knightley, Jane, and myself. We have nothing clever to say—not one of us.>

<Yes, yes, pray pass me,> added her husband, with a sort of sneering consciousness; <I have nothing to say that can entertain Miss Woodhouse, or any other young lady. An old married man—quite good for nothing. Shall we walk, Augusta?>

<With all my heart. I am really tired of exploring so long on one spot. Come, Jane, take my other arm.>

Jane declined it, however, and the husband and wife walked off. <Happy couple!> said Frank Churchill, as soon as they were out of hearing:—<How well they suit one another!—Very lucky—marrying as they did, upon an acquaintance formed only in a public place!—They only knew each other, I think, a few weeks in Bath! Peculiarly lucky!—for as to any real knowledge of a person's disposition that Bath, or any public place, can give—it is all nothing; there can be no knowledge. It is only by seeing women in their own homes, among their own set, just as they always are, that you can form any just judgment. Short of that, it is all guess and luck—and will generally be ill-luck. How many a man has committed himself on a short acquaintance, and rued it all the rest of his life!>

Miss Fairfax, who had seldom spoken before, except among her own confederates, spoke now.

<Such things do occur, undoubtedly.>—She was stopped by a cough. Frank Churchill turned towards her to listen.

<You were speaking,> said he, gravely. She recovered her voice.

<I was only going to observe, that though such unfortunate circumstances do sometimes occur both to men and women, I cannot imagine them to be very frequent. A hasty and imprudent attachment may arise—but there is generally time to recover from it afterwards. I would be understood to mean, that it can be only weak, irresolute characters, (whose happiness must be always at the mercy of chance,) who will suffer an unfortunate acquaintance to be an inconvenience, an oppression for ever.>

He made no answer; merely looked, and bowed in submission; and soon afterwards said, in a lively tone,

<Well, I have so little confidence in my own judgment, that whenever I marry, I hope some body will chuse my wife for me. Will you? (turning to Emma.) Will you chuse a wife for me?—I am sure I should like any body fixed on by you. You provide for the family, you know, (with a smile at his father). Find some body for me. I am in no hurry. Adopt her, educate her.>

<And make her like myself.>

<By all means, if you can.>

<Very well. I undertake the commission. You shall have a charming wife.>

<She must be very lively, and have hazle eyes. I care for nothing else. I shall go abroad for a couple of years—and when I return, I shall come to you for my wife. Remember.>

Emma was in no danger of forgetting. It was a commission to touch every favourite feeling. Would not Harriet be the very creature described? Hazle eyes excepted, two years more might make her all that he wished. He might even have Harriet in his thoughts at the moment; who could say? Referring the education to her seemed to imply it.

<Now, ma'am,> said Jane to her aunt, <shall we join Mrs Elton?>

<If you please, my dear. With all my heart. I am quite ready. I was ready to have gone with her, but this will do just as well. We shall soon overtake her. There she is—no, that's somebody else. That's one of the ladies in the Irish car party, not at all like her.—Well, I declare\longdash>

They walked off, followed in half a minute by Mr Knightley. Mr Weston, his son, Emma, and Harriet, only remained; and the young man's spirits now rose to a pitch almost unpleasant. Even Emma grew tired at last of flattery and merriment, and wished herself rather walking quietly about with any of the others, or sitting almost alone, and quite unattended to, in tranquil observation of the beautiful views beneath her. The appearance of the servants looking out for them to give notice of the carriages was a joyful sight; and even the bustle of collecting and preparing to depart, and the solicitude of Mrs Elton to have her carriage first, were gladly endured, in the prospect of the quiet drive home which was to close the very questionable enjoyments of this day of pleasure. Such another scheme, composed of so many ill-assorted people, she hoped never to be betrayed into again.

While waiting for the carriage, she found Mr Knightley by her side. He looked around, as if to see that no one were near, and then said,

<Emma, I must once more speak to you as I have been used to do: a privilege rather endured than allowed, perhaps, but I must still use it. I cannot see you acting wrong, without a remonstrance. How could you be so unfeeling to Miss Bates? How could you be so insolent in your wit to a woman of her character, age, and situation?—Emma, I had not thought it possible.>

Emma recollected, blushed, was sorry, but tried to laugh it off.

<Nay, how could I help saying what I did?—Nobody could have helped it. It was not so very bad. I dare say she did not understand me.>

<I assure you she did. She felt your full meaning. She has talked of it since. I wish you could have heard how she talked of it—with what candour and generosity. I wish you could have heard her honouring your forbearance, in being able to pay her such attentions, as she was for ever receiving from yourself and your father, when her society must be so irksome.>

<Oh!> cried Emma, <I know there is not a better creature in the world: but you must allow, that what is good and what is ridiculous are most unfortunately blended in her.>

<They are blended,> said he, <I acknowledge; and, were she prosperous, I could allow much for the occasional prevalence of the ridiculous over the good. Were she a woman of fortune, I would leave every harmless absurdity to take its chance, I would not quarrel with you for any liberties of manner. Were she your equal in situation—but, Emma, consider how far this is from being the case. She is poor; she has sunk from the comforts she was born to; and, if she live to old age, must probably sink more. Her situation should secure your compassion. It was badly done, indeed! You, whom she had known from an infant, whom she had seen grow up from a period when her notice was an honour, to have you now, in thoughtless spirits, and the pride of the moment, laugh at her, humble her—and before her niece, too—and before others, many of whom (certainly some,) would be entirely guided by your treatment of her.—This is not pleasant to you, Emma—and it is very far from pleasant to me; but I must, I will,—I will tell you truths while I can; satisfied with proving myself your friend by very faithful counsel, and trusting that you will some time or other do me greater justice than you can do now.>

While they talked, they were advancing towards the carriage; it was ready; and, before she could speak again, he had handed her in. He had misinterpreted the feelings which had kept her face averted, and her tongue motionless. They were combined only of anger against herself, mortification, and deep concern. She had not been able to speak; and, on entering the carriage, sunk back for a moment overcome—then reproaching herself for having taken no leave, making no acknowledgment, parting in apparent sullenness, she looked out with voice and hand eager to shew a difference; but it was just too late. He had turned away, and the horses were in motion. She continued to look back, but in vain; and soon, with what appeared unusual speed, they were half way down the hill, and every thing left far behind. She was vexed beyond what could have been expressed—almost beyond what she could conceal. Never had she felt so agitated, mortified, grieved, at any circumstance in her life. She was most forcibly struck. The truth of this representation there was no denying. She felt it at her heart. How could she have been so brutal, so cruel to Miss Bates! How could she have exposed herself to such ill opinion in any one she valued! And how suffer him to leave her without saying one word of gratitude, of concurrence, of common kindness!

Time did not compose her. As she reflected more, she seemed but to feel it more. She never had been so depressed. Happily it was not necessary to speak. There was only Harriet, who seemed not in spirits herself, fagged, and very willing to be silent; and Emma felt the tears running down her cheeks almost all the way home, without being at any trouble to check them, extraordinary as they were.