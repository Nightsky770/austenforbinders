\chapter[Chapter \thechapter]{} 

 \lettrine{T}{he} visions of romance were over. Catherine was completely awakened. Henry's address, short as it had been, had more thoroughly opened her eyes to the extravagance of her late fancies than all their several disappointments had done. Most grievously was she humbled. Most bitterly did she cry. It was not only with herself that she was sunk—but with Henry. Her folly, which now seemed even criminal, was all exposed to him, and he must despise her forever. The liberty which her imagination had dared to take with the character of his father—could he ever forgive it? The absurdity of her curiosity and her fears—could they ever be forgotten? She hated herself more than she could express. He had—she thought he had, once or twice before this fatal morning, shown something like affection for her. But now—in short, she made herself as miserable as possible for about half an hour, went down when the clock struck five, with a broken heart, and could scarcely give an intelligible answer to Eleanor's inquiry if she was well. The formidable Henry soon followed her into the room, and the only difference in his behaviour to her was that he paid her rather more attention than usual. Catherine had never wanted comfort more, and he looked as if he was aware of it. 

 The evening wore away with no abatement of this soothing politeness; and her spirits were gradually raised to a modest tranquillity. She did not learn either to forget or defend the past; but she learned to hope that it would never transpire farther, and that it might not cost her Henry's entire regard. Her thoughts being still chiefly fixed on what she had with such causeless terror felt and done, nothing could shortly be clearer than that it had been all a voluntary, self-created delusion, each trifling circumstance receiving importance from an imagination resolved on alarm, and everything forced to bend to one purpose by a mind which, before she entered the abbey, had been craving to be frightened. She remembered with what feelings she had prepared for a knowledge of Northanger. She saw that the infatuation had been created, the mischief settled, long before her quitting Bath, and it seemed as if the whole might be traced to the influence of that sort of reading which she had there indulged. 

 Charming as were all Mrs~Radcliffe's works, and charming even as were the works of all her imitators, it was not in them perhaps that human nature, at least in the Midland counties of England, was to be looked for. Of the Alps and Pyrenees, with their pine forests and their vices, they might give a faithful delineation; and Italy, Switzerland, and the south of France might be as fruitful in horrors as they were there represented. Catherine dared not doubt beyond her own country, and even of that, if hard pressed, would have yielded the northern and western extremities. But in the central part of England there was surely some security for the existence even of a wife not beloved, in the laws of the land, and the manners of the age. Murder was not tolerated, servants were not slaves, and neither poison nor sleeping potions to be procured, like rhubarb, from every druggist. Among the Alps and Pyrenees, perhaps, there were no mixed characters. There, such as were not as spotless as an angel might have the dispositions of a fiend. But in England it was not so; among the English, she believed, in their hearts and habits, there was a general though unequal mixture of good and bad. Upon this conviction, she would not be surprised if even in Henry and Eleanor Tilney, some slight imperfection might hereafter appear; and upon this conviction she need not fear to acknowledge some actual specks in the character of their father, who, though cleared from the grossly injurious suspicions which she must ever blush to have entertained, she did believe, upon serious consideration, to be not perfectly amiable. 

 Her mind made up on these several points, and her resolution formed, of always judging and acting in future with the greatest good sense, she had nothing to do but to forgive herself and be happier than ever; and the lenient hand of time did much for her by insensible gradations in the course of another day. Henry's astonishing generosity and nobleness of conduct, in never alluding in the slightest way to what had passed, was of the greatest assistance to her; and sooner than she could have supposed it possible in the beginning of her distress, her spirits became absolutely comfortable, and capable, as heretofore, of continual improvement by anything he said. There were still some subjects, indeed, under which she believed they must always tremble—the mention of a chest or a cabinet, for instance—and she did not love the sight of japan in any shape: but even \textit{she} could allow that an occasional memento of past folly, however painful, might not be without use. 

 The anxieties of common life began soon to succeed to the alarms of romance. Her desire of hearing from Isabella grew every day greater. She was quite impatient to know how the Bath world went on, and how the rooms were attended; and especially was she anxious to be assured of Isabella's having matched some fine netting-cotton, on which she had left her intent; and of her continuing on the best terms with James. Her only dependence for information of any kind was on Isabella. James had protested against writing to her till his return to Oxford; and Mrs~Allen had given her no hopes of a letter till she had got back to Fullerton. But Isabella had promised and promised again; and when she promised a thing, she was so scrupulous in performing it! This made it so particularly strange! 

 For nine successive mornings, Catherine wondered over the repetition of a disappointment, which each morning became more severe: but, on the tenth, when she entered the breakfast-room, her first object was a letter, held out by Henry's willing hand. She thanked him as heartily as if he had written it himself. <'Tis only from James, however,> as she looked at the direction. She opened it; it was from Oxford; and to this purpose:  
 
 \begin{mail}{}{Dear Catherine,}
 Though, God knows, with little inclination for writing, I think it my duty to tell you that everything is at an end between Miss Thorpe and me. I left her and Bath yesterday, never to see either again. I shall not enter into particulars—they would only pain you more. You will soon hear enough from another quarter to know where lies the blame; and I hope will acquit your brother of everything but the folly of too easily thinking his affection returned. Thank God! I am undeceived in time! But it is a heavy blow! After my father's consent had been so kindly given—but no more of this. She has made me miserable forever! Let me soon hear from you, dear Catherine; you are my only friend; \textit{your} love I do build upon. I wish your visit at Northanger may be over before Captain Tilney makes his engagement known, or you will be uncomfortably circumstanced. Poor Thorpe is in town: I dread the sight of him; his honest heart would feel so much. I have written to him and my father. Her duplicity hurts me more than all; till the very last, if I reasoned with her, she declared herself as much attached to me as ever, and laughed at my fears. I am ashamed to think how long I bore with it; but if ever man had reason to believe himself loved, I was that man. I cannot understand even now what she would be at, for there could be no need of my being played off to make her secure of Tilney. We parted at last by mutual consent—happy for me had we never met! I can never expect to know such another woman! Dearest Catherine, beware how you give your heart.  
 
 \closeletter[Believe me, \&c.]{}
\end{mail}


 Catherine had not read three lines before her sudden change of countenance, and short exclamations of sorrowing wonder, declared her to be receiving unpleasant news; and Henry, earnestly watching her through the whole letter, saw plainly that it ended no better than it began. He was prevented, however, from even looking his surprise by his father's entrance. They went to breakfast directly; but Catherine could hardly eat anything. Tears filled her eyes, and even ran down her cheeks as she sat. The letter was one moment in her hand, then in her lap, and then in her pocket; and she looked as if she knew not what she did. The general, between his cocoa and his newspaper, had luckily no leisure for noticing her; but to the other two her distress was equally visible. As soon as she dared leave the table she hurried away to her own room; but the housemaids were busy in it, and she was obliged to come down again. She turned into the drawing-room for privacy, but Henry and Eleanor had likewise retreated thither, and were at that moment deep in consultation about her. She drew back, trying to beg their pardon, but was, with gentle violence, forced to return; and the others withdrew, after Eleanor had affectionately expressed a wish of being of use or comfort to her. 

 After half an hour's free indulgence of grief and reflection, Catherine felt equal to encountering her friends; but whether she should make her distress known to them was another consideration. Perhaps, if particularly questioned, she might just give an idea—just distantly hint at it—but not more. To expose a friend, such a friend as Isabella had been to her—and then their own brother so closely concerned in it! She believed she must waive the subject altogether. Henry and Eleanor were by themselves in the breakfast-room; and each, as she entered it, looked at her anxiously. Catherine took her place at the table, and, after a short silence, Eleanor said, <No bad news from Fullerton, I hope? Mr~and Mrs~Morland—your brothers and sisters—I hope they are none of them ill?> 

 <No, I thank you> (sighing as she spoke); <they are all very well. My letter was from my brother at Oxford.> 

 Nothing further was said for a few minutes; and then speaking through her tears, she added, <I do not think I shall ever wish for a letter again!> 

 <I am sorry,> said Henry, closing the book he had just opened; <if I had suspected the letter of containing anything unwelcome, I should have given it with very different feelings.> 

 <It contained something worse than anybody could suppose! Poor James is so unhappy! You will soon know why.> 

 <To have so kind-hearted, so affectionate a sister,> replied Henry warmly, <must be a comfort to him under any distress.> 

 <I have one favour to beg,> said Catherine, shortly afterwards, in an agitated manner, <that, if your brother should be coming here, you will give me notice of it, that I may go away.> 

 <Our brother! Frederick!> 

 <Yes; I am sure I should be very sorry to leave you so soon, but something has happened that would make it very dreadful for me to be in the same house with Captain Tilney.> 

 Eleanor's work was suspended while she gazed with increasing astonishment; but Henry began to suspect the truth, and something, in which Miss Thorpe's name was included, passed his lips. 

 <How quick you are!> cried Catherine: <you have guessed it, I declare! And yet, when we talked about it in Bath, you little thought of its ending so. Isabella—no wonder \textit{now} I have not heard from her—Isabella has deserted my brother, and is to marry yours! Could you have believed there had been such inconstancy and fickleness, and everything that is bad in the world?> 

 <I hope, so far as concerns my brother, you are misinformed. I hope he has not had any material share in bringing on Mr~Morland's disappointment. His marrying Miss Thorpe is not probable. I think you must be deceived so far. I am very sorry for Mr~Morland—sorry that anyone you love should be unhappy; but my surprise would be greater at Frederick's marrying her than at any other part of the story.> 

 <It is very true, however; you shall read James's letter yourself. Stay—There is one part\longdash> recollecting with a blush the last line. 

 <Will you take the trouble of reading to us the passages which concern my brother?> 

 <No, read it yourself,> cried Catherine, whose second thoughts were clearer. <I do not know what I was thinking of> (blushing again that she had blushed before); <James only means to give me good advice.> 

 He gladly received the letter, and, having read it through, with close attention, returned it saying, <Well, if it is to be so, I can only say that I am sorry for it. Frederick will not be the first man who has chosen a wife with less sense than his family expected. I do not envy his situation, either as a lover or a son.> 

 Miss Tilney, at Catherine's invitation, now read the letter likewise, and, having expressed also her concern and surprise, began to inquire into Miss Thorpe's connections and fortune. 

 <Her mother is a very good sort of woman,> was Catherine's answer. 

 <What was her father?> 

 <A lawyer, I believe. They live at Putney.> 

 <Are they a wealthy family?> 

 <No, not very. I do not believe Isabella has any fortune at all: but that will not signify in your family. Your father is so very liberal! He told me the other day that he only valued money as it allowed him to promote the happiness of his children.> The brother and sister looked at each other. <But,> said Eleanor, after a short pause, <would it be to promote his happiness, to enable him to marry such a girl? She must be an unprincipled one, or she could not have used your brother so. And how strange an infatuation on Frederick's side! A girl who, before his eyes, is violating an engagement voluntarily entered into with another man! Is not it inconceivable, Henry? Frederick too, who always wore his heart so proudly! Who found no woman good enough to be loved!> 

 <That is the most unpromising circumstance, the strongest presumption against him. When I think of his past declarations, I give him up. Moreover, I have too good an opinion of Miss Thorpe's prudence to suppose that she would part with one gentleman before the other was secured. It is all over with Frederick indeed! He is a deceased man—defunct in understanding. Prepare for your sister-in-law, Eleanor, and such a sister-in-law as you must delight in! Open, candid, artless, guileless, with affections strong but simple, forming no pretensions, and knowing no disguise.> 

 <Such a sister-in-law, Henry, I should delight in,> said Eleanor with a smile. 

 <But perhaps,> observed Catherine, <though she has behaved so ill by our family, she may behave better by yours. Now she has really got the man she likes, she may be constant.> 

 <Indeed I am afraid she will,> replied Henry; <I am afraid she will be very constant, unless a baronet should come in her way; that is Frederick's only chance. I will get the Bath paper, and look over the arrivals.> 

 <You think it is all for ambition, then? And, upon my word, there are some things that seem very like it. I cannot forget that, when she first knew what my father would do for them, she seemed quite disappointed that it was not more. I never was so deceived in anyone's character in my life before.> 

 <Among all the great variety that you have known and studied.> 

 <My own disappointment and loss in her is very great; but, as for poor James, I suppose he will hardly ever recover it.> 

 <Your brother is certainly very much to be pitied at present; but we must not, in our concern for his sufferings, undervalue yours. You feel, I suppose, that in losing Isabella, you lose half yourself: you feel a void in your heart which nothing else can occupy. Society is becoming irksome; and as for the amusements in which you were wont to share at Bath, the very idea of them without her is abhorrent. You would not, for instance, now go to a ball for the world. You feel that you have no longer any friend to whom you can speak with unreserve, on whose regard you can place dependence, or whose counsel, in any difficulty, you could rely on. You feel all this?> 

 <No,> said Catherine, after a few moments' reflection, <I do not—ought I? To say the truth, though I am hurt and grieved, that I cannot still love her, that I am never to hear from her, perhaps never to see her again, I do not feel so very, very much afflicted as one would have thought.> 

 <You feel, as you always do, what is most to the credit of human nature. Such feelings ought to be investigated, that they may know themselves.> 

 Catherine, by some chance or other, found her spirits so very much relieved by this conversation that she could not regret her being led on, though so unaccountably, to mention the circumstance which had produced it. 