\chapter[Chapter \thechapter]{} 

 \lettrine{S}{oon} after this, the general found himself obliged to go to London for a week; and he left Northanger earnestly regretting that any necessity should rob him even for an hour of Miss Morland's company, and anxiously recommending the study of her comfort and amusement to his children as their chief object in his absence. His departure gave Catherine the first experimental conviction that a loss may be sometimes a gain. The happiness with which their time now passed, every employment voluntary, every laugh indulged, every meal a scene of ease and good humour, walking where they liked and when they liked, their hours, pleasures, and fatigues at their own command, made her thoroughly sensible of the restraint which the General's presence had imposed, and most thankfully feel their present release from it. Such ease and such delights made her love the place and the people more and more every day; and had it not been for a dread of its soon becoming expedient to leave the one, and an apprehension of not being equally beloved by the other, she would at each moment of each day have been perfectly happy; but she was now in the fourth week of her visit; before the general came home, the fourth week would be turned, and perhaps it might seem an intrusion if she stayed much longer. This was a painful consideration whenever it occurred; and eager to get rid of such a weight on her mind, she very soon resolved to speak to Eleanor about it at once, propose going away, and be guided in her conduct by the manner in which her proposal might be taken. 

 Aware that if she gave herself much time, she might feel it difficult to bring forward so unpleasant a subject, she took the first opportunity of being suddenly alone with Eleanor, and of Eleanor's being in the middle of a speech about something very different, to start forth her obligation of going away very soon. Eleanor looked and declared herself much concerned. She had <hoped for the pleasure of her company for a much longer time—had been misled (perhaps by her wishes) to suppose that a much longer visit had been promised—and could not but think that if Mr~and Mrs~Morland were aware of the pleasure it was to her to have her there, they would be too generous to hasten her return.> Catherine explained: <Oh! As to \textit{that}, Papa and Mamma were in no hurry at all. As long as she was happy, they would always be satisfied.> 

 <Then why, might she ask, in such a hurry herself to leave them?> 

 <Oh! Because she had been there so long.> 

 <Nay, if you can use such a word, I can urge you no farther. If you think it long\longdash> 

 <Oh! No, I do not indeed. For my own pleasure, I could stay with you as long again.> And it was directly settled that, till she had, her leaving them was not even to be thought of. In having this cause of uneasiness so pleasantly removed, the force of the other was likewise weakened. The kindness, the earnestness of Eleanor's manner in pressing her to stay, and Henry's gratified look on being told that her stay was determined, were such sweet proofs of her importance with them, as left her only just so much solicitude as the human mind can never do comfortably without. She did—almost always—believe that Henry loved her, and quite always that his father and sister loved and even wished her to belong to them; and believing so far, her doubts and anxieties were merely sportive irritations. 

 Henry was not able to obey his father's injunction of remaining wholly at Northanger in attendance on the ladies, during his absence in London, the engagements of his curate at Woodston obliging him to leave them on Saturday for a couple of nights. His loss was not now what it had been while the general was at home; it lessened their gaiety, but did not ruin their comfort; and the two girls agreeing in occupation, and improving in intimacy, found themselves so well sufficient for the time to themselves, that it was eleven o'clock, rather a late hour at the abbey, before they quitted the supper-room on the day of Henry's departure. They had just reached the head of the stairs when it seemed, as far as the thickness of the walls would allow them to judge, that a carriage was driving up to the door, and the next moment confirmed the idea by the loud noise of the house-bell. After the first perturbation of surprise had passed away, in a <Good heaven! What can be the matter?> it was quickly decided by Eleanor to be her eldest brother, whose arrival was often as sudden, if not quite so unseasonable, and accordingly she hurried down to welcome him. 

 Catherine walked on to her chamber, making up her mind as well as she could, to a further acquaintance with Captain Tilney, and comforting herself under the unpleasant impression his conduct had given her, and the persuasion of his being by far too fine a gentleman to approve of her, that at least they should not meet under such circumstances as would make their meeting materially painful. She trusted he would never speak of Miss Thorpe; and indeed, as he must by this time be ashamed of the part he had acted, there could be no danger of it; and as long as all mention of Bath scenes were avoided, she thought she could behave to him very civilly. In such considerations time passed away, and it was certainly in his favour that Eleanor should be so glad to see him, and have so much to say, for half an hour was almost gone since his arrival, and Eleanor did not come up. 

 At that moment Catherine thought she heard her step in the gallery, and listened for its continuance; but all was silent. Scarcely, however, had she convicted her fancy of error, when the noise of something moving close to her door made her start; it seemed as if someone was touching the very doorway—and in another moment a slight motion of the lock proved that some hand must be on it. She trembled a little at the idea of anyone's approaching so cautiously; but resolving not to be again overcome by trivial appearances of alarm, or misled by a raised imagination, she stepped quietly forward, and opened the door. Eleanor, and only Eleanor, stood there. Catherine's spirits, however, were tranquillized but for an instant, for Eleanor's cheeks were pale, and her manner greatly agitated. Though evidently intending to come in, it seemed an effort to enter the room, and a still greater to speak when there. Catherine, supposing some uneasiness on Captain Tilney's account, could only express her concern by silent attention, obliged her to be seated, rubbed her temples with lavender water, and hung over her with affectionate solicitude. <My dear Catherine, you must not—you must not indeed\longdash> were Eleanor's first connected words. <I am quite well. This kindness distracts me—I cannot bear it—I come to you on such an errand!> 

 <Errand! To me!> 

 <How shall I tell you! Oh! How shall I tell you!> 

 A new idea now darted into Catherine's mind, and turning as pale as her friend, she exclaimed, <'Tis a messenger from Woodston!> 

 <You are mistaken, indeed,> returned Eleanor, looking at her most compassionately; <it is no one from Woodston. It is my father himself.> Her voice faltered, and her eyes were turned to the ground as she mentioned his name. His unlooked-for return was enough in itself to make Catherine's heart sink, and for a few moments she hardly supposed there were anything worse to be told. She said nothing; and Eleanor, endeavouring to collect herself and speak with firmness, but with eyes still cast down, soon went on. <You are too good, I am sure, to think the worse of me for the part I am obliged to perform. I am indeed a most unwilling messenger. After what has so lately passed, so lately been settled between us—how joyfully, how thankfully on my side!—as to your continuing here as I hoped for many, many weeks longer, how can I tell you that your kindness is not to be accepted—and that the happiness your company has hitherto given us is to be repaid by—But I must not trust myself with words. My dear Catherine, we are to part. My father has recollected an engagement that takes our whole family away on Monday. We are going to Lord Longtown's, near Hereford, for a fortnight. Explanation and apology are equally impossible. I cannot attempt either.> 

 <My dear Eleanor,> cried Catherine, suppressing her feelings as well as she could, <do not be so distressed. A second engagement must give way to a first. I am very, very sorry we are to part—so soon, and so suddenly too; but I am not offended, indeed I am not. I can finish my visit here, you know, at any time; or I hope you will come to me. Can you, when you return from this lord's, come to Fullerton?> 

 <It will not be in my power, Catherine.> 

 <Come when you can, then.> 

 Eleanor made no answer; and Catherine's thoughts recurring to something more directly interesting, she added, thinking aloud, <Monday—so soon as Monday; and you \textit{all} go. Well, I am certain of—I shall be able to take leave, however. I need not go till just before you do, you know. Do not be distressed, Eleanor, I can go on Monday very well. My father and mother's having no notice of it is of very little consequence. The general will send a servant with me, I dare say, half the way—and then I shall soon be at Salisbury, and then I am only nine miles from home.> 

 <Ah, Catherine! Were it settled so, it would be somewhat less intolerable, though in such common attentions you would have received but half what you ought. But—how can I tell you?—to-morrow morning is fixed for your leaving us, and not even the hour is left to your choice; the very carriage is ordered, and will be here at seven o'clock, and no servant will be offered you.> 

 Catherine sat down, breathless and speechless. <I could hardly believe my senses, when I heard it; and no displeasure, no resentment that you can feel at this moment, however justly great, can be more than I myself—but I must not talk of what I felt. Oh! That I could suggest anything in extenuation! Good God! What will your father and mother say! After courting you from the protection of real friends to this—almost double distance from your home, to have you driven out of the house, without the considerations even of decent civility! Dear, dear Catherine, in being the bearer of such a message, I seem guilty myself of all its insult; yet, I trust you will acquit me, for you must have been long enough in this house to see that I am but a nominal mistress of it, that my real power is nothing.> 

 <Have I offended the general?> said Catherine in a faltering voice. 

 <Alas! For my feelings as a daughter, all that I know, all that I answer for, is that you can have given him no just cause of offence. He certainly is greatly, very greatly discomposed; I have seldom seen him more so. His temper is not happy, and something has now occurred to ruffle it in an uncommon degree; some disappointment, some vexation, which just at this moment seems important, but which I can hardly suppose you to have any concern in, for how is it possible?> 

 It was with pain that Catherine could speak at all; and it was only for Eleanor's sake that she attempted it. <I am sure,> said she, <I am very sorry if I have offended him. It was the last thing I would willingly have done. But do not be unhappy, Eleanor. An engagement, you know, must be kept. I am only sorry it was not recollected sooner, that I might have written home. But it is of very little consequence.> 

 <I hope, I earnestly hope, that to your real safety it will be of none; but to everything else it is of the greatest consequence: to comfort, appearance, propriety, to your family, to the world. Were your friends, the Allens, still in Bath, you might go to them with comparative ease; a few hours would take you there; but a journey of seventy miles, to be taken post by you, at your age, alone, unattended!> 

 <Oh, the journey is nothing. Do not think about that. And if we are to part, a few hours sooner or later, you know, makes no difference. I can be ready by seven. Let me be called in time.> Eleanor saw that she wished to be alone; and believing it better for each that they should avoid any further conversation, now left her with, <I shall see you in the morning.> 

 Catherine's swelling heart needed relief. In Eleanor's presence friendship and pride had equally restrained her tears, but no sooner was she gone than they burst forth in torrents. Turned from the house, and in such a way! Without any reason that could justify, any apology that could atone for the abruptness, the rudeness, nay, the insolence of it. Henry at a distance—not able even to bid him farewell. Every hope, every expectation from him suspended, at least, and who could say how long? Who could say when they might meet again? And all this by such a man as General Tilney, so polite, so well bred, and heretofore so particularly fond of her! It was as incomprehensible as it was mortifying and grievous. From what it could arise, and where it would end, were considerations of equal perplexity and alarm. The manner in which it was done so grossly uncivil, hurrying her away without any reference to her own convenience, or allowing her even the appearance of choice as to the time or mode of her travelling; of two days, the earliest fixed on, and of that almost the earliest hour, as if resolved to have her gone before he was stirring in the morning, that he might not be obliged even to see her. What could all this mean but an intentional affront? By some means or other she must have had the misfortune to offend him. Eleanor had wished to spare her from so painful a notion, but Catherine could not believe it possible that any injury or any misfortune could provoke such ill will against a person not connected, or, at least, not supposed to be connected with it. 

 Heavily passed the night. Sleep, or repose that deserved the name of sleep, was out of the question. That room, in which her disturbed imagination had tormented her on her first arrival, was again the scene of agitated spirits and unquiet slumbers. Yet how different now the source of her inquietude from what it had been then—how mournfully superior in reality and substance! Her anxiety had foundation in fact, her fears in probability; and with a mind so occupied in the contemplation of actual and natural evil, the solitude of her situation, the darkness of her chamber, the antiquity of the building, were felt and considered without the smallest emotion; and though the wind was high, and often produced strange and sudden noises throughout the house, she heard it all as she lay awake, hour after hour, without curiosity or terror. 

 Soon after six Eleanor entered her room, eager to show attention or give assistance where it was possible; but very little remained to be done. Catherine had not loitered; she was almost dressed, and her packing almost finished. The possibility of some conciliatory message from the general occurred to her as his daughter appeared. What so natural, as that anger should pass away and repentance succeed it? And she only wanted to know how far, after what had passed, an apology might properly be received by her. But the knowledge would have been useless here; it was not called for; neither clemency nor dignity was put to the trial—Eleanor brought no message. Very little passed between them on meeting; each found her greatest safety in silence, and few and trivial were the sentences exchanged while they remained upstairs, Catherine in busy agitation completing her dress, and Eleanor with more goodwill than experience intent upon filling the trunk. When everything was done they left the room, Catherine lingering only half a minute behind her friend to throw a parting glance on every well-known, cherished object, and went down to the breakfast-parlour, where breakfast was prepared. She tried to eat, as well to save herself from the pain of being urged as to make her friend comfortable; but she had no appetite, and could not swallow many mouthfuls. The contrast between this and her last breakfast in that room gave her fresh misery, and strengthened her distaste for everything before her. It was not four and twenty hours ago since they had met there to the same repast, but in circumstances how different! With what cheerful ease, what happy, though false, security, had she then looked around her, enjoying everything present, and fearing little in future, beyond Henry's going to Woodston for a day! Happy, happy breakfast! For Henry had been there; Henry had sat by her and helped her. These reflections were long indulged undisturbed by any address from her companion, who sat as deep in thought as herself; and the appearance of the carriage was the first thing to startle and recall them to the present moment. Catherine's colour rose at the sight of it; and the indignity with which she was treated, striking at that instant on her mind with peculiar force, made her for a short time sensible only of resentment. Eleanor seemed now impelled into resolution and speech. 

 <You \textit{must} write to me, Catherine,> she cried; <you \textit{must} let me hear from you as soon as possible. Till I know you to be safe at home, I shall not have an hour's comfort. For \textit{one} letter, at all risks, all hazards, I must entreat. Let me have the satisfaction of knowing that you are safe at Fullerton, and have found your family well, and then, till I can ask for your correspondence as I ought to do, I will not expect more. Direct to me at Lord Longtown's, and, I must ask it, under cover to Alice.> 

 <No, Eleanor, if you are not allowed to receive a letter from me, I am sure I had better not write. There can be no doubt of my getting home safe.> 

 Eleanor only replied, <I cannot wonder at your feelings. I will not importune you. I will trust to your own kindness of heart when I am at a distance from you.> But this, with the look of sorrow accompanying it, was enough to melt Catherine's pride in a moment, and she instantly said, <Oh, Eleanor, I \textit{will} write to you indeed.> 

 There was yet another point which Miss Tilney was anxious to settle, though somewhat embarrassed in speaking of. It had occurred to her that after so long an absence from home, Catherine might not be provided with money enough for the expenses of her journey, and, upon suggesting it to her with most affectionate offers of accommodation, it proved to be exactly the case. Catherine had never thought on the subject till that moment, but, upon examining her purse, was convinced that but for this kindness of her friend, she might have been turned from the house without even the means of getting home; and the distress in which she must have been thereby involved filling the minds of both, scarcely another word was said by either during the time of their remaining together. Short, however, was that time. The carriage was soon announced to be ready; and Catherine, instantly rising, a long and affectionate embrace supplied the place of language in bidding each other adieu; and, as they entered the hall, unable to leave the house without some mention of one whose name had not yet been spoken by either, she paused a moment, and with quivering lips just made it intelligible that she left <her kind remembrance for her absent friend.> But with this approach to his name ended all possibility of restraining her feelings; and, hiding her face as well as she could with her handkerchief, she darted across the hall, jumped into the chaise, and in a moment was driven from the door. 