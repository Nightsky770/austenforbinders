\chapter[Chapter \thechapter]{} 

 \lettrine{M}{onday}, Tuesday, Wednesday, Thursday, Friday, and Saturday have now passed in review before the reader; the events of each day, its hopes and fears, mortifications and pleasures, have been separately stated, and the pangs of Sunday only now remain to be described, and close the week. The Clifton scheme had been deferred, not relinquished, and on the afternoon's Crescent of this day, it was brought forward again. In a private consultation between Isabella and James, the former of whom had particularly set her heart upon going, and the latter no less anxiously placed his upon pleasing her, it was agreed that, provided the weather were fair, the party should take place on the following morning; and they were to set off very early, in order to be at home in good time. The affair thus determined, and Thorpe's approbation secured, Catherine only remained to be apprised of it. She had left them for a few minutes to speak to Miss Tilney. In that interval the plan was completed, and as soon as she came again, her agreement was demanded; but instead of the gay acquiescence expected by Isabella, Catherine looked grave, was very sorry, but could not go. The engagement which ought to have kept her from joining in the former attempt would make it impossible for her to accompany them now. She had that moment settled with Miss Tilney to take their proposed walk to-morrow; it was quite determined, and she would not, upon any account, retract. But that she \textit{must} and \textit{should} retract, was instantly the eager cry of both the Thorpes; they must go to Clifton to-morrow, they would not go without her, it would be nothing to put off a mere walk for one day longer, and they would not hear of a refusal. Catherine was distressed, but not subdued. <Do not urge me, Isabella. I am engaged to Miss Tilney. I cannot go.> This availed nothing. The same arguments assailed her again; she must go, she should go, and they would not hear of a refusal. <It would be so easy to tell Miss Tilney that you had just been reminded of a prior engagement, and must only beg to put off the walk till Tuesday.> 
 
 <No, it would not be easy. I could not do it. There has been no prior engagement.> But Isabella became only more and more urgent, calling on her in the most affectionate manner, addressing her by the most endearing names. She was sure her dearest, sweetest Catherine would not seriously refuse such a trifling request to a friend who loved her so dearly. She knew her beloved Catherine to have so feeling a heart, so sweet a temper, to be so easily persuaded by those she loved. But all in vain; Catherine felt herself to be in the right, and though pained by such tender, such flattering supplication, could not allow it to influence her. Isabella then tried another method. She reproached her with having more affection for Miss Tilney, though she had known her so little a while, than for her best and oldest friends, with being grown cold and indifferent, in short, towards herself. <I cannot help being jealous, Catherine, when I see myself slighted for strangers, I, who love you so excessively! When once my affections are placed, it is not in the power of anything to change them. But I believe my feelings are stronger than anybody's; I am sure they are too strong for my own peace; and to see myself supplanted in your friendship by strangers does cut me to the quick, I own. These Tilneys seem to swallow up everything else.> 

 Catherine thought this reproach equally strange and unkind. Was it the part of a friend thus to expose her feelings to the notice of others? Isabella appeared to her ungenerous and selfish, regardless of everything but her own gratification. These painful ideas crossed her mind, though she said nothing. Isabella, in the meanwhile, had applied her handkerchief to her eyes; and Morland, miserable at such a sight, could not help saying, <Nay, Catherine. I think you cannot stand out any longer now. The sacrifice is not much; and to oblige such a friend—I shall think you quite unkind, if you still refuse.> 

 This was the first time of her brother's openly siding against her, and anxious to avoid his displeasure, she proposed a compromise. If they would only put off their scheme till Tuesday, which they might easily do, as it depended only on themselves, she could go with them, and everybody might then be satisfied. But <No, no, no!> was the immediate answer; <that could not be, for Thorpe did not know that he might not go to town on Tuesday.> Catherine was sorry, but could do no more; and a short silence ensued, which was broken by Isabella, who in a voice of cold resentment said, <Very well, then there is an end of the party. If Catherine does not go, I cannot. I cannot be the only woman. I would not, upon any account in the world, do so improper a thing.> 

 <Catherine, you must go,> said James. 

 <But why cannot Mr~Thorpe drive one of his other sisters? I dare say either of them would like to go.> 

 <Thank ye,> cried Thorpe, <but I did not come to Bath to drive my sisters about, and look like a fool. No, if you do not go, d\doubleemdash me if I do. I only go for the sake of driving you.> 

 <That is a compliment which gives me no pleasure.> But her words were lost on Thorpe, who had turned abruptly away. 

 The three others still continued together, walking in a most uncomfortable manner to poor Catherine; sometimes not a word was said, sometimes she was again attacked with supplications or reproaches, and her arm was still linked within Isabella's, though their hearts were at war. At one moment she was softened, at another irritated; always distressed, but always steady. 

 <I did not think you had been so obstinate, Catherine,> said James; <you were not used to be so hard to persuade; you once were the kindest, best-tempered of my sisters.> 

 <I hope I am not less so now,> she replied, very feelingly; <but indeed I cannot go. If I am wrong, I am doing what I believe to be right.> 

 <I suspect,> said Isabella, in a low voice, <there is no great struggle.> 

 Catherine's heart swelled; she drew away her arm, and Isabella made no opposition. Thus passed a long ten minutes, till they were again joined by Thorpe, who, coming to them with a gayer look, said, <Well, I have settled the matter, and now we may all go to-morrow with a safe conscience. I have been to Miss Tilney, and made your excuses.> 

 <You have not!> cried Catherine. 

 <I have, upon my soul. Left her this moment. Told her you had sent me to say that, having just recollected a prior engagement of going to Clifton with us to-morrow, you could not have the pleasure of walking with her till Tuesday. She said very well, Tuesday was just as convenient to her; so there is an end of all our difficulties. A pretty good thought of mine—hey?> 

 Isabella's countenance was once more all smiles and good humour, and James too looked happy again. 

 <A most heavenly thought indeed! Now, my sweet Catherine, all our distresses are over; you are honourably acquitted, and we shall have a most delightful party.> 

 <This will not do,> said Catherine; <I cannot submit to this. I must run after Miss Tilney directly and set her right.> 

 Isabella, however, caught hold of one hand, Thorpe of the other, and remonstrances poured in from all three. Even James was quite angry. When everything was settled, when Miss Tilney herself said that Tuesday would suit her as well, it was quite ridiculous, quite absurd, to make any further objection. 

 <I do not care. Mr~Thorpe had no business to invent any such message. If I had thought it right to put it off, I could have spoken to Miss Tilney myself. This is only doing it in a ruder way; and how do I know that Mr~Thorpe has—He may be mistaken again perhaps; he led me into one act of rudeness by his mistake on Friday. Let me go, Mr~Thorpe; Isabella, do not hold me.> 

 Thorpe told her it would be in vain to go after the Tilneys; they were turning the corner into Brock Street, when he had overtaken them, and were at home by this time. 

 <Then I will go after them,> said Catherine; <wherever they are I will go after them. It does not signify talking. If I could not be persuaded into doing what I thought wrong, I never will be tricked into it.> And with these words she broke away and hurried off. Thorpe would have darted after her, but Morland withheld him. <Let her go, let her go, if she will go.> 

 <She is as obstinate as\longdash> 

 Thorpe never finished the simile, for it could hardly have been a proper one. 

 Away walked Catherine in great agitation, as fast as the crowd would permit her, fearful of being pursued, yet determined to persevere. As she walked, she reflected on what had passed. It was painful to her to disappoint and displease them, particularly to displease her brother; but she could not repent her resistance. Setting her own inclination apart, to have failed a second time in her engagement to Miss Tilney, to have retracted a promise voluntarily made only five minutes before, and on a false pretence too, must have been wrong. She had not been withstanding them on selfish principles alone, she had not consulted merely her own gratification; \textit{that} might have been ensured in some degree by the excursion itself, by seeing Blaize Castle; no, she had attended to what was due to others, and to her own character in their opinion. Her conviction of being right, however, was not enough to restore her composure; till she had spoken to Miss Tilney she could not be at ease; and quickening her pace when she got clear of the Crescent, she almost ran over the remaining ground till she gained the top of Milsom Street. So rapid had been her movements that in spite of the Tilneys' advantage in the outset, they were but just turning into their lodgings as she came within view of them; and the servant still remaining at the open door, she used only the ceremony of saying that she must speak with Miss Tilney that moment, and hurrying by him proceeded upstairs. Then, opening the first door before her, which happened to be the right, she immediately found herself in the drawing-room with General Tilney, his son, and daughter. Her explanation, defective only in being—from her irritation of nerves and shortness of breath—no explanation at all, was instantly given. <I am come in a great hurry—It was all a mistake—I never promised to go—I told them from the first I could not go.—I ran away in a great hurry to explain it.—I did not care what you thought of me.—I would not stay for the servant.> 

 The business, however, though not perfectly elucidated by this speech, soon ceased to be a puzzle. Catherine found that John Thorpe \textit{had} given the message; and Miss Tilney had no scruple in owning herself greatly surprised by it. But whether her brother had still exceeded her in resentment, Catherine, though she instinctively addressed herself as much to one as to the other in her vindication, had no means of knowing. Whatever might have been felt before her arrival, her eager declarations immediately made every look and sentence as friendly as she could desire. 

 The affair thus happily settled, she was introduced by Miss Tilney to her father, and received by him with such ready, such solicitous politeness as recalled Thorpe's information to her mind, and made her think with pleasure that he might be sometimes depended on. To such anxious attention was the General's civility carried, that not aware of her extraordinary swiftness in entering the house, he was quite angry with the servant whose neglect had reduced her to open the door of the apartment herself. <What did William mean by it? He should make a point of inquiring into the matter.> And if Catherine had not most warmly asserted his innocence, it seemed likely that William would lose the favour of his master forever, if not his place, by her rapidity. 

 After sitting with them a quarter of an hour, she rose to take leave, and was then most agreeably surprised by General Tilney's asking her if she would do his daughter the honour of dining and spending the rest of the day with her. Miss Tilney added her own wishes. Catherine was greatly obliged; but it was quite out of her power. Mr~and Mrs~Allen would expect her back every moment. The general declared he could say no more; the claims of Mr~and Mrs~Allen were not to be superseded; but on some other day he trusted, when longer notice could be given, they would not refuse to spare her to her friend. <Oh, no; Catherine was sure they would not have the least objection, and she should have great pleasure in coming.> The general attended her himself to the street-door, saying everything gallant as they went downstairs, admiring the elasticity of her walk, which corresponded exactly with the spirit of her dancing, and making her one of the most graceful bows she had ever beheld, when they parted. 

 Catherine, delighted by all that had passed, proceeded gaily to Pulteney Street, walking, as she concluded, with great elasticity, though she had never thought of it before. She reached home without seeing anything more of the offended party; and now that she had been triumphant throughout, had carried her point, and was secure of her walk, she began (as the flutter of her spirits subsided) to doubt whether she had been perfectly right. A sacrifice was always noble; and if she had given way to their entreaties, she should have been spared the distressing idea of a friend displeased, a brother angry, and a scheme of great happiness to both destroyed, perhaps through her means. To ease her mind, and ascertain by the opinion of an unprejudiced person what her own conduct had really been, she took occasion to mention before Mr~Allen the half-settled scheme of her brother and the Thorpes for the following day. Mr~Allen caught at it directly. <Well,> said he, <and do you think of going too?> 

 <No; I had just engaged myself to walk with Miss Tilney before they told me of it; and therefore you know I could not go with them, could I?> 

 <No, certainly not; and I am glad you do not think of it. These schemes are not at all the thing. Young men and women driving about the country in open carriages! Now and then it is very well; but going to inns and public places together! It is not right; and I wonder Mrs~Thorpe should allow it. I am glad you do not think of going; I am sure Mrs~Morland would not be pleased. Mrs~Allen, are not you of my way of thinking? Do not you think these kind of projects objectionable?> 

 <Yes, very much so indeed. Open carriages are nasty things. A clean gown is not five minutes' wear in them. You are splashed getting in and getting out; and the wind takes your hair and your bonnet in every direction. I hate an open carriage myself.> 

 <I know you do; but that is not the question. Do not you think it has an odd appearance, if young ladies are frequently driven about in them by young men, to whom they are not even related?> 

 <Yes, my dear, a very odd appearance indeed. I cannot bear to see it.> 

 <Dear madam,> cried Catherine, <then why did not you tell me so before? I am sure if I had known it to be improper, I would not have gone with Mr~Thorpe at all; but I always hoped you would tell me, if you thought I was doing wrong.> 

 <And so I should, my dear, you may depend on it; for as I told Mrs~Morland at parting, I would always do the best for you in my power. But one must not be over particular. Young people \textit{will} be young people, as your good mother says herself. You know I wanted you, when we first came, not to buy that sprigged muslin, but you would. Young people do not like to be always thwarted.> 

 <But this was something of real consequence; and I do not think you would have found me hard to persuade.> 

 <As far as it has gone hitherto, there is no harm done,> said Mr~Allen; <and I would only advise you, my dear, not to go out with Mr~Thorpe any more.> 

 <That is just what I was going to say,> added his wife. 

 Catherine, relieved for herself, felt uneasy for Isabella, and after a moment's thought, asked Mr~Allen whether it would not be both proper and kind in her to write to Miss Thorpe, and explain the indecorum of which she must be as insensible as herself; for she considered that Isabella might otherwise perhaps be going to Clifton the next day, in spite of what had passed. Mr~Allen, however, discouraged her from doing any such thing. <You had better leave her alone, my dear; she is old enough to know what she is about, and if not, has a mother to advise her. Mrs~Thorpe is too indulgent beyond a doubt; but, however, you had better not interfere. She and your brother choose to go, and you will be only getting ill will.> 

 Catherine submitted, and though sorry to think that Isabella should be doing wrong, felt greatly relieved by Mr~Allen's approbation of her own conduct, and truly rejoiced to be preserved by his advice from the danger of falling into such an error herself. Her escape from being one of the party to Clifton was now an escape indeed; for what would the Tilneys have thought of her, if she had broken her promise to them in order to do what was wrong in itself, if she had been guilty of one breach of propriety, only to enable her to be guilty of another? 