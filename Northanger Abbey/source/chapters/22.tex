\chapter[Chapter \thechapter]{} 

 \lettrine{T}{he} housemaid's folding back her window-shutters at eight o'clock the next day was the sound which first roused Catherine; and she opened her eyes, wondering that they could ever have been closed, on objects of cheerfulness; her fire was already burning, and a bright morning had succeeded the tempest of the night. Instantaneously, with the consciousness of existence, returned her recollection of the manuscript; and springing from the bed in the very moment of the maid's going away, she eagerly collected every scattered sheet which had burst from the roll on its falling to the ground, and flew back to enjoy the luxury of their perusal on her pillow. She now plainly saw that she must not expect a manuscript of equal length with the generality of what she had shuddered over in books, for the roll, seeming to consist entirely of small disjointed sheets, was altogether but of trifling size, and much less than she had supposed it to be at first. 

 Her greedy eye glanced rapidly over a page. She started at its import. Could it be possible, or did not her senses play her false? An inventory of linen, in coarse and modern characters, seemed all that was before her! If the evidence of sight might be trusted, she held a washing-bill in her hand. She seized another sheet, and saw the same articles with little variation; a third, a fourth, and a fifth presented nothing new. Shirts, stockings, cravats, and waistcoats faced her in each. Two others, penned by the same hand, marked an expenditure scarcely more interesting, in letters, hair-powder, shoe-string, and breeches-ball. And the larger sheet, which had enclosed the rest, seemed by its first cramp line, <To poultice chestnut mare>—a farrier's bill! Such was the collection of papers (left perhaps, as she could then suppose, by the negligence of a servant in the place whence she had taken them) which had filled her with expectation and alarm, and robbed her of half her night's rest! She felt humbled to the dust. Could not the adventure of the chest have taught her wisdom? A corner of it, catching her eye as she lay, seemed to rise up in judgment against her. Nothing could now be clearer than the absurdity of her recent fancies. To suppose that a manuscript of many generations back could have remained undiscovered in a room such as that, so modern, so habitable!—Or that she should be the first to possess the skill of unlocking a cabinet, the key of which was open to all! 

 How could she have so imposed on herself? Heaven forbid that Henry Tilney should ever know her folly! And it was in a great measure his own doing, for had not the cabinet appeared so exactly to agree with his description of her adventures, she should never have felt the smallest curiosity about it. This was the only comfort that occurred. Impatient to get rid of those hateful evidences of her folly, those detestable papers then scattered over the bed, she rose directly, and folding them up as nearly as possible in the same shape as before, returned them to the same spot within the cabinet, with a very hearty wish that no untoward accident might ever bring them forward again, to disgrace her even with herself. 

 Why the locks should have been so difficult to open, however, was still something remarkable, for she could now manage them with perfect ease. In this there was surely something mysterious, and she indulged in the flattering suggestion for half a minute, till the possibility of the door's having been at first unlocked, and of being herself its fastener, darted into her head, and cost her another blush. 

 She got away as soon as she could from a room in which her conduct produced such unpleasant reflections, and found her way with all speed to the breakfast-parlour, as it had been pointed out to her by Miss Tilney the evening before. Henry was alone in it; and his immediate hope of her having been undisturbed by the tempest, with an arch reference to the character of the building they inhabited, was rather distressing. For the world would she not have her weakness suspected, and yet, unequal to an absolute falsehood, was constrained to acknowledge that the wind had kept her awake a little. <But we have a charming morning after it,> she added, desiring to get rid of the subject; <and storms and sleeplessness are nothing when they are over. What beautiful hyacinths! I have just learnt to love a hyacinth.> 

 <And how might you learn? By accident or argument?> 

 <Your sister taught me; I cannot tell how. Mrs~Allen used to take pains, year after year, to make me like them; but I never could, till I saw them the other day in Milsom Street; I am naturally indifferent about flowers.> 

 <But now you love a hyacinth. So much the better. You have gained a new source of enjoyment, and it is well to have as many holds upon happiness as possible. Besides, a taste for flowers is always desirable in your sex, as a means of getting you out of doors, and tempting you to more frequent exercise than you would otherwise take. And though the love of a hyacinth may be rather domestic, who can tell, the sentiment once raised, but you may in time come to love a rose?> 

 <But I do not want any such pursuit to get me out of doors. The pleasure of walking and breathing fresh air is enough for me, and in fine weather I am out more than half my time. Mamma says I am never within.> 

 <At any rate, however, I am pleased that you have learnt to love a hyacinth. The mere habit of learning to love is the thing; and a teachableness of disposition in a young lady is a great blessing. Has my sister a pleasant mode of instruction?> 

 Catherine was saved the embarrassment of attempting an answer by the entrance of the general, whose smiling compliments announced a happy state of mind, but whose gentle hint of sympathetic early rising did not advance her composure. 

 The elegance of the breakfast set forced itself on Catherine's notice when they were seated at table; and, luckily, it had been the General's choice. He was enchanted by her approbation of his taste, confessed it to be neat and simple, thought it right to encourage the manufacture of his country; and for his part, to his uncritical palate, the tea was as well flavoured from the clay of Staffordshire, as from that of Dresden or Sêve. But this was quite an old set, purchased two years ago. The manufacture was much improved since that time; he had seen some beautiful specimens when last in town, and had he not been perfectly without vanity of that kind, might have been tempted to order a new set. He trusted, however, that an opportunity might ere long occur of selecting one—though not for himself. Catherine was probably the only one of the party who did not understand him. 

 Shortly after breakfast Henry left them for Woodston, where business required and would keep him two or three days. They all attended in the hall to see him mount his horse, and immediately on re-entering the breakfast-room, Catherine walked to a window in the hope of catching another glimpse of his figure. <This is a somewhat heavy call upon your brother's fortitude,> observed the general to Eleanor. <Woodston will make but a sombre appearance to-day.> 

 <Is it a pretty place?> asked Catherine. 

 <What say you, Eleanor? Speak your opinion, for ladies can best tell the taste of ladies in regard to places as well as men. I think it would be acknowledged by the most impartial eye to have many recommendations. The house stands among fine meadows facing the south-east, with an excellent kitchen-garden in the same aspect; the walls surrounding which I built and stocked myself about ten years ago, for the benefit of my son. It is a family living, Miss Morland; and the property in the place being chiefly my own, you may believe I take care that it shall not be a bad one. Did Henry's income depend solely on this living, he would not be ill-provided for. Perhaps it may seem odd, that with only two younger children, I should think any profession necessary for him; and certainly there are moments when we could all wish him disengaged from every tie of business. But though I may not exactly make converts of you young ladies, I am sure your father, Miss Morland, would agree with me in thinking it expedient to give every young man some employment. The money is nothing, it is not an object, but employment is the thing. Even Frederick, my eldest son, you see, who will perhaps inherit as considerable a landed property as any private man in the county, has his profession.> 

 The imposing effect of this last argument was equal to his wishes. The silence of the lady proved it to be unanswerable. 

 Something had been said the evening before of her being shown over the house, and he now offered himself as her conductor; and though Catherine had hoped to explore it accompanied only by his daughter, it was a proposal of too much happiness in itself, under any circumstances, not to be gladly accepted; for she had been already eighteen hours in the abbey, and had seen only a few of its rooms. The netting-box, just leisurely drawn forth, was closed with joyful haste, and she was ready to attend him in a moment. <And when they had gone over the house, he promised himself moreover the pleasure of accompanying her into the shrubberies and garden.> She curtsied her acquiescence. <But perhaps it might be more agreeable to her to make those her first object. The weather was at present favourable, and at this time of year the uncertainty was very great of its continuing so. Which would she prefer? He was equally at her service. Which did his daughter think would most accord with her fair friend's wishes? But he thought he could discern. Yes, he certainly read in Miss Morland's eyes a judicious desire of making use of the present smiling weather. But when did she judge amiss? The abbey would be always safe and dry. He yielded implicitly, and would fetch his hat and attend them in a moment.> He left the room, and Catherine, with a disappointed, anxious face, began to speak of her unwillingness that he should be taking them out of doors against his own inclination, under a mistaken idea of pleasing her; but she was stopped by Miss Tilney's saying, with a little confusion, <I believe it will be wisest to take the morning while it is so fine; and do not be uneasy on my father's account; he always walks out at this time of day.> 

 Catherine did not exactly know how this was to be understood. Why was Miss Tilney embarrassed? Could there be any unwillingness on the General's side to show her over the abbey? The proposal was his own. And was not it odd that he should \textit{always} take his walk so early? Neither her father nor Mr~Allen did so. It was certainly very provoking. She was all impatience to see the house, and had scarcely any curiosity about the grounds. If Henry had been with them indeed! But now she should not know what was picturesque when she saw it. Such were her thoughts, but she kept them to herself, and put on her bonnet in patient discontent. 

 She was struck, however, beyond her expectation, by the grandeur of the abbey, as she saw it for the first time from the lawn. The whole building enclosed a large court; and two sides of the quadrangle, rich in Gothic ornaments, stood forward for admiration. The remainder was shut off by knolls of old trees, or luxuriant plantations, and the steep woody hills rising behind, to give it shelter, were beautiful even in the leafless month of March. Catherine had seen nothing to compare with it; and her feelings of delight were so strong, that without waiting for any better authority, she boldly burst forth in wonder and praise. The general listened with assenting gratitude; and it seemed as if his own estimation of Northanger had waited unfixed till that hour. 

 The kitchen-garden was to be next admired, and he led the way to it across a small portion of the park. 

 The number of acres contained in this garden was such as Catherine could not listen to without dismay, being more than double the extent of all Mr~Allen's, as well as her father's, including church-yard and orchard. The walls seemed countless in number, endless in length; a village of hot-houses seemed to arise among them, and a whole parish to be at work within the enclosure. The general was flattered by her looks of surprise, which told him almost as plainly, as he soon forced her to tell him in words, that she had never seen any gardens at all equal to them before; and he then modestly owned that, <without any ambition of that sort himself—without any solicitude about it—he did believe them to be unrivalled in the kingdom. If he had a hobby-horse, it was \textit{that}. He loved a garden. Though careless enough in most matters of eating, he loved good fruit—or if he did not, his friends and children did. There were great vexations, however, attending such a garden as his. The utmost care could not always secure the most valuable fruits. The pinery had yielded only one hundred in the last year. Mr~Allen, he supposed, must feel these inconveniences as well as himself.> 

 <No, not at all. Mr~Allen did not care about the garden, and never went into it.> 

 With a triumphant smile of self-satisfaction, the general wished he could do the same, for he never entered his, without being vexed in some way or other, by its falling short of his plan. 

 <How were Mr~Allen's succession-houses worked?> describing the nature of his own as they entered them. 

 <Mr~Allen had only one small hot-house, which Mrs~Allen had the use of for her plants in winter, and there was a fire in it now and then.> 

 <He is a happy man!> said the general, with a look of very happy contempt. 

 Having taken her into every division, and led her under every wall, till she was heartily weary of seeing and wondering, he suffered the girls at last to seize the advantage of an outer door, and then expressing his wish to examine the effect of some recent alterations about the tea-house, proposed it as no unpleasant extension of their walk, if Miss Morland were not tired. <But where are you going, Eleanor? Why do you choose that cold, damp path to it? Miss Morland will get wet. Our best way is across the park.> 

 <This is so favourite a walk of mine,> said Miss Tilney, <that I always think it the best and nearest way. But perhaps it may be damp.> 

 It was a narrow winding path through a thick grove of old Scotch firs; and Catherine, struck by its gloomy aspect, and eager to enter it, could not, even by the General's disapprobation, be kept from stepping forward. He perceived her inclination, and having again urged the plea of health in vain, was too polite to make further opposition. He excused himself, however, from attending them: <The rays of the sun were not too cheerful for him, and he would meet them by another course.> He turned away; and Catherine was shocked to find how much her spirits were relieved by the separation. The shock, however, being less real than the relief, offered it no injury; and she began to talk with easy gaiety of the delightful melancholy which such a grove inspired. 

 <I am particularly fond of this spot,> said her companion, with a sigh. <It was my mother's favourite walk.> 

 Catherine had never heard Mrs~Tilney mentioned in the family before, and the interest excited by this tender remembrance showed itself directly in her altered countenance, and in the attentive pause with which she waited for something more. 

 <I used to walk here so often with her!> added Eleanor; <though I never loved it then, as I have loved it since. At that time indeed I used to wonder at her choice. But her memory endears it now.> 

 <And ought it not,> reflected Catherine, <to endear it to her husband? Yet the general would not enter it.> Miss Tilney continuing silent, she ventured to say, <Her death must have been a great affliction!> 

 <A great and increasing one,> replied the other, in a low voice. <I was only thirteen when it happened; and though I felt my loss perhaps as strongly as one so young could feel it, I did not, I could not, then know what a loss it was.> She stopped for a moment, and then added, with great firmness, <I have no sister, you know—and though Henry—though my brothers are very affectionate, and Henry is a great deal here, which I am most thankful for, it is impossible for me not to be often solitary.> 

 <To be sure you must miss him very much.> 

 <A mother would have been always present. A mother would have been a constant friend; her influence would have been beyond all other.> 

 <Was she a very charming woman? Was she handsome? Was there any picture of her in the abbey? And why had she been so partial to that grove? Was it from dejection of spirits?>—were questions now eagerly poured forth; the first three received a ready affirmative, the two others were passed by; and Catherine's interest in the deceased Mrs~Tilney augmented with every question, whether answered or not. Of her unhappiness in marriage, she felt persuaded. The general certainly had been an unkind husband. He did not love her walk: could he therefore have loved her? And besides, handsome as he was, there was a something in the turn of his features which spoke his not having behaved well to her. 

 <Her picture, I suppose,> blushing at the consummate art of her own question, <hangs in your father's room?> 

 <No; it was intended for the drawing-room; but my father was dissatisfied with the painting, and for some time it had no place. Soon after her death I obtained it for my own, and hung it in my bed-chamber—where I shall be happy to show it you; it is very like.> Here was another proof. A portrait—very like—of a departed wife, not valued by the husband! He must have been dreadfully cruel to her! 

 Catherine attempted no longer to hide from herself the nature of the feelings which, in spite of all his attentions, he had previously excited; and what had been terror and dislike before, was now absolute aversion. Yes, aversion! His cruelty to such a charming woman made him odious to her. She had often read of such characters, characters which Mr~Allen had been used to call unnatural and overdrawn; but here was proof positive of the contrary. 

 She had just settled this point when the end of the path brought them directly upon the general; and in spite of all her virtuous indignation, she found herself again obliged to walk with him, listen to him, and even to smile when he smiled. Being no longer able, however, to receive pleasure from the surrounding objects, she soon began to walk with lassitude; the general perceived it, and with a concern for her health, which seemed to reproach her for her opinion of him, was most urgent for returning with his daughter to the house. He would follow them in a quarter of an hour. Again they parted—but Eleanor was called back in half a minute to receive a strict charge against taking her friend round the abbey till his return. This second instance of his anxiety to delay what she so much wished for struck Catherine as very remarkable. 