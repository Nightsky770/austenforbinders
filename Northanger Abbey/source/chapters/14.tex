\chapter[Chapter \thechapter]{} 

 \lettrine{T}{he} next morning was fair, and Catherine almost expected another attack from the assembled party. With Mr~Allen to support her, she felt no dread of the event: but she would gladly be spared a contest, where victory itself was painful, and was heartily rejoiced therefore at neither seeing nor hearing anything of them. The Tilneys called for her at the appointed time; and no new difficulty arising, no sudden recollection, no unexpected summons, no impertinent intrusion to disconcert their measures, my heroine was most unnaturally able to fulfil her engagement, though it was made with the hero himself. They determined on walking round Beechen Cliff, that noble hill whose beautiful verdure and hanging coppice render it so striking an object from almost every opening in Bath. 

 <I never look at it,> said Catherine, as they walked along the side of the river, <without thinking of the south of France.> 

 <You have been abroad then?> said Henry, a little surprised. 

 <Oh! No, I only mean what I have read about. It always puts me in mind of the country that Emily and her father travelled through, in \textit{The Mysteries of Udolpho}. But you never read novels, I dare say?> 

 <Why not?> 

 <Because they are not clever enough for you—gentlemen read better books.> 

 <The person, be it gentleman or lady, who has not pleasure in a good novel, must be intolerably stupid. I have read all Mrs~Radcliffe's works, and most of them with great pleasure. \textit{The Mysteries of Udolpho}, when I had once begun it, I could not lay down again; I remember finishing it in two days—my hair standing on end the whole time.> 

 <Yes,> added Miss Tilney, <and I remember that you undertook to read it aloud to me, and that when I was called away for only five minutes to answer a note, instead of waiting for me, you took the volume into the Hermitage Walk, and I was obliged to stay till you had finished it.> 

 <Thank you, Eleanor—a most honourable testimony. You see, Miss Morland, the injustice of your suspicions. Here was I, in my eagerness to get on, refusing to wait only five minutes for my sister, breaking the promise I had made of reading it aloud, and keeping her in suspense at a most interesting part, by running away with the volume, which, you are to observe, was her own, particularly her own. I am proud when I reflect on it, and I think it must establish me in your good opinion.> 

 <I am very glad to hear it indeed, and now I shall never be ashamed of liking \textit{Udolpho} myself. But I really thought before, young men despised novels amazingly.> 

 <It is \textit{amazingly;} it may well suggest \textit{amazement} if they do—for they read nearly as many as women. I myself have read hundreds and hundreds. Do not imagine that you can cope with me in a knowledge of Julias and Louisas. If we proceed to particulars, and engage in the never-ceasing inquiry of <Have you read this?> and <Have you read that?> I shall soon leave you as far behind me as—what shall I say?—I want an appropriate simile.—as far as your friend Emily herself left poor Valancourt when she went with her aunt into Italy. Consider how many years I have had the start of you. I had entered on my studies at Oxford, while you were a good little girl working your sampler at home!> 

 <Not very good, I am afraid. But now really, do not you think \textit{Udolpho} the nicest book in the world?> 

 <The nicest—by which I suppose you mean the neatest. That must depend upon the binding.> 

 <Henry,> said Miss Tilney, <you are very impertinent. Miss Morland, he is treating you exactly as he does his sister. He is forever finding fault with me, for some incorrectness of language, and now he is taking the same liberty with you. The word <nicest,> as you used it, did not suit him; and you had better change it as soon as you can, or we shall be overpowered with Johnson and Blair all the rest of the way.> 

 <I am sure,> cried Catherine, <I did not mean to say anything wrong; but it \textit{is} a nice book, and why should not I call it so?> 

 <Very true,> said Henry, <and this is a very nice day, and we are taking a very nice walk, and you are two very nice young ladies. Oh! It is a very nice word indeed! It does for everything. Originally perhaps it was applied only to express neatness, propriety, delicacy, or refinement—people were nice in their dress, in their sentiments, or their choice. But now every commendation on every subject is comprised in that one word.> 

 <While, in fact,> cried his sister, <it ought only to be applied to you, without any commendation at all. You are more nice than wise. Come, Miss Morland, let us leave him to meditate over our faults in the utmost propriety of diction, while we praise \textit{Udolpho} in whatever terms we like best. It is a most interesting work. You are fond of that kind of reading?> 

 <To say the truth, I do not much like any other.> 

 <Indeed!> 

 <That is, I can read poetry and plays, and things of that sort, and do not dislike travels. But history, real solemn history, I cannot be interested in. Can you?> 

 <Yes, I am fond of history.> 

 <I wish I were too. I read it a little as a duty, but it tells me nothing that does not either vex or weary me. The quarrels of popes and kings, with wars or pestilences, in every page; the men all so good for nothing, and hardly any women at all—it is very tiresome: and yet I often think it odd that it should be so dull, for a great deal of it must be invention. The speeches that are put into the heroes' mouths, their thoughts and designs—the chief of all this must be invention, and invention is what delights me in other books.> 

 <Historians, you think,> said Miss Tilney, <are not happy in their flights of fancy. They display imagination without raising interest. I am fond of history—and am very well contented to take the false with the true. In the principal facts they have sources of intelligence in former histories and records, which may be as much depended on, I conclude, as anything that does not actually pass under one's own observation; and as for the little embellishments you speak of, they are embellishments, and I like them as such. If a speech be well drawn up, I read it with pleasure, by whomsoever it may be made—and probably with much greater, if the production of Mr~Hume or Mr~Robertson, than if the genuine words of Caractacus, Agricola, or Alfred the Great.> 

 <You are fond of history! And so are Mr~Allen and my father; and I have two brothers who do not dislike it. So many instances within my small circle of friends is remarkable! At this rate, I shall not pity the writers of history any longer. If people like to read their books, it is all very well, but to be at so much trouble in filling great volumes, which, as I used to think, nobody would willingly ever look into, to be labouring only for the torment of little boys and girls, always struck me as a hard fate; and though I know it is all very right and necessary, I have often wondered at the person's courage that could sit down on purpose to do it.> 

 <That little boys and girls should be tormented,> said Henry, <is what no one at all acquainted with human nature in a civilized state can deny; but in behalf of our most distinguished historians, I must observe that they might well be offended at being supposed to have no higher aim, and that by their method and style, they are perfectly well qualified to torment readers of the most advanced reason and mature time of life. I use the verb <to torment,> as I observed to be your own method, instead of <to instruct,> supposing them to be now admitted as synonymous.> 

 <You think me foolish to call instruction a torment, but if you had been as much used as myself to hear poor little children first learning their letters and then learning to spell, if you had ever seen how stupid they can be for a whole morning together, and how tired my poor mother is at the end of it, as I am in the habit of seeing almost every day of my life at home, you would allow that to \textit{torment} and to \textit{instruct} might sometimes be used as synonymous words.> 

 <Very probably. But historians are not accountable for the difficulty of learning to read; and even you yourself, who do not altogether seem particularly friendly to very severe, very intense application, may perhaps be brought to acknowledge that it is very well worth-while to be tormented for two or three years of one's life, for the sake of being able to read all the rest of it. Consider—if reading had not been taught, Mrs~Radcliffe would have written in vain—or perhaps might not have written at all.> 

 Catherine assented—and a very warm panegyric from her on that lady's merits closed the subject. The Tilneys were soon engaged in another on which she had nothing to say. They were viewing the country with the eyes of persons accustomed to drawing, and decided on its capability of being formed into pictures, with all the eagerness of real taste. Here Catherine was quite lost. She knew nothing of drawing—nothing of taste: and she listened to them with an attention which brought her little profit, for they talked in phrases which conveyed scarcely any idea to her. The little which she could understand, however, appeared to contradict the very few notions she had entertained on the matter before. It seemed as if a good view were no longer to be taken from the top of an high hill, and that a clear blue sky was no longer a proof of a fine day. She was heartily ashamed of her ignorance. A misplaced shame. Where people wish to attach, they should always be ignorant. To come with a well-informed mind is to come with an inability of administering to the vanity of others, which a sensible person would always wish to avoid. A woman especially, if she have the misfortune of knowing anything, should conceal it as well as she can. 

 The advantages of natural folly in a beautiful girl have been already set forth by the capital pen of a sister author; and to her treatment of the subject I will only add, in justice to men, that though to the larger and more trifling part of the sex, imbecility in females is a great enhancement of their personal charms, there is a portion of them too reasonable and too well informed themselves to desire anything more in woman than ignorance. But Catherine did not know her own advantages—did not know that a good-looking girl, with an affectionate heart and a very ignorant mind, cannot fail of attracting a clever young man, unless circumstances are particularly untoward. In the present instance, she confessed and lamented her want of knowledge, declared that she would give anything in the world to be able to draw; and a lecture on the picturesque immediately followed, in which his instructions were so clear that she soon began to see beauty in everything admired by him, and her attention was so earnest that he became perfectly satisfied of her having a great deal of natural taste. He talked of foregrounds, distances, and second distances—side-screens and perspectives—lights and shades; and Catherine was so hopeful a scholar that when they gained the top of Beechen Cliff, she voluntarily rejected the whole city of Bath as unworthy to make part of a landscape. Delighted with her progress, and fearful of wearying her with too much wisdom at once, Henry suffered the subject to decline, and by an easy transition from a piece of rocky fragment and the withered oak which he had placed near its summit, to oaks in general, to forests, the enclosure of them, waste lands, crown lands and government, he shortly found himself arrived at politics; and from politics, it was an easy step to silence. The general pause which succeeded his short disquisition on the state of the nation was put an end to by Catherine, who, in rather a solemn tone of voice, uttered these words, <I have heard that something very shocking indeed will soon come out in London.> 

 Miss Tilney, to whom this was chiefly addressed, was startled, and hastily replied, <Indeed! And of what nature?> 

 <That I do not know, nor who is the author. I have only heard that it is to be more horrible than anything we have met with yet.> 

 <Good heaven! Where could you hear of such a thing?> 

 <A particular friend of mine had an account of it in a letter from London yesterday. It is to be uncommonly dreadful. I shall expect murder and everything of the kind.> 

 <You speak with astonishing composure! But I hope your friend's accounts have been exaggerated; and if such a design is known beforehand, proper measures will undoubtedly be taken by government to prevent its coming to effect.> 

 <Government,> said Henry, endeavouring not to smile, <neither desires nor dares to interfere in such matters. There must be murder; and government cares not how much.> 

 The ladies stared. He laughed, and added, <Come, shall I make you understand each other, or leave you to puzzle out an explanation as you can? No—I will be noble. I will prove myself a man, no less by the generosity of my soul than the clearness of my head. I have no patience with such of my sex as disdain to let themselves sometimes down to the comprehension of yours. Perhaps the abilities of women are neither sound nor acute—neither vigorous nor keen. Perhaps they may want observation, discernment, judgment, fire, genius, and wit.> 

 <Miss Morland, do not mind what he says; but have the goodness to satisfy me as to this dreadful riot.> 

 <Riot! What riot?> 

 <My dear Eleanor, the riot is only in your own brain. The confusion there is scandalous. Miss Morland has been talking of nothing more dreadful than a new publication which is shortly to come out, in three duodecimo volumes, two hundred and seventy-six pages in each, with a frontispiece to the first, of two tombstones and a lantern—do you understand? And you, Miss Morland—my stupid sister has mistaken all your clearest expressions. You talked of expected horrors in London—and instead of instantly conceiving, as any rational creature would have done, that such words could relate only to a circulating library, she immediately pictured to herself a mob of three thousand men assembling in St~George's Fields, the Bank attacked, the Tower threatened, the streets of London flowing with blood, a detachment of the Twelfth Light Dragoons (the hopes of the nation) called up from Northampton to quell the insurgents, and the gallant Captain Frederick Tilney, in the moment of charging at the head of his troop, knocked off his horse by a brickbat from an upper window. Forgive her stupidity. The fears of the sister have added to the weakness of the woman; but she is by no means a simpleton in general.> 

 Catherine looked grave. <And now, Henry,> said Miss Tilney, <that you have made us understand each other, you may as well make Miss Morland understand yourself—unless you mean to have her think you intolerably rude to your sister, and a great brute in your opinion of women in general. Miss Morland is not used to your odd ways.> 

 <I shall be most happy to make her better acquainted with them.> 

 <No doubt; but that is no explanation of the present.> 

 <What am I to do?> 

 <You know what you ought to do. Clear your character handsomely before her. Tell her that you think very highly of the understanding of women.> 

 <Miss Morland, I think very highly of the understanding of all the women in the world—especially of those—whoever they may be—with whom I happen to be in company.> 

 <That is not enough. Be more serious.> 

 <Miss Morland, no one can think more highly of the understanding of women than I do. In my opinion, nature has given them so much that they never find it necessary to use more than half.> 

 <We shall get nothing more serious from him now, Miss Morland. He is not in a sober mood. But I do assure you that he must be entirely misunderstood, if he can ever appear to say an unjust thing of any woman at all, or an unkind one of me.> 

 It was no effort to Catherine to believe that Henry Tilney could never be wrong. His manner might sometimes surprise, but his meaning must always be just: and what she did not understand, she was almost as ready to admire, as what she did. The whole walk was delightful, and though it ended too soon, its conclusion was delightful too; her friends attended her into the house, and Miss Tilney, before they parted, addressing herself with respectful form, as much to Mrs~Allen as to Catherine, petitioned for the pleasure of her company to dinner on the day after the next. No difficulty was made on Mrs~Allen's side, and the only difficulty on Catherine's was in concealing the excess of her pleasure. 

 The morning had passed away so charmingly as to banish all her friendship and natural affection, for no thought of Isabella or James had crossed her during their walk. When the Tilneys were gone, she became amiable again, but she was amiable for some time to little effect; Mrs~Allen had no intelligence to give that could relieve her anxiety; she had heard nothing of any of them. Towards the end of the morning, however, Catherine, having occasion for some indispensable yard of ribbon which must be bought without a moment's delay, walked out into the town, and in Bond Street overtook the second Miss Thorpe as she was loitering towards Edgar's Buildings between two of the sweetest girls in the world, who had been her dear friends all the morning. From her, she soon learned that the party to Clifton had taken place. <They set off at eight this morning,> said Miss Anne, <and I am sure I do not envy them their drive. I think you and I are very well off to be out of the scrape. It must be the dullest thing in the world, for there is not a soul at Clifton at this time of year. Belle went with your brother, and John drove Maria.> 

 Catherine spoke the pleasure she really felt on hearing this part of the arrangement. 

 <Oh! yes,> rejoined the other, <Maria is gone. She was quite wild to go. She thought it would be something very fine. I cannot say I admire her taste; and for my part, I was determined from the first not to go, if they pressed me ever so much.> 

 Catherine, a little doubtful of this, could not help answering, <I wish you could have gone too. It is a pity you could not all go.> 

 <Thank you; but it is quite a matter of indifference to me. Indeed, I would not have gone on any account. I was saying so to Emily and Sophia when you overtook us.> 

 Catherine was still unconvinced; but glad that Anne should have the friendship of an Emily and a Sophia to console her, she bade her adieu without much uneasiness, and returned home, pleased that the party had not been prevented by her refusing to join it, and very heartily wishing that it might be too pleasant to allow either James or Isabella to resent her resistance any longer. 