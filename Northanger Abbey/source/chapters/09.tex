\chapter[Chapter \thechapter]{} 

 \lettrine{T}{he} progress of Catherine's unhappiness from the events of the evening was as follows. It appeared first in a general dissatisfaction with everybody about her, while she remained in the rooms, which speedily brought on considerable weariness and a violent desire to go home. This, on arriving in Pulteney Street, took the direction of extraordinary hunger, and when that was appeased, changed into an earnest longing to be in bed; such was the extreme point of her distress; for when there she immediately fell into a sound sleep which lasted nine hours, and from which she awoke perfectly revived, in excellent spirits, with fresh hopes and fresh schemes. The first wish of her heart was to improve her acquaintance with Miss Tilney, and almost her first resolution, to seek her for that purpose, in the pump-room at noon. In the pump-room, one so newly arrived in Bath must be met with, and that building she had already found so favourable for the discovery of female excellence, and the completion of female intimacy, so admirably adapted for secret discourses and unlimited confidence, that she was most reasonably encouraged to expect another friend from within its walls. Her plan for the morning thus settled, she sat quietly down to her book after breakfast, resolving to remain in the same place and the same employment till the clock struck one; and from habitude very little incommoded by the remarks and ejaculations of Mrs~Allen, whose vacancy of mind and incapacity for thinking were such, that as she never talked a great deal, so she could never be entirely silent; and, therefore, while she sat at her work, if she lost her needle or broke her thread, if she heard a carriage in the street, or saw a speck upon her gown, she must observe it aloud, whether there were anyone at leisure to answer her or not. At about half past twelve, a remarkably loud rap drew her in haste to the window, and scarcely had she time to inform Catherine of there being two open carriages at the door, in the first only a servant, her brother driving Miss Thorpe in the second, before John Thorpe came running upstairs, calling out, <Well, Miss Morland, here I am. Have you been waiting long? We could not come before; the old devil of a coachmaker was such an eternity finding out a thing fit to be got into, and now it is ten thousand to one but they break down before we are out of the street. How do you do, Mrs~Allen? A famous ball last night, was not it? Come, Miss Morland, be quick, for the others are in a confounded hurry to be off. They want to get their tumble over.> 

 <What do you mean?> said Catherine. <Where are you all going to?> 

 <Going to? Why, you have not forgot our engagement! Did not we agree together to take a drive this morning? What a head you have! We are going up Claverton Down.> 

 <Something was said about it, I remember,> said Catherine, looking at Mrs~Allen for her opinion; <but really I did not expect you.> 

 <Not expect me! That's a good one! And what a dust you would have made, if I had not come.> 

 Catherine's silent appeal to her friend, meanwhile, was entirely thrown away, for Mrs~Allen, not being at all in the habit of conveying any expression herself by a look, was not aware of its being ever intended by anybody else; and Catherine, whose desire of seeing Miss Tilney again could at that moment bear a short delay in favour of a drive, and who thought there could be no impropriety in her going with Mr~Thorpe, as Isabella was going at the same time with James, was therefore obliged to speak plainer. <Well, ma'am, what do you say to it? Can you spare me for an hour or two? Shall I go?> 

 <Do just as you please, my dear,> replied Mrs~Allen, with the most placid indifference. Catherine took the advice, and ran off to get ready. In a very few minutes she reappeared, having scarcely allowed the two others time enough to get through a few short sentences in her praise, after Thorpe had procured Mrs~Allen's admiration of his gig; and then receiving her friend's parting good wishes, they both hurried downstairs. <My dearest creature,> cried Isabella, to whom the duty of friendship immediately called her before she could get into the carriage, <you have been at least three hours getting ready. I was afraid you were ill. What a delightful ball we had last night. I have a thousand things to say to you; but make haste and get in, for I long to be off.> 

 Catherine followed her orders and turned away, but not too soon to hear her friend exclaim aloud to James, <What a sweet girl she is! I quite dote on her.> 

 <You will not be frightened, Miss Morland,> said Thorpe, as he handed her in, <if my horse should dance about a little at first setting off. He will, most likely, give a plunge or two, and perhaps take the rest for a minute; but he will soon know his master. He is full of spirits, playful as can be, but there is no vice in him.> 

 Catherine did not think the portrait a very inviting one, but it was too late to retreat, and she was too young to own herself frightened; so, resigning herself to her fate, and trusting to the animal's boasted knowledge of its owner, she sat peaceably down, and saw Thorpe sit down by her. Everything being then arranged, the servant who stood at the horse's head was bid in an important voice <to let him go,> and off they went in the quietest manner imaginable, without a plunge or a caper, or anything like one. Catherine, delighted at so happy an escape, spoke her pleasure aloud with grateful surprise; and her companion immediately made the matter perfectly simple by assuring her that it was entirely owing to the peculiarly judicious manner in which he had then held the reins, and the singular discernment and dexterity with which he had directed his whip. Catherine, though she could not help wondering that with such perfect command of his horse, he should think it necessary to alarm her with a relation of its tricks, congratulated herself sincerely on being under the care of so excellent a coachman; and perceiving that the animal continued to go on in the same quiet manner, without showing the smallest propensity towards any unpleasant vivacity, and (considering its inevitable pace was ten miles an hour) by no means alarmingly fast, gave herself up to all the enjoyment of air and exercise of the most invigorating kind, in a fine mild day of February, with the consciousness of safety. A silence of several minutes succeeded their first short dialogue; it was broken by Thorpe's saying very abruptly, <Old Allen is as rich as a Jew—is not he?> Catherine did not understand him—and he repeated his question, adding in explanation, <Old Allen, the man you are with.> 

 <Oh! Mr~Allen, you mean. Yes, I believe, he is very rich.> 

 <And no children at all?> 

 <No—not any.> 

 <A famous thing for his next heirs. He is \textit{your} godfather, is not he?> 

 <My godfather! No.> 

 <But you are always very much with them.> 

 <Yes, very much.> 

 <Aye, that is what I meant. He seems a good kind of old fellow enough, and has lived very well in his time, I dare say; he is not gouty for nothing. Does he drink his bottle a day now?> 

 <His bottle a day! No. Why should you think of such a thing? He is a very temperate man, and you could not fancy him in liquor last night?> 

 <Lord help you! You women are always thinking of men's being in liquor. Why, you do not suppose a man is overset by a bottle? I am sure of \textit{this}—that if everybody was to drink their bottle a day, there would not be half the disorders in the world there are now. It would be a famous good thing for us all.> 

 <I cannot believe it.> 

 <Oh! Lord, it would be the saving of thousands. There is not the hundredth part of the wine consumed in this kingdom that there ought to be. Our foggy climate wants help.> 

 <And yet I have heard that there is a great deal of wine drunk in Oxford.> 

 <Oxford! There is no drinking at Oxford now, I assure you. Nobody drinks there. You would hardly meet with a man who goes beyond his four pints at the utmost. Now, for instance, it was reckoned a remarkable thing, at the last party in my rooms, that upon an average we cleared about five pints a head. It was looked upon as something out of the common way. \textit{Mine} is famous good stuff, to be sure. You would not often meet with anything like it in Oxford—and that may account for it. But this will just give you a notion of the general rate of drinking there.> 

 <Yes, it does give a notion,> said Catherine warmly, <and that is, that you all drink a great deal more wine than I thought you did. However, I am sure James does not drink so much.> 

 This declaration brought on a loud and overpowering reply, of which no part was very distinct, except the frequent exclamations, amounting almost to oaths, which adorned it, and Catherine was left, when it ended, with rather a strengthened belief of there being a great deal of wine drunk in Oxford, and the same happy conviction of her brother's comparative sobriety. 

 Thorpe's ideas then all reverted to the merits of his own equipage, and she was called on to admire the spirit and freedom with which his horse moved along, and the ease which his paces, as well as the excellence of the springs, gave the motion of the carriage. She followed him in all his admiration as well as she could. To go before or beyond him was impossible. His knowledge and her ignorance of the subject, his rapidity of expression, and her diffidence of herself put that out of her power; she could strike out nothing new in commendation, but she readily echoed whatever he chose to assert, and it was finally settled between them without any difficulty that his equipage was altogether the most complete of its kind in England, his carriage the neatest, his horse the best goer, and himself the best coachman. <You do not really think, Mr~Thorpe,> said Catherine, venturing after some time to consider the matter as entirely decided, and to offer some little variation on the subject, <that James's gig will break down?> 

 <Break down! Oh, lord! Did you ever see such a little tittuppy thing in your life? There is not a sound piece of iron about it. The wheels have been fairly worn out these ten years at least—and as for the body! Upon my soul, you might shake it to pieces yourself with a touch. It is the most devilish little rickety business I ever beheld! Thank God! we have got a better. I would not be bound to go two miles in it for fifty thousand pounds.> 

 <Good heavens!> cried Catherine, quite frightened. <Then pray let us turn back; they will certainly meet with an accident if we go on. Do let us turn back, Mr~Thorpe; stop and speak to my brother, and tell him how very unsafe it is.> 

 <Unsafe! Oh, lord! What is there in that? They will only get a roll if it does break down; and there is plenty of dirt; it will be excellent falling. Oh, curse it! The carriage is safe enough, if a man knows how to drive it; a thing of that sort in good hands will last above twenty years after it is fairly worn out. Lord bless you! I would undertake for five pounds to drive it to York and back again, without losing a nail.> 

 Catherine listened with astonishment; she knew not how to reconcile two such very different accounts of the same thing; for she had not been brought up to understand the propensities of a rattle, nor to know to how many idle assertions and impudent falsehoods the excess of vanity will lead. Her own family were plain, matter-of-fact people who seldom aimed at wit of any kind; her father, at the utmost, being contented with a pun, and her mother with a proverb; they were not in the habit therefore of telling lies to increase their importance, or of asserting at one moment what they would contradict the next. She reflected on the affair for some time in much perplexity, and was more than once on the point of requesting from Mr~Thorpe a clearer insight into his real opinion on the subject; but she checked herself, because it appeared to her that he did not excel in giving those clearer insights, in making those things plain which he had before made ambiguous; and, joining to this, the consideration that he would not really suffer his sister and his friend to be exposed to a danger from which he might easily preserve them, she concluded at last that he must know the carriage to be in fact perfectly safe, and therefore would alarm herself no longer. By him the whole matter seemed entirely forgotten; and all the rest of his conversation, or rather talk, began and ended with himself and his own concerns. He told her of horses which he had bought for a trifle and sold for incredible sums; of racing matches, in which his judgment had infallibly foretold the winner; of shooting parties, in which he had killed more birds (though without having one good shot) than all his companions together; and described to her some famous day's sport, with the fox-hounds, in which his foresight and skill in directing the dogs had repaired the mistakes of the most experienced huntsman, and in which the boldness of his riding, though it had never endangered his own life for a moment, had been constantly leading others into difficulties, which he calmly concluded had broken the necks of many. 

 Little as Catherine was in the habit of judging for herself, and unfixed as were her general notions of what men ought to be, she could not entirely repress a doubt, while she bore with the effusions of his endless conceit, of his being altogether completely agreeable. It was a bold surmise, for he was Isabella's brother; and she had been assured by James that his manners would recommend him to all her sex; but in spite of this, the extreme weariness of his company, which crept over her before they had been out an hour, and which continued unceasingly to increase till they stopped in Pulteney Street again, induced her, in some small degree, to resist such high authority, and to distrust his powers of giving universal pleasure. 

 When they arrived at Mrs~Allen's door, the astonishment of Isabella was hardly to be expressed, on finding that it was too late in the day for them to attend her friend into the house: <Past three o'clock!> It was inconceivable, incredible, impossible! And she would neither believe her own watch, nor her brother's, nor the servant's; she would believe no assurance of it founded on reason or reality, till Morland produced his watch, and ascertained the fact; to have doubted a moment longer \textit{then}, would have been equally inconceivable, incredible, and impossible; and she could only protest, over and over again, that no two hours and a half had ever gone off so swiftly before, as Catherine was called on to confirm; Catherine could not tell a falsehood even to please Isabella; but the latter was spared the misery of her friend's dissenting voice, by not waiting for her answer. Her own feelings entirely engrossed her; her wretchedness was most acute on finding herself obliged to go directly home. It was ages since she had had a moment's conversation with her dearest Catherine; and, though she had such thousands of things to say to her, it appeared as if they were never to be together again; so, with smiles of most exquisite misery, and the laughing eye of utter despondency, she bade her friend adieu and went on. 

 Catherine found Mrs~Allen just returned from all the busy idleness of the morning, and was immediately greeted with, <Well, my dear, here you are,> a truth which she had no greater inclination than power to dispute; <and I hope you have had a pleasant airing?> 

 <Yes, ma'am, I thank you; we could not have had a nicer day.> 

 <So Mrs~Thorpe said; she was vastly pleased at your all going.> 

 <You have seen Mrs~Thorpe, then?> 

 <Yes, I went to the pump-room as soon as you were gone, and there I met her, and we had a great deal of talk together. She says there was hardly any veal to be got at market this morning, it is so uncommonly scarce.> 

 <Did you see anybody else of our acquaintance?> 

 <Yes; we agreed to take a turn in the Crescent, and there we met Mrs~Hughes, and Mr~and Miss Tilney walking with her.> 

 <Did you indeed? And did they speak to you?> 

 <Yes, we walked along the Crescent together for half an hour. They seem very agreeable people. Miss Tilney was in a very pretty spotted muslin, and I fancy, by what I can learn, that she always dresses very handsomely. Mrs~Hughes talked to me a great deal about the family.> 

 <And what did she tell you of them?> 

 <Oh! A vast deal indeed; she hardly talked of anything else.> 

 <Did she tell you what part of Gloucestershire they come from?> 

 <Yes, she did; but I cannot recollect now. But they are very good kind of people, and very rich. Mrs~Tilney was a Miss Drummond, and she and Mrs~Hughes were schoolfellows; and Miss Drummond had a very large fortune; and, when she married, her father gave her twenty thousand pounds, and five hundred to buy wedding-clothes. Mrs~Hughes saw all the clothes after they came from the warehouse.> 

 <And are Mr~and Mrs~Tilney in Bath?> 

 <Yes, I fancy they are, but I am not quite certain. Upon recollection, however, I have a notion they are both dead; at least the mother is; yes, I am sure Mrs~Tilney is dead, because Mrs~Hughes told me there was a very beautiful set of pearls that Mr~Drummond gave his daughter on her wedding-day and that Miss Tilney has got now, for they were put by for her when her mother died.> 

 <And is Mr~Tilney, my partner, the only son?> 

 <I cannot be quite positive about that, my dear; I have some idea he is; but, however, he is a very fine young man, Mrs~Hughes says, and likely to do very well.> 

 Catherine inquired no further; she had heard enough to feel that Mrs~Allen had no real intelligence to give, and that she was most particularly unfortunate herself in having missed such a meeting with both brother and sister. Could she have foreseen such a circumstance, nothing should have persuaded her to go out with the others; and, as it was, she could only lament her ill luck, and think over what she had lost, till it was clear to her that the drive had by no means been very pleasant and that John Thorpe himself was quite disagreeable. 