\chapter[Chapter \thechapter]{} 

 \lettrine{T}{he} Allens had now entered on the sixth week of their stay in Bath; and whether it should be the last was for some time a question, to which Catherine listened with a beating heart. To have her acquaintance with the Tilneys end so soon was an evil which nothing could counterbalance. Her whole happiness seemed at stake, while the affair was in suspense, and everything secured when it was determined that the lodgings should be taken for another fortnight. What this additional fortnight was to produce to her beyond the pleasure of sometimes seeing Henry Tilney made but a small part of Catherine's speculation. Once or twice indeed, since James's engagement had taught her what \textit{could} be done, she had got so far as to indulge in a secret <perhaps,> but in general the felicity of being with him for the present bounded her views: the present was now comprised in another three weeks, and her happiness being certain for that period, the rest of her life was at such a distance as to excite but little interest. In the course of the morning which saw this business arranged, she visited Miss Tilney, and poured forth her joyful feelings. It was doomed to be a day of trial. No sooner had she expressed her delight in Mr~Allen's lengthened stay than Miss Tilney told her of her father's having just determined upon quitting Bath by the end of another week. Here was a blow! The past suspense of the morning had been ease and quiet to the present disappointment. Catherine's countenance fell, and in a voice of most sincere concern she echoed Miss Tilney's concluding words, <By the end of another week!> 

 <Yes, my father can seldom be prevailed on to give the waters what I think a fair trial. He has been disappointed of some friends' arrival whom he expected to meet here, and as he is now pretty well, is in a hurry to get home.> 

 <I am very sorry for it,> said Catherine dejectedly; <if I had known this before\longdash> 

 <Perhaps,> said Miss Tilney in an embarrassed manner, <you would be so good—it would make me very happy if\longdash> 

 The entrance of her father put a stop to the civility, which Catherine was beginning to hope might introduce a desire of their corresponding. After addressing her with his usual politeness, he turned to his daughter and said, <Well, Eleanor, may I congratulate you on being successful in your application to your fair friend?> 

 <I was just beginning to make the request, sir, as you came in.> 

 <Well, proceed by all means. I know how much your heart is in it. My daughter, Miss Morland,> he continued, without leaving his daughter time to speak, <has been forming a very bold wish. We leave Bath, as she has perhaps told you, on Saturday se'nnight. A letter from my steward tells me that my presence is wanted at home; and being disappointed in my hope of seeing the Marquis of Longtown and General Courteney here, some of my very old friends, there is nothing to detain me longer in Bath. And could we carry our selfish point with you, we should leave it without a single regret. Can you, in short, be prevailed on to quit this scene of public triumph and oblige your friend Eleanor with your company in Gloucestershire? I am almost ashamed to make the request, though its presumption would certainly appear greater to every creature in Bath than yourself. Modesty such as yours—but not for the world would I pain it by open praise. If you can be induced to honour us with a visit, you will make us happy beyond expression. 'Tis true, we can offer you nothing like the gaieties of this lively place; we can tempt you neither by amusement nor splendour, for our mode of living, as you see, is plain and unpretending; yet no endeavours shall be wanting on our side to make Northanger Abbey not wholly disagreeable.> 

 Northanger Abbey! These were thrilling words, and wound up Catherine's feelings to the highest point of ecstasy. Her grateful and gratified heart could hardly restrain its expressions within the language of tolerable calmness. To receive so flattering an invitation! To have her company so warmly solicited! Everything honourable and soothing, every present enjoyment, and every future hope was contained in it; and her acceptance, with only the saving clause of Papa and Mamma's approbation, was eagerly given. <I will write home directly,> said she, <and if they do not object, as I dare say they will not\longdash> 

 General Tilney was not less sanguine, having already waited on her excellent friends in Pulteney Street, and obtained their sanction of his wishes. <Since they can consent to part with you,> said he, <we may expect philosophy from all the world.> 

 Miss Tilney was earnest, though gentle, in her secondary civilities, and the affair became in a few minutes as nearly settled as this necessary reference to Fullerton would allow. 

 The circumstances of the morning had led Catherine's feelings through the varieties of suspense, security, and disappointment; but they were now safely lodged in perfect bliss; and with spirits elated to rapture, with Henry at her heart, and Northanger Abbey on her lips, she hurried home to write her letter. Mr~and Mrs~Morland, relying on the discretion of the friends to whom they had already entrusted their daughter, felt no doubt of the propriety of an acquaintance which had been formed under their eye, and sent therefore by return of post their ready consent to her visit in Gloucestershire. This indulgence, though not more than Catherine had hoped for, completed her conviction of being favoured beyond every other human creature, in friends and fortune, circumstance and chance. Everything seemed to cooperate for her advantage. By the kindness of her first friends, the Allens, she had been introduced into scenes where pleasures of every kind had met her. Her feelings, her preferences, had each known the happiness of a return. Wherever she felt attachment, she had been able to create it. The affection of Isabella was to be secured to her in a sister. The Tilneys, they, by whom, above all, she desired to be favourably thought of, outstripped even her wishes in the flattering measures by which their intimacy was to be continued. She was to be their chosen visitor, she was to be for weeks under the same roof with the person whose society she mostly prized—and, in addition to all the rest, this roof was to be the roof of an abbey! Her passion for ancient edifices was next in degree to her passion for Henry Tilney—and castles and abbeys made usually the charm of those reveries which his image did not fill. To see and explore either the ramparts and keep of the one, or the cloisters of the other, had been for many weeks a darling wish, though to be more than the visitor of an hour had seemed too nearly impossible for desire. And yet, this was to happen. With all the chances against her of house, hall, place, park, court, and cottage, Northanger turned up an abbey, and she was to be its inhabitant. Its long, damp passages, its narrow cells and ruined chapel, were to be within her daily reach, and she could not entirely subdue the hope of some traditional legends, some awful memorials of an injured and ill-fated nun. 

 It was wonderful that her friends should seem so little elated by the possession of such a home, that the consciousness of it should be so meekly borne. The power of early habit only could account for it. A distinction to which they had been born gave no pride. Their superiority of abode was no more to them than their superiority of person. 

 Many were the inquiries she was eager to make of Miss Tilney; but so active were her thoughts, that when these inquiries were answered, she was hardly more assured than before, of Northanger Abbey having been a richly endowed convent at the time of the Reformation, of its having fallen into the hands of an ancestor of the Tilneys on its dissolution, of a large portion of the ancient building still making a part of the present dwelling although the rest was decayed, or of its standing low in a valley, sheltered from the north and east by rising woods of oak. 