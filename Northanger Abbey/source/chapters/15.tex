\chapter[Chapter \thechapter]{} 

 \lettrine{E}{arly} the next day, a note from Isabella, speaking peace and tenderness in every line, and entreating the immediate presence of her friend on a matter of the utmost importance, hastened Catherine, in the happiest state of confidence and curiosity, to Edgar's Buildings. The two youngest Miss Thorpes were by themselves in the parlour; and, on Anne's quitting it to call her sister, Catherine took the opportunity of asking the other for some particulars of their yesterday's party. Maria desired no greater pleasure than to speak of it; and Catherine immediately learnt that it had been altogether the most delightful scheme in the world, that nobody could imagine how charming it had been, and that it had been more delightful than anybody could conceive. Such was the information of the first five minutes; the second unfolded thus much in detail—that they had driven directly to the York Hotel, ate some soup, and bespoke an early dinner, walked down to the pump-room, tasted the water, and laid out some shillings in purses and spars; thence adjourned to eat ice at a pastry-cook's, and hurrying back to the hotel, swallowed their dinner in haste, to prevent being in the dark; and then had a delightful drive back, only the moon was not up, and it rained a little, and Mr~Morland's horse was so tired he could hardly get it along. 

 Catherine listened with heartfelt satisfaction. It appeared that Blaize Castle had never been thought of; and, as for all the rest, there was nothing to regret for half an instant. Maria's intelligence concluded with a tender effusion of pity for her sister Anne, whom she represented as insupportably cross, from being excluded the party. 

 <She will never forgive me, I am sure; but, you know, how could I help it? John would have me go, for he vowed he would not drive her, because she had such thick ankles. I dare say she will not be in good humour again this month; but I am determined I will not be cross; it is not a little matter that puts me out of temper.> 

 Isabella now entered the room with so eager a step, and a look of such happy importance, as engaged all her friend's notice. Maria was without ceremony sent away, and Isabella, embracing Catherine, thus began: <Yes, my dear Catherine, it is so indeed; your penetration has not deceived you. Oh, that arch eye of yours! It sees through everything.> 

 Catherine replied only by a look of wondering ignorance. 

 <Nay, my beloved, sweetest friend,> continued the other, <compose yourself. I am amazingly agitated, as you perceive. Let us sit down and talk in comfort. Well, and so you guessed it the moment you had my note? Sly creature! Oh! My dear Catherine, you alone, who know my heart, can judge of my present happiness. Your brother is the most charming of men. I only wish I were more worthy of him. But what will your excellent father and mother say? Oh! Heavens! When I think of them I am so agitated!> 

 Catherine's understanding began to awake: an idea of the truth suddenly darted into her mind; and, with the natural blush of so new an emotion, she cried out, <Good heaven! My dear Isabella, what do you mean? Can you—can you really be in love with James?> 

 This bold surmise, however, she soon learnt comprehended but half the fact. The anxious affection, which she was accused of having continually watched in Isabella's every look and action, had, in the course of their yesterday's party, received the delightful confession of an equal love. Her heart and faith were alike engaged to James. Never had Catherine listened to anything so full of interest, wonder, and joy. Her brother and her friend engaged! New to such circumstances, the importance of it appeared unspeakably great, and she contemplated it as one of those grand events, of which the ordinary course of life can hardly afford a return. The strength of her feelings she could not express; the nature of them, however, contented her friend. The happiness of having such a sister was their first effusion, and the fair ladies mingled in embraces and tears of joy. 

 Delighting, however, as Catherine sincerely did, in the prospect of the connection, it must be acknowledged that Isabella far surpassed her in tender anticipations. <You will be so infinitely dearer to me, my Catherine, than either Anne or Maria: I feel that I shall be so much more attached to my dear Morland's family than to my own.> 

 This was a pitch of friendship beyond Catherine. 

 <You are so like your dear brother,> continued Isabella, <that I quite doted on you the first moment I saw you. But so it always is with me; the first moment settles everything. The very first day that Morland came to us last Christmas—the very first moment I beheld him—my heart was irrecoverably gone. I remember I wore my yellow gown, with my hair done up in braids; and when I came into the drawing-room, and John introduced him, I thought I never saw anybody so handsome before.> 

 Here Catherine secretly acknowledged the power of love; for, though exceedingly fond of her brother, and partial to all his endowments, she had never in her life thought him handsome. 

 <I remember too, Miss Andrews drank tea with us that evening, and wore her puce-coloured sarsenet; and she looked so heavenly that I thought your brother must certainly fall in love with her; I could not sleep a wink all night for thinking of it. Oh! Catherine, the many sleepless nights I have had on your brother's account! I would not have you suffer half what I have done! I am grown wretchedly thin, I know; but I will not pain you by describing my anxiety; you have seen enough of it. I feel that I have betrayed myself perpetually—so unguarded in speaking of my partiality for the church! But my secret I was always sure would be safe with \textit{you}.> 

 Catherine felt that nothing could have been safer; but ashamed of an ignorance little expected, she dared no longer contest the point, nor refuse to have been as full of arch penetration and affectionate sympathy as Isabella chose to consider her. Her brother, she found, was preparing to set off with all speed to Fullerton, to make known his situation and ask consent; and here was a source of some real agitation to the mind of Isabella. Catherine endeavoured to persuade her, as she was herself persuaded, that her father and mother would never oppose their son's wishes. <It is impossible,> said she, <for parents to be more kind, or more desirous of their children's happiness; I have no doubt of their consenting immediately.> 

 <Morland says exactly the same,> replied Isabella; <and yet I dare not expect it; my fortune will be so small; they never can consent to it. Your brother, who might marry anybody!> 

 Here Catherine again discerned the force of love. 

 <Indeed, Isabella, you are too humble. The difference of fortune can be nothing to signify.> 

 <Oh! My sweet Catherine, in \textit{your} generous heart I know it would signify nothing; but we must not expect such disinterestedness in many. As for myself, I am sure I only wish our situations were reversed. Had I the command of millions, were I mistress of the whole world, your brother would be my only choice.> 

 This charming sentiment, recommended as much by sense as novelty, gave Catherine a most pleasing remembrance of all the heroines of her acquaintance; and she thought her friend never looked more lovely than in uttering the grand idea. <I am sure they will consent,> was her frequent declaration; <I am sure they will be delighted with you.> 

 <For my own part,> said Isabella, <my wishes are so moderate that the smallest income in nature would be enough for me. Where people are really attached, poverty itself is wealth; grandeur I detest: I would not settle in London for the universe. A cottage in some retired village would be ecstasy. There are some charming little villas about Richmond.> 

 <Richmond!> cried Catherine. <You must settle near Fullerton. You must be near us.> 

 <I am sure I shall be miserable if we do not. If I can but be near \textit{you}, I shall be satisfied. But this is idle talking! I will not allow myself to think of such things, till we have your father's answer. Morland says that by sending it to-night to Salisbury, we may have it to-morrow. To-morrow? I know I shall never have courage to open the letter. I know it will be the death of me.> 

 A reverie succeeded this conviction—and when Isabella spoke again, it was to resolve on the quality of her wedding-gown. 

 Their conference was put an end to by the anxious young lover himself, who came to breathe his parting sigh before he set off for Wiltshire. Catherine wished to congratulate him, but knew not what to say, and her eloquence was only in her eyes. From them, however, the eight parts of speech shone out most expressively, and James could combine them with ease. Impatient for the realization of all that he hoped at home, his adieus were not long; and they would have been yet shorter, had he not been frequently detained by the urgent entreaties of his fair one that he would go. Twice was he called almost from the door by her eagerness to have him gone. <Indeed, Morland, I must drive you away. Consider how far you have to ride. I cannot bear to see you linger so. For heaven's sake, waste no more time. There, go, go—I insist on it.> 

 The two friends, with hearts now more united than ever, were inseparable for the day; and in schemes of sisterly happiness the hours flew along. Mrs~Thorpe and her son, who were acquainted with everything, and who seemed only to want Mr~Morland's consent, to consider Isabella's engagement as the most fortunate circumstance imaginable for their family, were allowed to join their counsels, and add their quota of significant looks and mysterious expressions to fill up the measure of curiosity to be raised in the unprivileged younger sisters. To Catherine's simple feelings, this odd sort of reserve seemed neither kindly meant, nor consistently supported; and its unkindness she would hardly have forborne pointing out, had its inconsistency been less their friend; but Anne and Maria soon set her heart at ease by the sagacity of their <I know what>; and the evening was spent in a sort of war of wit, a display of family ingenuity, on one side in the mystery of an affected secret, on the other of undefined discovery, all equally acute. 

 Catherine was with her friend again the next day, endeavouring to support her spirits and while away the many tedious hours before the delivery of the letters; a needful exertion, for as the time of reasonable expectation drew near, Isabella became more and more desponding, and before the letter arrived, had worked herself into a state of real distress. But when it did come, where could distress be found? <I have had no difficulty in gaining the consent of my kind parents, and am promised that everything in their power shall be done to forward my happiness,> were the first three lines, and in one moment all was joyful security. The brightest glow was instantly spread over Isabella's features, all care and anxiety seemed removed, her spirits became almost too high for control, and she called herself without scruple the happiest of mortals. 

 Mrs~Thorpe, with tears of joy, embraced her daughter, her son, her visitor, and could have embraced half the inhabitants of Bath with satisfaction. Her heart was overflowing with tenderness. It was <dear John> and <dear Catherine> at every word; <dear Anne and dear Maria> must immediately be made sharers in their felicity; and two <dears> at once before the name of Isabella were not more than that beloved child had now well earned. John himself was no skulker in joy. He not only bestowed on Mr~Morland the high commendation of being one of the finest fellows in the world, but swore off many sentences in his praise. 

 The letter, whence sprang all this felicity, was short, containing little more than this assurance of success; and every particular was deferred till James could write again. But for particulars Isabella could well afford to wait. The needful was comprised in Mr~Morland's promise; his honour was pledged to make everything easy; and by what means their income was to be formed, whether landed property were to be resigned, or funded money made over, was a matter in which her disinterested spirit took no concern. She knew enough to feel secure of an honourable and speedy establishment, and her imagination took a rapid flight over its attendant felicities. She saw herself at the end of a few weeks, the gaze and admiration of every new acquaintance at Fullerton, the envy of every valued old friend in Putney, with a carriage at her command, a new name on her tickets, and a brilliant exhibition of hoop rings on her finger. 

 When the contents of the letter were ascertained, John Thorpe, who had only waited its arrival to begin his journey to London, prepared to set off. <Well, Miss Morland,> said he, on finding her alone in the parlour, <I am come to bid you good-bye.> Catherine wished him a good journey. Without appearing to hear her, he walked to the window, fidgeted about, hummed a tune, and seemed wholly self-occupied. 

 <Shall not you be late at Devizes?> said Catherine. He made no answer; but after a minute's silence burst out with, <A famous good thing this marrying scheme, upon my soul! A clever fancy of Morland's and Belle's. What do you think of it, Miss Morland? \textit{I} say it is no bad notion.> 

 <I am sure I think it a very good one.> 

 <Do you? That's honest, by heavens! I am glad you are no enemy to matrimony, however. Did you ever hear the old song, <Going to One Wedding Brings on Another>? I say, you will come to Belle's wedding, I hope.> 

 <Yes; I have promised your sister to be with her, if possible.> 

 <And then you know>—twisting himself about and forcing a foolish laugh—<I say, then you know, we may try the truth of this same old song.> 

 <May we? But I never sing. Well, I wish you a good journey. I dine with Miss Tilney to-day, and must now be going home.> 

 <Nay, but there is no such confounded hurry. Who knows when we may be together again? Not but that I shall be down again by the end of a fortnight, and a devilish long fortnight it will appear to me.> 

 <Then why do you stay away so long?> replied Catherine—finding that he waited for an answer. 

 <That is kind of you, however—kind and good-natured. I shall not forget it in a hurry. But you have more good nature and all that, than anybody living, I believe. A monstrous deal of good nature, and it is not only good nature, but you have so much, so much of everything; and then you have such—upon my soul, I do not know anybody like you.> 

 <Oh! dear, there are a great many people like me, I dare say, only a great deal better. Good morning to you.> 

 <But I say, Miss Morland, I shall come and pay my respects at Fullerton before it is long, if not disagreeable.> 

 <Pray do. My father and mother will be very glad to see you.> 

 <And I hope—I hope, Miss Morland, \textit{you} will not be sorry to see me.> 

 <Oh! dear, not at all. There are very few people I am sorry to see. Company is always cheerful.> 

 <That is just my way of thinking. Give me but a little cheerful company, let me only have the company of the people I love, let me only be where I like and with whom I like, and the devil take the rest, say I. And I am heartily glad to hear you say the same. But I have a notion, Miss Morland, you and I think pretty much alike upon most matters.> 

 <Perhaps we may; but it is more than I ever thought of. And as to \textit{most matters}, to say the truth, there are not many that I know my own mind about.> 

 <By Jove, no more do I. It is not my way to bother my brains with what does not concern me. My notion of things is simple enough. Let me only have the girl I like, say I, with a comfortable house over my head, and what care I for all the rest? Fortune is nothing. I am sure of a good income of my own; and if she had not a penny, why, so much the better.> 

 <Very true. I think like you there. If there is a good fortune on one side, there can be no occasion for any on the other. No matter which has it, so that there is enough. I hate the idea of one great fortune looking out for another. And to marry for money I think the wickedest thing in existence. Good day. We shall be very glad to see you at Fullerton, whenever it is convenient.> And away she went. It was not in the power of all his gallantry to detain her longer. With such news to communicate, and such a visit to prepare for, her departure was not to be delayed by anything in his nature to urge; and she hurried away, leaving him to the undivided consciousness of his own happy address, and her explicit encouragement. 

 The agitation which she had herself experienced on first learning her brother's engagement made her expect to raise no inconsiderable emotion in Mr~and Mrs~Allen, by the communication of the wonderful event. How great was her disappointment! The important affair, which many words of preparation ushered in, had been foreseen by them both ever since her brother's arrival; and all that they felt on the occasion was comprehended in a wish for the young people's happiness, with a remark, on the gentleman's side, in favour of Isabella's beauty, and on the lady's, of her great good luck. It was to Catherine the most surprising insensibility. The disclosure, however, of the great secret of James's going to Fullerton the day before, did raise some emotion in Mrs~Allen. She could not listen to that with perfect calmness, but repeatedly regretted the necessity of its concealment, wished she could have known his intention, wished she could have seen him before he went, as she should certainly have troubled him with her best regards to his father and mother, and her kind compliments to all the Skinners. 