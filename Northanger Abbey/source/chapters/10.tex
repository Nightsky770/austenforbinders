\chapter[Chapter \thechapter]{} 

 \lettrine{T}{he} Allens, Thorpes, and Morlands all met in the evening at the theatre; and, as Catherine and Isabella sat together, there was then an opportunity for the latter to utter some few of the many thousand things which had been collecting within her for communication in the immeasurable length of time which had divided them. <Oh, heavens! My beloved Catherine, have I got you at last?> was her address on Catherine's entering the box and sitting by her. <Now, Mr~Morland,> for he was close to her on the other side, <I shall not speak another word to you all the rest of the evening; so I charge you not to expect it. My sweetest Catherine, how have you been this long age? But I need not ask you, for you look delightfully. You really have done your hair in a more heavenly style than ever; you mischievous creature, do you want to attract everybody? I assure you, my brother is quite in love with you already; and as for Mr~Tilney—but \textit{that} is a settled thing—even \textit{your} modesty cannot doubt his attachment now; his coming back to Bath makes it too plain. Oh! What would not I give to see him! I really am quite wild with impatience. My mother says he is the most delightful young man in the world; she saw him this morning, you know; you must introduce him to me. Is he in the house now? Look about, for heaven's sake! I assure you, I can hardly exist till I see him.> 

 <No,> said Catherine, <he is not here; I cannot see him anywhere.> 

 <Oh, horrid! Am I never to be acquainted with him? How do you like my gown? I think it does not look amiss; the sleeves were entirely my own thought. Do you know, I get so immoderately sick of Bath; your brother and I were agreeing this morning that, though it is vastly well to be here for a few weeks, we would not live here for millions. We soon found out that our tastes were exactly alike in preferring the country to every other place; really, our opinions were so exactly the same, it was quite ridiculous! There was not a single point in which we differed; I would not have had you by for the world; you are such a sly thing, I am sure you would have made some droll remark or other about it.> 

 <No, indeed I should not.> 

 <Oh, yes you would indeed; I know you better than you know yourself. You would have told us that we seemed born for each other, or some nonsense of that kind, which would have distressed me beyond conception; my cheeks would have been as red as your roses; I would not have had you by for the world.> 

 <Indeed you do me injustice; I would not have made so improper a remark upon any account; and besides, I am sure it would never have entered my head.> 

 Isabella smiled incredulously and talked the rest of the evening to James. 

 Catherine's resolution of endeavouring to meet Miss Tilney again continued in full force the next morning; and till the usual moment of going to the pump-room, she felt some alarm from the dread of a second prevention. But nothing of that kind occurred, no visitors appeared to delay them, and they all three set off in good time for the pump-room, where the ordinary course of events and conversation took place; Mr~Allen, after drinking his glass of water, joined some gentlemen to talk over the politics of the day and compare the accounts of their newspapers; and the ladies walked about together, noticing every new face, and almost every new bonnet in the room. The female part of the Thorpe family, attended by James Morland, appeared among the crowd in less than a quarter of an hour, and Catherine immediately took her usual place by the side of her friend. James, who was now in constant attendance, maintained a similar position, and separating themselves from the rest of their party, they walked in that manner for some time, till Catherine began to doubt the happiness of a situation which, confining her entirely to her friend and brother, gave her very little share in the notice of either. They were always engaged in some sentimental discussion or lively dispute, but their sentiment was conveyed in such whispering voices, and their vivacity attended with so much laughter, that though Catherine's supporting opinion was not unfrequently called for by one or the other, she was never able to give any, from not having heard a word of the subject. At length however she was empowered to disengage herself from her friend, by the avowed necessity of speaking to Miss Tilney, whom she most joyfully saw just entering the room with Mrs~Hughes, and whom she instantly joined, with a firmer determination to be acquainted, than she might have had courage to command, had she not been urged by the disappointment of the day before. Miss Tilney met her with great civility, returned her advances with equal goodwill, and they continued talking together as long as both parties remained in the room; and though in all probability not an observation was made, nor an expression used by either which had not been made and used some thousands of times before, under that roof, in every Bath season, yet the merit of their being spoken with simplicity and truth, and without personal conceit, might be something uncommon. 

 <How well your brother dances!> was an artless exclamation of Catherine's towards the close of their conversation, which at once surprised and amused her companion. 

 <Henry!> she replied with a smile. <Yes, he does dance very well.> 

 <He must have thought it very odd to hear me say I was engaged the other evening, when he saw me sitting down. But I really had been engaged the whole day to Mr~Thorpe.> Miss Tilney could only bow. <You cannot think,> added Catherine after a moment's silence, <how surprised I was to see him again. I felt so sure of his being quite gone away.> 

 <When Henry had the pleasure of seeing you before, he was in Bath but for a couple of days. He came only to engage lodgings for us.> 

 <\textit{That} never occurred to me; and of course, not seeing him anywhere, I thought he must be gone. Was not the young lady he danced with on Monday a Miss Smith?> 

 <Yes, an acquaintance of Mrs~Hughes.> 

 <I dare say she was very glad to dance. Do you think her pretty?> 

 <Not very.> 

 <He never comes to the pump-room, I suppose?> 

 <Yes, sometimes; but he has rid out this morning with my father.> 

 Mrs~Hughes now joined them, and asked Miss Tilney if she was ready to go. <I hope I shall have the pleasure of seeing you again soon,> said Catherine. <Shall you be at the cotillion ball to-morrow?> 

 <Perhaps we—Yes, I think we certainly shall.> 

 <I am glad of it, for we shall all be there.> This civility was duly returned; and they parted—on Miss Tilney's side with some knowledge of her new acquaintance's feelings, and on Catherine's, without the smallest consciousness of having explained them. 

 She went home very happy. The morning had answered all her hopes, and the evening of the following day was now the object of expectation, the future good. What gown and what head-dress she should wear on the occasion became her chief concern. She cannot be justified in it. Dress is at all times a frivolous distinction, and excessive solicitude about it often destroys its own aim. Catherine knew all this very well; her great aunt had read her a lecture on the subject only the Christmas before; and yet she lay awake ten minutes on Wednesday night debating between her spotted and her tamboured muslin, and nothing but the shortness of the time prevented her buying a new one for the evening. This would have been an error in judgment, great though not uncommon, from which one of the other sex rather than her own, a brother rather than a great aunt, might have warned her, for man only can be aware of the insensibility of man towards a new gown. It would be mortifying to the feelings of many ladies, could they be made to understand how little the heart of man is affected by what is costly or new in their attire; how little it is biased by the texture of their muslin, and how unsusceptible of peculiar tenderness towards the spotted, the sprigged, the mull, or the jackonet. Woman is fine for her own satisfaction alone. No man will admire her the more, no woman will like her the better for it. Neatness and fashion are enough for the former, and a something of shabbiness or impropriety will be most endearing to the latter. But not one of these grave reflections troubled the tranquillity of Catherine. 

 She entered the rooms on Thursday evening with feelings very different from what had attended her thither the Monday before. She had then been exulting in her engagement to Thorpe, and was now chiefly anxious to avoid his sight, lest he should engage her again; for though she could not, dared not expect that Mr~Tilney should ask her a third time to dance, her wishes, hopes, and plans all centred in nothing less. Every young lady may feel for my heroine in this critical moment, for every young lady has at some time or other known the same agitation. All have been, or at least all have believed themselves to be, in danger from the pursuit of someone whom they wished to avoid; and all have been anxious for the attentions of someone whom they wished to please. As soon as they were joined by the Thorpes, Catherine's agony began; she fidgeted about if John Thorpe came towards her, hid herself as much as possible from his view, and when he spoke to her pretended not to hear him. The cotillions were over, the country-dancing beginning, and she saw nothing of the Tilneys. 

 <Do not be frightened, my dear Catherine,> whispered Isabella, <but I am really going to dance with your brother again. I declare positively it is quite shocking. I tell him he ought to be ashamed of himself, but you and John must keep us in countenance. Make haste, my dear creature, and come to us. John is just walked off, but he will be back in a moment.> 

 Catherine had neither time nor inclination to answer. The others walked away, John Thorpe was still in view, and she gave herself up for lost. That she might not appear, however, to observe or expect him, she kept her eyes intently fixed on her fan; and a self-condemnation for her folly, in supposing that among such a crowd they should even meet with the Tilneys in any reasonable time, had just passed through her mind, when she suddenly found herself addressed and again solicited to dance, by Mr~Tilney himself. With what sparkling eyes and ready motion she granted his request, and with how pleasing a flutter of heart she went with him to the set, may be easily imagined. To escape, and, as she believed, so narrowly escape John Thorpe, and to be asked, so immediately on his joining her, asked by Mr~Tilney, as if he had sought her on purpose!—it did not appear to her that life could supply any greater felicity. 

 Scarcely had they worked themselves into the quiet possession of a place, however, when her attention was claimed by John Thorpe, who stood behind her. <Heyday, Miss Morland!> said he. <What is the meaning of this? I thought you and I were to dance together.> 

 <I wonder you should think so, for you never asked me.> 

 <That is a good one, by Jove! I asked you as soon as I came into the room, and I was just going to ask you again, but when I turned round, you were gone! This is a cursed shabby trick! I only came for the sake of dancing with \textit{you}, and I firmly believe you were engaged to me ever since Monday. Yes; I remember, I asked you while you were waiting in the lobby for your cloak. And here have I been telling all my acquaintance that I was going to dance with the prettiest girl in the room; and when they see you standing up with somebody else, they will quiz me famously.> 

 <Oh, no; they will never think of \textit{me}, after such a description as that.> 

 <By heavens, if they do not, I will kick them out of the room for blockheads. What chap have you there?> Catherine satisfied his curiosity. <Tilney,> he repeated. <Hum—I do not know him. A good figure of a man; well put together. Does he want a horse? Here is a friend of mine, Sam Fletcher, has got one to sell that would suit anybody. A famous clever animal for the road—only forty guineas. I had fifty minds to buy it myself, for it is one of my maxims always to buy a good horse when I meet with one; but it would not answer my purpose, it would not do for the field. I would give any money for a real good hunter. I have three now, the best that ever were backed. I would not take eight hundred guineas for them. Fletcher and I mean to get a house in Leicestershire, against the next season. It is so d\doubleemdash uncomfortable, living at an inn.> 

 This was the last sentence by which he could weary Catherine's attention, for he was just then borne off by the resistless pressure of a long string of passing ladies. Her partner now drew near, and said, <That gentleman would have put me out of patience, had he stayed with you half a minute longer. He has no business to withdraw the attention of my partner from me. We have entered into a contract of mutual agreeableness for the space of an evening, and all our agreeableness belongs solely to each other for that time. Nobody can fasten themselves on the notice of one, without injuring the rights of the other. I consider a country-dance as an emblem of marriage. Fidelity and complaisance are the principal duties of both; and those men who do not choose to dance or marry themselves, have no business with the partners or wives of their neighbours.> 

 <But they are such very different things!> 

 <—That you think they cannot be compared together.> 

 <To be sure not. People that marry can never part, but must go and keep house together. People that dance only stand opposite each other in a long room for half an hour.> 

 <And such is your definition of matrimony and dancing. Taken in that light certainly, their resemblance is not striking; but I think I could place them in such a view. You will allow, that in both, man has the advantage of choice, woman only the power of refusal; that in both, it is an engagement between man and woman, formed for the advantage of each; and that when once entered into, they belong exclusively to each other till the moment of its dissolution; that it is their duty, each to endeavour to give the other no cause for wishing that he or she had bestowed themselves elsewhere, and their best interest to keep their own imaginations from wandering towards the perfections of their neighbours, or fancying that they should have been better off with anyone else. You will allow all this?> 

 <Yes, to be sure, as you state it, all this sounds very well; but still they are so very different. I cannot look upon them at all in the same light, nor think the same duties belong to them.> 

 <In one respect, there certainly is a difference. In marriage, the man is supposed to provide for the support of the woman, the woman to make the home agreeable to the man; he is to purvey, and she is to smile. But in dancing, their duties are exactly changed; the agreeableness, the compliance are expected from him, while she furnishes the fan and the lavender water. \textit{That}, I suppose, was the difference of duties which struck you, as rendering the conditions incapable of comparison.> 

 <No, indeed, I never thought of that.> 

 <Then I am quite at a loss. One thing, however, I must observe. This disposition on your side is rather alarming. You totally disallow any similarity in the obligations; and may I not thence infer that your notions of the duties of the dancing state are not so strict as your partner might wish? Have I not reason to fear that if the gentleman who spoke to you just now were to return, or if any other gentleman were to address you, there would be nothing to restrain you from conversing with him as long as you chose?> 

 <Mr~Thorpe is such a very particular friend of my brother's, that if he talks to me, I must talk to him again; but there are hardly three young men in the room besides him that I have any acquaintance with.> 

 <And is that to be my only security? Alas, alas!> 

 <Nay, I am sure you cannot have a better; for if I do not know anybody, it is impossible for me to talk to them; and, besides, I do not \textit{want} to talk to anybody.> 

 <Now you have given me a security worth having; and I shall proceed with courage. Do you find Bath as agreeable as when I had the honour of making the inquiry before?> 

 <Yes, quite—more so, indeed.> 

 <More so! Take care, or you will forget to be tired of it at the proper time. You ought to be tired at the end of six weeks.> 

 <I do not think I should be tired, if I were to stay here six months.> 

 <Bath, compared with London, has little variety, and so everybody finds out every year. <For six weeks, I allow Bath is pleasant enough; but beyond \textit{that}, it is the most tiresome place in the world.> You would be told so by people of all descriptions, who come regularly every winter, lengthen their six weeks into ten or twelve, and go away at last because they can afford to stay no longer.> 

 <Well, other people must judge for themselves, and those who go to London may think nothing of Bath. But I, who live in a small retired village in the country, can never find greater sameness in such a place as this than in my own home; for here are a variety of amusements, a variety of things to be seen and done all day long, which I can know nothing of there.> 

 <You are not fond of the country.> 

 <Yes, I am. I have always lived there, and always been very happy. But certainly there is much more sameness in a country life than in a Bath life. One day in the country is exactly like another.> 

 <But then you spend your time so much more rationally in the country.> 

 <Do I?> 

 <Do you not?> 

 <I do not believe there is much difference.> 

 <Here you are in pursuit only of amusement all day long.> 

 <And so I am at home—only I do not find so much of it. I walk about here, and so I do there; but here I see a variety of people in every street, and there I can only go and call on Mrs~Allen.> 

 Mr~Tilney was very much amused. 

 <Only go and call on Mrs~Allen!> he repeated. <What a picture of intellectual poverty! However, when you sink into this abyss again, you will have more to say. You will be able to talk of Bath, and of all that you did here.> 

 <Oh! Yes. I shall never be in want of something to talk of again to Mrs~Allen, or anybody else. I really believe I shall always be talking of Bath, when I am at home again—I \textit{do} like it so very much. If I could but have Papa and Mamma, and the rest of them here, I suppose I should be too happy! James's coming (my eldest brother) is quite delightful—and especially as it turns out that the very family we are just got so intimate with are his intimate friends already. Oh! Who can ever be tired of Bath?> 

 <Not those who bring such fresh feelings of every sort to it as you do. But papas and mammas, and brothers, and intimate friends are a good deal gone by, to most of the frequenters of Bath—and the honest relish of balls and plays, and everyday sights, is past with them.> 

 Here their conversation closed, the demands of the dance becoming now too importunate for a divided attention. 

 Soon after their reaching the bottom of the set, Catherine perceived herself to be earnestly regarded by a gentleman who stood among the lookers-on, immediately behind her partner. He was a very handsome man, of a commanding aspect, past the bloom, but not past the vigour of life; and with his eye still directed towards her, she saw him presently address Mr~Tilney in a familiar whisper. Confused by his notice, and blushing from the fear of its being excited by something wrong in her appearance, she turned away her head. But while she did so, the gentleman retreated, and her partner, coming nearer, said, <I see that you guess what I have just been asked. That gentleman knows your name, and you have a right to know his. It is General Tilney, my father.> 

 Catherine's answer was only <Oh!>—but it was an <Oh!> expressing everything needful: attention to his words, and perfect reliance on their truth. With real interest and strong admiration did her eye now follow the general, as he moved through the crowd, and <How handsome a family they are!> was her secret remark. 

 In chatting with Miss Tilney before the evening concluded, a new source of felicity arose to her. She had never taken a country walk since her arrival in Bath. Miss Tilney, to whom all the commonly frequented environs were familiar, spoke of them in terms which made her all eagerness to know them too; and on her openly fearing that she might find nobody to go with her, it was proposed by the brother and sister that they should join in a walk, some morning or other. <I shall like it,> she cried, <beyond anything in the world; and do not let us put it off—let us go to-morrow.> This was readily agreed to, with only a proviso of Miss Tilney's, that it did not rain, which Catherine was sure it would not. At twelve o'clock, they were to call for her in Pulteney Street; and <Remember—twelve o'clock,> was her parting speech to her new friend. Of her other, her older, her more established friend, Isabella, of whose fidelity and worth she had enjoyed a fortnight's experience, she scarcely saw anything during the evening. Yet, though longing to make her acquainted with her happiness, she cheerfully submitted to the wish of Mr~Allen, which took them rather early away, and her spirits danced within her, as she danced in her chair all the way home. 