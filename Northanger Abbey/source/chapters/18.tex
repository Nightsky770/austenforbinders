\chapter[Chapter \thechapter]{} 

 \lettrine{W}{ith} a mind thus full of happiness, Catherine was hardly aware that two or three days had passed away, without her seeing Isabella for more than a few minutes together. She began first to be sensible of this, and to sigh for her conversation, as she walked along the pump-room one morning, by Mrs~Allen's side, without anything to say or to hear; and scarcely had she felt a five minutes' longing of friendship, before the object of it appeared, and inviting her to a secret conference, led the way to a seat. <This is my favourite place,> said she as they sat down on a bench between the doors, which commanded a tolerable view of everybody entering at either; <it is so out of the way.> 

 Catherine, observing that Isabella's eyes were continually bent towards one door or the other, as in eager expectation, and remembering how often she had been falsely accused of being arch, thought the present a fine opportunity for being really so; and therefore gaily said, <Do not be uneasy, Isabella, James will soon be here.> 

 <Psha! My dear creature,> she replied, <do not think me such a simpleton as to be always wanting to confine him to my elbow. It would be hideous to be always together; we should be the jest of the place. And so you are going to Northanger! I am amazingly glad of it. It is one of the finest old places in England, I understand. I shall depend upon a most particular description of it.> 

 <You shall certainly have the best in my power to give. But who are you looking for? Are your sisters coming?> 

 <I am not looking for anybody. One's eyes must be somewhere, and you know what a foolish trick I have of fixing mine, when my thoughts are an hundred miles off. I am amazingly absent; I believe I am the most absent creature in the world. Tilney says it is always the case with minds of a certain stamp.> 

 <But I thought, Isabella, you had something in particular to tell me?> 

 <Oh yes, and so I have. But here is a proof of what I was saying. My poor head, I had quite forgot it. Well, the thing is this: I have just had a letter from John; you can guess the contents.> 

 <No, indeed, I cannot.> 

 <My sweet love, do not be so abominably affected. What can he write about, but yourself? You know he is over head and ears in love with you.> 

 <With \textit{me}, dear Isabella!> 

 <Nay, my sweetest Catherine, this is being quite absurd! Modesty, and all that, is very well in its way, but really a little common honesty is sometimes quite as becoming. I have no idea of being so overstrained! It is fishing for compliments. His attentions were such as a child must have noticed. And it was but half an hour before he left Bath that you gave him the most positive encouragement. He says so in this letter, says that he as good as made you an offer, and that you received his advances in the kindest way; and now he wants me to urge his suit, and say all manner of pretty things to you. So it is in vain to affect ignorance.> 

 Catherine, with all the earnestness of truth, expressed her astonishment at such a charge, protesting her innocence of every thought of Mr~Thorpe's being in love with her, and the consequent impossibility of her having ever intended to encourage him. <As to any attentions on his side, I do declare, upon my honour, I never was sensible of them for a moment—except just his asking me to dance the first day of his coming. And as to making me an offer, or anything like it, there must be some unaccountable mistake. I could not have misunderstood a thing of that kind, you know! And, as I ever wish to be believed, I solemnly protest that no syllable of such a nature ever passed between us. The last half hour before he went away! It must be all and completely a mistake—for I did not see him once that whole morning.> 

 <But \textit{that} you certainly did, for you spent the whole morning in Edgar's Buildings—it was the day your father's consent came—and I am pretty sure that you and John were alone in the parlour some time before you left the house.> 

 <Are you? Well, if you say it, it was so, I dare say—but for the life of me, I cannot recollect it. I \textit{do} remember now being with you, and seeing him as well as the rest—but that we were ever alone for five minutes— However, it is not worth arguing about, for whatever might pass on his side, you must be convinced, by my having no recollection of it, that I never thought, nor expected, nor wished for anything of the kind from him. I am excessively concerned that he should have any regard for me—but indeed it has been quite unintentional on my side; I never had the smallest idea of it. Pray undeceive him as soon as you can, and tell him I beg his pardon—that is—I do not know what I ought to say—but make him understand what I mean, in the properest way. I would not speak disrespectfully of a brother of yours, Isabella, I am sure; but you know very well that if I could think of one man more than another—\textit{he} is not the person.> Isabella was silent. <My dear friend, you must not be angry with me. I cannot suppose your brother cares so very much about me. And, you know, we shall still be sisters.> 

 <Yes, yes> (with a blush), <there are more ways than one of our being sisters. But where am I wandering to? Well, my dear Catherine, the case seems to be that you are determined against poor John—is not it so?> 

 <I certainly cannot return his affection, and as certainly never meant to encourage it.> 

 <Since that is the case, I am sure I shall not tease you any further. John desired me to speak to you on the subject, and therefore I have. But I confess, as soon as I read his letter, I thought it a very foolish, imprudent business, and not likely to promote the good of either; for what were you to live upon, supposing you came together? You have both of you something, to be sure, but it is not a trifle that will support a family nowadays; and after all that romancers may say, there is no doing without money. I only wonder John could think of it; he could not have received my last.> 

 <You \textit{do} acquit me, then, of anything wrong?—You are convinced that I never meant to deceive your brother, never suspected him of liking me till this moment?> 

 <Oh! As to that,> answered Isabella laughingly, <I do not pretend to determine what your thoughts and designs in time past may have been. All that is best known to yourself. A little harmless flirtation or so will occur, and one is often drawn on to give more encouragement than one wishes to stand by. But you may be assured that I am the last person in the world to judge you severely. All those things should be allowed for in youth and high spirits. What one means one day, you know, one may not mean the next. Circumstances change, opinions alter.> 

 <But my opinion of your brother never did alter; it was always the same. You are describing what never happened.> 

 <My dearest Catherine,> continued the other without at all listening to her, <I would not for all the world be the means of hurrying you into an engagement before you knew what you were about. I do not think anything would justify me in wishing you to sacrifice all your happiness merely to oblige my brother, because he is my brother, and who perhaps after all, you know, might be just as happy without you, for people seldom know what they would be at, young men especially, they are so amazingly changeable and inconstant. What I say is, why should a brother's happiness be dearer to me than a friend's? You know I carry my notions of friendship pretty high. But, above all things, my dear Catherine, do not be in a hurry. Take my word for it, that if you are in too great a hurry, you will certainly live to repent it. Tilney says there is nothing people are so often deceived in as the state of their own affections, and I believe he is very right. Ah! Here he comes; never mind, he will not see us, I am sure.> 

 Catherine, looking up, perceived Captain Tilney; and Isabella, earnestly fixing her eye on him as she spoke, soon caught his notice. He approached immediately, and took the seat to which her movements invited him. His first address made Catherine start. Though spoken low, she could distinguish, <What! Always to be watched, in person or by proxy!> 

 <Psha, nonsense!> was Isabella's answer in the same half whisper. <Why do you put such things into my head? If I could believe it—my spirit, you know, is pretty independent.> 

 <I wish your heart were independent. That would be enough for me.> 

 <My heart, indeed! What can you have to do with hearts? You men have none of you any hearts.> 

 <If we have not hearts, we have eyes; and they give us torment enough.> 

 <Do they? I am sorry for it; I am sorry they find anything so disagreeable in me. I will look another way. I hope this pleases you> (turning her back on him); <I hope your eyes are not tormented now.> 

 <Never more so; for the edge of a blooming cheek is still in view—at once too much and too little.> 

 Catherine heard all this, and quite out of countenance, could listen no longer. Amazed that Isabella could endure it, and jealous for her brother, she rose up, and saying she should join Mrs~Allen, proposed their walking. But for this Isabella showed no inclination. She was so amazingly tired, and it was so odious to parade about the pump-room; and if she moved from her seat she should miss her sisters; she was expecting her sisters every moment; so that her dearest Catherine must excuse her, and must sit quietly down again. But Catherine could be stubborn too; and Mrs~Allen just then coming up to propose their returning home, she joined her and walked out of the pump-room, leaving Isabella still sitting with Captain Tilney. With much uneasiness did she thus leave them. It seemed to her that Captain Tilney was falling in love with Isabella, and Isabella unconsciously encouraging him; unconsciously it must be, for Isabella's attachment to James was as certain and well acknowledged as her engagement. To doubt her truth or good intentions was impossible; and yet, during the whole of their conversation her manner had been odd. She wished Isabella had talked more like her usual self, and not so much about money, and had not looked so well pleased at the sight of Captain Tilney. How strange that she should not perceive his admiration! Catherine longed to give her a hint of it, to put her on her guard, and prevent all the pain which her too lively behaviour might otherwise create both for him and her brother. 

 The compliment of John Thorpe's affection did not make amends for this thoughtlessness in his sister. She was almost as far from believing as from wishing it to be sincere; for she had not forgotten that he could mistake, and his assertion of the offer and of her encouragement convinced her that his mistakes could sometimes be very egregious. In vanity, therefore, she gained but little; her chief profit was in wonder. That he should think it worth his while to fancy himself in love with her was a matter of lively astonishment. Isabella talked of his attentions; \textit{she} had never been sensible of any; but Isabella had said many things which she hoped had been spoken in haste, and would never be said again; and upon this she was glad to rest altogether for present ease and comfort. 