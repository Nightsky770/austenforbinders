\chapter{Lady~Susan to Mrs~Johnson}
  
  	\begin{a4}
	\vspace{5em}
	\end{a4}
	
  \begin{mail}{Churchhill.}{}

I call on you, dear Alicia, for congratulations: I am my own self, gay and triumphant! When I wrote to you the other day I was, in truth, in high irritation, and with ample cause. Nay, I know not whether I ought to be quite tranquil now, for I have had more trouble in restoring peace than I ever intended to submit to—a spirit, too, resulting from a fancied sense of superior integrity, which is peculiarly insolent! I shall not easily forgive him, I assure you. He was actually on the point of leaving Churchhill! I had scarcely concluded my last, when Wilson brought me word of it. I found, therefore, that something must be done; for I did not choose to leave my character at the mercy of a man whose passions are so violent and so revengeful. It would have been trifling with my reputation to allow of his departing with such an impression in my disfavour; in this light, condescension was necessary. I sent Wilson to say that I desired to speak with him before he went; he came immediately. The angry emotions which had marked every feature when we last parted were partially subdued. He seemed astonished at the summons, and looked as if half wishing and half fearing to be softened by what I might say. If my countenance expressed what I aimed at, it was composed and dignified; and yet, with a degree of pensiveness which might convince him that I was not quite happy. <I beg your pardon, sir, for the liberty I have taken in sending for you,> said I; <but as I have just learnt your intention of leaving this place to-day, I feel it my duty to entreat that you will not on my account shorten your visit here even an hour. I am perfectly aware that after what has passed between us it would ill suit the feelings of either to remain longer in the same house: so very great, so total a change from the intimacy of friendship must render any future intercourse the severest punishment; and your resolution of quitting Churchhill is undoubtedly in unison with our situation, and with those lively feelings which I know you to possess. But, at the same time, it is not for me to suffer such a sacrifice as it must be to leave relations to whom you are so much attached, and are so dear. My remaining here cannot give that pleasure to Mr~and Mrs~Vernon which your society must; and my visit has already perhaps been too long. My removal, therefore, which must, at any rate, take place soon, may, with perfect convenience, be hastened; and I make it my particular request that I may not in any way be instrumental in separating a family so affectionately attached to each other. Where I go is of no consequence to anyone; of very little to myself; but you are of importance to all your connections.> Here I concluded, and I hope you will be satisfied with my speech. Its effect on Reginald justifies some portion of vanity, for it was no less favourable than instantaneous. Oh, how delightful it was to watch the variations of his countenance while I spoke! to see the struggle between returning tenderness and the remains of displeasure. There is something agreeable in feelings so easily worked on; not that I envy him their possession, nor would, for the world, have such myself; but they are very convenient when one wishes to influence the passions of another. And yet this Reginald, whom a very few words from me softened at once into the utmost submission, and rendered more tractable, more attached, more devoted than ever, would have left me in the first angry swelling of his proud heart without deigning to seek an explanation. Humbled as he now is, I cannot forgive him such an instance of pride, and am doubtful whether I ought not to punish him by dismissing him at once after this reconciliation, or by marrying and teazing him for ever. But these measures are each too violent to be adopted without some deliberation; at present my thoughts are fluctuating between various schemes. I have many things to compass: I must punish Frederica, and pretty severely too, for her application to Reginald; I must punish him for receiving it so favourably, and for the rest of his conduct. I must torment my sister-in-law for the insolent triumph of her look and manner since Sir~James has been dismissed; for, in reconciling Reginald to me, I was not able to save that ill-fated young man; and I must make myself amends for the humiliation to which I have stooped within these few days. To effect all this I have various plans. I have also an idea of being soon in town; and whatever may be my determination as to the rest, I shall probably put \textit{that} project in execution; for London will be always the fairest field of action, however my views may be directed; and at any rate I shall there be rewarded by your society, and a little dissipation, for a ten weeks' penance at Churchhill. I believe I owe it to my character to complete the match between my daughter and Sir~James after having so long intended it. Let me know your opinion on this point. Flexibility of mind, a disposition easily biassed by others, is an attribute which you know I am not very desirous of obtaining; nor has Frederica any claim to the indulgence of her notions at the expense of her mother's inclinations. Her idle love for Reginald, too! It is surely my duty to discourage such romantic nonsense. All things considered, therefore, it seems incumbent on me to take her to town and marry her immediately to Sir~James. When my own will is effected contrary to his, I shall have some credit in being on good terms with Reginald, which at present, in fact, I have not; for though he is still in my power, I have given up the very article by which our quarrel was produced, and at best the honour of victory is doubtful. Send me your opinion on all these matters, my dear Alicia, and let me know whether you can get lodgings to suit me within a short distance of you. 

\closeletter[Your most attached]{S. Vernon.} 
\end{mail}