\chapter{Mrs~Vernon to Lady~De Courcy}
  
  \begin{mail}{Churchhill.}{My dear Mother,}

Mr~Vernon returned on Thursday night, bringing his niece with him. Lady~Susan had received a line from him by that day's post, informing her that Miss~Summers had absolutely refused to allow of Miss~Vernon's continuance in her academy; we were therefore prepared for her arrival, and expected them impatiently the whole evening. They came while we were at tea, and I never saw any creature look so frightened as Frederica when she entered the room. Lady~Susan, who had been shedding tears before, and showing great agitation at the idea of the meeting, received her with perfect self-command, and without betraying the least tenderness of spirit. She hardly spoke to her, and on Frederica's bursting into tears as soon as we were seated, took her out of the room, and did not return for some time. When she did, her eyes looked very red and she was as much agitated as before. We saw no more of her daughter. Poor Reginald was beyond measure concerned to see his fair friend in such distress, and watched her with so much tender solicitude, that I, who occasionally caught her observing his countenance with exultation, was quite out of patience. This pathetic representation lasted the whole evening, and so ostentatious and artful a display has entirely convinced me that she did in fact feel nothing. I am more angry with her than ever since I have seen her daughter; the poor girl looks so unhappy that my heart aches for her. Lady~Susan is surely too severe, for Frederica does not seem to have the sort of temper to make severity necessary. She looks perfectly timid, dejected, and penitent. She is very pretty, though not so handsome as her mother, nor at all like her. Her complexion is delicate, but neither so fair nor so blooming as Lady~Susan's, and she has quite the Vernon cast of countenance, the oval face and mild dark eyes, and there is peculiar sweetness in her look when she speaks either to her uncle or me, for as we behave kindly to her we have of course engaged her gratitude.

Her mother has insinuated that her temper is intractable, but I never saw a face less indicative of any evil disposition than hers; and from what I can see of the behaviour of each to the other, the invariable severity of Lady~Susan and the silent dejection of Frederica, I am led to believe as heretofore that the former has no real love for her daughter, and has never done her justice or treated her affectionately. I have not been able to have any conversation with my niece; she is shy, and I think I can see that some pains are taken to prevent her being much with me. Nothing satisfactory transpires as to her reason for running away. Her kind-hearted uncle, you may be sure, was too fearful of distressing her to ask many questions as they travelled. I wish it had been possible for me to fetch her instead of him. I think I should have discovered the truth in the course of a thirty-mile journey. The small pianoforte has been removed within these few days, at Lady~Susan's request, into her dressing-room, and Frederica spends great part of the day there, practising as it is called; but I seldom hear any noise when I pass that way; what she does with herself there I do not know. There are plenty of books, but it is not every girl who has been running wild the first fifteen years of her life, that can or will read. Poor creature! the prospect from her window is not very instructive, for that room overlooks the lawn, you know, with the shrubbery on one side, where she may see her mother walking for an hour together in earnest conversation with Reginald. A girl of Frederica's age must be childish indeed, if such things do not strike her. Is it not inexcusable to give such an example to a daughter? Yet Reginald still thinks Lady~Susan the best of mothers, and still condemns Frederica as a worthless girl! He is convinced that her attempt to run away proceeded from no justifiable cause, and had no provocation. I am sure I cannot say that it \textit{had}, but while Miss~Summers declares that Miss~Vernon showed no signs of obstinacy or perverseness during her whole stay in Wigmore Street, till she was detected in this scheme, I cannot so readily credit what Lady~Susan has made him, and wants to make me believe, that it was merely an impatience of restraint and a desire of escaping from the tuition of masters which brought on the plan of an elopement. O Reginald, how is your judgment enslaved! He scarcely dares even allow her to be handsome, and when I speak of her beauty, replies only that her eyes have no brilliancy! Sometimes he is sure she is deficient in understanding, and at others that her temper only is in fault. In short, when a person is always to deceive, it is impossible to be consistent. Lady~Susan finds it necessary that Frederica should be to blame, and probably has sometimes judged it expedient to accuse her of ill-nature and sometimes to lament her want of sense. Reginald is only repeating after her ladyship. 

\closeletter[I remain, \&c., \&c.,]{Catherine Vernon.} 
\end{mail}