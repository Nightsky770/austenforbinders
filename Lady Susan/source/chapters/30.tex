\chapter{Lady~Susan Vernon to Mr~De Courcy}
  
  \begin{mail}{Upper Seymour Street.}{}

I have received your letter, and though I do not attempt to conceal that I am gratified by your impatience for the hour of meeting, I yet feel myself under the necessity of delaying that hour beyond the time originally fixed. Do not think me unkind for such an exercise of my power, nor accuse me of instability without first hearing my reasons. In the course of my journey from Churchhill I had ample leisure for reflection on the present state of our affairs, and every review has served to convince me that they require a delicacy and cautiousness of conduct to which we have hitherto been too little attentive. We have been hurried on by our feelings to a degree of precipitation which ill accords with the claims of our friends or the opinion of the world. We have been unguarded in forming this hasty engagement, but we must not complete the imprudence by ratifying it while there is so much reason to fear the connection would be opposed by those friends on whom you depend. It is not for us to blame any expectations on your father's side of your marrying to advantage; where possessions are so extensive as those of your family, the wish of increasing them, if not strictly reasonable, is too common to excite surprize or resentment. He has a right to require a woman of fortune in his daughter-in-law, and I am sometimes quarrelling with myself for suffering you to form a connection so imprudent; but the influence of reason is often acknowledged too late by those who feel like me. I have now been but a few months a widow, and, however little indebted to my husband's memory for any happiness derived from him during a union of some years, I cannot forget that the indelicacy of so early a second marriage must subject me to the censure of the world, and incur, what would be still more insupportable, the displeasure of Mr~Vernon. I might perhaps harden myself in time against the injustice of general reproach, but the loss of \textit{his} valued esteem I am, as you well know, ill-fitted to endure; and when to this may be added the consciousness of having injured you with your family, how am I to support myself? With feelings so poignant as mine, the conviction of having divided the son from his parents would make me, even with you, the most miserable of beings. It will surely, therefore, be advisable to delay our union—to delay it till appearances are more promising—till affairs have taken a more favourable turn. To assist us in such a resolution I feel that absence will be necessary. We must not meet. Cruel as this sentence may appear, the necessity of pronouncing it, which can alone reconcile it to myself, will be evident to you when you have considered our situation in the light in which I have found myself imperiously obliged to place it. You may be—you must be—well assured that nothing but the strongest conviction of duty could induce me to wound my own feelings by urging a lengthened separation, and of insensibility to yours you will hardly suspect me. Again, therefore, I say that we ought not, we must not, yet meet. By a removal for some months from each other we shall tranquillise the sisterly fears of Mrs~Vernon, who, accustomed herself to the enjoyment of riches, considers fortune as necessary everywhere, and whose sensibilities are not of a nature to comprehend ours. Let me hear from you soon—very soon. Tell me that you submit to my arguments, and do not reproach me for using such. I cannot bear reproaches: my spirits are not so high as to need being repressed. I must endeavour to seek amusement, and fortunately many of my friends are in town; amongst them the Mainwarings; you know how sincerely I regard both husband and wife. 

\closeletter[I am, very faithfully yours,]{S. Vernon.} 
\end{mail}