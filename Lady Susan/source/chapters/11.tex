\chapter{Mrs~Vernon to Lady~De Courcy}
  
  \begin{mail}{Churchhill}{}

I really grow quite uneasy, my dearest mother, about Reginald, from witnessing the very rapid increase of Lady~Susan's influence. They are now on terms of the most particular friendship, frequently engaged in long conversations together; and she has contrived by the most artful coquetry to subdue his judgment to her own purposes. It is impossible to see the intimacy between them so very soon established without some alarm, though I can hardly suppose that Lady~Susan's plans extend to marriage. I wish you could get Reginald home again on any plausible pretence; he is not at all disposed to leave us, and I have given him as many hints of my father's precarious state of health as common decency will allow me to do in my own house. Her power over him must now be boundless, as she has entirely effaced all his former ill-opinion, and persuaded him not merely to forget but to justify her conduct. Mr~Smith's account of her proceedings at Langford, where he accused her of having made Mr~Mainwaring and a young man engaged to Miss~Mainwaring distractedly in love with her, which Reginald firmly believed when he came here, is now, he is persuaded, only a scandalous invention. He has told me so with a warmth of manner which spoke his regret at having believed the contrary himself. How sincerely do I grieve that she ever entered this house! I always looked forward to her coming with uneasiness; but very far was it from originating in anxiety for Reginald. I expected a most disagreeable companion for myself, but could not imagine that my brother would be in the smallest danger of being captivated by a woman with whose principles he was so well acquainted, and whose character he so heartily despised. If you can get him away it will be a good thing. 

\closeletter[Yours, \&c.,]{Catherine Vernon.} 
\end{mail}