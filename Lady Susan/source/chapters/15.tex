\chapter{Mrs~Vernon to Lady~De Courcy}
  
  	\begin{a4}
	\vspace{5em}
	\end{a4}
	
	\begin{letter}
	\vspace{5em}
	\end{letter}
	
  \begin{mail}{Churchhill}{My dear Mother,}

I return you Reginald's letter, and rejoice with all my heart that my father is made easy by it: tell him so, with my congratulations; but, between ourselves, I must own it has only convinced \textit{me} of my brother's having no \textit{present} intention of marrying Lady~Susan, not that he is in no danger of doing so three months hence. He gives a very plausible account of her behaviour at Langford; I wish it may be true, but his intelligence must come from herself, and I am less disposed to believe it than to lament the degree of intimacy subsisting between them, implied by the discussion of such a subject. I am sorry to have incurred his displeasure, but can expect nothing better while he is so very eager in Lady~Susan's justification. He is very severe against me indeed, and yet I hope I have not been hasty in my judgment of her. Poor woman! though I have reasons enough for my dislike, I cannot help pitying her at present, as she is in real distress, and with too much cause. She had this morning a letter from the Lady~with whom she has placed her daughter, to request that Miss~Vernon might be immediately removed, as she had been detected in an attempt to run away. Why, or whither she intended to go, does not appear; but, as her situation seems to have been unexceptionable, it is a sad thing, and of course highly distressing to Lady~Susan. Frederica must be as much as sixteen, and ought to know better; but from what her mother insinuates, I am afraid she is a perverse girl. She has been sadly neglected, however, and her mother ought to remember it. Mr~Vernon set off for London as soon as she had determined what should be done. He is, if possible, to prevail on Miss~Summers to let Frederica continue with her; and if he cannot succeed, to bring her to Churchhill for the present, till some other situation can be found for her. Her ladyship is comforting herself meanwhile by strolling along the shrubbery with Reginald, calling forth all his tender feelings, I suppose, on this distressing occasion. She has been talking a great deal about it to me. She talks vastly well; I am afraid of being ungenerous, or I should say, \textit{too} well to feel so very deeply; but I will not look for her faults; she may be Reginald's wife! Heaven forbid it! but why should I be quicker-sighted than anyone else? Mr~Vernon declares that he never saw deeper distress than hers, on the receipt of the letter; and is his judgment inferior to mine? She was very unwilling that Frederica should be allowed to come to Churchhill, and justly enough, as it seems a sort of reward to behaviour deserving very differently; but it was impossible to take her anywhere else, and she is not to remain here long. <It will be absolutely necessary,> said she, “as you, my dear sister, must be sensible, to treat my daughter with some severity while she is here; a most painful necessity, but I will \textit{endeavour} to submit to it. I am afraid I have often been too indulgent, but my poor Frederica's temper could never bear opposition well: you must support and encourage me; you must urge the necessity of reproof if you see me too lenient.” All this sounds very reasonable. Reginald is so incensed against the poor silly girl! Surely it is not to Lady~Susan's credit that he should be so bitter against her daughter; his idea of her must be drawn from the mother's description. Well, whatever may be his fate, we have the comfort of knowing that we have done our utmost to save him. We must commit the event to a higher power. 

\closeletter[Yours ever, \&c.,]{Catherine Vernon.} 
\end{mail}