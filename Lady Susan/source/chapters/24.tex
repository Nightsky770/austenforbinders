\chapter{From the same to the same}
  
  	\begin{a4}
	\vspace{5em}
	\end{a4}
	
  \begin{mail}{Churchhill.}{}

Little did I imagine, my dear Mother, when I sent off my last letter, that the delightful perturbation of spirits I was then in would undergo so speedy, so melancholy a reverse. I never can sufficiently regret that I wrote to you at all. Yet who could have foreseen what has happened? My dear mother, every hope which made me so happy only two hours ago has vanished. The quarrel between Lady~Susan and Reginald is made up, and we are all as we were before. One point only is gained. Sir~James Martin is dismissed. What are we now to look forward to? I am indeed disappointed; Reginald was all but gone, his horse was ordered and all but brought to the door; who would not have felt safe? For half an hour I was in momentary expectation of his departure. After I had sent off my letter to you, I went to Mr~Vernon, and sat with him in his room talking over the whole matter, and then determined to look for Frederica, whom I had not seen since breakfast. I met her on the stairs, and saw that she was crying. <My dear aunt,> said she, <he is going—Mr~De Courcy is going, and it is all my fault. I am afraid you will be very angry with me, but indeed I had no idea it would end so.> <My love,> I replied, <do not think it necessary to apologize to me on that account. I shall feel myself under an obligation to anyone who is the means of sending my brother home, because,> recollecting myself, <I know my father wants very much to see him. But what is it you have done to occasion all this?> She blushed deeply as she answered: <I was so unhappy about Sir~James that I could not help—I have done something very wrong, I know; but you have not an idea of the misery I have been in: and mamma had ordered me never to speak to you or my uncle about it, and—> <You therefore spoke to my brother to engage his interference,> said I, to save her the explanation. <No, but I wrote to him—I did indeed, I got up this morning before it was light, and was two hours about it; and when my letter was done I thought I never should have courage to give it. After breakfast however, as I was going to my room, I met him in the passage, and then, as I knew that everything must depend on that moment, I forced myself to give it. He was so good as to take it immediately. I dared not look at him, and ran away directly. I was in such a fright I could hardly breathe. My dear aunt, you do not know how miserable I have been.> <Frederica> said I, <you ought to have told me all your distresses. You would have found in me a friend always ready to assist you. Do you think that your uncle or I should not have espoused your cause as warmly as my brother?> <Indeed, I did not doubt your kindness,> said she, colouring again, <but I thought Mr~De Courcy could do anything with my mother; but I was mistaken: they have had a dreadful quarrel about it, and he is going away. Mamma will never forgive me, and I shall be worse off than ever.> <No, you shall not,> I replied; “in such a point as this your mother's prohibition ought not to have prevented your speaking to me on the subject. She has no right to make you unhappy, and she shall \textit{not} do it. Your applying, however, to Reginald can be productive only of good to all parties. I believe it is best as it is. Depend upon it that you shall not be made unhappy any longer.” At that moment how great was my astonishment at seeing Reginald come out of Lady~Susan's dressing-room. My heart misgave me instantly. His confusion at seeing me was very evident. Frederica immediately disappeared. <Are you going?> I said; <you will find Mr~Vernon in his own room.> <No, Catherine,> he replied, <I am not going. Will you let me speak to you a moment?> We went into my room. <I find,> he continued, his confusion increasing as he spoke, <that I have been acting with my usual foolish impetuosity. I have entirely misunderstood Lady~Susan, and was on the point of leaving the house under a false impression of her conduct. There has been some very great mistake; we have been all mistaken, I fancy. Frederica does not know her mother. Lady~Susan means nothing but her good, but she will not make a friend of her. Lady~Susan does not always know, therefore, what will make her daughter happy. Besides, I could have no right to interfere. Miss~Vernon was mistaken in applying to me. In short, Catherine, everything has gone wrong, but it is now all happily settled. Lady~Susan, I believe, wishes to speak to you about it, if you are at leisure.> <Certainly,> I replied, deeply sighing at the recital of so lame a story. I made no comments, however, for words would have been vain.

Reginald was glad to get away, and I went to Lady~Susan, curious, indeed, to hear her account of it. <Did I not tell you,> said she with a smile, <that your brother would not leave us after all?> <You did, indeed,> replied I very gravely; <but I flattered myself you would be mistaken.> <I should not have hazarded such an opinion,> returned she, <if it had not at that moment occurred to me that his resolution of going might be occasioned by a conversation in which we had been this morning engaged, and which had ended very much to his dissatisfaction, from our not rightly understanding each other's meaning. This idea struck me at the moment, and I instantly determined that an accidental dispute, in which I might probably be as much to blame as himself, should not deprive you of your brother. If you remember, I left the room almost immediately. I was resolved to lose no time in clearing up those mistakes as far as I could. The case was this—Frederica had set herself violently against marrying Sir~James.> <And can your ladyship wonder that she should?> cried I with some warmth; <Frederica has an excellent understanding, and Sir~James has none.> <I am at least very far from regretting it, my dear sister,> said she; <on the contrary, I am grateful for so favourable a sign of my daughter's sense. Sir~James is certainly below par (his boyish manners make him appear worse); and had Frederica possessed the penetration and the abilities which I could have wished in my daughter, or had I even known her to possess as much as she does, I should not have been anxious for the match.> <It is odd that you should alone be ignorant of your daughter's sense!> <Frederica never does justice to herself; her manners are shy and childish, and besides she is afraid of me. During her poor father's life she was a spoilt child; the severity which it has since been necessary for me to show has alienated her affection; neither has she any of that brilliancy of intellect, that genius or vigour of mind which will force itself forward.> <Say rather that she has been unfortunate in her education!> <Heaven knows, my dearest Mrs~Vernon, how fully I am aware of that; but I would wish to forget every circumstance that might throw blame on the memory of one whose name is sacred with me.> Here she pretended to cry; I was out of patience with her. <But what,> said I, <was your ladyship going to tell me about your disagreement with my brother?> <It originated in an action of my daughter's, which equally marks her want of judgment and the unfortunate dread of me I have been mentioning—she wrote to Mr~De Courcy.> <I know she did; you had forbidden her speaking to Mr~Vernon or to me on the cause of her distress; what could she do, therefore, but apply to my brother?> <Good God!> she exclaimed, “what an opinion you must have of me! Can you possibly suppose that I was aware of her unhappiness! that it was my object to make my own child miserable, and that I had forbidden her speaking to you on the subject from a fear of your interrupting the diabolical scheme? Do you think me destitute of every honest, every natural feeling? Am I capable of consigning \textit{her} to everlasting misery whose welfare it is my first earthly duty to promote? The idea is horrible!” <What, then, was your intention when you insisted on her silence?> <Of what use, my dear sister, could be any application to you, however the affair might stand? Why should I subject you to entreaties which I refused to attend to myself? Neither for your sake nor for hers, nor for my own, could such a thing be desirable. When my own resolution was taken I could not wish for the interference, however friendly, of another person. I was mistaken, it is true, but I believed myself right.> <But what was this mistake to which your ladyship so often alludes? from whence arose so astonishing a misconception of your daughter's feelings? Did you not know that she disliked Sir~James?> <I knew that he was not absolutely the man she would have chosen, but I was persuaded that her objections to him did not arise from any perception of his deficiency. You must not question me, however, my dear sister, too minutely on this point,> continued she, taking me affectionately by the hand; <I honestly own that there is something to conceal. Frederica makes me very unhappy! Her applying to Mr~De Courcy hurt me particularly.> <What is it you mean to infer,> said I, <by this appearance of mystery? If you think your daughter at all attached to Reginald, her objecting to Sir~James could not less deserve to be attended to than if the cause of her objecting had been a consciousness of his folly; and why should your ladyship, at any rate, quarrel with my brother for an interference which, you must know, it is not in his nature to refuse when urged in such a manner?>

<His disposition, you know, is warm, and he came to expostulate with me; his compassion all alive for this ill-used girl, this heroine in distress! We misunderstood each other: he believed me more to blame than I really was; I considered his interference less excusable than I now find it. I have a real regard for him, and was beyond expression mortified to find it, as I thought, so ill bestowed. We were both warm, and of course both to blame. His resolution of leaving Churchhill is consistent with his general eagerness. When I understood his intention, however, and at the same time began to think that we had been perhaps equally mistaken in each other's meaning, I resolved to have an explanation before it was too late. For any member of your family I must always feel a degree of affection, and I own it would have sensibly hurt me if my acquaintance with Mr~De Courcy had ended so gloomily. I have now only to say further, that as I am convinced of Frederica's having a reasonable dislike to Sir~James, I shall instantly inform him that he must give up all hope of her. I reproach myself for having, even though innocently, made her unhappy on that score. She shall have all the retribution in my power to make; if she value her own happiness as much as I do, if she judge wisely, and command herself as she ought, she may now be easy. Excuse me, my dearest sister, for thus trespassing on your time, but I owe it to my own character; and after this explanation I trust I am in no danger of sinking in your opinion.> I could have said, <Not much, indeed!> but I left her almost in silence. It was the greatest stretch of forbearance I could practise. I could not have stopped myself had I begun. Her assurance! her deceit! but I will not allow myself to dwell on them; they will strike you sufficiently. My heart sickens within me. As soon as I was tolerably composed I returned to the parlour. Sir~James's carriage was at the door, and he, merry as usual, soon afterwards took his leave. How easily does her ladyship encourage or disMiss~a lover! In spite of this release, Frederica still looks unhappy: still fearful, perhaps, of her mother's anger; and though dreading my brother's departure, jealous, it may be, of his staying. I see how closely she observes him and Lady~Susan, poor girl! I have now no hope for her. There is not a chance of her affection being returned. He thinks very differently of her from what he used to do; he does her some justice, but his reconciliation with her mother precludes every dearer hope. Prepare, my dear mother, for the worst! The probability of their marrying is surely heightened! He is more securely hers than ever. When that wretched event takes place, Frederica must belong wholly to us. I am thankful that my last letter will precede this by so little, as every moment that you can be saved from feeling a joy which leads only to disappointment is of consequence. 

\closeletter[Yours ever, \&c.,]{Catherine Vernon.} 
\end{mail}