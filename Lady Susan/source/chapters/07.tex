\chapter{Lady~Susan Vernon to Mrs~Johnson}
  
  \begin{mail}{Churchhill.}{My dear Alicia,}
You are very good in taking notice of Frederica, and I am grateful for it as a mark of your friendship; but as I cannot have any doubt of the warmth of your affection, I am far from exacting so heavy a sacrifice. She is a stupid girl, and has nothing to recommend her. I would not, therefore, on my account, have you encumber one moment of your precious time by sending for her to Edward Street, especially as every visit is so much deducted from the grand affair of education, which I really wish to have attended to while she remains at Miss~Summers's. I want her to play and sing with some portion of taste and a good deal of assurance, as she has my hand and arm and a tolerable voice. I was so much indulged in my infant years that I was never obliged to attend to anything, and consequently am without the accomplishments which are now necessary to finish a pretty woman. Not that I am an advocate for the prevailing fashion of acquiring a perfect knowledge of all languages, arts, and sciences. It is throwing time away to be mistress of French, Italian, and German: music, singing, and drawing, \&c., will gain a woman some applause, but will not add one lover to her list—grace and manner, after all, are of the greatest importance. I do not mean, therefore, that Frederica's acquirements should be more than superficial, and I flatter myself that she will not remain long enough at school to understand anything thoroughly. I hope to see her the wife of Sir~James within a twelvemonth. You know on what I ground my hope, and it is certainly a good foundation, for school must be very humiliating to a girl of Frederica's age. And, by-the-by, you had better not invite her any more on that account, as I wish her to find her situation as unpleasant as possible. I am sure of Sir~James at any time, and could make him renew his application by a line. I shall trouble you meanwhile to prevent his forming any other attachment when he comes to town. Ask him to your house occasionally, and talk to him of Frederica, that he may not forget her. Upon the whole, I commend my own conduct in this affair extremely, and regard it as a very happy instance of circumspection and tenderness. Some mothers would have insisted on their daughter's accepting so good an offer on the first overture; but I could not reconcile it to myself to force Frederica into a marriage from which her heart revolted, and instead of adopting so harsh a measure merely propose to make it her own choice, by rendering her thoroughly uncomfortable till she does accept him—but enough of this tiresome girl. You may well wonder how I contrive to pass my time here, and for the first week it was insufferably dull. Now, however, we begin to mend, our party is enlarged by Mrs~Vernon's brother, a handsome young man, who promises me some amusement. There is something about him which rather interests me, a sort of sauciness and familiarity which I shall teach him to correct. He is lively, and seems clever, and when I have inspired him with greater respect for me than his sister's kind offices have implanted, he may be an agreeable flirt. There is exquisite pleasure in subduing an insolent spirit, in making a person predetermined to dislike acknowledge one's superiority. I have disconcerted him already by my calm reserve, and it shall be my endeavour to humble the pride of these self important De Courcys still lower, to convince Mrs~Vernon that her sisterly cautions have been bestowed in vain, and to persuade Reginald that she has scandalously belied me. This project will serve at least to amuse me, and prevent my feeling so acutely this dreadful separation from you and all whom I love.

\closeletter[Yours ever,]{S. Vernon.} 
\end{mail}