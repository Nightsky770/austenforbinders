\chapter{Sir~Reginald De Courcy to his Son}
  
  \begin{mail}{Parklands.}{}

I know that young men in general do not admit of any enquiry even from their nearest relations into affairs of the heart, but I hope, my dear Reginald, that you will be superior to such as allow nothing for a father's anxiety, and think themselves privileged to refuse him their confidence and slight his advice. You must be sensible that as an only son, and the representative of an ancient family, your conduct in life is most interesting to your connections; and in the very important concern of marriage especially, there is everything at stake—your own happiness, that of your parents, and the credit of your name. I do not suppose that you would deliberately form an absolute engagement of that nature without acquainting your mother and myself, or at least, without being convinced that we should approve of your choice; but I cannot help fearing that you may be drawn in, by the Lady~who has lately attached you, to a marriage which the whole of your family, far and near, must highly reprobate. Lady~Susan's age is itself a material objection, but her want of character is one so much more serious, that the difference of even twelve years becomes in comparison of small amount. Were you not blinded by a sort of fascination, it would be ridiculous in me to repeat the instances of great misconduct on her side so very generally known.

Her neglect of her husband, her encouragement of other men, her extravagance and dissipation, were so gross and notorious that no one could be ignorant of them at the time, nor can now have forgotten them. To our family she has always been represented in softened colours by the benevolence of Mr~Charles Vernon, and yet, in spite of his generous endeavours to excuse her, we know that she did, from the most selfish motives, take all possible pains to prevent his marriage with Catherine.

My years and increasing infirmities make me very desirous of seeing you settled in the world. To the fortune of a wife, the goodness of my own will make me indifferent, but her family and character must be equally unexceptionable. When your choice is fixed so that no objection can be made to it, then I can promise you a ready and cheerful consent; but it is my duty to oppose a match which deep art only could render possible, and must in the end make wretched. It is possible her behaviour may arise only from vanity, or the wish of gaining the admiration of a man whom she must imagine to be particularly prejudiced against her; but it is more likely that she should aim at something further. She is poor, and may naturally seek an alliance which must be advantageous to herself; you know your own rights, and that it is out of my power to prevent your inheriting the family estate. My ability of distressing you during my life would be a species of revenge to which I could hardly stoop under any circumstances.

I honestly tell you my sentiments and intentions: I do not wish to work on your fears, but on your sense and affection. It would destroy every comfort of my life to know that you were married to Lady~Susan Vernon; it would be the death of that honest pride with which I have hitherto considered my son; I should blush to see him, to hear of him, to think of him. I may perhaps do no good but that of relieving my own mind by this letter, but I felt it my duty to tell you that your partiality for Lady~Susan is no secret to your friends, and to warn you against her. I should be glad to hear your reasons for disbelieving Mr~Smith's intelligence; you had no doubt of its authenticity a month ago. If you can give me your assurance of having no design beyond enjoying the conversation of a clever woman for a short period, and of yielding admiration only to her beauty and abilities, without being blinded by them to her faults, you will restore me to happiness; but, if you cannot do this, explain to me, at least, what has occasioned so great an alteration in your opinion of her. 

\closeletter[I am, \&c., \&c,]{Reginald De Courcy.}
\end{mail}