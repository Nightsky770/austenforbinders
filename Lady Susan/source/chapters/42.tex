\chapter{Conclusion} 

\lettrine[lraise=0.3]{T}{his} correspondence, by a meeting between some of the parties, and a separation between the others, could not, to the great detriment of the Post Office revenue, be continued any longer. Very little assistance to the State could be derived from the epistolary intercourse of Mrs~Vernon and her niece; for the former soon perceived, by the style of Frederica's letters, that they were written under her mother's inspection! and therefore, deferring all particular enquiry till she could make it personally in London, ceased writing minutely or often. Having learnt enough, in the meanwhile, from her open-hearted brother, of what had passed between him and Lady~Susan to sink the latter lower than ever in her opinion, she was proportionably more anxious to get Frederica removed from such a mother, and placed under her own care; and, though with little hope of success, was resolved to leave nothing unattempted that might offer a chance of obtaining her sister-in-law's consent to it. Her anxiety on the subject made her press for an early visit to London; and Mr~Vernon, who, as it must already have appeared, lived only to do whatever he was desired, soon found some accommodating business to call him thither. With a heart full of the matter, Mrs~Vernon waited on Lady~Susan shortly after her arrival in town, and was met with such an easy and cheerful affection, as made her almost turn from her with horror. No remembrance of Reginald, no consciousness of guilt, gave one look of embarrassment; she was in excellent spirits, and seemed eager to show at once by every possible attention to her brother and sister her sense of their kindness, and her pleasure in their society. Frederica was no more altered than Lady~Susan; the same restrained manners, the same timid look in the presence of her mother as heretofore, assured her aunt of her situation being uncomfortable, and confirmed her in the plan of altering it. No unkindness, however, on the part of Lady~Susan appeared. Persecution on the subject of Sir~James was entirely at an end; his name merely mentioned to say that he was not in London; and indeed, in all her conversation, she was solicitous only for the welfare and improvement of her daughter, acknowledging, in terms of grateful delight, that Frederica was now growing every day more and more what a parent could desire. Mrs~Vernon, surprized and incredulous, knew not what to suspect, and, without any change in her own views, only feared greater difficulty in accomplishing them. The first hope of anything better was derived from Lady~Susan's asking her whether she thought Frederica looked quite as well as she had done at Churchhill, as she must confess herself to have sometimes an anxious doubt of London's perfectly agreeing with her. Mrs~Vernon, encouraging the doubt, directly proposed her niece's returning with them into the country. Lady~Susan was unable to express her sense of such kindness, yet knew not, from a variety of reasons, how to part with her daughter; and as, though her own plans were not yet wholly fixed, she trusted it would ere long be in her power to take Frederica into the country herself, concluded by declining entirely to profit by such unexampled attention. Mrs~Vernon persevered, however, in the offer of it, and though Lady~Susan continued to resist, her resistance in the course of a few days seemed somewhat less formidable. The lucky alarm of an influenza decided what might not have been decided quite so soon. Lady~Susan's maternal fears were then too much awakened for her to think of anything but Frederica's removal from the risk of infection; above all disorders in the world she most dreaded the influenza for her daughter's constitution!

Frederica returned to Churchhill with her uncle and aunt; and three weeks afterwards, Lady~Susan announced her being married to Sir~James Martin. Mrs~Vernon was then convinced of what she had only suspected before, that she might have spared herself all the trouble of urging a removal which Lady~Susan had doubtless resolved on from the first. Frederica's visit was nominally for six weeks, but her mother, though inviting her to return in one or two affectionate letters, was very ready to oblige the whole party by consenting to a prolongation of her stay, and in the course of two months ceased to write of her absence, and in the course of two more to write to her at all. Frederica was therefore fixed in the family of her uncle and aunt till such time as Reginald De Courcy could be talked, flattered, and finessed into an affection for her which, allowing leisure for the conquest of his attachment to her mother, for his abjuring all future attachments, and detesting the sex, might be reasonably looked for in the course of a twelvemonth. Three months might have done it in general, but Reginald's feelings were no less lasting than lively. Whether Lady~Susan was or was not happy in her second choice, I do not see how it can ever be ascertained; for who would take her assurance of it on either side of the question? The world must judge from probabilities; she had nothing against her but her husband, and her conscience. Sir~James may seem to have drawn a harder lot than mere folly merited; I leave him, therefore, to all the pity that anybody can give him. For myself, I confess that \textit{I} can pity only Miss~Mainwaring; who, coming to town, and putting herself to an expense in clothes which impoverished her for two years, on purpose to secure him, was defrauded of her due by a woman ten years older than herself. 