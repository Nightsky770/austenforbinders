\chapter{Lady~Susan Vernon to Mrs~Johnson}
  
  \begin{mail}{Churchhill.}{}

I am much obliged to you, my dear Friend, for your advice respecting Mr~De Courcy, which I know was given with the full conviction of its expediency, though I am not quite determined on following it. I cannot easily resolve on anything so serious as marriage; especially as I am not at present in want of money, and might perhaps, till the old gentleman's death, be very little benefited by the match. It is true that I am vain enough to believe it within my reach. I have made him sensible of my power, and can now enjoy the pleasure of triumphing over a mind prepared to dislike me, and prejudiced against all my past actions. His sister, too, is, I hope, convinced how little the ungenerous representations of anyone to the disadvantage of another will avail when opposed by the immediate influence of intellect and manner. I see plainly that she is uneasy at my progress in the good opinion of her brother, and conclude that nothing will be wanting on her part to counteract me; but having once made him doubt the justice of her opinion of me, I think I may defy her. It has been delightful to me to watch his advances towards intimacy, especially to observe his altered manner in consequence of my repressing by the cool dignity of my deportment his insolent approach to direct familiarity. My conduct has been equally guarded from the first, and I never behaved less like a coquette in the whole course of my life, though perhaps my desire of dominion was never more decided. I have subdued him entirely by sentiment and serious conversation, and made him, I may venture to say, at least half in love with me, without the semblance of the most commonplace flirtation. Mrs~Vernon's consciousness of deserving every sort of revenge that it can be in my power to inflict for her ill-offices could alone enable her to perceive that I am actuated by any design in behaviour so gentle and unpretending. Let her think and act as she chooses, however. I have never yet found that the advice of a sister could prevent a young man's being in love if he chose. We are advancing now to some kind of confidence, and in short are likely to be engaged in a sort of platonic friendship. On my side you may be sure of its never being more, for if I were not attached to another person as much as I can be to anyone, I should make a point of not bestowing my affection on a man who had dared to think so meanly of me. Reginald has a good figure and is not unworthy the praise you have heard given him, but is still greatly inferior to our friend at Langford. He is less polished, less insinuating than Mainwaring, and is comparatively deficient in the power of saying those delightful things which put one in good humour with oneself and all the world. He is quite agreeable enough, however, to afford me amusement, and to make many of those hours pass very pleasantly which would otherwise be spent in endeavouring to overcome my sister-in-law's reserve, and listening to the insipid talk of her husband. Your account of Sir~James is most satisfactory, and I mean to give Miss~Frederica a hint of my intentions very soon. 

\closeletter[Yours, \&c.,]{S. Vernon.} 
\end{mail}