\chapter{Lady~Susan Vernon to Mrs~Johnson}
  
  \begin{mail}{Churchhill.}{}

I received your note, my dear Alicia, just before I left town, and rejoice to be assured that Mr~Johnson suspected nothing of your engagement the evening before. It is undoubtedly better to deceive him entirely, and since he will be stubborn he must be tricked. I arrived here in safety, and have no reason to complain of my reception from Mr~Vernon; but I confess myself not equally satisfied with the behaviour of his lady. She is perfectly well-bred, indeed, and has the air of a woman of fashion, but her manners are not such as can persuade me of her being prepossessed in my favour. I wanted her to be delighted at seeing me. I was as amiable as possible on the occasion, but all in vain. She does not like me. To be sure, when we consider that I \textit{did} take some pains to prevent my brother-in-law's marrying her, this want of cordiality is not very surprizing, and yet it shows an illiberal and vindictive spirit to resent a project which influenced me six years ago, and which never succeeded at last.

I am sometimes disposed to repent that I did not let Charles buy Vernon Castle, when we were obliged to sell it; but it was a trying circumstance, especially as the sale took place exactly at the time of his marriage; and everybody ought to respect the delicacy of those feelings which could not endure that my husband's dignity should be lessened by his younger brother's having possession of the family estate. Could matters have been so arranged as to prevent the necessity of our leaving the castle, could we have lived with Charles and kept him single, I should have been very far from persuading my husband to dispose of it elsewhere; but Charles was on the point of marrying Miss~De Courcy, and the event has justified me. Here are children in abundance, and what benefit could have accrued to me from his purchasing Vernon? My having prevented it may perhaps have given his wife an unfavourable impression, but where there is a disposition to dislike, a motive will never be wanting; and as to money matters it has not withheld him from being very useful to me. I really have a regard for him, he is so easily imposed upon! The house is a good one, the furniture fashionable, and everything announces plenty and elegance. Charles is very rich I am sure; when a man has once got his name in a banking-house he rolls in money; but they do not know what to do with it, keep very little company, and never go to London but on business. We shall be as stupid as possible. I mean to win my sister-in-law's heart through the children; I know all their names already, and am going to attach myself with the greatest sensibility to one in particular, a young Frederic, whom I take on my lap and sigh over for his dear uncle's sake.

Poor Mainwaring! I need not tell you how much I Miss~him, how perpetually he is in my thoughts. I found a dismal letter from him on my arrival here, full of complaints of his wife and sister, and lamentations on the cruelty of his fate. I passed off the letter as his wife's, to the Vernons, and when I write to him it must be under cover to you. 

\closeletter[Ever yours,]{S. Vernon.} 
\end{mail}