\chapter{Mr~De Courcy to Sir~Reginald}
  
  \begin{mail}{Churchhill.}{My dear Sir,}

I have this moment received your letter, which has given me more astonishment than I ever felt before. I am to thank my sister, I suppose, for having represented me in such a light as to injure me in your opinion, and give you all this alarm. I know not why she should choose to make herself and her family uneasy by apprehending an event which no one but herself, I can affirm, would ever have thought possible. To impute such a design to Lady~Susan would be taking from her every claim to that excellent understanding which her bitterest enemies have never denied her; and equally low must sink my pretensions to common sense if I am suspected of matrimonial views in my behaviour to her. Our difference of age must be an insuperable objection, and I entreat you, my dear father, to quiet your mind, and no longer harbour a suspicion which cannot be more injurious to your own peace than to our understandings. I can have no other view in remaining with Lady~Susan, than to enjoy for a short time (as you have yourself expressed it) the conversation of a woman of high intellectual powers. If Mrs~Vernon would allow something to my affection for herself and her husband in the length of my visit, she would do more justice to us all; but my sister is unhappily prejudiced beyond the hope of conviction against Lady~Susan. From an attachment to her husband, which in itself does honour to both, she cannot forgive the endeavours at preventing their union, which have been attributed to selfishness in Lady~Susan; but in this case, as well as in many others, the world has most grossly injured that lady, by supposing the worst where the motives of her conduct have been doubtful. Lady~Susan had heard something so materially to the disadvantage of my sister as to persuade her that the happiness of Mr~Vernon, to whom she was always much attached, would be wholly destroyed by the marriage. And this circumstance, while it explains the true motives of Lady~Susan's conduct, and removes all the blame which has been so lavished on her, may also convince us how little the general report of anyone ought to be credited; since no character, however upright, can escape the malevolence of slander. If my sister, in the security of retirement, with as little opportunity as inclination to do evil, could not avoid censure, we must not rashly condemn those who, living in the world and surrounded with temptations, should be accused of errors which they are known to have the power of committing.

I blame myself severely for having so easily believed the slanderous tales invented by Charles Smith to the prejudice of Lady~Susan, as I am now convinced how greatly they have traduced her. As to Mrs~Mainwaring's jealousy it was totally his own invention, and his account of her attaching Miss~Mainwaring's lover was scarcely better founded. Sir~James Martin had been drawn in by that young Lady~to pay her some attention; and as he is a man of fortune, it was easy to see \textit{her} views extended to marriage. It is well known that Miss~M. is absolutely on the catch for a husband, and no one therefore can pity her for losing, by the superior attractions of another woman, the chance of being able to make a worthy man completely wretched. Lady~Susan was far from intending such a conquest, and on finding how warmly Miss~Mainwaring resented her lover's defection, determined, in spite of Mr~and Mrs~Mainwaring's most urgent entreaties, to leave the family. I have reason to imagine she did receive serious proposals from Sir~James, but her removing from Langford immediately on the discovery of his attachment, must acquit her on that article with any mind of common candour. You will, I am sure, my dear Sir, feel the truth of this, and will hereby learn to do justice to the character of a very injured woman. I know that Lady~Susan in coming to Churchhill was governed only by the most honourable and amiable intentions; her prudence and economy are exemplary, her regard for Mr~Vernon equal even to \textit{his} deserts; and her wish of obtaining my sister's good opinion merits a better return than it has received. As a mother she is unexceptionable; her solid affection for her child is shown by placing her in hands where her education will be properly attended to; but because she has not the blind and weak partiality of most mothers, she is accused of wanting maternal tenderness. Every person of sense, however, will know how to value and commend her well-directed affection, and will join me in wishing that Frederica Vernon may prove more worthy than she has yet done of her mother's tender care. I have now, my dear father, written my real sentiments of Lady~Susan; you will know from this letter how highly I admire her abilities, and esteem her character; but if you are not equally convinced by my full and solemn assurance that your fears have been most idly created, you will deeply mortify and distress me. 

\closeletter[I am, \&c., \&c.,]{R. De Courcy.}
\end{mail}