\chapter{Lady~Susan to Mrs~Johnson}
  
  \begin{mail}{Churchhill.}{}

You will be eager, I know, to hear something further of Frederica, and perhaps may think me negligent for not writing before. She arrived with her uncle last Thursday fortnight, when, of course, I lost no time in demanding the cause of her behaviour; and soon found myself to have been perfectly right in attributing it to my own letter. The prospect of it frightened her so thoroughly, that, with a mixture of true girlish perverseness and folly, she resolved on getting out of the house and proceeding directly by the stage to her friends, the Clarkes; and had really got as far as the length of two streets in her journey when she was fortunately missed, pursued, and overtaken. Such was the first distinguished exploit of Miss~Frederica Vernon; and, if we consider that it was achieved at the tender age of sixteen, we shall have room for the most flattering prognostics of her future renown. I am excessively provoked, however, at the parade of propriety which prevented Miss~Summers from keeping the girl; and it seems so extraordinary a piece of nicety, considering my daughter's family connections, that I can only suppose the Lady~to be governed by the fear of never getting her money. Be that as it may, however, Frederica is returned on my hands; and, having nothing else to employ her, is busy in pursuing the plan of romance begun at Langford. She is actually falling in love with Reginald De Courcy! To disobey her mother by refusing an unexceptionable offer is not enough; her affections must also be given without her mother's approbation. I never saw a girl of her age bid fairer to be the sport of mankind. Her feelings are tolerably acute, and she is so charmingly artless in their display as to afford the most reasonable hope of her being ridiculous, and despised by every man who sees her.

Artlessness will never do in love matters; and that girl is born a simpleton who has it either by nature or affectation. I am not yet certain that Reginald sees what she is about, nor is it of much consequence. She is now an object of indifference to him, and she would be one of contempt were he to understand her emotions. Her beauty is much admired by the Vernons, but it has no effect on him. She is in high favour with her aunt altogether, because she is so little like myself, of course. She is exactly the companion for Mrs~Vernon, who dearly loves to be firm, and to have all the sense and all the wit of the conversation to herself: Frederica will never eclipse her. When she first came I was at some pains to prevent her seeing much of her aunt; but I have relaxed, as I believe I may depend on her observing the rules I have laid down for their discourse. But do not imagine that with all this lenity I have for a moment given up my plan of her marriage. No; I am unalterably fixed on this point, though I have not yet quite decided on the manner of bringing it about. I should not chuse to have the business brought on here, and canvassed by the wise heads of Mr~and Mrs~Vernon; and I cannot just now afford to go to town. Miss~Frederica must therefore wait a little. 

\closeletter[Yours ever,]{S. Vernon.} 
\end{mail}