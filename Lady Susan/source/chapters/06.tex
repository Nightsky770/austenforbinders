\chapter{Mrs~Vernon to Mr~De Courcy}
  
  \begin{mail}{Churchhill.}{}

Well, my dear Reginald, I have seen this dangerous creature, and must give you some description of her, though I hope you will soon be able to form your own judgment. She is really excessively pretty; however you may choose to question the allurements of a Lady~no longer young, I must, for my own part, declare that I have seldom seen so lovely a woman as Lady~Susan. She is delicately fair, with fine grey eyes and dark eyelashes; and from her appearance one would not suppose her more than five and twenty, though she must in fact be ten years older. I was certainly not disposed to admire her, though always hearing she was beautiful; but I cannot help feeling that she possesses an uncommon union of symmetry, brilliancy, and grace. Her address to me was so gentle, frank, and even affectionate, that, if I had not known how much she has always disliked me for marrying Mr~Vernon, and that we had never met before, I should have imagined her an attached friend. One is apt, I believe, to connect assurance of manner with coquetry, and to expect that an impudent address will naturally attend an impudent mind; at least I was myself prepared for an improper degree of confidence in Lady~Susan; but her countenance is absolutely sweet, and her voice and manner winningly mild. I am sorry it is so, for what is this but deceit? Unfortunately, one knows her too well. She is clever and agreeable, has all that knowledge of the world which makes conversation easy, and talks very well, with a happy command of language, which is too often used, I believe, to make black appear white. She has already almost persuaded me of her being warmly attached to her daughter, though I have been so long convinced to the contrary. She speaks of her with so much tenderness and anxiety, lamenting so bitterly the neglect of her education, which she represents however as wholly unavoidable, that I am forced to recollect how many successive springs her ladyship spent in town, while her daughter was left in Staffordshire to the care of servants, or a governess very little better, to prevent my believing what she says.

If her manners have so great an influence on my resentful heart, you may judge how much more strongly they operate on Mr~Vernon's generous temper. I wish I could be as well satisfied as he is, that it was really her choice to leave Langford for Churchhill; and if she had not stayed there for months before she discovered that her friend's manner of living did not suit her situation or feelings, I might have believed that concern for the loss of such a husband as Mr~Vernon, to whom her own behaviour was far from unexceptionable, might for a time make her wish for retirement. But I cannot forget the length of her visit to the Mainwarings, and when I reflect on the different mode of life which she led with them from that to which she must now submit, I can only suppose that the wish of establishing her reputation by following though late the path of propriety, occasioned her removal from a family where she must in reality have been particularly happy. Your friend Mr~Smith's story, however, cannot be quite correct, as she corresponds regularly with Mrs~Mainwaring. At any rate it must be exaggerated. It is scarcely possible that two men should be so grossly deceived by her at once. 

\closeletter[Yours, \&c.,]{Catherine Vernon}
\end{mail}