\chapter[Chapter \thechapter]{} 

 \lettrine[lraise=0.3]{M}{iss} Crawford's uneasiness was much lightened by this conversation, and she walked home again in spirits which might have defied almost another week of the same small party in the same bad weather, had they been put to the proof; but as that very evening brought her brother down from London again in quite, or more than quite, his usual cheerfulness, she had nothing farther to try her own. His still refusing to tell her what he had gone for was but the promotion of gaiety; a day before it might have irritated, but now it was a pleasant joke—suspected only of concealing something planned as a pleasant surprise to herself. And the next day \textit{did}  bring a surprise to her. Henry had said he should just go and ask the Bertrams how they did, and be back in ten minutes, but he was gone above an hour; and when his sister, who had been waiting for him to walk with her in the garden, met him at last most impatiently in the sweep, and cried out, <My dear Henry, where can you have been all this time?> he had only to say that he had been sitting with Lady Bertram and Fanny.

<Sitting with them an hour and a half!> exclaimed Mary.

But this was only the beginning of her surprise.

<Yes, Mary,> said he, drawing her arm within his, and walking along the sweep as if not knowing where he was: <I could not get away sooner; Fanny looked so lovely! I am quite determined, Mary. My mind is entirely made up. Will it astonish you? No: you must be aware that I am quite determined to marry Fanny Price.>

The surprise was now complete; for, in spite of whatever his consciousness might suggest, a suspicion of his having any such views had never entered his sister's imagination; and she looked so truly the astonishment she felt, that he was obliged to repeat what he had said, and more fully and more solemnly. The conviction of his determination once admitted, it was not unwelcome. There was even pleasure with the surprise. Mary was in a state of mind to rejoice in a connexion with the Bertram family, and to be not displeased with her brother's marrying a little beneath him.

<Yes, Mary,> was Henry's concluding assurance. <I am fairly caught. You know with what idle designs I began; but this is the end of them. I have, I flatter myself, made no inconsiderable progress in her affections; but my own are entirely fixed.>

<Lucky, lucky girl!> cried Mary, as soon as she could speak; <what a match for her! My dearest Henry, this must be my \textit{first}  feeling; but my \textit{second}, which you shall have as sincerely, is, that I approve your choice from my soul, and foresee your happiness as heartily as I wish and desire it. You will have a sweet little wife; all gratitude and devotion. Exactly what you deserve. What an amazing match for her! Mrs~Norris often talks of her luck; what will she say now? The delight of all the family, indeed! And she has some \textit{true}  friends in it! How \textit{they}  will rejoice! But tell me all about it! Talk to me for ever. When did you begin to think seriously about her?>

Nothing could be more impossible than to answer such a question, though nothing could be more agreeable than to have it asked. <How the pleasing plague had stolen on him> he could not say; and before he had expressed the same sentiment with a little variation of words three times over, his sister eagerly interrupted him with, <Ah, my dear Henry, and this is what took you to London! This was your business! You chose to consult the Admiral before you made up your mind.>

But this he stoutly denied. He knew his uncle too well to consult him on any matrimonial scheme. The Admiral hated marriage, and thought it never pardonable in a young man of independent fortune.

<When Fanny is known to him,> continued Henry, <he will doat on her. She is exactly the woman to do away every prejudice of such a man as the Admiral, for she is exactly such a woman as he thinks does not exist in the world. She is the very impossibility he would describe, if indeed he has now delicacy of language enough to embody his own ideas. But till it is absolutely settled—settled beyond all interference, he shall know nothing of the matter. No, Mary, you are quite mistaken. You have not discovered my business yet.>

<Well, well, I am satisfied. I know now to whom it must relate, and am in no hurry for the rest. Fanny Price! wonderful, quite wonderful! That Mansfield should have done so much for—that \textit{you}  should have found your fate in Mansfield! But you are quite right; you could not have chosen better. There is not a better girl in the world, and you do not want for fortune; and as to her connexions, they are more than good. The Bertrams are undoubtedly some of the first people in this country. She is niece to Sir~Thomas Bertram; that will be enough for the world. But go on, go on. Tell me more. What are your plans? Does she know her own happiness?>

<No.>

<What are you waiting for?>

<For—for very little more than opportunity. Mary, she is not like her cousins; but I think I shall not ask in vain.>

<Oh no! you cannot. Were you even less pleasing—supposing her not to love you already (of which, however, I can have little doubt)—you would be safe. The gentleness and gratitude of her disposition would secure her all your own immediately. From my soul I do not think she would marry you \textit{without}  love; that is, if there is a girl in the world capable of being uninfluenced by ambition, I can suppose it her; but ask her to love you, and she will never have the heart to refuse.>

As soon as her eagerness could rest in silence, he was as happy to tell as she could be to listen; and a conversation followed almost as deeply interesting to her as to himself, though he had in fact nothing to relate but his own sensations, nothing to dwell on but Fanny's charms. Fanny's beauty of face and figure, Fanny's graces of manner and goodness of heart, were the exhaustless theme. The gentleness, modesty, and sweetness of her character were warmly expatiated on; that sweetness which makes so essential a part of every woman's worth in the judgment of man, that though he sometimes loves where it is not, he can never believe it absent. Her temper he had good reason to depend on and to praise. He had often seen it tried. Was there one of the family, excepting Edmund, who had not in some way or other continually exercised her patience and forbearance? Her affections were evidently strong. To see her with her brother! What could more delightfully prove that the warmth of her heart was equal to its gentleness? What could be more encouraging to a man who had her love in view? Then, her understanding was beyond every suspicion, quick and clear; and her manners were the mirror of her own modest and elegant mind. Nor was this all. Henry Crawford had too much sense not to feel the worth of good principles in a wife, though he was too little accustomed to serious reflection to know them by their proper name; but when he talked of her having such a steadiness and regularity of conduct, such a high notion of honour, and such an observance of decorum as might warrant any man in the fullest dependence on her faith and integrity, he expressed what was inspired by the knowledge of her being well principled and religious.

<I could so wholly and absolutely confide in her,> said he; <and \textit{that}  is what I want.>

Well might his sister, believing as she really did that his opinion of Fanny Price was scarcely beyond her merits, rejoice in her prospects.

<The more I think of it,> she cried, <the more am I convinced that you are doing quite right; and though I should never have selected Fanny Price as the girl most likely to attach you, I am now persuaded she is the very one to make you happy. Your wicked project upon her peace turns out a clever thought indeed. You will both find your good in it.>

<It was bad, very bad in me against such a creature; but I did not know her then; and she shall have no reason to lament the hour that first put it into my head. I will make her very happy, Mary; happier than she has ever yet been herself, or ever seen anybody else. I will not take her from Northamptonshire. I shall let Everingham, and rent a place in this neighbourhood; perhaps Stanwix Lodge. I shall let a seven years' lease of Everingham. I am sure of an excellent tenant at half a word. I could name three people now, who would give me my own terms and thank me.>

<Ha!> cried Mary; <settle in Northamptonshire! That is pleasant! Then we shall be all together.>

When she had spoken it, she recollected herself, and wished it unsaid; but there was no need of confusion; for her brother saw her only as the supposed inmate of Mansfield parsonage, and replied but to invite her in the kindest manner to his own house, and to claim the best right in her.

<You must give us more than half your time,> said he. <I cannot admit Mrs~Grant to have an equal claim with Fanny and myself, for we shall both have a right in you. Fanny will be so truly your sister!>

Mary had only to be grateful and give general assurances; but she was now very fully purposed to be the guest of neither brother nor sister many months longer.

<You will divide your year between London and Northamptonshire?>

<Yes.>

<That's right; and in London, of course, a house of your own: no longer with the Admiral. My dearest Henry, the advantage to you of getting away from the Admiral before your manners are hurt by the contagion of his, before you have contracted any of his foolish opinions, or learned to sit over your dinner as if it were the best blessing of life! \textit{You}  are not sensible of the gain, for your regard for him has blinded you; but, in my estimation, your marrying early may be the saving of you. To have seen you grow like the Admiral in word or deed, look or gesture, would have broken my heart.>

<Well, well, we do not think quite alike here. The Admiral has his faults, but he is a very good man, and has been more than a father to me. Few fathers would have let me have my own way half so much. You must not prejudice Fanny against him. I must have them love one another.>

Mary refrained from saying what she felt, that there could not be two persons in existence whose characters and manners were less accordant: time would discover it to him; but she could not help \textit{this}  reflection on the Admiral. <Henry, I think so highly of Fanny Price, that if I could suppose the next Mrs~Crawford would have half the reason which my poor ill-used aunt had to abhor the very name, I would prevent the marriage, if possible; but I know you: I know that a wife you \textit{loved}  would be the happiest of women, and that even when you ceased to love, she would yet find in you the liberality and good-breeding of a gentleman.>

The impossibility of not doing everything in the world to make Fanny Price happy, or of ceasing to love Fanny Price, was of course the groundwork of his eloquent answer.

<Had you seen her this morning, Mary,> he continued, <attending with such ineffable sweetness and patience to all the demands of her aunt's stupidity, working with her, and for her, her colour beautifully heightened as she leant over the work, then returning to her seat to finish a note which she was previously engaged in writing for that stupid woman's service, and all this with such unpretending gentleness, so much as if it were a matter of course that she was not to have a moment at her own command, her hair arranged as neatly as it always is, and one little curl falling forward as she wrote, which she now and then shook back, and in the midst of all this, still speaking at intervals to \textit{me}, or listening, and as if she liked to listen, to what I said. Had you seen her so, Mary, you would not have implied the possibility of her power over my heart ever ceasing.>

<My dearest Henry,> cried Mary, stopping short, and smiling in his face, <how glad I am to see you so much in love! It quite delights me. But what will Mrs~Rushworth and Julia say?>

<I care neither what they say nor what they feel. They will now see what sort of woman it is that can attach me, that can attach a man of sense. I wish the discovery may do them any good. And they will now see their cousin treated as she ought to be, and I wish they may be heartily ashamed of their own abominable neglect and unkindness. They will be angry,> he added, after a moment's silence, and in a cooler tone; <Mrs~Rushworth will be very angry. It will be a bitter pill to her; that is, like other bitter pills, it will have two moments' ill flavour, and then be swallowed and forgotten; for I am not such a coxcomb as to suppose her feelings more lasting than other women's, though \textit{I}  was the object of them. Yes, Mary, my Fanny will feel a difference indeed: a daily, hourly difference, in the behaviour of every being who approaches her; and it will be the completion of my happiness to know that I am the doer of it, that I am the person to give the consequence so justly her due. Now she is dependent, helpless, friendless, neglected, forgotten.>

<Nay, Henry, not by all; not forgotten by all; not friendless or forgotten. Her cousin Edmund never forgets her.>

<Edmund! True, I believe he is, generally speaking, kind to her, and so is Sir~Thomas in his way; but it is the way of a rich, superior, long-worded, arbitrary uncle. What can Sir~Thomas and Edmund together do, what \textit{do}  they do for her happiness, comfort, honour, and dignity in the world, to what I \textit{shall}  do?> 