\chapter[Chapter \thechapter]{} 

 \lettrine[lraise=0.3]{T}{he} first event of any importance in the family was the death of Mr~Norris, which happened when Fanny was about fifteen, and necessarily introduced alterations and novelties. Mrs~Norris, on quitting the Parsonage, removed first to the Park, and afterwards to a small house of Sir~Thomas's in the village, and consoled herself for the loss of her husband by considering that she could do very well without him; and for her reduction of income by the evident necessity of stricter economy.

The living was hereafter for Edmund; and, had his uncle died a few years sooner, it would have been duly given to some friend to hold till he were old enough for orders. But Tom's extravagance had, previous to that event, been so great as to render a different disposal of the next presentation necessary, and the younger brother must help to pay for the pleasures of the elder. There was another family living actually held for Edmund; but though this circumstance had made the arrangement somewhat easier to Sir~Thomas's conscience, he could not but feel it to be an act of injustice, and he earnestly tried to impress his eldest son with the same conviction, in the hope of its producing a better effect than anything he had yet been able to say or do.

<I blush for you, Tom,> said he, in his most dignified manner; <I blush for the expedient which I am driven on, and I trust I may pity your feelings as a brother on the occasion. You have robbed Edmund for ten, twenty, thirty years, perhaps for life, of more than half the income which ought to be his. It may hereafter be in my power, or in yours (I hope it will), to procure him better preferment; but it must not be forgotten that no benefit of that sort would have been beyond his natural claims on us, and that nothing can, in fact, be an equivalent for the certain advantage which he is now obliged to forego through the urgency of your debts.>

Tom listened with some shame and some sorrow; but escaping as quickly as possible, could soon with cheerful selfishness reflect, firstly, that he had not been half so much in debt as some of his friends; secondly, that his father had made a most tiresome piece of work of it; and, thirdly, that the future incumbent, whoever he might be, would, in all probability, die very soon.

On Mr~Norris's death the presentation became the right of a Dr~Grant, who came consequently to reside at Mansfield; and on proving to be a hearty man of forty-five, seemed likely to disappoint Mr~Bertram's calculations. But <no, he was a short-necked, apoplectic sort of fellow, and, plied well with good things, would soon pop off.>

He had a wife about fifteen years his junior, but no children; and they entered the neighbourhood with the usual fair report of being very respectable, agreeable people.

The time was now come when Sir~Thomas expected his sister-in-law to claim her share in their niece, the change in Mrs~Norris's situation, and the improvement in Fanny's age, seeming not merely to do away any former objection to their living together, but even to give it the most decided eligibility; and as his own circumstances were rendered less fair than heretofore, by some recent losses on his West India estate, in addition to his eldest son's extravagance, it became not undesirable to himself to be relieved from the expense of her support, and the obligation of her future provision. In the fullness of his belief that such a thing must be, he mentioned its probability to his wife; and the first time of the subject's occurring to her again happening to be when Fanny was present, she calmly observed to her, <So, Fanny, you are going to leave us, and live with my sister. How shall you like it?>

Fanny was too much surprised to do more than repeat her aunt's words, <Going to leave you?>

<Yes, my dear; why should you be astonished? You have been five years with us, and my sister always meant to take you when Mr~Norris died. But you must come up and tack on my patterns all the same.>

The news was as disagreeable to Fanny as it had been unexpected. She had never received kindness from her aunt Norris, and could not love her.

<I shall be very sorry to go away,> said she, with a faltering voice.

<Yes, I dare say you will; \textit{that's}  natural enough. I suppose you have had as little to vex you since you came into this house as any creature in the world.>

<I hope I am not ungrateful, aunt,> said Fanny modestly.

<No, my dear; I hope not. I have always found you a very good girl.>

<And am I never to live here again?>

<Never, my dear; but you are sure of a comfortable home. It can make very little difference to you, whether you are in one house or the other.>

Fanny left the room with a very sorrowful heart; she could not feel the difference to be so small, she could not think of living with her aunt with anything like satisfaction. As soon as she met with Edmund she told him her distress.

<Cousin,> said she, <something is going to happen which I do not like at all; and though you have often persuaded me into being reconciled to things that I disliked at first, you will not be able to do it now. I am going to live entirely with my aunt Norris.>

<Indeed!>

<Yes; my aunt Bertram has just told me so. It is quite settled. I am to leave Mansfield Park, and go to the White House, I suppose, as soon as she is removed there.>

<Well, Fanny, and if the plan were not unpleasant to you, I should call it an excellent one.>

<Oh, cousin!>

<It has everything else in its favour. My aunt is acting like a sensible woman in wishing for you. She is choosing a friend and companion exactly where she ought, and I am glad her love of money does not interfere. You will be what you ought to be to her. I hope it does not distress you very much, Fanny?>

<Indeed it does: I cannot like it. I love this house and everything in it: I shall love nothing there. You know how uncomfortable I feel with her.>

<I can say nothing for her manner to you as a child; but it was the same with us all, or nearly so. She never knew how to be pleasant to children. But you are now of an age to be treated better; I think she is behaving better already; and when you are her only companion, you \textit{must}  be important to her.>

<I can never be important to any one.>

<What is to prevent you?>

<Everything. My situation, my foolishness and awkwardness.>

<As to your foolishness and awkwardness, my dear Fanny, believe me, you never have a shadow of either, but in using the words so improperly. There is no reason in the world why you should not be important where you are known. You have good sense, and a sweet temper, and I am sure you have a grateful heart, that could never receive kindness without wishing to return it. I do not know any better qualifications for a friend and companion.>

<You are too kind,> said Fanny, colouring at such praise; <how shall I ever thank you as I ought, for thinking so well of me. Oh! cousin, if I am to go away, I shall remember your goodness to the last moment of my life.>

<Why, indeed, Fanny, I should hope to be remembered at such a distance as the White House. You speak as if you were going two hundred miles off instead of only across the park; but you will belong to us almost as much as ever. The two families will be meeting every day in the year. The only difference will be that, living with your aunt, you will necessarily be brought forward as you ought to be. \textit{Here}  there are too many whom you can hide behind; but with \textit{her}  you will be forced to speak for yourself.>

<Oh! do not say so.>

<I must say it, and say it with pleasure. Mrs~Norris is much better fitted than my mother for having the charge of you now. She is of a temper to do a great deal for anybody she really interests herself about, and she will force you to do justice to your natural powers.>

Fanny sighed, and said, <I cannot see things as you do; but I ought to believe you to be right rather than myself, and I am very much obliged to you for trying to reconcile me to what must be. If I could suppose my aunt really to care for me, it would be delightful to feel myself of consequence to anybody. \textit{Here}, I know, I am of none, and yet I love the place so well.>

<The place, Fanny, is what you will not quit, though you quit the house. You will have as free a command of the park and gardens as ever. Even \textit{your}  constant little heart need not take fright at such a nominal change. You will have the same walks to frequent, the same library to choose from, the same people to look at, the same horse to ride.>

<Very true. Yes, dear old grey pony! Ah! cousin, when I remember how much I used to dread riding, what terrors it gave me to hear it talked of as likely to do me good (oh! how I have trembled at my uncle's opening his lips if horses were talked of), and then think of the kind pains you took to reason and persuade me out of my fears, and convince me that I should like it after a little while, and feel how right you proved to be, I am inclined to hope you may always prophesy as well.>

<And I am quite convinced that your being with Mrs~Norris will be as good for your mind as riding has been for your health, and as much for your ultimate happiness too.>

So ended their discourse, which, for any very appropriate service it could render Fanny, might as well have been spared, for Mrs~Norris had not the smallest intention of taking her. It had never occurred to her, on the present occasion, but as a thing to be carefully avoided. To prevent its being expected, she had fixed on the smallest habitation which could rank as genteel among the buildings of Mansfield parish, the White House being only just large enough to receive herself and her servants, and allow a spare room for a friend, of which she made a very particular point. The spare rooms at the Parsonage had never been wanted, but the absolute necessity of a spare room for a friend was now never forgotten. Not all her precautions, however, could save her from being suspected of something better; or, perhaps, her very display of the importance of a spare room might have misled Sir~Thomas to suppose it really intended for Fanny. Lady Bertram soon brought the matter to a certainty by carelessly observing to Mrs~Norris—

<I think, sister, we need not keep Miss~Lee any longer, when Fanny goes to live with you.>

Mrs~Norris almost started. <Live with me, dear Lady Bertram! what do you mean?>

<Is she not to live with you? I thought you had settled it with Sir~Thomas.>

<Me! never. I never spoke a syllable about it to Sir~Thomas, nor he to me. Fanny live with me! the last thing in the world for me to think of, or for anybody to wish that really knows us both. Good heaven! what could I do with Fanny? Me! a poor, helpless, forlorn widow, unfit for anything, my spirits quite broke down; what could I do with a girl at her time of life? A girl of fifteen! the very age of all others to need most attention and care, and put the cheerfullest spirits to the test! Sure Sir~Thomas could not seriously expect such a thing! Sir~Thomas is too much my friend. Nobody that wishes me well, I am sure, would propose it. How came Sir~Thomas to speak to you about it?>

<Indeed, I do not know. I suppose he thought it best.>

<But what did he say? He could not say he \textit{wished}  me to take Fanny. I am sure in his heart he could not wish me to do it.>

<No; he only said he thought it very likely; and I thought so too. We both thought it would be a comfort to you. But if you do not like it, there is no more to be said. She is no encumbrance here.>

<Dear sister, if you consider my unhappy state, how can she be any comfort to me? Here am I, a poor desolate widow, deprived of the best of husbands, my health gone in attending and nursing him, my spirits still worse, all my peace in this world destroyed, with hardly enough to support me in the rank of a gentlewoman, and enable me to live so as not to disgrace the memory of the dear departed—what possible comfort could I have in taking such a charge upon me as Fanny? If I could wish it for my own sake, I would not do so unjust a thing by the poor girl. She is in good hands, and sure of doing well. I must struggle through my sorrows and difficulties as I can.>

<Then you will not mind living by yourself quite alone?>

<Lady Bertram, I do not complain. I know I cannot live as I have done, but I must retrench where I can, and learn to be a better manager. I \textit{have been}  a liberal housekeeper enough, but I shall not be ashamed to practise economy now. My situation is as much altered as my income. A great many things were due from poor Mr~Norris, as clergyman of the parish, that cannot be expected from me. It is unknown how much was consumed in our kitchen by odd comers and goers. At the White House, matters must be better looked after. I \textit{must}  live within my income, or I shall be miserable; and I own it would give me great satisfaction to be able to do rather more, to lay by a little at the end of the year.>

<I dare say you will. You always do, don't you?>

<My object, Lady Bertram, is to be of use to those that come after me. It is for your children's good that I wish to be richer. I have nobody else to care for, but I should be very glad to think I could leave a little trifle among them worth their having.>

<You are very good, but do not trouble yourself about them. They are sure of being well provided for. Sir~Thomas will take care of that.>

<Why, you know, Sir~Thomas's means will be rather straitened if the Antigua estate is to make such poor returns.>

<Oh! \textit{that}  will soon be settled. Sir~Thomas has been writing about it, I know.>

<Well, Lady Bertram,> said Mrs~Norris, moving to go, <I can only say that my sole desire is to be of use to your family: and so, if Sir~Thomas should ever speak again about my taking Fanny, you will be able to say that my health and spirits put it quite out of the question; besides that, I really should not have a bed to give her, for I must keep a spare room for a friend.>

Lady Bertram repeated enough of this conversation to her husband to convince him how much he had mistaken his sister-in-law's views; and she was from that moment perfectly safe from all expectation, or the slightest allusion to it from him. He could not but wonder at her refusing to do anything for a niece whom she had been so forward to adopt; but, as she took early care to make him, as well as Lady Bertram, understand that whatever she possessed was designed for their family, he soon grew reconciled to a distinction which, at the same time that it was advantageous and complimentary to them, would enable him better to provide for Fanny himself.

Fanny soon learnt how unnecessary had been her fears of a removal; and her spontaneous, untaught felicity on the discovery, conveyed some consolation to Edmund for his disappointment in what he had expected to be so essentially serviceable to her. Mrs~Norris took possession of the White House, the Grants arrived at the Parsonage, and these events over, everything at Mansfield went on for some time as usual.

The Grants showing a disposition to be friendly and sociable, gave great satisfaction in the main among their new acquaintance. They had their faults, and Mrs~Norris soon found them out. The Doctor was very fond of eating, and would have a good dinner every day; and Mrs~Grant, instead of contriving to gratify him at little expense, gave her cook as high wages as they did at Mansfield Park, and was scarcely ever seen in her offices. Mrs~Norris could not speak with any temper of such grievances, nor of the quantity of butter and eggs that were regularly consumed in the house. <Nobody loved plenty and hospitality more than herself; nobody more hated pitiful doings; the Parsonage, she believed, had never been wanting in comforts of any sort, had never borne a bad character in \textit{her time}, but this was a way of going on that she could not understand. A fine lady in a country parsonage was quite out of place. \textit{Her}  store-room, she thought, might have been good enough for Mrs~Grant to go into. Inquire where she would, she could not find out that Mrs~Grant had ever had more than five thousand pounds.>

Lady Bertram listened without much interest to this sort of invective. She could not enter into the wrongs of an economist, but she felt all the injuries of beauty in Mrs~Grant's being so well settled in life without being handsome, and expressed her astonishment on that point almost as often, though not so diffusely, as Mrs~Norris discussed the other.

These opinions had been hardly canvassed a year before another event arose of such importance in the family, as might fairly claim some place in the thoughts and conversation of the ladies. Sir~Thomas found it expedient to go to Antigua himself, for the better arrangement of his affairs, and he took his eldest son with him, in the hope of detaching him from some bad connexions at home. They left England with the probability of being nearly a twelvemonth absent.

The necessity of the measure in a pecuniary light, and the hope of its utility to his son, reconciled Sir~Thomas to the effort of quitting the rest of his family, and of leaving his daughters to the direction of others at their present most interesting time of life. He could not think Lady Bertram quite equal to supply his place with them, or rather, to perform what should have been her own; but, in Mrs~Norris's watchful attention, and in Edmund's judgment, he had sufficient confidence to make him go without fears for their conduct.

Lady Bertram did not at all like to have her husband leave her; but she was not disturbed by any alarm for his safety, or solicitude for his comfort, being one of those persons who think nothing can be dangerous, or difficult, or fatiguing to anybody but themselves.

The Miss~Bertrams were much to be pitied on the occasion: not for their sorrow, but for their want of it. Their father was no object of love to them; he had never seemed the friend of their pleasures, and his absence was unhappily most welcome. They were relieved by it from all restraint; and without aiming at one gratification that would probably have been forbidden by Sir~Thomas, they felt themselves immediately at their own disposal, and to have every indulgence within their reach. Fanny's relief, and her consciousness of it, were quite equal to her cousins'; but a more tender nature suggested that her feelings were ungrateful, and she really grieved because she could not grieve. <Sir~Thomas, who had done so much for her and her brothers, and who was gone perhaps never to return! that she should see him go without a tear! it was a shameful insensibility.> He had said to her, moreover, on the very last morning, that he hoped she might see William again in the course of the ensuing winter, and had charged her to write and invite him to Mansfield as soon as the squadron to which he belonged should be known to be in England. <This was so thoughtful and kind!> and would he only have smiled upon her, and called her <my dear Fanny,> while he said it, every former frown or cold address might have been forgotten. But he had ended his speech in a way to sink her in sad mortification, by adding, <If William does come to Mansfield, I hope you may be able to convince him that the many years which have passed since you parted have not been spent on your side entirely without improvement; though, I fear, he must find his sister at sixteen in some respects too much like his sister at ten.> She cried bitterly over this reflection when her uncle was gone; and her cousins, on seeing her with red eyes, set her down as a hypocrite. 