\chapter[Chapter \thechapter]{} 

 \lettrine[lraise=0.3]{H}{enry} Crawford was at Mansfield Park again the next morning, and at an earlier hour than common visiting warrants. The two ladies were together in the breakfast-room, and, fortunately for him, Lady Bertram was on the very point of quitting it as he entered. She was almost at the door, and not chusing by any means to take so much trouble in vain, she still went on, after a civil reception, a short sentence about being waited for, and a <Let Sir~Thomas know> to the servant.

Henry, overjoyed to have her go, bowed and watched her off, and without losing another moment, turned instantly to Fanny, and, taking out some letters, said, with a most animated look, <I must acknowledge myself infinitely obliged to any creature who gives me such an opportunity of seeing you alone: I have been wishing it more than you can have any idea. Knowing as I do what your feelings as a sister are, I could hardly have borne that any one in the house should share with you in the first knowledge of the news I now bring. He is made. Your brother is a lieutenant. I have the infinite satisfaction of congratulating you on your brother's promotion. Here are the letters which announce it, this moment come to hand. You will, perhaps, like to see them.>

Fanny could not speak, but he did not want her to speak. To see the expression of her eyes, the change of her complexion, the progress of her feelings, their doubt, confusion, and felicity, was enough. She took the letters as he gave them. The first was from the Admiral to inform his nephew, in a few words, of his having succeeded in the object he had undertaken, the promotion of young Price, and enclosing two more, one from the Secretary of the First Lord to a friend, whom the Admiral had set to work in the business, the other from that friend to himself, by which it appeared that his lordship had the very great happiness of attending to the recommendation of Sir~Charles; that Sir~Charles was much delighted in having such an opportunity of proving his regard for Admiral Crawford, and that the circumstance of Mr~William Price's commission as Second Lieutenant of H.M. Sloop Thrush being made out was spreading general joy through a wide circle of great people.

While her hand was trembling under these letters, her eye running from one to the other, and her heart swelling with emotion, Crawford thus continued, with unfeigned eagerness, to express his interest in the event—

<I will not talk of my own happiness,> said he, <great as it is, for I think only of yours. Compared with you, who has a right to be happy? I have almost grudged myself my own prior knowledge of what you ought to have known before all the world. I have not lost a moment, however. The post was late this morning, but there has not been since a moment's delay. How impatient, how anxious, how wild I have been on the subject, I will not attempt to describe; how severely mortified, how cruelly disappointed, in not having it finished while I was in London! I was kept there from day to day in the hope of it, for nothing less dear to me than such an object would have detained me half the time from Mansfield. But though my uncle entered into my wishes with all the warmth I could desire, and exerted himself immediately, there were difficulties from the absence of one friend, and the engagements of another, which at last I could no longer bear to stay the end of, and knowing in what good hands I left the cause, I came away on Monday, trusting that many posts would not pass before I should be followed by such very letters as these. My uncle, who is the very best man in the world, has exerted himself, as I knew he would, after seeing your brother. He was delighted with him. I would not allow myself yesterday to say how delighted, or to repeat half that the Admiral said in his praise. I deferred it all till his praise should be proved the praise of a friend, as this day \textit{does}  prove it. \textit{Now}  I may say that even I could not require William Price to excite a greater interest, or be followed by warmer wishes and higher commendation, than were most voluntarily bestowed by my uncle after the evening they had passed together.>

<Has this been all \textit{your}  doing, then?> cried Fanny. <Good heaven! how very, very kind! Have you really—was it by \textit{your}  desire? I beg your pardon, but I am bewildered. Did Admiral Crawford apply? How was it? I am stupefied.>

Henry was most happy to make it more intelligible, by beginning at an earlier stage, and explaining very particularly what he had done. His last journey to London had been undertaken with no other view than that of introducing her brother in Hill Street, and prevailing on the Admiral to exert whatever interest he might have for getting him on. This had been his business. He had communicated it to no creature: he had not breathed a syllable of it even to Mary; while uncertain of the issue, he could not have borne any participation of his feelings, but this had been his business; and he spoke with such a glow of what his solicitude had been, and used such strong expressions, was so abounding in the \textit{deepest interest}, in \textit{twofold motives}, in \textit{views and wishes more than could be told}, that Fanny could not have remained insensible of his drift, had she been able to attend; but her heart was so full and her senses still so astonished, that she could listen but imperfectly even to what he told her of William, and saying only when he paused, <How kind! how very kind! Oh, Mr~Crawford, we are infinitely obliged to you! Dearest, dearest William!> She jumped up and moved in haste towards the door, crying out, <I will go to my uncle. My uncle ought to know it as soon as possible.> But this could not be suffered. The opportunity was too fair, and his feelings too impatient. He was after her immediately. <She must not go, she must allow him five minutes longer,> and he took her hand and led her back to her seat, and was in the middle of his farther explanation, before she had suspected for what she was detained. When she did understand it, however, and found herself expected to believe that she had created sensations which his heart had never known before, and that everything he had done for William was to be placed to the account of his excessive and unequalled attachment to her, she was exceedingly distressed, and for some moments unable to speak. She considered it all as nonsense, as mere trifling and gallantry, which meant only to deceive for the hour; she could not but feel that it was treating her improperly and unworthily, and in such a way as she had not deserved; but it was like himself, and entirely of a piece with what she had seen before; and she would not allow herself to shew half the displeasure she felt, because he had been conferring an obligation, which no want of delicacy on his part could make a trifle to her. While her heart was still bounding with joy and gratitude on William's behalf, she could not be severely resentful of anything that injured only herself; and after having twice drawn back her hand, and twice attempted in vain to turn away from him, she got up, and said only, with much agitation, <Don't, Mr~Crawford, pray don't! I beg you would not. This is a sort of talking which is very unpleasant to me. I must go away. I cannot bear it.> But he was still talking on, describing his affection, soliciting a return, and, finally, in words so plain as to bear but one meaning even to her, offering himself, hand, fortune, everything, to her acceptance. It was so; he had said it. Her astonishment and confusion increased; and though still not knowing how to suppose him serious, she could hardly stand. He pressed for an answer.

<No, no, no!> she cried, hiding her face. <This is all nonsense. Do not distress me. I can hear no more of this. Your kindness to William makes me more obliged to you than words can express; but I do not want, I cannot bear, I must not listen to such—No, no, don't think of me. But you are \textit{not}  thinking of me. I know it is all nothing.>

She had burst away from him, and at that moment Sir~Thomas was heard speaking to a servant in his way towards the room they were in. It was no time for farther assurances or entreaty, though to part with her at a moment when her modesty alone seemed, to his sanguine and preassured mind, to stand in the way of the happiness he sought, was a cruel necessity. She rushed out at an opposite door from the one her uncle was approaching, and was walking up and down the East room in the utmost confusion of contrary feeling, before Sir~Thomas's politeness or apologies were over, or he had reached the beginning of the joyful intelligence which his visitor came to communicate.

She was feeling, thinking, trembling about everything; agitated, happy, miserable, infinitely obliged, absolutely angry. It was all beyond belief! He was inexcusable, incomprehensible! But such were his habits that he could do nothing without a mixture of evil. He had previously made her the happiest of human beings, and now he had insulted—she knew not what to say, how to class, or how to regard it. She would not have him be serious, and yet what could excuse the use of such words and offers, if they meant but to trifle?

But William was a lieutenant. \textit{That}  was a fact beyond a doubt, and without an alloy. She would think of it for ever and forget all the rest. Mr~Crawford would certainly never address her so again: he must have seen how unwelcome it was to her; and in that case, how gratefully she could esteem him for his friendship to William!

She would not stir farther from the East room than the head of the great staircase, till she had satisfied herself of Mr~Crawford's having left the house; but when convinced of his being gone, she was eager to go down and be with her uncle, and have all the happiness of his joy as well as her own, and all the benefit of his information or his conjectures as to what would now be William's destination. Sir~Thomas was as joyful as she could desire, and very kind and communicative; and she had so comfortable a talk with him about William as to make her feel as if nothing had occurred to vex her, till she found, towards the close, that Mr~Crawford was engaged to return and dine there that very day. This was a most unwelcome hearing, for though he might think nothing of what had passed, it would be quite distressing to her to see him again so soon.

She tried to get the better of it; tried very hard, as the dinner hour approached, to feel and appear as usual; but it was quite impossible for her not to look most shy and uncomfortable when their visitor entered the room. She could not have supposed it in the power of any concurrence of circumstances to give her so many painful sensations on the first day of hearing of William's promotion.

Mr~Crawford was not only in the room—he was soon close to her. He had a note to deliver from his sister. Fanny could not look at him, but there was no consciousness of past folly in his voice. She opened her note immediately, glad to have anything to do, and happy, as she read it, to feel that the fidgetings of her aunt Norris, who was also to dine there, screened her a little from view.  
\begin{mail}{}{My dear Fanny,}
—for so I may now always call you, to the infinite relief of a tongue that has been stumbling at \textit{Miss~Price}  for at least the last six weeks—I cannot let my brother go without sending you a few lines of general congratulation, and giving my most joyful consent and approval. Go on, my dear Fanny, and without fear; there can be no difficulties worth naming. I chuse to suppose that the assurance of my consent will be something; so you may smile upon him with your sweetest smiles this afternoon, and send him back to me even happier than he goes.  

\closeletter[Yours affectionately,]{M. C.}
\end{mail}

These were not expressions to do Fanny any good; for though she read in too much haste and confusion to form the clearest judgment of Miss~Crawford's meaning, it was evident that she meant to compliment her on her brother's attachment, and even to \textit{appear}  to believe it serious. She did not know what to do, or what to think. There was wretchedness in the idea of its being serious; there was perplexity and agitation every way. She was distressed whenever Mr~Crawford spoke to her, and he spoke to her much too often; and she was afraid there was a something in his voice and manner in addressing her very different from what they were when he talked to the others. Her comfort in that day's dinner was quite destroyed: she could hardly eat anything; and when Sir~Thomas good-humouredly observed that joy had taken away her appetite, she was ready to sink with shame, from the dread of Mr~Crawford's interpretation; for though nothing could have tempted her to turn her eyes to the right hand, where he sat, she felt that \textit{his}  were immediately directed towards her.

She was more silent than ever. She would hardly join even when William was the subject, for his commission came all from the right hand too, and there was pain in the connexion.

She thought Lady Bertram sat longer than ever, and began to be in despair of ever getting away; but at last they were in the drawing-room, and she was able to think as she would, while her aunts finished the subject of William's appointment in their own style.

Mrs~Norris seemed as much delighted with the saving it would be to Sir~Thomas as with any part of it. <\textit{Now}  William would be able to keep himself, which would make a vast difference to his uncle, for it was unknown how much he had cost his uncle; and, indeed, it would make some difference in \textit{her}  presents too. She was very glad that she had given William what she did at parting, very glad, indeed, that it had been in her power, without material inconvenience, just at that time to give him something rather considerable; that is, for \textit{her}, with \textit{her}  limited means, for now it would all be useful in helping to fit up his cabin. She knew he must be at some expense, that he would have many things to buy, though to be sure his father and mother would be able to put him in the way of getting everything very cheap; but she was very glad she had contributed her mite towards it.>

<I am glad you gave him something considerable,> said Lady Bertram, with most unsuspicious calmness, “for \textit{I}  gave him only \textsterling 10.”

<Indeed!> cried Mrs~Norris, reddening. <Upon my word, he must have gone off with his pockets well lined, and at no expense for his journey to London either!>

“Sir~Thomas told me \textsterling 10 would be enough.”

Mrs~Norris, being not at all inclined to question its sufficiency, began to take the matter in another point.

<It is amazing,> said she, <how much young people cost their friends, what with bringing them up and putting them out in the world! They little think how much it comes to, or what their parents, or their uncles and aunts, pay for them in the course of the year. Now, here are my sister Price's children; take them all together, I dare say nobody would believe what a sum they cost Sir~Thomas every year, to say nothing of what \textit{I}  do for them.>

<Very true, sister, as you say. But, poor things! they cannot help it; and you know it makes very little difference to Sir~Thomas. Fanny, William must not forget my shawl if he goes to the East Indies; and I shall give him a commission for anything else that is worth having. I wish he may go to the East Indies, that I may have my shawl. I think I will have two shawls, Fanny.>

Fanny, meanwhile, speaking only when she could not help it, was very earnestly trying to understand what Mr~and Miss~Crawford were at. There was everything in the world \textit{against}  their being serious but his words and manner. Everything natural, probable, reasonable, was against it; all their habits and ways of thinking, and all her own demerits. How could \textit{she}  have excited serious attachment in a man who had seen so many, and been admired by so many, and flirted with so many, infinitely her superiors; who seemed so little open to serious impressions, even where pains had been taken to please him; who thought so slightly, so carelessly, so unfeelingly on all such points; who was everything to everybody, and seemed to find no one essential to him? And farther, how could it be supposed that his sister, with all her high and worldly notions of matrimony, would be forwarding anything of a serious nature in such a quarter? Nothing could be more unnatural in either. Fanny was ashamed of her own doubts. Everything might be possible rather than serious attachment, or serious approbation of it toward her. She had quite convinced herself of this before Sir~Thomas and Mr~Crawford joined them. The difficulty was in maintaining the conviction quite so absolutely after Mr~Crawford was in the room; for once or twice a look seemed forced on her which she did not know how to class among the common meaning; in any other man, at least, she would have said that it meant something very earnest, very pointed. But she still tried to believe it no more than what he might often have expressed towards her cousins and fifty other women.

She thought he was wishing to speak to her unheard by the rest. She fancied he was trying for it the whole evening at intervals, whenever Sir~Thomas was out of the room, or at all engaged with Mrs~Norris, and she carefully refused him every opportunity.

At last—it seemed an at last to Fanny's nervousness, though not remarkably late—he began to talk of going away; but the comfort of the sound was impaired by his turning to her the next moment, and saying, <Have you nothing to send to Mary? No answer to her note? She will be disappointed if she receives nothing from you. Pray write to her, if it be only a line.>

<Oh yes! certainly,> cried Fanny, rising in haste, the haste of embarrassment and of wanting to get away—<I will write directly.>

She went accordingly to the table, where she was in the habit of writing for her aunt, and prepared her materials without knowing what in the world to say. She had read Miss~Crawford's note only once, and how to reply to anything so imperfectly understood was most distressing. Quite unpractised in such sort of note-writing, had there been time for scruples and fears as to style she would have felt them in abundance: but something must be instantly written; and with only one decided feeling, that of wishing not to appear to think anything really intended, she wrote thus, in great trembling both of spirits and hand—  

\begin{mail}{}{}
I am very much obliged to you, my dear Miss~Crawford, for your kind congratulations, as far as they relate to my dearest William. The rest of your note I know means nothing; but I am so unequal to anything of the sort, that I hope you will excuse my begging you to take no farther notice. I have seen too much of Mr~Crawford not to understand his manners; if he understood me as well, he would, I dare say, behave differently. I do not know what I write, but it would be a great favour of you never to mention the subject again. With thanks for the honour of your note, 

\closeletter[I remain, dear Miss~Crawford, \&c. \&c.]{}
\end{mail}

The conclusion was scarcely intelligible from increasing fright, for she found that Mr~Crawford, under pretence of receiving the note, was coming towards her.

<You cannot think I mean to hurry you,> said he, in an undervoice, perceiving the amazing trepidation with which she made up the note, <you cannot think I have any such object. Do not hurry yourself, I entreat.>

<Oh! I thank you; I have quite done, just done; it will be ready in a moment; I am very much obliged to you; if you will be so good as to give \textit{that}  to Miss~Crawford.>

The note was held out, and must be taken; and as she instantly and with averted eyes walked towards the fireplace, where sat the others, he had nothing to do but to go in good earnest.

Fanny thought she had never known a day of greater agitation, both of pain and pleasure; but happily the pleasure was not of a sort to die with the day; for every day would restore the knowledge of William's advancement, whereas the pain, she hoped, would return no more. She had no doubt that her note must appear excessively ill-written, that the language would disgrace a child, for her distress had allowed no arrangement; but at least it would assure them both of her being neither imposed on nor gratified by Mr~Crawford's attentions. 