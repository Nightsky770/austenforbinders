\chapter[Chapter \thechapter]{} 

 \lettrine[lraise=0.3]{T}{he} Prices were just setting off for church the next day when Mr~Crawford appeared again. He came, not to stop, but to join them; he was asked to go with them to the Garrison chapel, which was exactly what he had intended, and they all walked thither together.

The family were now seen to advantage. Nature had given them no inconsiderable share of beauty, and every Sunday dressed them in their cleanest skins and best attire. Sunday always brought this comfort to Fanny, and on this Sunday she felt it more than ever. Her poor mother now did not look so very unworthy of being Lady Bertram's sister as she was but too apt to look. It often grieved her to the heart to think of the contrast between them; to think that where nature had made so little difference, circumstances should have made so much, and that her mother, as handsome as Lady Bertram, and some years her junior, should have an appearance so much more worn and faded, so comfortless, so slatternly, so shabby. But Sunday made her a very creditable and tolerably cheerful-looking Mrs~Price, coming abroad with a fine family of children, feeling a little respite of her weekly cares, and only discomposed if she saw her boys run into danger, or Rebecca pass by with a flower in her hat.

In chapel they were obliged to divide, but Mr~Crawford took care not to be divided from the female branch; and after chapel he still continued with them, and made one in the family party on the ramparts.

Mrs~Price took her weekly walk on the ramparts every fine Sunday throughout the year, always going directly after morning service and staying till dinner-time. It was her public place: there she met her acquaintance, heard a little news, talked over the badness of the Portsmouth servants, and wound up her spirits for the six days ensuing.

Thither they now went; Mr~Crawford most happy to consider the Miss~Prices as his peculiar charge; and before they had been there long, somehow or other, there was no saying how, Fanny could not have believed it, but he was walking between them with an arm of each under his, and she did not know how to prevent or put an end to it. It made her uncomfortable for a time, but yet there were enjoyments in the day and in the view which would be felt.

The day was uncommonly lovely. It was really March; but it was April in its mild air, brisk soft wind, and bright sun, occasionally clouded for a minute; and everything looked so beautiful under the influence of such a sky, the effects of the shadows pursuing each other on the ships at Spithead and the island beyond, with the ever-varying hues of the sea, now at high water, dancing in its glee and dashing against the ramparts with so fine a sound, produced altogether such a combination of charms for Fanny, as made her gradually almost careless of the circumstances under which she felt them. Nay, had she been without his arm, she would soon have known that she needed it, for she wanted strength for a two hours' saunter of this kind, coming, as it generally did, upon a week's previous inactivity. Fanny was beginning to feel the effect of being debarred from her usual regular exercise; she had lost ground as to health since her being in Portsmouth; and but for Mr~Crawford and the beauty of the weather would soon have been knocked up now.

The loveliness of the day, and of the view, he felt like herself. They often stopt with the same sentiment and taste, leaning against the wall, some minutes, to look and admire; and considering he was not Edmund, Fanny could not but allow that he was sufficiently open to the charms of nature, and very well able to express his admiration. She had a few tender reveries now and then, which he could sometimes take advantage of to look in her face without detection; and the result of these looks was, that though as bewitching as ever, her face was less blooming than it ought to be. She \textit{said}  she was very well, and did not like to be supposed otherwise; but take it all in all, he was convinced that her present residence could not be comfortable, and therefore could not be salutary for her, and he was growing anxious for her being again at Mansfield, where her own happiness, and his in seeing her, must be so much greater.

<You have been here a month, I think?> said he.

<No; not quite a month. It is only four weeks to-morrow since I left Mansfield.>

<You are a most accurate and honest reckoner. I should call that a month.>

<I did not arrive here till Tuesday evening.>

<And it is to be a two months' visit, is not?>

<Yes. My uncle talked of two months. I suppose it will not be less.>

<And how are you to be conveyed back again? Who comes for you?>

<I do not know. I have heard nothing about it yet from my aunt. Perhaps I may be to stay longer. It may not be convenient for me to be fetched exactly at the two months' end.>

After a moment's reflection, Mr~Crawford replied, <I know Mansfield, I know its way, I know its faults towards \textit{you}. I know the danger of your being so far forgotten, as to have your comforts give way to the imaginary convenience of any single being in the family. I am aware that you may be left here week after week, if Sir~Thomas cannot settle everything for coming himself, or sending your aunt's maid for you, without involving the slightest alteration of the arrangements which he may have laid down for the next quarter of a year. This will not do. Two months is an ample allowance; I should think six weeks quite enough. I am considering your sister's health,> said he, addressing himself to Susan, <which I think the confinement of Portsmouth unfavourable to. She requires constant air and exercise. When you know her as well as I do, I am sure you will agree that she does, and that she ought never to be long banished from the free air and liberty of the country. If, therefore> (turning again to Fanny), <you find yourself growing unwell, and any difficulties arise about your returning to Mansfield, without waiting for the two months to be ended, \textit{that}  must not be regarded as of any consequence, if you feel yourself at all less strong or comfortable than usual, and will only let my sister know it, give her only the slightest hint, she and I will immediately come down, and take you back to Mansfield. You know the ease and the pleasure with which this would be done. You know all that would be felt on the occasion.>

Fanny thanked him, but tried to laugh it off.

<I am perfectly serious,> he replied, <as you perfectly know. And I hope you will not be cruelly concealing any tendency to indisposition. Indeed, you shall \textit{not}; it shall not be in your power; for so long only as you positively say, in every letter to Mary, <I am well,> and I know you cannot speak or write a falsehood, so long only shall you be considered as well.>

Fanny thanked him again, but was affected and distressed to a degree that made it impossible for her to say much, or even to be certain of what she ought to say. This was towards the close of their walk. He attended them to the last, and left them only at the door of their own house, when he knew them to be going to dinner, and therefore pretended to be waited for elsewhere.

<I wish you were not so tired,> said he, still detaining Fanny after all the others were in the house—<I wish I left you in stronger health. Is there anything I can do for you in town? I have half an idea of going into Norfolk again soon. I am not satisfied about Maddison. I am sure he still means to impose on me if possible, and get a cousin of his own into a certain mill, which I design for somebody else. I must come to an understanding with him. I must make him know that I will not be tricked on the south side of Everingham, any more than on the north: that I will be master of my own property. I was not explicit enough with him before. The mischief such a man does on an estate, both as to the credit of his employer and the welfare of the poor, is inconceivable. I have a great mind to go back into Norfolk directly, and put everything at once on such a footing as cannot be afterwards swerved from. Maddison is a clever fellow; I do not wish to displace him, provided he does not try to displace \textit{me}; but it would be simple to be duped by a man who has no right of creditor to dupe me, and worse than simple to let him give me a hard-hearted, griping fellow for a tenant, instead of an honest man, to whom I have given half a promise already. Would it not be worse than simple? Shall I go? Do you advise it?>

<I advise! You know very well what is right.>

<Yes. When you give me your opinion, I always know what is right. Your judgment is my rule of right.>

<Oh, no! do not say so. We have all a better guide in ourselves, if we would attend to it, than any other person can be. Good-bye; I wish you a pleasant journey to-morrow.>

<Is there nothing I can do for you in town?>

<Nothing; I am much obliged to you.>

<Have you no message for anybody?>

<My love to your sister, if you please; and when you see my cousin, my cousin Edmund, I wish you would be so good as to say that I suppose I shall soon hear from him.>

<Certainly; and if he is lazy or negligent, I will write his excuses myself.>

He could say no more, for Fanny would be no longer detained. He pressed her hand, looked at her, and was gone. \textit{He}  went to while away the next three hours as he could, with his other acquaintance, till the best dinner that a capital inn afforded was ready for their enjoyment, and \textit{she}  turned in to her more simple one immediately.

Their general fare bore a very different character; and could he have suspected how many privations, besides that of exercise, she endured in her father's house, he would have wondered that her looks were not much more affected than he found them. She was so little equal to Rebecca's puddings and Rebecca's hashes, brought to table, as they all were, with such accompaniments of half-cleaned plates, and not half-cleaned knives and forks, that she was very often constrained to defer her heartiest meal till she could send her brothers in the evening for biscuits and buns. After being nursed up at Mansfield, it was too late in the day to be hardened at Portsmouth; and though Sir~Thomas, had he known all, might have thought his niece in the most promising way of being starved, both mind and body, into a much juster value for Mr~Crawford's good company and good fortune, he would probably have feared to push his experiment farther, lest she might die under the cure.

Fanny was out of spirits all the rest of the day. Though tolerably secure of not seeing Mr~Crawford again, she could not help being low. It was parting with somebody of the nature of a friend; and though, in one light, glad to have him gone, it seemed as if she was now deserted by everybody; it was a sort of renewed separation from Mansfield; and she could not think of his returning to town, and being frequently with Mary and Edmund, without feelings so near akin to envy as made her hate herself for having them.

Her dejection had no abatement from anything passing around her; a friend or two of her father's, as always happened if he was not with them, spent the long, long evening there; and from six o'clock till half-past nine, there was little intermission of noise or grog. She was very low. The wonderful improvement which she still fancied in Mr~Crawford was the nearest to administering comfort of anything within the current of her thoughts. Not considering in how different a circle she had been just seeing him, nor how much might be owing to contrast, she was quite persuaded of his being astonishingly more gentle and regardful of others than formerly. And, if in little things, must it not be so in great? So anxious for her health and comfort, so very feeling as he now expressed himself, and really seemed, might not it be fairly supposed that he would not much longer persevere in a suit so distressing to her? 