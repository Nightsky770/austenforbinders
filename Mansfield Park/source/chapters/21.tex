\chapter[Chapter \thechapter]{} 

 \lettrine[lraise=0.3]{S}{ir} Thomas's return made a striking change in the ways of the family, independent of Lovers' Vows. Under his government, Mansfield was an altered place. Some members of their society sent away, and the spirits of many others saddened—it was all sameness and gloom compared with the past—a sombre family party rarely enlivened. There was little intercourse with the Parsonage. Sir~Thomas, drawing back from intimacies in general, was particularly disinclined, at this time, for any engagements but in one quarter. The Rushworths were the only addition to his own domestic circle which he could solicit.

Edmund did not wonder that such should be his father's feelings, nor could he regret anything but the exclusion of the Grants. <But they,> he observed to Fanny, <have a claim. They seem to belong to us; they seem to be part of ourselves. I could wish my father were more sensible of their very great attention to my mother and sisters while he was away. I am afraid they may feel themselves neglected. But the truth is, that my father hardly knows them. They had not been here a twelvemonth when he left England. If he knew them better, he would value their society as it deserves; for they are in fact exactly the sort of people he would like. We are sometimes a little in want of animation among ourselves: my sisters seem out of spirits, and Tom is certainly not at his ease. Dr~and Mrs~Grant would enliven us, and make our evenings pass away with more enjoyment even to my father.>

<Do you think so?> said Fanny: <in my opinion, my uncle would not like \textit{any}  addition. I think he values the very quietness you speak of, and that the repose of his own family circle is all he wants. And it does not appear to me that we are more serious than we used to be—I mean before my uncle went abroad. As well as I can recollect, it was always much the same. There was never much laughing in his presence; or, if there is any difference, it is not more, I think, than such an absence has a tendency to produce at first. There must be a sort of shyness; but I cannot recollect that our evenings formerly were ever merry, except when my uncle was in town. No young people's are, I suppose, when those they look up to are at home>.

<I believe you are right, Fanny,> was his reply, after a short consideration. <I believe our evenings are rather returned to what they were, than assuming a new character. The novelty was in their being lively. Yet, how strong the impression that only a few weeks will give! I have been feeling as if we had never lived so before.>

<I suppose I am graver than other people,> said Fanny. <The evenings do not appear long to me. I love to hear my uncle talk of the West Indies. I could listen to him for an hour together. It entertains \textit{me}  more than many other things have done; but then I am unlike other people, I dare say.>

<Why should you dare say \textit{that} ?> (smiling). <Do you want to be told that you are only unlike other people in being more wise and discreet? But when did you, or anybody, ever get a compliment from me, Fanny? Go to my father if you want to be complimented. He will satisfy you. Ask your uncle what he thinks, and you will hear compliments enough: and though they may be chiefly on your person, you must put up with it, and trust to his seeing as much beauty of mind in time.>

Such language was so new to Fanny that it quite embarrassed her.

<Your uncle thinks you very pretty, dear Fanny—and that is the long and the short of the matter. Anybody but myself would have made something more of it, and anybody but you would resent that you had not been thought very pretty before; but the truth is, that your uncle never did admire you till now—and now he does. Your complexion is so improved!—and you have gained so much countenance!—and your figure—nay, Fanny, do not turn away about it—it is but an uncle. If you cannot bear an uncle's admiration, what is to become of you? You must really begin to harden yourself to the idea of being worth looking at. You must try not to mind growing up into a pretty woman.>

<Oh! don't talk so, don't talk so,> cried Fanny, distressed by more feelings than he was aware of; but seeing that she was distressed, he had done with the subject, and only added more seriously—

<Your uncle is disposed to be pleased with you in every respect; and I only wish you would talk to him more. You are one of those who are too silent in the evening circle.>

<But I do talk to him more than I used. I am sure I do. Did not you hear me ask him about the slave-trade last night?>

<I did—and was in hopes the question would be followed up by others. It would have pleased your uncle to be inquired of farther.>

<And I longed to do it—but there was such a dead silence! And while my cousins were sitting by without speaking a word, or seeming at all interested in the subject, I did not like—I thought it would appear as if I wanted to set myself off at their expense, by shewing a curiosity and pleasure in his information which he must wish his own daughters to feel.>

<Miss~Crawford was very right in what she said of you the other day: that you seemed almost as fearful of notice and praise as other women were of neglect. We were talking of you at the Parsonage, and those were her words. She has great discernment. I know nobody who distinguishes characters better. For so young a woman it is remarkable! She certainly understands \textit{you}  better than you are understood by the greater part of those who have known you so long; and with regard to some others, I can perceive, from occasional lively hints, the unguarded expressions of the moment, that she could define \textit{many}  as accurately, did not delicacy forbid it. I wonder what she thinks of my father! She must admire him as a fine-looking man, with most gentlemanlike, dignified, consistent manners; but perhaps, having seen him so seldom, his reserve may be a little repulsive. Could they be much together, I feel sure of their liking each other. He would enjoy her liveliness and she has talents to value his powers. I wish they met more frequently! I hope she does not suppose there is any dislike on his side.>

<She must know herself too secure of the regard of all the rest of you,> said Fanny, with half a sigh, <to have any such apprehension. And Sir~Thomas's wishing just at first to be only with his family, is so very natural, that she can argue nothing from that. After a little while, I dare say, we shall be meeting again in the same sort of way, allowing for the difference of the time of year.>

<This is the first October that she has passed in the country since her infancy. I do not call Tunbridge or Cheltenham the country; and November is a still more serious month, and I can see that Mrs~Grant is very anxious for her not finding Mansfield dull as winter comes on.>

Fanny could have said a great deal, but it was safer to say nothing, and leave untouched all Miss~Crawford's resources—her accomplishments, her spirits, her importance, her friends, lest it should betray her into any observations seemingly unhandsome. Miss~Crawford's kind opinion of herself deserved at least a grateful forbearance, and she began to talk of something else.

<To-morrow, I think, my uncle dines at Sotherton, and you and Mr~Bertram too. We shall be quite a small party at home. I hope my uncle may continue to like Mr~Rushworth.>

<That is impossible, Fanny. He must like him less after to-morrow's visit, for we shall be five hours in his company. I should dread the stupidity of the day, if there were not a much greater evil to follow—the impression it must leave on Sir~Thomas. He cannot much longer deceive himself. I am sorry for them all, and would give something that Rushworth and Maria had never met.>

In this quarter, indeed, disappointment was impending over Sir~Thomas. Not all his good-will for Mr~Rushworth, not all Mr~Rushworth's deference for him, could prevent him from soon discerning some part of the truth—that Mr~Rushworth was an inferior young man, as ignorant in business as in books, with opinions in general unfixed, and without seeming much aware of it himself.

He had expected a very different son-in-law; and beginning to feel grave on Maria's account, tried to understand \textit{her}  feelings. Little observation there was necessary to tell him that indifference was the most favourable state they could be in. Her behaviour to Mr~Rushworth was careless and cold. She could not, did not like him. Sir~Thomas resolved to speak seriously to her. Advantageous as would be the alliance, and long standing and public as was the engagement, her happiness must not be sacrificed to it. Mr~Rushworth had, perhaps, been accepted on too short an acquaintance, and, on knowing him better, she was repenting.

With solemn kindness Sir~Thomas addressed her: told her his fears, inquired into her wishes, entreated her to be open and sincere, and assured her that every inconvenience should be braved, and the connexion entirely given up, if she felt herself unhappy in the prospect of it. He would act for her and release her. Maria had a moment's struggle as she listened, and only a moment's: when her father ceased, she was able to give her answer immediately, decidedly, and with no apparent agitation. She thanked him for his great attention, his paternal kindness, but he was quite mistaken in supposing she had the smallest desire of breaking through her engagement, or was sensible of any change of opinion or inclination since her forming it. She had the highest esteem for Mr~Rushworth's character and disposition, and could not have a doubt of her happiness with him.

Sir~Thomas was satisfied; too glad to be satisfied, perhaps, to urge the matter quite so far as his judgment might have dictated to others. It was an alliance which he could not have relinquished without pain; and thus he reasoned. Mr~Rushworth was young enough to improve. Mr~Rushworth must and would improve in good society; and if Maria could now speak so securely of her happiness with him, speaking certainly without the prejudice, the blindness of love, she ought to be believed. Her feelings, probably, were not acute; he had never supposed them to be so; but her comforts might not be less on that account; and if she could dispense with seeing her husband a leading, shining character, there would certainly be everything else in her favour. A well-disposed young woman, who did not marry for love, was in general but the more attached to her own family; and the nearness of Sotherton to Mansfield must naturally hold out the greatest temptation, and would, in all probability, be a continual supply of the most amiable and innocent enjoyments. Such and such-like were the reasonings of Sir~Thomas, happy to escape the embarrassing evils of a rupture, the wonder, the reflections, the reproach that must attend it; happy to secure a marriage which would bring him such an addition of respectability and influence, and very happy to think anything of his daughter's disposition that was most favourable for the purpose.

To her the conference closed as satisfactorily as to him. She was in a state of mind to be glad that she had secured her fate beyond recall: that she had pledged herself anew to Sotherton; that she was safe from the possibility of giving Crawford the triumph of governing her actions, and destroying her prospects; and retired in proud resolve, determined only to behave more cautiously to Mr~Rushworth in future, that her father might not be again suspecting her.

Had Sir~Thomas applied to his daughter within the first three or four days after Henry Crawford's leaving Mansfield, before her feelings were at all tranquillised, before she had given up every hope of him, or absolutely resolved on enduring his rival, her answer might have been different; but after another three or four days, when there was no return, no letter, no message, no symptom of a softened heart, no hope of advantage from separation, her mind became cool enough to seek all the comfort that pride and self revenge could give.

Henry Crawford had destroyed her happiness, but he should not know that he had done it; he should not destroy her credit, her appearance, her prosperity, too. He should not have to think of her as pining in the retirement of Mansfield for \textit{him}, rejecting Sotherton and London, independence and splendour, for \textit{his}  sake. Independence was more needful than ever; the want of it at Mansfield more sensibly felt. She was less and less able to endure the restraint which her father imposed. The liberty which his absence had given was now become absolutely necessary. She must escape from him and Mansfield as soon as possible, and find consolation in fortune and consequence, bustle and the world, for a wounded spirit. Her mind was quite determined, and varied not.

To such feelings delay, even the delay of much preparation, would have been an evil, and Mr~Rushworth could hardly be more impatient for the marriage than herself. In all the important preparations of the mind she was complete: being prepared for matrimony by an hatred of home, restraint, and tranquillity; by the misery of disappointed affection, and contempt of the man she was to marry. The rest might wait. The preparations of new carriages and furniture might wait for London and spring, when her own taste could have fairer play.

The principals being all agreed in this respect, it soon appeared that a very few weeks would be sufficient for such arrangements as must precede the wedding.

Mrs~Rushworth was quite ready to retire, and make way for the fortunate young woman whom her dear son had selected; and very early in November removed herself, her maid, her footman, and her chariot, with true dowager propriety, to Bath, there to parade over the wonders of Sotherton in her evening parties; enjoying them as thoroughly, perhaps, in the animation of a card-table, as she had ever done on the spot; and before the middle of the same month the ceremony had taken place which gave Sotherton another mistress.

It was a very proper wedding. The bride was elegantly dressed; the two bridesmaids were duly inferior; her father gave her away; her mother stood with salts in her hand, expecting to be agitated; her aunt tried to cry; and the service was impressively read by Dr~Grant. Nothing could be objected to when it came under the discussion of the neighbourhood, except that the carriage which conveyed the bride and bridegroom and Julia from the church-door to Sotherton was the same chaise which Mr~Rushworth had used for a twelvemonth before. In everything else the etiquette of the day might stand the strictest investigation.

It was done, and they were gone. Sir~Thomas felt as an anxious father must feel, and was indeed experiencing much of the agitation which his wife had been apprehensive of for herself, but had fortunately escaped. Mrs~Norris, most happy to assist in the duties of the day, by spending it at the Park to support her sister's spirits, and drinking the health of Mr~and Mrs~Rushworth in a supernumerary glass or two, was all joyous delight; for she had made the match; she had done everything; and no one would have supposed, from her confident triumph, that she had ever heard of conjugal infelicity in her life, or could have the smallest insight into the disposition of the niece who had been brought up under her eye.

The plan of the young couple was to proceed, after a few days, to Brighton, and take a house there for some weeks. Every public place was new to Maria, and Brighton is almost as gay in winter as in summer. When the novelty of amusement there was over, it would be time for the wider range of London.

Julia was to go with them to Brighton. Since rivalry between the sisters had ceased, they had been gradually recovering much of their former good understanding; and were at least sufficiently friends to make each of them exceedingly glad to be with the other at such a time. Some other companion than Mr~Rushworth was of the first consequence to his lady; and Julia was quite as eager for novelty and pleasure as Maria, though she might not have struggled through so much to obtain them, and could better bear a subordinate situation.

Their departure made another material change at Mansfield, a chasm which required some time to fill up. The family circle became greatly contracted; and though the Miss~Bertrams had latterly added little to its gaiety, they could not but be missed. Even their mother missed them; and how much more their tenderhearted cousin, who wandered about the house, and thought of them, and felt for them, with a degree of affectionate regret which they had never done much to deserve! 