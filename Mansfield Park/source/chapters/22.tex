\chapter[Chapter \thechapter]{} 

 \lettrine[lraise=0.3]{F}{anny's} consequence increased on the departure of her cousins. Becoming, as she then did, the only young woman in the drawing-room, the only occupier of that interesting division of a family in which she had hitherto held so humble a third, it was impossible for her not to be more looked at, more thought of and attended to, than she had ever been before; and <Where is Fanny?> became no uncommon question, even without her being wanted for any one's convenience.

Not only at home did her value increase, but at the Parsonage too. In that house, which she had hardly entered twice a year since Mr~Norris's death, she became a welcome, an invited guest, and in the gloom and dirt of a November day, most acceptable to Mary Crawford. Her visits there, beginning by chance, were continued by solicitation. Mrs~Grant, really eager to get any change for her sister, could, by the easiest self-deceit, persuade herself that she was doing the kindest thing by Fanny, and giving her the most important opportunities of improvement in pressing her frequent calls.

Fanny, having been sent into the village on some errand by her aunt Norris, was overtaken by a heavy shower close to the Parsonage; and being descried from one of the windows endeavouring to find shelter under the branches and lingering leaves of an oak just beyond their premises, was forced, though not without some modest reluctance on her part, to come in. A civil servant she had withstood; but when Dr~Grant himself went out with an umbrella, there was nothing to be done but to be very much ashamed, and to get into the house as fast as possible; and to poor Miss~Crawford, who had just been contemplating the dismal rain in a very desponding state of mind, sighing over the ruin of all her plan of exercise for that morning, and of every chance of seeing a single creature beyond themselves for the next twenty-four hours, the sound of a little bustle at the front door, and the sight of Miss~Price dripping with wet in the vestibule, was delightful. The value of an event on a wet day in the country was most forcibly brought before her. She was all alive again directly, and among the most active in being useful to Fanny, in detecting her to be wetter than she would at first allow, and providing her with dry clothes; and Fanny, after being obliged to submit to all this attention, and to being assisted and waited on by mistresses and maids, being also obliged, on returning downstairs, to be fixed in their drawing-room for an hour while the rain continued, the blessing of something fresh to see and think of was thus extended to Miss~Crawford, and might carry on her spirits to the period of dressing and dinner.

The two sisters were so kind to her, and so pleasant, that Fanny might have enjoyed her visit could she have believed herself not in the way, and could she have foreseen that the weather would certainly clear at the end of the hour, and save her from the shame of having Dr~Grant's carriage and horses out to take her home, with which she was threatened. As to anxiety for any alarm that her absence in such weather might occasion at home, she had nothing to suffer on that score; for as her being out was known only to her two aunts, she was perfectly aware that none would be felt, and that in whatever cottage aunt Norris might chuse to establish her during the rain, her being in such cottage would be indubitable to aunt Bertram.

It was beginning to look brighter, when Fanny, observing a harp in the room, asked some questions about it, which soon led to an acknowledgment of her wishing very much to hear it, and a confession, which could hardly be believed, of her having never yet heard it since its being in Mansfield. To Fanny herself it appeared a very simple and natural circumstance. She had scarcely ever been at the Parsonage since the instrument's arrival, there had been no reason that she should; but Miss~Crawford, calling to mind an early expressed wish on the subject, was concerned at her own neglect; and <Shall I play to you now?> and <What will you have?> were questions immediately following with the readiest good-humour.

She played accordingly; happy to have a new listener, and a listener who seemed so much obliged, so full of wonder at the performance, and who shewed herself not wanting in taste. She played till Fanny's eyes, straying to the window on the weather's being evidently fair, spoke what she felt must be done.

<Another quarter of an hour,> said Miss~Crawford, <and we shall see how it will be. Do not run away the first moment of its holding up. Those clouds look alarming.>

<But they are passed over,> said Fanny. <I have been watching them. This weather is all from the south.>

<South or north, I know a black cloud when I see it; and you must not set forward while it is so threatening. And besides, I want to play something more to you—a very pretty piece—and your cousin Edmund's prime favourite. You must stay and hear your cousin's favourite.>

Fanny felt that she must; and though she had not waited for that sentence to be thinking of Edmund, such a memento made her particularly awake to his idea, and she fancied him sitting in that room again and again, perhaps in the very spot where she sat now, listening with constant delight to the favourite air, played, as it appeared to her, with superior tone and expression; and though pleased with it herself, and glad to like whatever was liked by him, she was more sincerely impatient to go away at the conclusion of it than she had been before; and on this being evident, she was so kindly asked to call again, to take them in her walk whenever she could, to come and hear more of the harp, that she felt it necessary to be done, if no objection arose at home.

Such was the origin of the sort of intimacy which took place between them within the first fortnight after the Miss~Bertrams' going away—an intimacy resulting principally from Miss~Crawford's desire of something new, and which had little reality in Fanny's feelings. Fanny went to her every two or three days: it seemed a kind of fascination: she could not be easy without going, and yet it was without loving her, without ever thinking like her, without any sense of obligation for being sought after now when nobody else was to be had; and deriving no higher pleasure from her conversation than occasional amusement, and \textit{that}  often at the expense of her judgment, when it was raised by pleasantry on people or subjects which she wished to be respected. She went, however, and they sauntered about together many an half-hour in Mrs~Grant's shrubbery, the weather being unusually mild for the time of year, and venturing sometimes even to sit down on one of the benches now comparatively unsheltered, remaining there perhaps till, in the midst of some tender ejaculation of Fanny's on the sweets of so protracted an autumn, they were forced, by the sudden swell of a cold gust shaking down the last few yellow leaves about them, to jump up and walk for warmth.

<This is pretty, very pretty,> said Fanny, looking around her as they were thus sitting together one day; <every time I come into this shrubbery I am more struck with its growth and beauty. Three years ago, this was nothing but a rough hedgerow along the upper side of the field, never thought of as anything, or capable of becoming anything; and now it is converted into a walk, and it would be difficult to say whether most valuable as a convenience or an ornament; and perhaps, in another three years, we may be forgetting—almost forgetting what it was before. How wonderful, how very wonderful the operations of time, and the changes of the human mind!> And following the latter train of thought, she soon afterwards added: <If any one faculty of our nature may be called \textit{more}  wonderful than the rest, I do think it is memory. There seems something more speakingly incomprehensible in the powers, the failures, the inequalities of memory, than in any other of our intelligences. The memory is sometimes so retentive, so serviceable, so obedient; at others, so bewildered and so weak; and at others again, so tyrannic, so beyond control! We are, to be sure, a miracle every way; but our powers of recollecting and of forgetting do seem peculiarly past finding out.>

Miss~Crawford, untouched and inattentive, had nothing to say; and Fanny, perceiving it, brought back her own mind to what she thought must interest.

<It may seem impertinent in \textit{me}  to praise, but I must admire the taste Mrs~Grant has shewn in all this. There is such a quiet simplicity in the plan of the walk! Not too much attempted!>

<Yes,> replied Miss~Crawford carelessly, <it does very well for a place of this sort. One does not think of extent \textit{here}; and between ourselves, till I came to Mansfield, I had not imagined a country parson ever aspired to a shrubbery, or anything of the kind.>

<I am so glad to see the evergreens thrive!> said Fanny, in reply. <My uncle's gardener always says the soil here is better than his own, and so it appears from the growth of the laurels and evergreens in general. The evergreen! How beautiful, how welcome, how wonderful the evergreen! When one thinks of it, how astonishing a variety of nature! In some countries we know the tree that sheds its leaf is the variety, but that does not make it less amazing that the same soil and the same sun should nurture plants differing in the first rule and law of their existence. You will think me rhapsodising; but when I am out of doors, especially when I am sitting out of doors, I am very apt to get into this sort of wondering strain. One cannot fix one's eyes on the commonest natural production without finding food for a rambling fancy.>

<To say the truth,> replied Miss~Crawford, <I am something like the famous Doge at the court of Lewis XIV.; and may declare that I see no wonder in this shrubbery equal to seeing myself in it. If anybody had told me a year ago that this place would be my home, that I should be spending month after month here, as I have done, I certainly should not have believed them. I have now been here nearly five months; and, moreover, the quietest five months I ever passed.>

<\textit{Too}  quiet for you, I believe.>

<I should have thought so \textit{theoretically}  myself, but,> and her eyes brightened as she spoke, <take it all and all, I never spent so happy a summer. But then,> with a more thoughtful air and lowered voice, <there is no saying what it may lead to.>

Fanny's heart beat quick, and she felt quite unequal to surmising or soliciting anything more. Miss~Crawford, however, with renewed animation, soon went on—

<I am conscious of being far better reconciled to a country residence than I had ever expected to be. I can even suppose it pleasant to spend \textit{half}  the year in the country, under certain circumstances, very pleasant. An elegant, moderate-sized house in the centre of family connexions; continual engagements among them; commanding the first society in the neighbourhood; looked up to, perhaps, as leading it even more than those of larger fortune, and turning from the cheerful round of such amusements to nothing worse than a \textit{tête-à-tête}  with the person one feels most agreeable in the world. There is nothing frightful in such a picture, is there, Miss~Price? One need not envy the new Mrs~Rushworth with such a home as \textit{that}.>

<Envy Mrs~Rushworth!> was all that Fanny attempted to say. <Come, come, it would be very un-handsome in us to be severe on Mrs~Rushworth, for I look forward to our owing her a great many gay, brilliant, happy hours. I expect we shall be all very much at Sotherton another year. Such a match as Miss~Bertram has made is a public blessing; for the first pleasures of Mr~Rushworth's wife must be to fill her house, and give the best balls in the country.>

Fanny was silent, and Miss~Crawford relapsed into thoughtfulness, till suddenly looking up at the end of a few minutes, she exclaimed, <Ah! here he is.> It was not Mr~Rushworth, however, but Edmund, who then appeared walking towards them with Mrs~Grant. <My sister and Mr~Bertram. I am so glad your eldest cousin is gone, that he may be Mr~Bertram again. There is something in the sound of Mr~\textit{Edmund}  Bertram so formal, so pitiful, so younger-brother-like, that I detest it.>

<How differently we feel!> cried Fanny. <To me, the sound of \textit{Mr.}  Bertram is so cold and nothing-meaning, so entirely without warmth or character! It just stands for a gentleman, and that's all. But there is nobleness in the name of Edmund. It is a name of heroism and renown; of kings, princes, and knights; and seems to breathe the spirit of chivalry and warm affections.>

<I grant you the name is good in itself, and \textit{Lord}  Edmund or \textit{Sir}  Edmund sound delightfully; but sink it under the chill, the annihilation of a Mr., and Mr~Edmund is no more than Mr~John or Mr~Thomas. Well, shall we join and disappoint them of half their lecture upon sitting down out of doors at this time of year, by being up before they can begin?>

Edmund met them with particular pleasure. It was the first time of his seeing them together since the beginning of that better acquaintance which he had been hearing of with great satisfaction. A friendship between two so very dear to him was exactly what he could have wished: and to the credit of the lover's understanding, be it stated, that he did not by any means consider Fanny as the only, or even as the greater gainer by such a friendship.

<Well,> said Miss~Crawford, <and do you not scold us for our imprudence? What do you think we have been sitting down for but to be talked to about it, and entreated and supplicated never to do so again?>

<Perhaps I might have scolded,> said Edmund, <if either of you had been sitting down alone; but while you do wrong together, I can overlook a great deal.>

<They cannot have been sitting long,> cried Mrs~Grant, <for when I went up for my shawl I saw them from the staircase window, and then they were walking.>

<And really,> added Edmund, <the day is so mild, that your sitting down for a few minutes can be hardly thought imprudent. Our weather must not always be judged by the calendar. We may sometimes take greater liberties in November than in May.>

<Upon my word,> cried Miss~Crawford, <you are two of the most disappointing and unfeeling kind friends I ever met with! There is no giving you a moment's uneasiness. You do not know how much we have been suffering, nor what chills we have felt! But I have long thought Mr~Bertram one of the worst subjects to work on, in any little manoeuvre against common sense, that a woman could be plagued with. I had very little hope of \textit{him}  from the first; but you, Mrs~Grant, my sister, my own sister, I think I had a right to alarm you a little.>

<Do not flatter yourself, my dearest Mary. You have not the smallest chance of moving me. I have my alarms, but they are quite in a different quarter; and if I could have altered the weather, you would have had a good sharp east wind blowing on you the whole time—for here are some of my plants which Robert \textit{will}  leave out because the nights are so mild, and I know the end of it will be, that we shall have a sudden change of weather, a hard frost setting in all at once, taking everybody (at least Robert) by surprise, and I shall lose every one; and what is worse, cook has just been telling me that the turkey, which I particularly wished not to be dressed till Sunday, because I know how much more Dr~Grant would enjoy it on Sunday after the fatigues of the day, will not keep beyond to-morrow. These are something like grievances, and make me think the weather most unseasonably close.>

<The sweets of housekeeping in a country village!> said Miss~Crawford archly. <Commend me to the nurseryman and the poulterer.>

<My dear child, commend Dr~Grant to the deanery of Westminster or St~Paul's, and I should be as glad of your nurseryman and poulterer as you could be. But we have no such people in Mansfield. What would you have me do?>

<Oh! you can do nothing but what you do already: be plagued very often, and never lose your temper.>

<Thank you; but there is no escaping these little vexations, Mary, live where we may; and when you are settled in town and I come to see you, I dare say I shall find you with yours, in spite of the nurseryman and the poulterer, perhaps on their very account. Their remoteness and unpunctuality, or their exorbitant charges and frauds, will be drawing forth bitter lamentations.>

<I mean to be too rich to lament or to feel anything of the sort. A large income is the best recipe for happiness I ever heard of. It certainly may secure all the myrtle and turkey part of it.>

<You intend to be very rich?> said Edmund, with a look which, to Fanny's eye, had a great deal of serious meaning.

<To be sure. Do not you? Do not we all?>

<I cannot intend anything which it must be so completely beyond my power to command. Miss~Crawford may chuse her degree of wealth. She has only to fix on her number of thousands a year, and there can be no doubt of their coming. My intentions are only not to be poor.>

<By moderation and economy, and bringing down your wants to your income, and all that. I understand you—and a very proper plan it is for a person at your time of life, with such limited means and indifferent connexions. What can \textit{you}  want but a decent maintenance? You have not much time before you; and your relations are in no situation to do anything for you, or to mortify you by the contrast of their own wealth and consequence. Be honest and poor, by all means—but I shall not envy you; I do not much think I shall even respect you. I have a much greater respect for those that are honest and rich.>

<Your degree of respect for honesty, rich or poor, is precisely what I have no manner of concern with. I do not mean to be poor. Poverty is exactly what I have determined against. Honesty, in the something between, in the middle state of worldly circumstances, is all that I am anxious for your not looking down on.>

<But I do look down upon it, if it might have been higher. I must look down upon anything contented with obscurity when it might rise to distinction.>

<But how may it rise? How may my honesty at least rise to any distinction?>

This was not so very easy a question to answer, and occasioned an <Oh!> of some length from the fair lady before she could add, <You ought to be in parliament, or you should have gone into the army ten years ago.>

<\textit{That}  is not much to the purpose now; and as to my being in parliament, I believe I must wait till there is an especial assembly for the representation of younger sons who have little to live on. No, Miss~Crawford,> he added, in a more serious tone, <there \textit{are}  distinctions which I should be miserable if I thought myself without any chance—absolutely without chance or possibility of obtaining—but they are of a different character.>

A look of consciousness as he spoke, and what seemed a consciousness of manner on Miss~Crawford's side as she made some laughing answer, was sorrowfull food for Fanny's observation; and finding herself quite unable to attend as she ought to Mrs~Grant, by whose side she was now following the others, she had nearly resolved on going home immediately, and only waited for courage to say so, when the sound of the great clock at Mansfield Park, striking three, made her feel that she had really been much longer absent than usual, and brought the previous self-inquiry of whether she should take leave or not just then, and how, to a very speedy issue. With undoubting decision she directly began her adieus; and Edmund began at the same time to recollect that his mother had been inquiring for her, and that he had walked down to the Parsonage on purpose to bring her back.

Fanny's hurry increased; and without in the least expecting Edmund's attendance, she would have hastened away alone; but the general pace was quickened, and they all accompanied her into the house, through which it was necessary to pass. Dr~Grant was in the vestibule, and as they stopt to speak to him she found, from Edmund's manner, that he \textit{did}  mean to go with her. He too was taking leave. She could not but be thankful. In the moment of parting, Edmund was invited by Dr~Grant to eat his mutton with him the next day; and Fanny had barely time for an unpleasant feeling on the occasion, when Mrs~Grant, with sudden recollection, turned to her and asked for the pleasure of her company too. This was so new an attention, so perfectly new a circumstance in the events of Fanny's life, that she was all surprise and embarrassment; and while stammering out her great obligation, and her <but she did not suppose it would be in her power,> was looking at Edmund for his opinion and help. But Edmund, delighted with her having such an happiness offered, and ascertaining with half a look, and half a sentence, that she had no objection but on her aunt's account, could not imagine that his mother would make any difficulty of sparing her, and therefore gave his decided open advice that the invitation should be accepted; and though Fanny would not venture, even on his encouragement, to such a flight of audacious independence, it was soon settled, that if nothing were heard to the contrary, Mrs~Grant might expect her.

<And you know what your dinner will be,> said Mrs~Grant, smiling—<the turkey, and I assure you a very fine one; for, my dear,> turning to her husband, <cook insists upon the turkey's being dressed to-morrow.>

<Very well, very well,> cried Dr~Grant, <all the better; I am glad to hear you have anything so good in the house. But Miss~Price and Mr~Edmund Bertram, I dare say, would take their chance. We none of us want to hear the bill of fare. A friendly meeting, and not a fine dinner, is all we have in view. A turkey, or a goose, or a leg of mutton, or whatever you and your cook chuse to give us.>

The two cousins walked home together; and, except in the immediate discussion of this engagement, which Edmund spoke of with the warmest satisfaction, as so particularly desirable for her in the intimacy which he saw with so much pleasure established, it was a silent walk; for having finished that subject, he grew thoughtful and indisposed for any other. 