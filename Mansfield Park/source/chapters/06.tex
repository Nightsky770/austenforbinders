\chapter[Chapter \thechapter]{} 

 \lettrine[lraise=0.3]{M}{r} Bertram set off for \doubleemdash, and Miss~Crawford was prepared to find a great chasm in their society, and to miss him decidedly in the meetings which were now becoming almost daily between the families; and on their all dining together at the Park soon after his going, she retook her chosen place near the bottom of the table, fully expecting to feel a most melancholy difference in the change of masters. It would be a very flat business, she was sure. In comparison with his brother, Edmund would have nothing to say. The soup would be sent round in a most spiritless manner, wine drank without any smiles or agreeable trifling, and the venison cut up without supplying one pleasant anecdote of any former haunch, or a single entertaining story, about <my friend such a one.> She must try to find amusement in what was passing at the upper end of the table, and in observing Mr~Rushworth, who was now making his appearance at Mansfield for the first time since the Crawfords' arrival. He had been visiting a friend in the neighbouring county, and that friend having recently had his grounds laid out by an improver, Mr~Rushworth was returned with his head full of the subject, and very eager to be improving his own place in the same way; and though not saying much to the purpose, could talk of nothing else. The subject had been already handled in the drawing-room; it was revived in the dining-parlour. Miss~Bertram's attention and opinion was evidently his chief aim; and though her deportment showed rather conscious superiority than any solicitude to oblige him, the mention of Sotherton Court, and the ideas attached to it, gave her a feeling of complacency, which prevented her from being very ungracious.

<I wish you could see Compton,> said he; <it is the most complete thing! I never saw a place so altered in my life. I told Smith I did not know where I was. The approach \textit{now}, is one of the finest things in the country: you see the house in the most surprising manner. I declare, when I got back to Sotherton yesterday, it looked like a prison—quite a dismal old prison.>

<Oh, for shame!> cried Mrs~Norris. <A prison indeed? Sotherton Court is the noblest old place in the world.>

<It wants improvement, ma'am, beyond anything. I never saw a place that wanted so much improvement in my life; and it is so forlorn that I do not know what can be done with it.>

<No wonder that Mr~Rushworth should think so at present,> said Mrs~Grant to Mrs~Norris, with a smile; <but depend upon it, Sotherton will have \textit{every}  improvement in time which his heart can desire.>

<I must try to do something with it,> said Mr~Rushworth, <but I do not know what. I hope I shall have some good friend to help me.>

<Your best friend upon such an occasion,> said Miss~Bertram calmly, <would be Mr~Repton, I imagine.>

<That is what I was thinking of. As he has done so well by Smith, I think I had better have him at once. His terms are five guineas a day.>

<Well, and if they were \textit{ten},> cried Mrs~Norris, <I am sure \textit{you}  need not regard it. The expense need not be any impediment. If I were you, I should not think of the expense. I would have everything done in the best style, and made as nice as possible. Such a place as Sotherton Court deserves everything that taste and money can do. You have space to work upon there, and grounds that will well reward you. For my own part, if I had anything within the fiftieth part of the size of Sotherton, I should be always planting and improving, for naturally I am excessively fond of it. It would be too ridiculous for me to attempt anything where I am now, with my little half acre. It would be quite a burlesque. But if I had more room, I should take a prodigious delight in improving and planting. We did a vast deal in that way at the Parsonage: we made it quite a different place from what it was when we first had it. You young ones do not remember much about it, perhaps; but if dear Sir~Thomas were here, he could tell you what improvements we made: and a great deal more would have been done, but for poor Mr~Norris's sad state of health. He could hardly ever get out, poor man, to enjoy anything, and \textit{that}  disheartened me from doing several things that Sir~Thomas and I used to talk of. If it had not been for \textit{that}, we should have carried on the garden wall, and made the plantation to shut out the churchyard, just as Dr~Grant has done. We were always doing something as it was. It was only the spring twelvemonth before Mr~Norris's death that we put in the apricot against the stable wall, which is now grown such a noble tree, and getting to such perfection, sir,> addressing herself then to Dr~Grant.

<The tree thrives well, beyond a doubt, madam,> replied Dr~Grant. <The soil is good; and I never pass it without regretting that the fruit should be so little worth the trouble of gathering.>

<Sir, it is a Moor Park, we bought it as a Moor Park, and it cost us—that is, it was a present from Sir~Thomas, but I saw the bill—and I know it cost seven shillings, and was charged as a Moor Park.>

<You were imposed on, ma'am,> replied Dr~Grant: <these potatoes have as much the flavour of a Moor Park apricot as the fruit from that tree. It is an insipid fruit at the best; but a good apricot is eatable, which none from my garden are.>

<The truth is, ma'am,> said Mrs~Grant, pretending to whisper across the table to Mrs~Norris, <that Dr~Grant hardly knows what the natural taste of our apricot is: he is scarcely ever indulged with one, for it is so valuable a fruit; with a little assistance, and ours is such a remarkably large, fair sort, that what with early tarts and preserves, my cook contrives to get them all.>

Mrs~Norris, who had begun to redden, was appeased; and, for a little while, other subjects took place of the improvements of Sotherton. Dr~Grant and Mrs~Norris were seldom good friends; their acquaintance had begun in dilapidations, and their habits were totally dissimilar.

After a short interruption Mr~Rushworth began again. <Smith's place is the admiration of all the country; and it was a mere nothing before Repton took it in hand. I think I shall have Repton.>

<Mr~Rushworth,> said Lady Bertram, <if I were you, I would have a very pretty shrubbery. One likes to get out into a shrubbery in fine weather.>

Mr~Rushworth was eager to assure her ladyship of his acquiescence, and tried to make out something complimentary; but, between his submission to \textit{her}  taste, and his having always intended the same himself, with the superadded objects of professing attention to the comfort of ladies in general, and of insinuating that there was one only whom he was anxious to please, he grew puzzled, and Edmund was glad to put an end to his speech by a proposal of wine. Mr~Rushworth, however, though not usually a great talker, had still more to say on the subject next his heart. <Smith has not much above a hundred acres altogether in his grounds, which is little enough, and makes it more surprising that the place can have been so improved. Now, at Sotherton we have a good seven hundred, without reckoning the water meadows; so that I think, if so much could be done at Compton, we need not despair. There have been two or three fine old trees cut down, that grew too near the house, and it opens the prospect amazingly, which makes me think that Repton, or anybody of that sort, would certainly have the avenue at Sotherton down: the avenue that leads from the west front to the top of the hill, you know,> turning to Miss~Bertram particularly as he spoke. But Miss~Bertram thought it most becoming to reply—

<The avenue! Oh! I do not recollect it. I really know very little of Sotherton.>

Fanny, who was sitting on the other side of Edmund, exactly opposite Miss~Crawford, and who had been attentively listening, now looked at him, and said in a low voice—

<Cut down an avenue! What a pity! Does it not make you think of Cowper? <Ye fallen avenues, once more I mourn your fate unmerited.>>

He smiled as he answered, <I am afraid the avenue stands a bad chance, Fanny.>

<I should like to see Sotherton before it is cut down, to see the place as it is now, in its old state; but I do not suppose I shall.>

<Have you never been there? No, you never can; and, unluckily, it is out of distance for a ride. I wish we could contrive it.>

<Oh! it does not signify. Whenever I do see it, you will tell me how it has been altered.>

<I collect,> said Miss~Crawford, <that Sotherton is an old place, and a place of some grandeur. In any particular style of building?>

<The house was built in Elizabeth's time, and is a large, regular, brick building; heavy, but respectable looking, and has many good rooms. It is ill placed. It stands in one of the lowest spots of the park; in that respect, unfavourable for improvement. But the woods are fine, and there is a stream, which, I dare say, might be made a good deal of. Mr~Rushworth is quite right, I think, in meaning to give it a modern dress, and I have no doubt that it will be all done extremely well.>

Miss~Crawford listened with submission, and said to herself, <He is a well-bred man; he makes the best of it.>

<I do not wish to influence Mr~Rushworth,> he continued; <but, had I a place to new fashion, I should not put myself into the hands of an improver. I would rather have an inferior degree of beauty, of my own choice, and acquired progressively. I would rather abide by my own blunders than by his.>

<\textit{You}  would know what you were about, of course; but that would not suit \textit{me}. I have no eye or ingenuity for such matters, but as they are before me; and had I a place of my own in the country, I should be most thankful to any Mr~Repton who would undertake it, and give me as much beauty as he could for my money; and I should never look at it till it was complete.>

<It would be delightful to \textit{me}  to see the progress of it all,> said Fanny.

<Ay, you have been brought up to it. It was no part of my education; and the only dose I ever had, being administered by not the first favourite in the world, has made me consider improvements \textit{in hand}  as the greatest of nuisances. Three years ago the Admiral, my honoured uncle, bought a cottage at Twickenham for us all to spend our summers in; and my aunt and I went down to it quite in raptures; but it being excessively pretty, it was soon found necessary to be improved, and for three months we were all dirt and confusion, without a gravel walk to step on, or a bench fit for use. I would have everything as complete as possible in the country, shrubberies and flower-gardens, and rustic seats innumerable: but it must all be done without my care. Henry is different; he loves to be doing.>

Edmund was sorry to hear Miss~Crawford, whom he was much disposed to admire, speak so freely of her uncle. It did not suit his sense of propriety, and he was silenced, till induced by further smiles and liveliness to put the matter by for the present.

<Mr~Bertram,> said she, <I have tidings of my harp at last. I am assured that it is safe at Northampton; and there it has probably been these ten days, in spite of the solemn assurances we have so often received to the contrary.> Edmund expressed his pleasure and surprise. <The truth is, that our inquiries were too direct; we sent a servant, we went ourselves: this will not do seventy miles from London; but this morning we heard of it in the right way. It was seen by some farmer, and he told the miller, and the miller told the butcher, and the butcher's son-in-law left word at the shop.>

<I am very glad that you have heard of it, by whatever means, and hope there will be no further delay.>

<I am to have it to-morrow; but how do you think it is to be conveyed? Not by a wagon or cart: oh no! nothing of that kind could be hired in the village. I might as well have asked for porters and a handbarrow.>

<You would find it difficult, I dare say, just now, in the middle of a very late hay harvest, to hire a horse and cart?>

<I was astonished to find what a piece of work was made of it! To want a horse and cart in the country seemed impossible, so I told my maid to speak for one directly; and as I cannot look out of my dressing-closet without seeing one farmyard, nor walk in the shrubbery without passing another, I thought it would be only ask and have, and was rather grieved that I could not give the advantage to all. Guess my surprise, when I found that I had been asking the most unreasonable, most impossible thing in the world; had offended all the farmers, all the labourers, all the hay in the parish! As for Dr~Grant's bailiff, I believe I had better keep out of \textit{his}  way; and my brother-in-law himself, who is all kindness in general, looked rather black upon me when he found what I had been at.>

<You could not be expected to have thought on the subject before; but when you \textit{do}  think of it, you must see the importance of getting in the grass. The hire of a cart at any time might not be so easy as you suppose: our farmers are not in the habit of letting them out; but, in harvest, it must be quite out of their power to spare a horse.>

<I shall understand all your ways in time; but, coming down with the true London maxim, that everything is to be got with money, I was a little embarrassed at first by the sturdy independence of your country customs. However, I am to have my harp fetched to-morrow. Henry, who is good-nature itself, has offered to fetch it in his barouche. Will it not be honourably conveyed?>

Edmund spoke of the harp as his favourite instrument, and hoped to be soon allowed to hear her. Fanny had never heard the harp at all, and wished for it very much.

<I shall be most happy to play to you both,> said Miss~Crawford; <at least as long as you can like to listen: probably much longer, for I dearly love music myself, and where the natural taste is equal the player must always be best off, for she is gratified in more ways than one. Now, Mr~Bertram, if you write to your brother, I entreat you to tell him that my harp is come: he heard so much of my misery about it. And you may say, if you please, that I shall prepare my most plaintive airs against his return, in compassion to his feelings, as I know his horse will lose.>

<If I write, I will say whatever you wish me; but I do not, at present, foresee any occasion for writing.>

<No, I dare say, nor if he were to be gone a twelvemonth, would you ever write to him, nor he to you, if it could be helped. The occasion would never be foreseen. What strange creatures brothers are! You would not write to each other but upon the most urgent necessity in the world; and when obliged to take up the pen to say that such a horse is ill, or such a relation dead, it is done in the fewest possible words. You have but one style among you. I know it perfectly. Henry, who is in every other respect exactly what a brother should be, who loves me, consults me, confides in me, and will talk to me by the hour together, has never yet turned the page in a letter; and very often it is nothing more than—<Dear Mary, I am just arrived. Bath seems full, and everything as usual. Yours sincerely.' That is the true manly style; that is a complete brother>s letter.>

<When they are at a distance from all their family,> said Fanny, colouring for William's sake, <they can write long letters.>

<Miss~Price has a brother at sea,> said Edmund, <whose excellence as a correspondent makes her think you too severe upon us.>

<At sea, has she? In the king's service, of course?>

Fanny would rather have had Edmund tell the story, but his determined silence obliged her to relate her brother's situation: her voice was animated in speaking of his profession, and the foreign stations he had been on; but she could not mention the number of years that he had been absent without tears in her eyes. Miss~Crawford civilly wished him an early promotion.

<Do you know anything of my cousin's captain?> said Edmund; <Captain Marshall? You have a large acquaintance in the navy, I conclude?>

<Among admirals, large enough; but,> with an air of grandeur, <we know very little of the inferior ranks. Post-captains may be very good sort of men, but they do not belong to \textit{us}. Of various admirals I could tell you a great deal: of them and their flags, and the gradation of their pay, and their bickerings and jealousies. But, in general, I can assure you that they are all passed over, and all very ill used. Certainly, my home at my uncle's brought me acquainted with a circle of admirals. Of \textit{Rears}  and \textit{Vices}  I saw enough. Now do not be suspecting me of a pun, I entreat.>

Edmund again felt grave, and only replied, <It is a noble profession.>

<Yes, the profession is well enough under two circumstances: if it make the fortune, and there be discretion in spending it; but, in short, it is not a favourite profession of mine. It has never worn an amiable form to \textit{me}.>

Edmund reverted to the harp, and was again very happy in the prospect of hearing her play.

The subject of improving grounds, meanwhile, was still under consideration among the others; and Mrs~Grant could not help addressing her brother, though it was calling his attention from Miss~Julia Bertram.

<My dear Henry, have \textit{you}  nothing to say? You have been an improver yourself, and from what I hear of Everingham, it may vie with any place in England. Its natural beauties, I am sure, are great. Everingham, as it \textit{used}  to be, was perfect in my estimation: such a happy fall of ground, and such timber! What would I not give to see it again!>

<Nothing could be so gratifying to me as to hear your opinion of it,> was his answer; <but I fear there would be some disappointment: you would not find it equal to your present ideas. In extent, it is a mere nothing; you would be surprised at its insignificance; and, as for improvement, there was very little for me to do—too little: I should like to have been busy much longer.>

<You are fond of the sort of thing?> said Julia.

<Excessively; but what with the natural advantages of the ground, which pointed out, even to a very young eye, what little remained to be done, and my own consequent resolutions, I had not been of age three months before Everingham was all that it is now. My plan was laid at Westminster, a little altered, perhaps, at Cambridge, and at one-and-twenty executed. I am inclined to envy Mr~Rushworth for having so much happiness yet before him. I have been a devourer of my own.>

<Those who see quickly, will resolve quickly, and act quickly,> said Julia. <\textit{You}  can never want employment. Instead of envying Mr~Rushworth, you should assist him with your opinion.>

Mrs~Grant, hearing the latter part of this speech, enforced it warmly, persuaded that no judgment could be equal to her brother's; and as Miss~Bertram caught at the idea likewise, and gave it her full support, declaring that, in her opinion, it was infinitely better to consult with friends and disinterested advisers, than immediately to throw the business into the hands of a professional man, Mr~Rushworth was very ready to request the favour of Mr~Crawford's assistance; and Mr~Crawford, after properly depreciating his own abilities, was quite at his service in any way that could be useful. Mr~Rushworth then began to propose Mr~Crawford's doing him the honour of coming over to Sotherton, and taking a bed there; when Mrs~Norris, as if reading in her two nieces' minds their little approbation of a plan which was to take Mr~Crawford away, interposed with an amendment.

<There can be no doubt of Mr~Crawford's willingness; but why should not more of us go? Why should not we make a little party? Here are many that would be interested in your improvements, my dear Mr~Rushworth, and that would like to hear Mr~Crawford's opinion on the spot, and that might be of some small use to you with \textit{their}  opinions; and, for my own part, I have been long wishing to wait upon your good mother again; nothing but having no horses of my own could have made me so remiss; but now I could go and sit a few hours with Mrs~Rushworth, while the rest of you walked about and settled things, and then we could all return to a late dinner here, or dine at Sotherton, just as might be most agreeable to your mother, and have a pleasant drive home by moonlight. I dare say Mr~Crawford would take my two nieces and me in his barouche, and Edmund can go on horseback, you know, sister, and Fanny will stay at home with you.>

Lady Bertram made no objection; and every one concerned in the going was forward in expressing their ready concurrence, excepting Edmund, who heard it all and said nothing. 