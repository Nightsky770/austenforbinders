\chapter[Chapter \thechapter]{} 

 \lettrine[lraise=0.3]{T}{he} novelty of travelling, and the happiness of being with William, soon produced their natural effect on Fanny's spirits, when Mansfield Park was fairly left behind; and by the time their first stage was ended, and they were to quit Sir~Thomas's carriage, she was able to take leave of the old coachman, and send back proper messages, with cheerful looks.

Of pleasant talk between the brother and sister there was no end. Everything supplied an amusement to the high glee of William's mind, and he was full of frolic and joke in the intervals of their higher-toned subjects, all of which ended, if they did not begin, in praise of the Thrush, conjectures how she would be employed, schemes for an action with some superior force, which (supposing the first lieutenant out of the way, and William was not very merciful to the first lieutenant) was to give himself the next step as soon as possible, or speculations upon prize-money, which was to be generously distributed at home, with only the reservation of enough to make the little cottage comfortable, in which he and Fanny were to pass all their middle and later life together.

Fanny's immediate concerns, as far as they involved Mr~Crawford, made no part of their conversation. William knew what had passed, and from his heart lamented that his sister's feelings should be so cold towards a man whom he must consider as the first of human characters; but he was of an age to be all for love, and therefore unable to blame; and knowing her wish on the subject, he would not distress her by the slightest allusion.

She had reason to suppose herself not yet forgotten by Mr~Crawford. She had heard repeatedly from his sister within the three weeks which had passed since their leaving Mansfield, and in each letter there had been a few lines from himself, warm and determined like his speeches. It was a correspondence which Fanny found quite as unpleasant as she had feared. Miss~Crawford's style of writing, lively and affectionate, was itself an evil, independent of what she was thus forced into reading from the brother's pen, for Edmund would never rest till she had read the chief of the letter to him; and then she had to listen to his admiration of her language, and the warmth of her attachments. There had, in fact, been so much of message, of allusion, of recollection, so much of Mansfield in every letter, that Fanny could not but suppose it meant for him to hear; and to find herself forced into a purpose of that kind, compelled into a correspondence which was bringing her the addresses of the man she did not love, and obliging her to administer to the adverse passion of the man she did, was cruelly mortifying. Here, too, her present removal promised advantage. When no longer under the same roof with Edmund, she trusted that Miss~Crawford would have no motive for writing strong enough to overcome the trouble, and that at Portsmouth their correspondence would dwindle into nothing.

With such thoughts as these, among ten hundred others, Fanny proceeded in her journey safely and cheerfully, and as expeditiously as could rationally be hoped in the dirty month of February. They entered Oxford, but she could take only a hasty glimpse of Edmund's college as they passed along, and made no stop anywhere till they reached Newbury, where a comfortable meal, uniting dinner and supper, wound up the enjoyments and fatigues of the day.

The next morning saw them off again at an early hour; and with no events, and no delays, they regularly advanced, and were in the environs of Portsmouth while there was yet daylight for Fanny to look around her, and wonder at the new buildings. They passed the drawbridge, and entered the town; and the light was only beginning to fail as, guided by William's powerful voice, they were rattled into a narrow street, leading from the High Street, and drawn up before the door of a small house now inhabited by Mr~Price.

Fanny was all agitation and flutter; all hope and apprehension. The moment they stopped, a trollopy-looking maidservant, seemingly in waiting for them at the door, stepped forward, and more intent on telling the news than giving them any help, immediately began with, <The Thrush is gone out of harbour, please sir, and one of the officers has been here to\longdash> She was interrupted by a fine tall boy of eleven years old, who, rushing out of the house, pushed the maid aside, and while William was opening the chaise-door himself, called out, <You are just in time. We have been looking for you this half-hour. The Thrush went out of harbour this morning. I saw her. It was a beautiful sight. And they think she will have her orders in a day or two. And Mr~Campbell was here at four o'clock to ask for you: he has got one of the Thrush's boats, and is going off to her at six, and hoped you would be here in time to go with him.>

A stare or two at Fanny, as William helped her out of the carriage, was all the voluntary notice which this brother bestowed; but he made no objection to her kissing him, though still entirely engaged in detailing farther particulars of the Thrush's going out of harbour, in which he had a strong right of interest, being to commence his career of seamanship in her at this very time.

Another moment and Fanny was in the narrow entrance-passage of the house, and in her mother's arms, who met her there with looks of true kindness, and with features which Fanny loved the more, because they brought her aunt Bertram's before her, and there were her two sisters: Susan, a well-grown fine girl of fourteen, and Betsey, the youngest of the family, about five—both glad to see her in their way, though with no advantage of manner in receiving her. But manner Fanny did not want. Would they but love her, she should be satisfied.

She was then taken into a parlour, so small that her first conviction was of its being only a passage-room to something better, and she stood for a moment expecting to be invited on; but when she saw there was no other door, and that there were signs of habitation before her, she called back her thoughts, reproved herself, and grieved lest they should have been suspected. Her mother, however, could not stay long enough to suspect anything. She was gone again to the street-door, to welcome William. <Oh! my dear William, how glad I am to see you. But have you heard about the Thrush? She is gone out of harbour already; three days before we had any thought of it; and I do not know what I am to do about Sam's things, they will never be ready in time; for she may have her orders to-morrow, perhaps. It takes me quite unawares. And now you must be off for Spithead too. Campbell has been here, quite in a worry about you; and now what shall we do? I thought to have had such a comfortable evening with you, and here everything comes upon me at once.>

Her son answered cheerfully, telling her that everything was always for the best; and making light of his own inconvenience in being obliged to hurry away so soon.

<To be sure, I had much rather she had stayed in harbour, that I might have sat a few hours with you in comfort; but as there is a boat ashore, I had better go off at once, and there is no help for it. Whereabouts does the Thrush lay at Spithead? Near the Canopus? But no matter; here's Fanny in the parlour, and why should we stay in the passage? Come, mother, you have hardly looked at your own dear Fanny yet.>

In they both came, and Mrs~Price having kindly kissed her daughter again, and commented a little on her growth, began with very natural solicitude to feel for their fatigues and wants as travellers.

<Poor dears! how tired you must both be! and now, what will you have? I began to think you would never come. Betsey and I have been watching for you this half-hour. And when did you get anything to eat? And what would you like to have now? I could not tell whether you would be for some meat, or only a dish of tea, after your journey, or else I would have got something ready. And now I am afraid Campbell will be here before there is time to dress a steak, and we have no butcher at hand. It is very inconvenient to have no butcher in the street. We were better off in our last house. Perhaps you would like some tea as soon as it can be got.>

They both declared they should prefer it to anything. <Then, Betsey, my dear, run into the kitchen and see if Rebecca has put the water on; and tell her to bring in the tea-things as soon as she can. I wish we could get the bell mended; but Betsey is a very handy little messenger.>

Betsey went with alacrity, proud to shew her abilities before her fine new sister.

<Dear me!> continued the anxious mother, <what a sad fire we have got, and I dare say you are both starved with cold. Draw your chair nearer, my dear. I cannot think what Rebecca has been about. I am sure I told her to bring some coals half an hour ago. Susan, you should have taken care of the fire.>

<I was upstairs, mama, moving my things,> said Susan, in a fearless, self-defending tone, which startled Fanny. <You know you had but just settled that my sister Fanny and I should have the other room; and I could not get Rebecca to give me any help.>

Farther discussion was prevented by various bustles: first, the driver came to be paid; then there was a squabble between Sam and Rebecca about the manner of carrying up his sister's trunk, which he would manage all his own way; and lastly, in walked Mr~Price himself, his own loud voice preceding him, as with something of the oath kind he kicked away his son's portmanteau and his daughter's bandbox in the passage, and called out for a candle; no candle was brought, however, and he walked into the room.

Fanny with doubting feelings had risen to meet him, but sank down again on finding herself undistinguished in the dusk, and unthought of. With a friendly shake of his son's hand, and an eager voice, he instantly began—<Ha! welcome back, my boy. Glad to see you. Have you heard the news? The Thrush went out of harbour this morning. Sharp is the word, you see! By G\emdashpunct{,} you are just in time! The doctor has been here inquiring for you: he has got one of the boats, and is to be off for Spithead by six, so you had better go with him. I have been to Turner's about your mess; it is all in a way to be done. I should not wonder if you had your orders to-morrow: but you cannot sail with this wind, if you are to cruise to the westward; and Captain Walsh thinks you will certainly have a cruise to the westward, with the Elephant. By G\emdashpunct{,} I wish you may! But old Scholey was saying, just now, that he thought you would be sent first to the Texel. Well, well, we are ready, whatever happens. But by G\emdashpunct{,} you lost a fine sight by not being here in the morning to see the Thrush go out of harbour! I would not have been out of the way for a thousand pounds. Old Scholey ran in at breakfast-time, to say she had slipped her moorings and was coming out, I jumped up, and made but two steps to the platform. If ever there was a perfect beauty afloat, she is one; and there she lays at Spithead, and anybody in England would take her for an eight-and-twenty. I was upon the platform two hours this afternoon looking at her. She lays close to the Endymion, between her and the Cleopatra, just to the eastward of the sheer hulk.>

<Ha!> cried William, <\textit{that's}  just where I should have put her myself. It's the best berth at Spithead. But here is my sister, sir; here is Fanny,> turning and leading her forward; <it is so dark you do not see her.>

With an acknowledgment that he had quite forgot her, Mr~Price now received his daughter; and having given her a cordial hug, and observed that she was grown into a woman, and he supposed would be wanting a husband soon, seemed very much inclined to forget her again. Fanny shrunk back to her seat, with feelings sadly pained by his language and his smell of spirits; and he talked on only to his son, and only of the Thrush, though William, warmly interested as he was in that subject, more than once tried to make his father think of Fanny, and her long absence and long journey.

After sitting some time longer, a candle was obtained; but as there was still no appearance of tea, nor, from Betsey's reports from the kitchen, much hope of any under a considerable period, William determined to go and change his dress, and make the necessary preparations for his removal on board directly, that he might have his tea in comfort afterwards.

As he left the room, two rosy-faced boys, ragged and dirty, about eight and nine years old, rushed into it just released from school, and coming eagerly to see their sister, and tell that the Thrush was gone out of harbour; Tom and Charles. Charles had been born since Fanny's going away, but Tom she had often helped to nurse, and now felt a particular pleasure in seeing again. Both were kissed very tenderly, but Tom she wanted to keep by her, to try to trace the features of the baby she had loved, and talked to, of his infant preference of herself. Tom, however, had no mind for such treatment: he came home not to stand and be talked to, but to run about and make a noise; and both boys had soon burst from her, and slammed the parlour-door till her temples ached.

She had now seen all that were at home; there remained only two brothers between herself and Susan, one of whom was a clerk in a public office in London, and the other midshipman on board an Indiaman. But though she had \textit{seen}  all the members of the family, she had not yet \textit{heard}  all the noise they could make. Another quarter of an hour brought her a great deal more. William was soon calling out from the landing-place of the second story for his mother and for Rebecca. He was in distress for something that he had left there, and did not find again. A key was mislaid, Betsey accused of having got at his new hat, and some slight, but essential alteration of his uniform waistcoat, which he had been promised to have done for him, entirely neglected.

Mrs~Price, Rebecca, and Betsey all went up to defend themselves, all talking together, but Rebecca loudest, and the job was to be done as well as it could in a great hurry; William trying in vain to send Betsey down again, or keep her from being troublesome where she was; the whole of which, as almost every door in the house was open, could be plainly distinguished in the parlour, except when drowned at intervals by the superior noise of Sam, Tom, and Charles chasing each other up and down stairs, and tumbling about and hallooing.

Fanny was almost stunned. The smallness of the house and thinness of the walls brought everything so close to her, that, added to the fatigue of her journey, and all her recent agitation, she hardly knew how to bear it. \textit{Within}  the room all was tranquil enough, for Susan having disappeared with the others, there were soon only her father and herself remaining; and he, taking out a newspaper, the accustomary loan of a neighbour, applied himself to studying it, without seeming to recollect her existence. The solitary candle was held between himself and the paper, without any reference to her possible convenience; but she had nothing to do, and was glad to have the light screened from her aching head, as she sat in bewildered, broken, sorrowful contemplation.

She was at home. But, alas! it was not such a home, she had not such a welcome, as—she checked herself; she was unreasonable. What right had she to be of importance to her family? She could have none, so long lost sight of! William's concerns must be dearest, they always had been, and he had every right. Yet to have so little said or asked about herself, to have scarcely an inquiry made after Mansfield! It did pain her to have Mansfield forgotten; the friends who had done so much—the dear, dear friends! But here, one subject swallowed up all the rest. Perhaps it must be so. The destination of the Thrush must be now preeminently interesting. A day or two might shew the difference. \textit{She}  only was to blame. Yet she thought it would not have been so at Mansfield. No, in her uncle's house there would have been a consideration of times and seasons, a regulation of subject, a propriety, an attention towards everybody which there was not here.

The only interruption which thoughts like these received for nearly half an hour was from a sudden burst of her father's, not at all calculated to compose them. At a more than ordinary pitch of thumping and hallooing in the passage, he exclaimed, <Devil take those young dogs! How they are singing out! Ay, Sam's voice louder than all the rest! That boy is fit for a boatswain. Holla, you there! Sam, stop your confounded pipe, or I shall be after you.>

This threat was so palpably disregarded, that though within five minutes afterwards the three boys all burst into the room together and sat down, Fanny could not consider it as a proof of anything more than their being for the time thoroughly fagged, which their hot faces and panting breaths seemed to prove, especially as they were still kicking each other's shins, and hallooing out at sudden starts immediately under their father's eye.

The next opening of the door brought something more welcome: it was for the tea-things, which she had begun almost to despair of seeing that evening. Susan and an attendant girl, whose inferior appearance informed Fanny, to her great surprise, that she had previously seen the upper servant, brought in everything necessary for the meal; Susan looking, as she put the kettle on the fire and glanced at her sister, as if divided between the agreeable triumph of shewing her activity and usefulness, and the dread of being thought to demean herself by such an office. <She had been into the kitchen,> she said, <to hurry Sally and help make the toast, and spread the bread and butter, or she did not know when they should have got tea, and she was sure her sister must want something after her journey.>

Fanny was very thankful. She could not but own that she should be very glad of a little tea, and Susan immediately set about making it, as if pleased to have the employment all to herself; and with only a little unnecessary bustle, and some few injudicious attempts at keeping her brothers in better order than she could, acquitted herself very well. Fanny's spirit was as much refreshed as her body; her head and heart were soon the better for such well-timed kindness. Susan had an open, sensible countenance; she was like William, and Fanny hoped to find her like him in disposition and goodwill towards herself.

In this more placid state of things William reentered, followed not far behind by his mother and Betsey. He, complete in his lieutenant's uniform, looking and moving all the taller, firmer, and more graceful for it, and with the happiest smile over his face, walked up directly to Fanny, who, rising from her seat, looked at him for a moment in speechless admiration, and then threw her arms round his neck to sob out her various emotions of pain and pleasure.

Anxious not to appear unhappy, she soon recovered herself; and wiping away her tears, was able to notice and admire all the striking parts of his dress; listening with reviving spirits to his cheerful hopes of being on shore some part of every day before they sailed, and even of getting her to Spithead to see the sloop.

The next bustle brought in Mr~Campbell, the surgeon of the Thrush, a very well-behaved young man, who came to call for his friend, and for whom there was with some contrivance found a chair, and with some hasty washing of the young tea-maker's, a cup and saucer; and after another quarter of an hour of earnest talk between the gentlemen, noise rising upon noise, and bustle upon bustle, men and boys at last all in motion together, the moment came for setting off; everything was ready, William took leave, and all of them were gone; for the three boys, in spite of their mother's entreaty, determined to see their brother and Mr~Campbell to the sally-port; and Mr~Price walked off at the same time to carry back his neighbour's newspaper.

Something like tranquillity might now be hoped for; and accordingly, when Rebecca had been prevailed on to carry away the tea-things, and Mrs~Price had walked about the room some time looking for a shirt-sleeve, which Betsey at last hunted out from a drawer in the kitchen, the small party of females were pretty well composed, and the mother having lamented again over the impossibility of getting Sam ready in time, was at leisure to think of her eldest daughter and the friends she had come from.

A few inquiries began: but one of the earliest—<How did sister Bertram manage about her servants?> <Was she as much plagued as herself to get tolerable servants?>—soon led her mind away from Northamptonshire, and fixed it on her own domestic grievances, and the shocking character of all the Portsmouth servants, of whom she believed her own two were the very worst, engrossed her completely. The Bertrams were all forgotten in detailing the faults of Rebecca, against whom Susan had also much to depose, and little Betsey a great deal more, and who did seem so thoroughly without a single recommendation, that Fanny could not help modestly presuming that her mother meant to part with her when her year was up.

<Her year!> cried Mrs~Price; <I am sure I hope I shall be rid of her before she has staid a year, for that will not be up till November. Servants are come to such a pass, my dear, in Portsmouth, that it is quite a miracle if one keeps them more than half a year. I have no hope of ever being settled; and if I was to part with Rebecca, I should only get something worse. And yet I do not think I am a very difficult mistress to please; and I am sure the place is easy enough, for there is always a girl under her, and I often do half the work myself.>

Fanny was silent; but not from being convinced that there might not be a remedy found for some of these evils. As she now sat looking at Betsey, she could not but think particularly of another sister, a very pretty little girl, whom she had left there not much younger when she went into Northamptonshire, who had died a few years afterwards. There had been something remarkably amiable about her. Fanny in those early days had preferred her to Susan; and when the news of her death had at last reached Mansfield, had for a short time been quite afflicted. The sight of Betsey brought the image of little Mary back again, but she would not have pained her mother by alluding to her for the world. While considering her with these ideas, Betsey, at a small distance, was holding out something to catch her eyes, meaning to screen it at the same time from Susan's.

<What have you got there, my love?> said Fanny; <come and shew it to me.>

It was a silver knife. Up jumped Susan, claiming it as her own, and trying to get it away; but the child ran to her mother's protection, and Susan could only reproach, which she did very warmly, and evidently hoping to interest Fanny on her side. <It was very hard that she was not to have her \textit{own}  knife; it was her own knife; little sister Mary had left it to her upon her deathbed, and she ought to have had it to keep herself long ago. But mama kept it from her, and was always letting Betsey get hold of it; and the end of it would be that Betsey would spoil it, and get it for her own, though mama had \textit{promised}  her that Betsey should not have it in her own hands.>

Fanny was quite shocked. Every feeling of duty, honour, and tenderness was wounded by her sister's speech and her mother's reply.

<Now, Susan,> cried Mrs~Price, in a complaining voice, <now, how can you be so cross? You are always quarrelling about that knife. I wish you would not be so quarrelsome. Poor little Betsey; how cross Susan is to you! But you should not have taken it out, my dear, when I sent you to the drawer. You know I told you not to touch it, because Susan is so cross about it. I must hide it another time, Betsey. Poor Mary little thought it would be such a bone of contention when she gave it me to keep, only two hours before she died. Poor little soul! she could but just speak to be heard, and she said so prettily, <Let sister Susan have my knife, mama, when I am dead and buried.> Poor little dear! she was so fond of it, Fanny, that she would have it lay by her in bed, all through her illness. It was the gift of her good godmother, old Mrs~Admiral Maxwell, only six weeks before she was taken for death. Poor little sweet creature! Well, she was taken away from evil to come. My own Betsey> (fondling her), <\textit{you}  have not the luck of such a good godmother. Aunt Norris lives too far off to think of such little people as you.>

Fanny had indeed nothing to convey from aunt Norris, but a message to say she hoped that her god-daughter was a good girl, and learnt her book. There had been at one moment a slight murmur in the drawing-room at Mansfield Park about sending her a prayer-book; but no second sound had been heard of such a purpose. Mrs~Norris, however, had gone home and taken down two old prayer-books of her husband with that idea; but, upon examination, the ardour of generosity went off. One was found to have too small a print for a child's eyes, and the other to be too cumbersome for her to carry about.

Fanny, fatigued and fatigued again, was thankful to accept the first invitation of going to bed; and before Betsey had finished her cry at being allowed to sit up only one hour extraordinary in honour of sister, she was off, leaving all below in confusion and noise again; the boys begging for toasted cheese, her father calling out for his rum and water, and Rebecca never where she ought to be.

There was nothing to raise her spirits in the confined and scantily furnished chamber that she was to share with Susan. The smallness of the rooms above and below, indeed, and the narrowness of the passage and staircase, struck her beyond her imagination. She soon learned to think with respect of her own little attic at Mansfield Park, in \textit{that}  house reckoned too small for anybody's comfort. 