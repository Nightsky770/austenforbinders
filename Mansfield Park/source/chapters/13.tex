\chapter[Chapter \thechapter]{} 

 \lettrine[lraise=0.3]{T}{he} Honourable John Yates, this new friend, had not much to recommend him beyond habits of fashion and expense, and being the younger son of a lord with a tolerable independence; and Sir~Thomas would probably have thought his introduction at Mansfield by no means desirable. Mr~Bertram's acquaintance with him had begun at Weymouth, where they had spent ten days together in the same society, and the friendship, if friendship it might be called, had been proved and perfected by Mr~Yates's being invited to take Mansfield in his way, whenever he could, and by his promising to come; and he did come rather earlier than had been expected, in consequence of the sudden breaking-up of a large party assembled for gaiety at the house of another friend, which he had left Weymouth to join. He came on the wings of disappointment, and with his head full of acting, for it had been a theatrical party; and the play in which he had borne a part was within two days of representation, when the sudden death of one of the nearest connexions of the family had destroyed the scheme and dispersed the performers. To be so near happiness, so near fame, so near the long paragraph in praise of the private theatricals at Ecclesford, the seat of the Right Hon. Lord Ravenshaw, in Cornwall, which would of course have immortalised the whole party for at least a twelvemonth! and being so near, to lose it all, was an injury to be keenly felt, and Mr~Yates could talk of nothing else. Ecclesford and its theatre, with its arrangements and dresses, rehearsals and jokes, was his never-failing subject, and to boast of the past his only consolation.

Happily for him, a love of the theatre is so general, an itch for acting so strong among young people, that he could hardly out-talk the interest of his hearers. From the first casting of the parts to the epilogue it was all bewitching, and there were few who did not wish to have been a party concerned, or would have hesitated to try their skill. The play had been Lovers' Vows, and Mr~Yates was to have been Count Cassel. <A trifling part,> said he, <and not at all to my taste, and such a one as I certainly would not accept again; but I was determined to make no difficulties. Lord Ravenshaw and the duke had appropriated the only two characters worth playing before I reached Ecclesford; and though Lord Ravenshaw offered to resign his to me, it was impossible to take it, you know. I was sorry for \textit{him}  that he should have so mistaken his powers, for he was no more equal to the Baron—a little man with a weak voice, always hoarse after the first ten minutes. It must have injured the piece materially; but \textit{I}  was resolved to make no difficulties. Sir~Henry thought the duke not equal to Frederick, but that was because Sir~Henry wanted the part himself; whereas it was certainly in the best hands of the two. I was surprised to see Sir~Henry such a stick. Luckily the strength of the piece did not depend upon him. Our Agatha was inimitable, and the duke was thought very great by many. And upon the whole, it would certainly have gone off wonderfully.>

<It was a hard case, upon my word>; and, <I do think you were very much to be pitied,> were the kind responses of listening sympathy.

<It is not worth complaining about; but to be sure the poor old dowager could not have died at a worse time; and it is impossible to help wishing that the news could have been suppressed for just the three days we wanted. It was but three days; and being only a grandmother, and all happening two hundred miles off, I think there would have been no great harm, and it was suggested, I know; but Lord Ravenshaw, who I suppose is one of the most correct men in England, would not hear of it.>

<An afterpiece instead of a comedy,> said Mr~Bertram. <Lovers' Vows were at an end, and Lord and Lady Ravenshaw left to act My Grandmother by themselves. Well, the jointure may comfort \textit{him}; and perhaps, between friends, he began to tremble for his credit and his lungs in the Baron, and was not sorry to withdraw; and to make \textit{you}  amends, Yates, I think we must raise a little theatre at Mansfield, and ask you to be our manager.>

This, though the thought of the moment, did not end with the moment; for the inclination to act was awakened, and in no one more strongly than in him who was now master of the house; and who, having so much leisure as to make almost any novelty a certain good, had likewise such a degree of lively talents and comic taste, as were exactly adapted to the novelty of acting. The thought returned again and again. <Oh for the Ecclesford theatre and scenery to try something with.> Each sister could echo the wish; and Henry Crawford, to whom, in all the riot of his gratifications it was yet an untasted pleasure, was quite alive at the idea. <I really believe,> said he, <I could be fool enough at this moment to undertake any character that ever was written, from Shylock or Richard III down to the singing hero of a farce in his scarlet coat and cocked hat. I feel as if I could be anything or everything; as if I could rant and storm, or sigh or cut capers, in any tragedy or comedy in the English language. Let us be doing something. Be it only half a play, an act, a scene; what should prevent us? Not these countenances, I am sure,> looking towards the Miss~Bertrams; <and for a theatre, what signifies a theatre? We shall be only amusing ourselves. Any room in this house might suffice.>

<We must have a curtain,> said Tom Bertram; <a few yards of green baize for a curtain, and perhaps that may be enough.>

<Oh, quite enough,> cried Mr~Yates, <with only just a side wing or two run up, doors in flat, and three or four scenes to be let down; nothing more would be necessary on such a plan as this. For mere amusement among ourselves we should want nothing more.>

<I believe we must be satisfied with \textit{less},> said Maria. <There would not be time, and other difficulties would arise. We must rather adopt Mr~Crawford's views, and make the \textit{performance}, not the \textit{theatre}, our object. Many parts of our best plays are independent of scenery.>

<Nay,> said Edmund, who began to listen with alarm. <Let us do nothing by halves. If we are to act, let it be in a theatre completely fitted up with pit, boxes, and gallery, and let us have a play entire from beginning to end; so as it be a German play, no matter what, with a good tricking, shifting afterpiece, and a figure-dance, and a hornpipe, and a song between the acts. If we do not outdo Ecclesford, we do nothing.>

<Now, Edmund, do not be disagreeable,> said Julia. <Nobody loves a play better than you do, or can have gone much farther to see one.>

<True, to see real acting, good hardened real acting; but I would hardly walk from this room to the next to look at the raw efforts of those who have not been bred to the trade: a set of gentlemen and ladies, who have all the disadvantages of education and decorum to struggle through.>

After a short pause, however, the subject still continued, and was discussed with unabated eagerness, every one's inclination increasing by the discussion, and a knowledge of the inclination of the rest; and though nothing was settled but that Tom Bertram would prefer a comedy, and his sisters and Henry Crawford a tragedy, and that nothing in the world could be easier than to find a piece which would please them all, the resolution to act something or other seemed so decided as to make Edmund quite uncomfortable. He was determined to prevent it, if possible, though his mother, who equally heard the conversation which passed at table, did not evince the least disapprobation.

The same evening afforded him an opportunity of trying his strength. Maria, Julia, Henry Crawford, and Mr~Yates were in the billiard-room. Tom, returning from them into the drawing-room, where Edmund was standing thoughtfully by the fire, while Lady Bertram was on the sofa at a little distance, and Fanny close beside her arranging her work, thus began as he entered—<Such a horribly vile billiard-table as ours is not to be met with, I believe, above ground. I can stand it no longer, and I think, I may say, that nothing shall ever tempt me to it again; but one good thing I have just ascertained: it is the very room for a theatre, precisely the shape and length for it; and the doors at the farther end, communicating with each other, as they may be made to do in five minutes, by merely moving the bookcase in my father's room, is the very thing we could have desired, if we had sat down to wish for it; and my father's room will be an excellent greenroom. It seems to join the billiard-room on purpose.>

<You are not serious, Tom, in meaning to act?> said Edmund, in a low voice, as his brother approached the fire.

<Not serious! never more so, I assure you. What is there to surprise you in it?>

<I think it would be very wrong. In a \textit{general}  light, private theatricals are open to some objections, but as \textit{we}  are circumstanced, I must think it would be highly injudicious, and more than injudicious to attempt anything of the kind. It would shew great want of feeling on my father's account, absent as he is, and in some degree of constant danger; and it would be imprudent, I think, with regard to Maria, whose situation is a very delicate one, considering everything, extremely delicate.>

<You take up a thing so seriously! as if we were going to act three times a week till my father's return, and invite all the country. But it is not to be a display of that sort. We mean nothing but a little amusement among ourselves, just to vary the scene, and exercise our powers in something new. We want no audience, no publicity. We may be trusted, I think, in chusing some play most perfectly unexceptionable; and I can conceive no greater harm or danger to any of us in conversing in the elegant written language of some respectable author than in chattering in words of our own. I have no fears and no scruples. And as to my father's being absent, it is so far from an objection, that I consider it rather as a motive; for the expectation of his return must be a very anxious period to my mother; and if we can be the means of amusing that anxiety, and keeping up her spirits for the next few weeks, I shall think our time very well spent, and so, I am sure, will he. It is a \textit{very}  anxious period for her.>

As he said this, each looked towards their mother. Lady Bertram, sunk back in one corner of the sofa, the picture of health, wealth, ease, and tranquillity, was just falling into a gentle doze, while Fanny was getting through the few difficulties of her work for her.

Edmund smiled and shook his head.

<By Jove! this won't do,> cried Tom, throwing himself into a chair with a hearty laugh. <To be sure, my dear mother, your anxiety—I was unlucky there.>

<What is the matter?> asked her ladyship, in the heavy tone of one half-roused; <I was not asleep.>

<Oh dear, no, ma'am, nobody suspected you! Well, Edmund,> he continued, returning to the former subject, posture, and voice, as soon as Lady Bertram began to nod again, <but \textit{this}  I \textit{will}  maintain, that we shall be doing no harm.>

<I cannot agree with you; I am convinced that my father would totally disapprove it.>

<And I am convinced to the contrary. Nobody is fonder of the exercise of talent in young people, or promotes it more, than my father, and for anything of the acting, spouting, reciting kind, I think he has always a decided taste. I am sure he encouraged it in us as boys. How many a time have we mourned over the dead body of Julius Caesar, and to \textit{be'd}  and not \textit{to be'd}, in this very room, for his amusement? And I am sure, \textit{my name was Norval}, every evening of my life through one Christmas holidays.>

<It was a very different thing. You must see the difference yourself. My father wished us, as schoolboys, to speak well, but he would never wish his grown-up daughters to be acting plays. His sense of decorum is strict.>

<I know all that,> said Tom, displeased. <I know my father as well as you do; and I'll take care that his daughters do nothing to distress him. Manage your own concerns, Edmund, and I'll take care of the rest of the family.>

<If you are resolved on acting,> replied the persevering Edmund, <I must hope it will be in a very small and quiet way; and I think a theatre ought not to be attempted. It would be taking liberties with my father's house in his absence which could not be justified.>

<For everything of that nature I will be answerable,> said Tom, in a decided tone. <His house shall not be hurt. I have quite as great an interest in being careful of his house as you can have; and as to such alterations as I was suggesting just now, such as moving a bookcase, or unlocking a door, or even as using the billiard-room for the space of a week without playing at billiards in it, you might just as well suppose he would object to our sitting more in this room, and less in the breakfast-room, than we did before he went away, or to my sister's pianoforte being moved from one side of the room to the other. Absolute nonsense!>

<The innovation, if not wrong as an innovation, will be wrong as an expense.>

<Yes, the expense of such an undertaking would be prodigious! Perhaps it might cost a whole twenty pounds. Something of a theatre we must have undoubtedly, but it will be on the simplest plan: a green curtain and a little carpenter's work, and that's all; and as the carpenter's work may be all done at home by Christopher Jackson himself, it will be too absurd to talk of expense; and as long as Jackson is employed, everything will be right with Sir~Thomas. Don't imagine that nobody in this house can see or judge but yourself. Don't act yourself, if you do not like it, but don't expect to govern everybody else.>

<No, as to acting myself,> said Edmund, <\textit{that}  I absolutely protest against.>

Tom walked out of the room as he said it, and Edmund was left to sit down and stir the fire in thoughtful vexation.

Fanny, who had heard it all, and borne Edmund company in every feeling throughout the whole, now ventured to say, in her anxiety to suggest some comfort, <Perhaps they may not be able to find any play to suit them. Your brother's taste and your sisters' seem very different.>

<I have no hope there, Fanny. If they persist in the scheme, they will find something. I shall speak to my sisters and try to dissuade \textit{them}, and that is all I can do.>

<I should think my aunt Norris would be on your side.>

<I dare say she would, but she has no influence with either Tom or my sisters that could be of any use; and if I cannot convince them myself, I shall let things take their course, without attempting it through her. Family squabbling is the greatest evil of all, and we had better do anything than be altogether by the ears.>

His sisters, to whom he had an opportunity of speaking the next morning, were quite as impatient of his advice, quite as unyielding to his representation, quite as determined in the cause of pleasure, as Tom. Their mother had no objection to the plan, and they were not in the least afraid of their father's disapprobation. There could be no harm in what had been done in so many respectable families, and by so many women of the first consideration; and it must be scrupulousness run mad that could see anything to censure in a plan like theirs, comprehending only brothers and sisters and intimate friends, and which would never be heard of beyond themselves. Julia \textit{did}  seem inclined to admit that Maria's situation might require particular caution and delicacy—but that could not extend to \textit{her}—she was at liberty; and Maria evidently considered her engagement as only raising her so much more above restraint, and leaving her less occasion than Julia to consult either father or mother. Edmund had little to hope, but he was still urging the subject when Henry Crawford entered the room, fresh from the Parsonage, calling out, <No want of hands in our theatre, Miss~Bertram. No want of understrappers: my sister desires her love, and hopes to be admitted into the company, and will be happy to take the part of any old duenna or tame confidante, that you may not like to do yourselves.>

Maria gave Edmund a glance, which meant, <What say you now? Can we be wrong if Mary Crawford feels the same?> And Edmund, silenced, was obliged to acknowledge that the charm of acting might well carry fascination to the mind of genius; and with the ingenuity of love, to dwell more on the obliging, accommodating purport of the message than on anything else.

The scheme advanced. Opposition was vain; and as to Mrs~Norris, he was mistaken in supposing she would wish to make any. She started no difficulties that were not talked down in five minutes by her eldest nephew and niece, who were all-powerful with her; and as the whole arrangement was to bring very little expense to anybody, and none at all to herself, as she foresaw in it all the comforts of hurry, bustle, and importance, and derived the immediate advantage of fancying herself obliged to leave her own house, where she had been living a month at her own cost, and take up her abode in theirs, that every hour might be spent in their service, she was, in fact, exceedingly delighted with the project. 