\chapter[Chapter \thechapter]{} 

 \lettrine[lraise=0.3]{E}{dmund} had great things to hear on his return. Many surprises were awaiting him. The first that occurred was not least in interest: the appearance of Henry Crawford and his sister walking together through the village as he rode into it. He had concluded—he had meant them to be far distant. His absence had been extended beyond a fortnight purposely to avoid Miss~Crawford. He was returning to Mansfield with spirits ready to feed on melancholy remembrances, and tender associations, when her own fair self was before him, leaning on her brother's arm, and he found himself receiving a welcome, unquestionably friendly, from the woman whom, two moments before, he had been thinking of as seventy miles off, and as farther, much farther, from him in inclination than any distance could express.

Her reception of him was of a sort which he could not have hoped for, had he expected to see her. Coming as he did from such a purport fulfilled as had taken him away, he would have expected anything rather than a look of satisfaction, and words of simple, pleasant meaning. It was enough to set his heart in a glow, and to bring him home in the properest state for feeling the full value of the other joyful surprises at hand.

William's promotion, with all its particulars, he was soon master of; and with such a secret provision of comfort within his own breast to help the joy, he found in it a source of most gratifying sensation and unvarying cheerfulness all dinner-time.

After dinner, when he and his father were alone, he had Fanny's history; and then all the great events of the last fortnight, and the present situation of matters at Mansfield were known to him.

Fanny suspected what was going on. They sat so much longer than usual in the dining-parlour, that she was sure they must be talking of her; and when tea at last brought them away, and she was to be seen by Edmund again, she felt dreadfully guilty. He came to her, sat down by her, took her hand, and pressed it kindly; and at that moment she thought that, but for the occupation and the scene which the tea-things afforded, she must have betrayed her emotion in some unpardonable excess.

He was not intending, however, by such action, to be conveying to her that unqualified approbation and encouragement which her hopes drew from it. It was designed only to express his participation in all that interested her, and to tell her that he had been hearing what quickened every feeling of affection. He was, in fact, entirely on his father's side of the question. His surprise was not so great as his father's at her refusing Crawford, because, so far from supposing her to consider him with anything like a preference, he had always believed it to be rather the reverse, and could imagine her to be taken perfectly unprepared, but Sir~Thomas could not regard the connexion as more desirable than he did. It had every recommendation to him; and while honouring her for what she had done under the influence of her present indifference, honouring her in rather stronger terms than Sir~Thomas could quite echo, he was most earnest in hoping, and sanguine in believing, that it would be a match at last, and that, united by mutual affection, it would appear that their dispositions were as exactly fitted to make them blessed in each other, as he was now beginning seriously to consider them. Crawford had been too precipitate. He had not given her time to attach herself. He had begun at the wrong end. With such powers as his, however, and such a disposition as hers, Edmund trusted that everything would work out a happy conclusion. Meanwhile, he saw enough of Fanny's embarrassment to make him scrupulously guard against exciting it a second time, by any word, or look, or movement.

Crawford called the next day, and on the score of Edmund's return, Sir~Thomas felt himself more than licensed to ask him to stay dinner; it was really a necessary compliment. He staid of course, and Edmund had then ample opportunity for observing how he sped with Fanny, and what degree of immediate encouragement for him might be extracted from her manners; and it was so little, so very, very little—every chance, every possibility of it, resting upon her embarrassment only; if there was not hope in her confusion, there was hope in nothing else—that he was almost ready to wonder at his friend's perseverance. Fanny was worth it all; he held her to be worth every effort of patience, every exertion of mind, but he did not think he could have gone on himself with any woman breathing, without something more to warm his courage than his eyes could discern in hers. He was very willing to hope that Crawford saw clearer, and this was the most comfortable conclusion for his friend that he could come to from all that he observed to pass before, and at, and after dinner.

In the evening a few circumstances occurred which he thought more promising. When he and Crawford walked into the drawing-room, his mother and Fanny were sitting as intently and silently at work as if there were nothing else to care for. Edmund could not help noticing their apparently deep tranquillity.

<We have not been so silent all the time,> replied his mother. <Fanny has been reading to me, and only put the book down upon hearing you coming.> And sure enough there was a book on the table which had the air of being very recently closed: a volume of Shakespeare. <She often reads to me out of those books; and she was in the middle of a very fine speech of that man's—what's his name, Fanny?—when we heard your footsteps.>

Crawford took the volume. <Let me have the pleasure of finishing that speech to your ladyship,> said he. <I shall find it immediately.> And by carefully giving way to the inclination of the leaves, he did find it, or within a page or two, quite near enough to satisfy Lady Bertram, who assured him, as soon as he mentioned the name of Cardinal Wolsey, that he had got the very speech. Not a look or an offer of help had Fanny given; not a syllable for or against. All her attention was for her work. She seemed determined to be interested by nothing else. But taste was too strong in her. She could not abstract her mind five minutes: she was forced to listen; his reading was capital, and her pleasure in good reading extreme. To \textit{good}  reading, however, she had been long used: her uncle read well, her cousins all, Edmund very well, but in Mr~Crawford's reading there was a variety of excellence beyond what she had ever met with. The King, the Queen, Buckingham, Wolsey, Cromwell, all were given in turn; for with the happiest knack, the happiest power of jumping and guessing, he could always alight at will on the best scene, or the best speeches of each; and whether it were dignity, or pride, or tenderness, or remorse, or whatever were to be expressed, he could do it with equal beauty. It was truly dramatic. His acting had first taught Fanny what pleasure a play might give, and his reading brought all his acting before her again; nay, perhaps with greater enjoyment, for it came unexpectedly, and with no such drawback as she had been used to suffer in seeing him on the stage with Miss~Bertram.

Edmund watched the progress of her attention, and was amused and gratified by seeing how she gradually slackened in the needlework, which at the beginning seemed to occupy her totally: how it fell from her hand while she sat motionless over it, and at last, how the eyes which had appeared so studiously to avoid him throughout the day were turned and fixed on Crawford—fixed on him for minutes, fixed on him, in short, till the attraction drew Crawford's upon her, and the book was closed, and the charm was broken. Then she was shrinking again into herself, and blushing and working as hard as ever; but it had been enough to give Edmund encouragement for his friend, and as he cordially thanked him, he hoped to be expressing Fanny's secret feelings too.

<That play must be a favourite with you,> said he; <you read as if you knew it well.>

<It will be a favourite, I believe, from this hour,> replied Crawford; <but I do not think I have had a volume of Shakespeare in my hand before since I was fifteen. I once saw Henry the Eighth acted, or I have heard of it from somebody who did, I am not certain which. But Shakespeare one gets acquainted with without knowing how. It is a part of an Englishman's constitution. His thoughts and beauties are so spread abroad that one touches them everywhere; one is intimate with him by instinct. No man of any brain can open at a good part of one of his plays without falling into the flow of his meaning immediately.>

<No doubt one is familiar with Shakespeare in a degree,> said Edmund, <from one's earliest years. His celebrated passages are quoted by everybody; they are in half the books we open, and we all talk Shakespeare, use his similes, and describe with his descriptions; but this is totally distinct from giving his sense as you gave it. To know him in bits and scraps is common enough; to know him pretty thoroughly is, perhaps, not uncommon; but to read him well aloud is no everyday talent.>

<Sir, you do me honour,> was Crawford's answer, with a bow of mock gravity.

Both gentlemen had a glance at Fanny, to see if a word of accordant praise could be extorted from her; yet both feeling that it could not be. Her praise had been given in her attention; \textit{that}  must content them.

Lady Bertram's admiration was expressed, and strongly too. <It was really like being at a play,> said she. <I wish Sir~Thomas had been here.>

Crawford was excessively pleased. If Lady Bertram, with all her incompetency and languor, could feel this, the inference of what her niece, alive and enlightened as she was, must feel, was elevating.

<You have a great turn for acting, I am sure, Mr~Crawford,> said her ladyship soon afterwards; <and I will tell you what, I think you will have a theatre, some time or other, at your house in Norfolk. I mean when you are settled there. I do indeed. I think you will fit up a theatre at your house in Norfolk.>

<Do you, ma'am?> cried he, with quickness. <No, no, that will never be. Your ladyship is quite mistaken. No theatre at Everingham! Oh no!> And he looked at Fanny with an expressive smile, which evidently meant, <That lady will never allow a theatre at Everingham.>

Edmund saw it all, and saw Fanny so determined \textit{not}  to see it, as to make it clear that the voice was enough to convey the full meaning of the protestation; and such a quick consciousness of compliment, such a ready comprehension of a hint, he thought, was rather favourable than not.

The subject of reading aloud was farther discussed. The two young men were the only talkers, but they, standing by the fire, talked over the too common neglect of the qualification, the total inattention to it, in the ordinary school-system for boys, the consequently natural, yet in some instances almost unnatural, degree of ignorance and uncouthness of men, of sensible and well-informed men, when suddenly called to the necessity of reading aloud, which had fallen within their notice, giving instances of blunders, and failures with their secondary causes, the want of management of the voice, of proper modulation and emphasis, of foresight and judgment, all proceeding from the first cause: want of early attention and habit; and Fanny was listening again with great entertainment.

<Even in my profession,> said Edmund, with a smile, <how little the art of reading has been studied! how little a clear manner, and good delivery, have been attended to! I speak rather of the past, however, than the present. There is now a spirit of improvement abroad; but among those who were ordained twenty, thirty, forty years ago, the larger number, to judge by their performance, must have thought reading was reading, and preaching was preaching. It is different now. The subject is more justly considered. It is felt that distinctness and energy may have weight in recommending the most solid truths; and besides, there is more general observation and taste, a more critical knowledge diffused than formerly; in every congregation there is a larger proportion who know a little of the matter, and who can judge and criticise.>

Edmund had already gone through the service once since his ordination; and upon this being understood, he had a variety of questions from Crawford as to his feelings and success; questions, which being made, though with the vivacity of friendly interest and quick taste, without any touch of that spirit of banter or air of levity which Edmund knew to be most offensive to Fanny, he had true pleasure in satisfying; and when Crawford proceeded to ask his opinion and give his own as to the properest manner in which particular passages in the service should be delivered, shewing it to be a subject on which he had thought before, and thought with judgment, Edmund was still more and more pleased. This would be the way to Fanny's heart. She was not to be won by all that gallantry and wit and good-nature together could do; or, at least, she would not be won by them nearly so soon, without the assistance of sentiment and feeling, and seriousness on serious subjects.

<Our liturgy,> observed Crawford, <has beauties, which not even a careless, slovenly style of reading can destroy; but it has also redundancies and repetitions which require good reading not to be felt. For myself, at least, I must confess being not always so attentive as I ought to be> (here was a glance at Fanny); <that nineteen times out of twenty I am thinking how such a prayer ought to be read, and longing to have it to read myself. Did you speak?> stepping eagerly to Fanny, and addressing her in a softened voice; and upon her saying <No,> he added, <Are you sure you did not speak? I saw your lips move. I fancied you might be going to tell me I ought to be more attentive, and not \textit{allow}  my thoughts to wander. Are not you going to tell me so?>

<No, indeed, you know your duty too well for me to—even supposing\longdash>

She stopt, felt herself getting into a puzzle, and could not be prevailed on to add another word, not by dint of several minutes of supplication and waiting. He then returned to his former station, and went on as if there had been no such tender interruption.

<A sermon, well delivered, is more uncommon even than prayers well read. A sermon, good in itself, is no rare thing. It is more difficult to speak well than to compose well; that is, the rules and trick of composition are oftener an object of study. A thoroughly good sermon, thoroughly well delivered, is a capital gratification. I can never hear such a one without the greatest admiration and respect, and more than half a mind to take orders and preach myself. There is something in the eloquence of the pulpit, when it is really eloquence, which is entitled to the highest praise and honour. The preacher who can touch and affect such an heterogeneous mass of hearers, on subjects limited, and long worn threadbare in all common hands; who can say anything new or striking, anything that rouses the attention without offending the taste, or wearing out the feelings of his hearers, is a man whom one could not, in his public capacity, honour enough. I should like to be such a man.>

Edmund laughed.

<I should indeed. I never listened to a distinguished preacher in my life without a sort of envy. But then, I must have a London audience. I could not preach but to the educated; to those who were capable of estimating my composition. And I do not know that I should be fond of preaching often; now and then, perhaps once or twice in the spring, after being anxiously expected for half a dozen Sundays together; but not for a constancy; it would not do for a constancy.>

Here Fanny, who could not but listen, involuntarily shook her head, and Crawford was instantly by her side again, entreating to know her meaning; and as Edmund perceived, by his drawing in a chair, and sitting down close by her, that it was to be a very thorough attack, that looks and undertones were to be well tried, he sank as quietly as possible into a corner, turned his back, and took up a newspaper, very sincerely wishing that dear little Fanny might be persuaded into explaining away that shake of the head to the satisfaction of her ardent lover; and as earnestly trying to bury every sound of the business from himself in murmurs of his own, over the various advertisements of <A most desirable Estate in South Wales>; <To Parents and Guardians>; and a <Capital season'd Hunter.>

Fanny, meanwhile, vexed with herself for not having been as motionless as she was speechless, and grieved to the heart to see Edmund's arrangements, was trying by everything in the power of her modest, gentle nature, to repulse Mr~Crawford, and avoid both his looks and inquiries; and he, unrepulsable, was persisting in both.

<What did that shake of the head mean?> said he. <What was it meant to express? Disapprobation, I fear. But of what? What had I been saying to displease you? Did you think me speaking improperly, lightly, irreverently on the subject? Only tell me if I was. Only tell me if I was wrong. I want to be set right. Nay, nay, I entreat you; for one moment put down your work. What did that shake of the head mean?>

In vain was her <Pray, sir, don't; pray, Mr~Crawford,> repeated twice over; and in vain did she try to move away. In the same low, eager voice, and the same close neighbourhood, he went on, reurging the same questions as before. She grew more agitated and displeased.

<How can you, sir? You quite astonish me; I wonder how you can\longdash>

<Do I astonish you?> said he. <Do you wonder? Is there anything in my present entreaty that you do not understand? I will explain to you instantly all that makes me urge you in this manner, all that gives me an interest in what you look and do, and excites my present curiosity. I will not leave you to wonder long.>

In spite of herself, she could not help half a smile, but she said nothing.

<You shook your head at my acknowledging that I should not like to engage in the duties of a clergyman always for a constancy. Yes, that was the word. Constancy: I am not afraid of the word. I would spell it, read it, write it with anybody. I see nothing alarming in the word. Did you think I ought?>

<Perhaps, sir,> said Fanny, wearied at last into speaking—<perhaps, sir, I thought it was a pity you did not always know yourself as well as you seemed to do at that moment.>

Crawford, delighted to get her to speak at any rate, was determined to keep it up; and poor Fanny, who had hoped to silence him by such an extremity of reproof, found herself sadly mistaken, and that it was only a change from one object of curiosity and one set of words to another. He had always something to entreat the explanation of. The opportunity was too fair. None such had occurred since his seeing her in her uncle's room, none such might occur again before his leaving Mansfield. Lady Bertram's being just on the other side of the table was a trifle, for she might always be considered as only half-awake, and Edmund's advertisements were still of the first utility.

<Well,> said Crawford, after a course of rapid questions and reluctant answers; <I am happier than I was, because I now understand more clearly your opinion of me. You think me unsteady: easily swayed by the whim of the moment, easily tempted, easily put aside. With such an opinion, no wonder that—But we shall see.—It is not by protestations that I shall endeavour to convince you I am wronged; it is not by telling you that my affections are steady. My conduct shall speak for me; absence, distance, time shall speak for me. \textit{They}  shall prove that, as far as you can be deserved by anybody, I do deserve you. You are infinitely my superior in merit; all \textit{that}  I know. You have qualities which I had not before supposed to exist in such a degree in any human creature. You have some touches of the angel in you beyond what—not merely beyond what one sees, because one never sees anything like it—but beyond what one fancies might be. But still I am not frightened. It is not by equality of merit that you can be won. That is out of the question. It is he who sees and worships your merit the strongest, who loves you most devotedly, that has the best right to a return. There I build my confidence. By that right I do and will deserve you; and when once convinced that my attachment is what I declare it, I know you too well not to entertain the warmest hopes. Yes, dearest, sweetest Fanny. Nay> (seeing her draw back displeased), <forgive me. Perhaps I have as yet no right; but by what other name can I call you? Do you suppose you are ever present to my imagination under any other? No, it is <Fanny> that I think of all day, and dream of all night. You have given the name such reality of sweetness, that nothing else can now be descriptive of you.>

Fanny could hardly have kept her seat any longer, or have refrained from at least trying to get away in spite of all the too public opposition she foresaw to it, had it not been for the sound of approaching relief, the very sound which she had been long watching for, and long thinking strangely delayed.

The solemn procession, headed by Baddeley, of tea-board, urn, and cake-bearers, made its appearance, and delivered her from a grievous imprisonment of body and mind. Mr~Crawford was obliged to move. She was at liberty, she was busy, she was protected.

Edmund was not sorry to be admitted again among the number of those who might speak and hear. But though the conference had seemed full long to him, and though on looking at Fanny he saw rather a flush of vexation, he inclined to hope that so much could not have been said and listened to without some profit to the speaker. 